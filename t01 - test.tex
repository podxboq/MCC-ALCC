\begin{questions}

  \question El conjugado de $z = 3 - 2i$ es:

  \begin{choices}
    \choice $-3 + 2i$
    \CorrectChoice $3 + 2i$
    \choice $-3 - 2i$
    \choice $2 + 3i$
  \end{choices}
  \begin{solution}
    El conjugado de $z = a + bi$ se obtiene cambiando el signo de la parte imaginaria, por tanto $\conj{3 - 2i} = 3 + 2i$.
  \end{solution}

  \question El módulo del número complejo $z = 3 + 4i$ es:

  \begin{choices}
    \choice $7$
    \choice $\sqrt{7}$
    \CorrectChoice $5$
    \choice $25$
  \end{choices}
  \begin{solution}
    El módulo se calcula como $|z| = \sqrt{3^2 + 4^2} = \sqrt{9 + 16} = \sqrt{25} = 5$.
  \end{solution}

  \question La fórmula de Euler establece que:

  \begin{choices}
    \choice $e^{i\theta} = \sin\theta + i\cos\theta$
    \CorrectChoice $e^{i\theta} = \cos\theta + i\sin\theta$
    \choice $e^{i\theta} = \cos\theta - i\sin\theta$
    \choice $e^{i\theta} = i\cos\theta + \sin\theta$
  \end{choices}
  \begin{solution}
    La fórmula de Euler es $e^{i\theta} = \cos\theta + i\sin\theta$, que relaciona la función exponencial compleja con las funciones trigonométricas.
  \end{solution}

  \question El producto $(1 + i)(2 - i)$ es igual a:

  \begin{choices}
    \choice $1 + i$
    \choice $2 - i$
    \CorrectChoice $3 + i$
    \choice $3 - i$
  \end{choices}
  \begin{solution}
    Aplicando la propiedad distributiva: $(1 + i)(2 - i) = 2 - i + 2i - i^2 = 2 + i - (-1) = 3 + i$.
  \end{solution}

  \question Las raíces cuartas de la unidad son:

  \begin{choices}
    \choice $1, -1, i, -i, e^{i\pi/4}$
    \CorrectChoice $1, -1, i, -i$
    \choice $1, i, -i, e^{i\pi/2}$
    \choice $1, -1, 2i, -2i$
  \end{choices}
  \begin{solution}
    Las raíces $n$-ésimas de la unidad son $e^{2\pi ik/n}$ para $k = 0, 1, \ldots, n-1$. Para $n = 4$: $e^{0} = 1$, $e^{i\pi/2} = i$, $e^{i\pi} = -1$, $e^{3i\pi/2} = -i$.
  \end{solution}

  \question El número complejo $z = 1 + i$ en forma exponencial es:

  \begin{choices}
    \choice $\sqrt{2}e^{i\pi}$
    \CorrectChoice $\sqrt{2}e^{i\pi/4}$
    \choice $2e^{i\pi/4}$
    \choice $e^{i\pi/2}$
  \end{choices}
  \begin{solution}
    El módulo es $|z| = \sqrt{1^2 + 1^2} = \sqrt{2}$. El argumento es $\theta = \arctan(1/1) = \pi/4$. Por tanto, $z = \sqrt{2}e^{i\pi/4}$.
  \end{solution}

  \question Si $z = 2e^{i\pi/3}$, entonces $z^3$ es igual a:

  \begin{choices}
    \choice $6e^{i\pi}$
    \CorrectChoice $-8$
    \choice $8e^{i\pi/9}$
    \choice $2e^{i\pi}$
  \end{choices}
  \begin{solution}
    Por la fórmula de De Moivre: $z^3 = (2e^{i\pi/3})^3 = 2^3 e^{i3\pi/3} = 8e^{i\pi} = -8$.
  \end{solution}

  \question ¿Cual es el valor de $\cos(i)$?

  \begin{choices}
    \choice $-1$
    \CorrectChoice $\frac{1+e^2}{2e}$
    \choice $i\cos(e)$
    \choice $1+\sin(e)$
  \end{choices}
  \begin{solution}
    Por la fórmula de Euler: $\cos(i) = \frac{e^{i^2} + e^{-i^2}}{2} = \frac{e^{-1} + e^{1}}{2} = \frac{1/e + e}{2} = \frac{1+e^2}{2e}$.
  \end{solution}

  \question Si $f(a+bi) = ia$ es una función compleja con valores complejos, entonces $\bar{f}$:

  \begin{choices}
    \choice $\bar{f}$ es una función real de variable compleja.
    \CorrectChoice $\bar{f}$ es una función compleja de variable compleja.
    \choice $\bar{f}$ es una función real de variable real.
    \choice $\bar{f}$ es una función compleja de variable real.
  \end{choices}
  \begin{solution}
    Por definición de conjugado de una función, $\bar{f}(a+bi) = \overline{f(a+bi)} = \overline{ia} = -ia$. Por lo tanto, $\bar{f}$ es una función compleja de variable compleja.
  \end{solution}

  \question Sea $f$ un función compleja de variable compleja, simétrica respecto del eje real, y sea $g$ la función definida por $g(z) = if(z)$. Entonces $g$ cumple que:

  \begin{choices}
    \choice Es antisimétrica respecto del eje real.
    \choice Es antisimétrica respecto del eje imaginario.
    \CorrectChoice Es simétrica respecto del eje real.
    \choice Es simétrica respecto del eje imaginario.
  \end{choices}
  \begin{solution}
    Por definición de conjugado, $g(\bar{z}) = if(\bar{z}) = if(z)=g(z)$. Por lo tanto, $g$ es simétrica respecto del eje real.
  \end{solution}
\end{questions}