\begin{questions}

  \question El conjugado de $z = 3 - 2i$ es:

  \begin{choices}
    \choice $-3 + 2i$
    \CorrectChoice $3 + 2i$
    \choice $-3 - 2i$
    \choice $2 + 3i$
  \end{choices}
  \begin{solution}
    El conjugado de $z = a + bi$ se obtiene cambiando el signo de la parte imaginaria, por tanto $\conj{3 - 2i} = 3 + 2i$.
  \end{solution}

  \question El módulo del número complejo $z = 3 + 4i$ es:

  \begin{choices}
    \choice $7$
    \choice $\sqrt{7}$
    \CorrectChoice $5$
    \choice $25$
  \end{choices}
  \begin{solution}
    El módulo se calcula como $|z| = \sqrt{3^2 + 4^2} = \sqrt{9 + 16} = \sqrt{25} = 5$.
  \end{solution}

  \question La fórmula de Euler establece que:

  \begin{choices}
    \choice $e^{i\theta} = \sin\theta + i\cos\theta$
    \CorrectChoice $e^{i\theta} = \cos\theta + i\sin\theta$
    \choice $e^{i\theta} = \cos\theta - i\sin\theta$
    \choice $e^{i\theta} = i\cos\theta + \sin\theta$
  \end{choices}
  \begin{solution}
    La fórmula de Euler es $e^{i\theta} = \cos\theta + i\sin\theta$, que relaciona la función exponencial compleja con las funciones trigonométricas.
  \end{solution}

  \question El producto $(1 + i)(2 - i)$ es igual a:

  \begin{choices}
    \choice $1 + i$
    \choice $2 - i$
    \CorrectChoice $3 + i$
    \choice $3 - i$
  \end{choices}
  \begin{solution}
    Aplicando la propiedad distributiva: $(1 + i)(2 - i) = 2 - i + 2i - i^2 = 2 + i - (-1) = 3 + i$.
  \end{solution}

  \question Las raíces cuartas de la unidad son:

  \begin{choices}
    \choice $1, -1, i, -i, e^{i\pi/4}$
    \CorrectChoice $1, -1, i, -i$
    \choice $1, i, -i, e^{i\pi/2}$
    \choice $1, -1, 2i, -2i$
  \end{choices}
  \begin{solution}
    Las raíces $n$-ésimas de la unidad son $e^{2\pi ik/n}$ para $k = 0, 1, \ldots, n-1$. Para $n = 4$: $e^{0} = 1$, $e^{i\pi/2} = i$, $e^{i\pi} = -1$, $e^{3i\pi/2} = -i$.
  \end{solution}

  \question Una amplitud cuántica $\alpha = \frac{3}{5} + \frac{4i}{5}$ tiene probabilidad asociada:

  \begin{choices}
    \choice $\frac{7}{5}$
    \CorrectChoice $1$
    \choice $\frac{9}{25}$
    \choice $\frac{16}{25}$
  \end{choices}
  \begin{solution}
    La probabilidad es $|\alpha|^2 = \left(\frac{3}{5}\right)^2 + \left(\frac{4}{5}\right)^2 = \frac{9}{25} + \frac{16}{25} = 1$.
  \end{solution}

  \question Para que dos amplitudes $\alpha_1 = \frac{1}{\sqrt{2}}e^{i\theta_1}$ y $\alpha_2 = \frac{1}{\sqrt{2}}e^{i\theta_2}$ interfieran destructivamente, la diferencia de fases debe ser:

  \begin{choices}
    \choice $0$
    \choice $\frac{\pi}{2}$
    \CorrectChoice $\pi$
    \choice $\frac{\pi}{4}$
  \end{choices}
  \begin{solution}
    Para interferencia destructiva necesitamos que $\alpha_1 + \alpha_2 = 0$, lo que ocurre cuando las fases difieren en $\pi$: $e^{i\theta_1} + e^{i(\theta_1 + \pi)} = e^{i\theta_1}(1 - 1) = 0$.
  \end{solution}

  \question El número complejo $z = 1 + i$ en forma exponencial es:

  \begin{choices}
    \choice $\sqrt{2}e^{i\pi}$
    \CorrectChoice $\sqrt{2}e^{i\pi/4}$
    \choice $2e^{i\pi/4}$
    \choice $e^{i\pi/2}$
  \end{choices}
  \begin{solution}
    El módulo es $|z| = \sqrt{1^2 + 1^2} = \sqrt{2}$. El argumento es $\theta = \arctan(1/1) = \pi/4$. Por tanto, $z = \sqrt{2}e^{i\pi/4}$.
  \end{solution}

  \question Si $z = 2e^{i\pi/3}$, entonces $z^3$ es igual a:

  \begin{choices}
    \choice $6e^{i\pi}$
    \CorrectChoice $8e^{i\pi} = -8$
    \choice $8e^{i\pi/9}$
    \choice $2e^{i\pi}$
  \end{choices}
  \begin{solution}
    Por la fórmula de De Moivre: $z^3 = (2e^{i\pi/3})^3 = 2^3 e^{i3\pi/3} = 8e^{i\pi} = -8$.
  \end{solution}

  \question Dos amplitudes cuánticas con magnitudes $\frac{1}{\sqrt{2}}$ y fases que difieren en $\frac{\pi}{2}$ al sumarse tienen probabilidad total:

  \begin{choices}
    \choice $0$
    \choice $\frac{1}{2}$
    \CorrectChoice $1$
    \choice $\frac{1}{4}$
  \end{choices}
  \begin{solution}
    Si $\alpha_1 = \frac{1}{\sqrt{2}}$ y $\alpha_2 = \frac{1}{\sqrt{2}}e^{i\pi/2} = \frac{i}{\sqrt{2}}$, entonces $|\alpha_1 + \alpha_2|^2 = \left|\frac{1 + i}{\sqrt{2}}\right|^2 = \frac{|1 + i|^2}{2} = \frac{2}{2} = 1$.
  \end{solution}

\end{questions}