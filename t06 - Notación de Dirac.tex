\portada

\begin{esquemaExplorador}
  \temaEsquema{Notación bra-ket}{
    \conceptoEsquema{Kets y bras}
    \conceptoEsquema{Productos internos}
    \conceptoEsquema{Operadores externos}
  }
  \temaEsquema{Estados cuánticos}{
    \conceptoEsquema{cúbits y normalización}
    \conceptoEsquema{Superposición cuántica}
    \conceptoEsquema{Interpretación probabilística}
  }
  \temaEsquema{Sistemas de múltiples cúbits}{
    \conceptoEsquema{Producto tensorial}
    \conceptoEsquema{Base computacional}
    \conceptoEsquema{Estados entrelazados}
  }
  \temaEsquema{Mediciones y observables}{
    \conceptoEsquema{Regla de Born}
    \conceptoEsquema{Colapso del estado}
    \conceptoEsquema{Medidas en diferentes bases}
  }
  \temaEsquema{Dinámicas cuánticas}{
    \conceptoEsquema{Acción de operadores}
    \conceptoEsquema{Evolución libre}
  }
\end{esquemaExplorador}

\unirsection{Ideas clave}

\subsection{Introducción y objetivos}

La notación de Dirac, también conocida como notación bra-ket, proporciona un formalismo elegante y poderoso para trabajar con estados cuánticos y operadores. Desarrollada por Paul Dirac, esta notación no solo simplifica los cálculos algebraicos, sino que también captura de manera intuitiva los conceptos físicos fundamentales de la mecánica cuántica.

En computación cuántica, la notación de Dirac es indispensable porque:

\begin{itemize}
  \item Proporciona una representación \textbf{concisa y clara} de estados cuánticos complejos
  \item Facilita el cálculo de \textbf{probabilidades cuánticas} mediante productos internos
  \item Permite expresar \textbf{operadores cuánticos} de manera natural y eficiente
  \item Conecta directamente la \textbf{estructura matemática} con la \textbf{interpretación física}
  \item Es el lenguaje estándar para describir \textbf{algoritmos cuánticos} y \textbf{protocolos cuánticos}
\end{itemize}

Este tema establece el puente definitivo entre la matemática abstracta de los espacios de Hilbert y la implementación práctica de sistemas cuánticos. Desarrollaremos desde los conceptos más básicos de la notación hasta aplicaciones avanzadas en sistemas de múltiples cúbits, proporcionando la base para la notación para todos los temas posteriores del curso.

\subsection{Fundamentos de la notación bra-ket}

\begin{defi}[Ket]
  Un \textbf{ket}, denotado $\ket{\psi}$, representa un vector unitario en un espacio de Hilbert complejo. Formalmente, $\ket{\psi} \in \H$ donde $\H$ es el espacio de estados del sistema cuántico.
\end{defi}

La base canónica $\set{e_1, e_2, \ldots, e_n}$ de $\C^n$, ahora se denota usando la notación de Dirac como los kets $\set{\ket{1}, \ket{2}, \ldots, \ket{n}}$ de $\H$. Aunque algunas veces es más conveniente usar la notación $\set{\ket{0}, \ket{1}, \ldots, \ket{n-1}}$.

En términos de coordenadas, si $\{\ket{i}\}$ es una base ortonormal:
$$\ket{\psi} = \sum_i c_i \ket{i}\so c_i = \braket{\psi}{i}$$
donde $c_i \in \C$ son las llamadas \textbf{amplitudes de probabilidad}.

\begin{defi}[Bracket]
  El \textbf{bracket} es el producto interno entre dos kets $\ket{\phi}$ y $\ket{\psi}$, se denota $\braket{\phi}{\psi}$:
  $$\braket{\phi}{\psi} = \langle \phi , \psi \rangle \in \C\,.$$
\end{defi}

Recordando las definiciones del dual de un espacio vectorial, dado un ket $\ket{\phi}$, el funcional asociado es un elemento del espacio dual $\H^*$, así, el producto interno se puede interpretar como la acción del funcional asociado.

\begin{defi}[Bra]
  Dado un ket $\ket{\phi}$, el funcional asociado del espacio dual lo llamamos \textbf{bra}, denotado $\bra{\phi}$ y que actúa sobre un ket $\ket{\psi}$ mediante la notación:
  $$\braket{\phi}{\psi} = \bra{\phi}(\ket{\psi})\,.$$
\end{defi}

Para cada ket $\ket{\psi}$, el bra correspondiente a nivel de coeficiente respecto de una base ortogonal y su base dual se cumple:
$$\bra{\psi} = \ket{\psi}^\dagger\,.$$

\begin{eje}[Notación básica en $\C^2$]
  Para un cúbit, los estados base se escriben:
  $$\ket{0} = \begin{pmatrix} 1 \\ 0 \end{pmatrix}, \quad \ket{1} = \begin{pmatrix} 0 \\ 1 \end{pmatrix}$$

  Los bras correspondientes son:
  $$\bra{0} = \begin{pmatrix} 1 & 0 \end{pmatrix}, \quad \bra{1} = \begin{pmatrix} 0 & 1 \end{pmatrix}$$

  Los productos internos fundamentales:
  \begin{align*}
    \braket{0}{0} & = 1, \quad \braket{1}{1} = 1 \\
    \braket{0}{1} & = 0, \quad \braket{1}{0} = 0
  \end{align*}
\end{eje}

\begin{defi}[Operador externo]
  El producto externo de un ket y un bra se denota $\ketbra{\psi}{\phi}$ y representa el operador lineal definido por:
  \[
    \ketbra{\psi}{\phi} (\ket{\chi}) = \braket{\phi}{\chi}\ket{\psi}
  \]
\end{defi}

Normalmente, no se escriben los paréntesis y se entiende que el operador actúa sobre un ket a su derecha.
\[
  \ketbra{\psi}{\phi} \ket{\chi} = \braket{\phi}{\chi}\ket{\psi}
\]

\begin{eje}[Operadores de proyección]
  Los operadores de proyección sobre los estados base son:
  \begin{align*}
    P_0 = \ketbra{0}{0} & = \begin{pmatrix} 1 \\ 0 \end{pmatrix} \begin{pmatrix} 1 & 0 \end{pmatrix} = \begin{pmatrix} 1 & 0 \\ 0 & 0 \end{pmatrix} \\
    P_1 = \ketbra{1}{1} & = \begin{pmatrix} 0 \\ 1 \end{pmatrix} \begin{pmatrix} 0 & 1 \end{pmatrix} = \begin{pmatrix} 0 & 0 \\ 0 & 1 \end{pmatrix}
  \end{align*}

  Se verifica que $P_0 + P_1 = I$ (relación de completitud).
\end{eje}

\begin{theo}[Relación de completitud]
  Para cualquier base ortonormal $\{\ket{i}\}$ de un espacio de Hilbert de dimensión finita:
  $$\sum_i \ketbra{i}{i} = I$$

  Esta identidad permite expresar cualquier operador o estado en términos de la base elegida.
\end{theo}

\subsection{Estados cuánticos y cúbits}

\begin{defi}[Estado cuántico puro]
  Un estado cuántico puro se representa mediante un ket normalizado:
  $$\ket{\psi} \in \H, \quad \braket{\psi}{\psi} = 1$$

  Para un cúbit, el estado general es:
  $$\ket{\psi} = \alpha \ket{0} + \beta \ket{1}$$
  donde $\alpha, \beta \in \C$ y $|\alpha|^2 + |\beta|^2 = 1$.
\end{defi}

\begin{eje}[Estados cuánticos importantes]
  \begin{enumerate}
    \item \textbf{Estados computacionales:}
          $$\ket{0} = \begin{pmatrix} 1 \\ 0 \end{pmatrix}, \quad \ket{1} = \begin{pmatrix} 0 \\ 1 \end{pmatrix}$$

    \item \textbf{Estados de superposición balanceada:}
          $$\ket{+} = \frac{\ket{0} + \ket{1}}{\sqrt{2}}, \quad \ket{-} = \frac{\ket{0} - \ket{1}}{\sqrt{2}}$$

    \item \textbf{Estados con fase compleja:}
          $$\ket{i+} = \frac{\ket{0} + i\ket{1}}{\sqrt{2}}, \quad \ket{i-} = \frac{\ket{0} - i\ket{1}}{\sqrt{2}}$$
  \end{enumerate}
\end{eje}

\begin{defi}[Interpretación probabilística - Regla de Born]
  Si un sistema se encuentra en el estado $\ket{\psi}$ y se mide un observable con vectores propios ortonormales $\{\ket{\phi_i}\}$, la probabilidad de obtener el resultado correspondiente al estado $\ket{\phi_i}$ es:
  $$P(\phi_i) = |\braket{\psi}{\phi_i}|^2$$
\end{defi}

\begin{eje}[Cálculo de probabilidades]
  Para el estado $\ket{\psi} = \frac{3}{5}\ket{0} + \frac{4i}{5}\ket{1}$:

  \textbf{Verificación de normalización:}
  $$\braket{\psi}{\psi} = \left|\frac{3}{5}\right|^2 + \left|\frac{4i}{5}\right|^2 = \frac{9}{25} + \frac{16}{25} = 1$$

  \textbf{Probabilidades en la base computacional:}
  \begin{align*}
    P(0) & = |\braket{\psi}{0}|^2 = \left|\frac{3}{5}\right|^2 = \frac{9}{25}   \\
    P(1) & = |\braket{\psi}{1}|^2 = \left|\frac{4i}{5}\right|^2 = \frac{16}{25}
  \end{align*}

  \textbf{Probabilidades en la base $\{\ket{+}, \ket{-}\}$:}
  $$\braket{\psi}{+} = \frac{1}{\sqrt{2}}\left(\frac{3}{5} + \frac{4i}{5}\right) = \frac{3 + 4i}{5\sqrt{2}}$$
  $$P(+) = \left|\frac{3 + 4i}{5\sqrt{2}}\right|^2 = \frac{9 + 16}{50} = \frac{1}{2}$$
\end{eje}

\begin{defi}[Colapso cuántico]
  Después de una medición que da como resultado el estado $\ket{\phi_i}$, el sistema colapsa al estado:
  $$\ket{\psi'} = \frac{P_i \ket{\psi}}{\|P_i \ket{\psi}\|} = \frac{\braket{\psi}{\phi_i}}{|\braket{\psi}{\phi_i}|} \ket{\phi_i}$$
  donde $P_i = \ketbra{\phi_i}{\phi_i}$ es el proyector sobre $\ket{\phi_i}$.
\end{defi}

\subsection{Operadores en notación de Dirac}

\begin{defi}[Valor esperado]
  El valor esperado de un operador $A$ en el estado $\ket{\psi}$ es:
  $$\langle A \rangle = \bra{\psi}A\ket{\psi}$$
\end{defi}

\begin{eje}[Matrices de Pauli en notación de Dirac]
  Las matrices de Pauli se pueden expresar como:
  \begin{align*}
    \sigma_x & = \ketbra{0}{1} + \ketbra{1}{0}    \\
    \sigma_y & = -i\ketbra{0}{1} + i\ketbra{1}{0} \\
    \sigma_z & = \ketbra{0}{0} - \ketbra{1}{1}
  \end{align*}

  \textbf{Verificación para $\sigma_x$:}
  \begin{align*}
    \sigma_x\ket{0} & = (\ketbra{0}{1} + \ketbra{1}{0})\ket{0} = \ket{0}\braket{1}{0} + \ket{1}\braket{0}{0} = \ket{1} \\
    \sigma_x\ket{1} & = (\ketbra{0}{1} + \ketbra{1}{0})\ket{1} = \ket{0}\braket{1}{1} + \ket{1}\braket{0}{1} = \ket{0}
  \end{align*}
\end{eje}

\begin{eje}[Valor esperado de $\sigma_z$]
  Para el estado $\ket{\psi} = \alpha\ket{0} + \beta\ket{1}$:
  \begin{align*}
    \langle \sigma_z \rangle & = \bra{\psi}\sigma_z\ket{\psi}                                                                              \\
                             & = (\conj{\alpha}\bra{0} + \conj{\beta}\bra{1})(\ketbra{0}{0} - \ketbra{1}{1})(\alpha\ket{0} + \beta\ket{1}) \\
                             & = |\alpha|^2 - |\beta|^2
  \end{align*}

  Este resultado tiene una interpretación física clara: es la diferencia entre las probabilidades de medir $+1$ y $-1$.
\end{eje}

\subsection{Sistemas de múltiples cúbits}

\begin{defi}[Producto tensorial de estados]
  Para sistemas compuestos, los estados se forman mediante el producto tensorial:
  $$\ket{\psi} \tensor \ket{\phi} = \ket{\psi\phi}$$

  Para $n$ cúbits, el espacio de estados es $(\C^2)^{\tensor n} \cong \C^{2^n}$.
\end{defi}

\begin{defi}[Base computacional para múltiples cúbits]
  Para $n$ cúbits, la base computacional está formada por:
  $$\{\ket{b_1 b_2 \cdots b_n} : b_i \in \{0,1\}\}\,.$$
\end{defi}

\begin{eje}[Sistema de dos cúbits]
  La base computacional para dos cúbits es:
  \begin{align*}
    \ket{00} = \ket{0} \tensor \ket{0} & = \begin{pmatrix} 1 \\ 0 \\ 0 \\ 0 \end{pmatrix} \\
    \ket{01} = \ket{0} \tensor \ket{1} & = \begin{pmatrix} 0 \\ 1 \\ 0 \\ 0 \end{pmatrix} \\
    \ket{10} = \ket{1} \tensor \ket{0} & = \begin{pmatrix} 0 \\ 0 \\ 1 \\ 0 \end{pmatrix} \\
    \ket{11} = \ket{1} \tensor \ket{1} & = \begin{pmatrix} 0 \\ 0 \\ 0 \\ 1 \end{pmatrix}
  \end{align*}

  Un estado general de dos cúbits se escribe:
  $$\ket{\psi} = \alpha_{00}\ket{00} + \alpha_{01}\ket{01} + \alpha_{10}\ket{10} + \alpha_{11}\ket{11}$$
  con $\sum_{ij} |\alpha_{ij}|^2 = 1$.
\end{eje}

\begin{eje}[Estados de Bell]
  Los cuatro estados de Bell forman una base ortonormal de estados entrelazados para dos cúbits:
  \begin{align*}
    \ket{\Phi^+} & = \frac{\ket{00} + \ket{11}}{\sqrt{2}} \\
    \ket{\Phi^-} & = \frac{\ket{00} - \ket{11}}{\sqrt{2}} \\
    \ket{\Psi^+} & = \frac{\ket{01} + \ket{10}}{\sqrt{2}} \\
    \ket{\Psi^-} & = \frac{\ket{01} - \ket{10}}{\sqrt{2}}
  \end{align*}
\end{eje}

\subsection{Mediciones y observables}

\begin{defi}[Observable en notación de Dirac]
  Un observable es un operador hermitiano $\hat{O}$ que puede escribirse en su descomposición espectral:
  $$\hat{O} = \sum_i \lambda_i \ketbra{\phi_i}{\phi_i}$$
  donde $\lambda_i \in \R$ son los valores propios y $\{\ket{\phi_i}\}$ es una base ortonormal de vectores propios.
\end{defi}

\begin{eje}[Medición de $\sigma_z$ en diferentes bases]
  \textbf{Base computacional:}
  $$\sigma_z = (+1)\ketbra{0}{0} + (-1)\ketbra{1}{1}$$

  Para el estado $\ket{\psi} = \alpha\ket{0} + \beta\ket{1}$:
  - $P(+1) = |\braket{0}{\psi}|^2 = |\alpha|^2$
  - $P(-1) = |\braket{1}{\psi}|^2 = |\beta|^2$

  \textbf{Base de $\sigma_x$ (base $\{\ket{+}, \ket{-}\}$):}
  $$\sigma_x = (+1)\ketbra{+}{+} + (-1)\ketbra{-}{-}$$

  Para el mismo estado:
  \begin{align*}
    \braket{+}{\psi} & = \frac{\alpha + \beta}{\sqrt{2}} \Rightarrow P(+1) = \frac{|\alpha + \beta|^2}{2} \\
    \braket{-}{\psi} & = \frac{\alpha - \beta}{\sqrt{2}} \Rightarrow P(-1) = \frac{|\alpha - \beta|^2}{2}
  \end{align*}
\end{eje}

\subsection{Dinámicas cuánticas en notación de Dirac}

\begin{defi}[Evolución unitaria]
  La evolución temporal de un sistema cuántico cerrado se describe mediante un operador unitario $U(t)$:
  $$\ket{\psi(t)} = U(t)\ket{\psi(0)}$$

  Para hamiltonianos independientes del tiempo:
  $$U(t) = e^{-i\hat{H}t/\hbar}$$
  donde $\hat{H}$ es el hamiltoniano del sistema.
\end{defi}

\begin{eje}[Evolución libre de un cúbit]
  Para un cúbit con hamiltoniano $\hat{H} = \frac{\omega}{2}\sigma_z$:
  $$U(t) = e^{-i\omega t\sigma_z/2} = \cos\frac{\omega t}{2}I - i\sin\frac{\omega t}{2}\sigma_z$$

  $$U(t) = \begin{pmatrix} e^{-i\omega t/2} & 0 \\ 0 & e^{i\omega t/2} \end{pmatrix}$$

  Si el estado inicial es $\ket{\psi(0)} = \alpha\ket{0} + \beta\ket{1}$:
  $$\ket{\psi(t)} = \alpha e^{-i\omega t/2}\ket{0} + \beta e^{i\omega t/2}\ket{1}$$

  Las probabilidades $|\alpha|^2$ y $|\beta|^2$ se mantienen constantes, pero las fases evolucionan.
\end{eje}

\begin{info}
  El Postulado V garantiza que la información cuántica se conserva durante la evolución temporal de sistemas cerrados. Esto es fundamental para el diseño de algoritmos cuánticos, donde las operaciones se implementan mediante secuencias de operadores unitarios.
\end{info}