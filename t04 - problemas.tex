\unirsection{Problemas}

\begin{questions}

  \question Verificar que las siguientes funciones definen productos internos y calcular la norma inducida:
  \begin{parts}
    \part En $\C^2$, $\langle u, v \rangle = 2\conj{u_1}v_1 + \conj{u_2}v_2 + \conj{u_1}v_1 + 3\conj{u_2}v_2$.
    \part En $\mathcal{M}_2(C)$, $\langle A, B \rangle = \tr(A^\dagger B)$.
  \end{parts}

  \question
  \label{ex:propiedadesProductoInterno}
  Demostrar el resultado de la propiedades del producto interno~(\ref{prop:propiedadesProductoInterno}).

  \question
  \label{ex:propiedadesNorma}
  Demostrar el resultado de la propiedades de la norma inducida~(\ref{prop:propiedadesNorma}).

  \question
  \label{ex:distanciaAsociadaNorma}
  Demostrar el resultado de la propiedades de la distancia asociada a la norma~(\ref{prop:distanciaAsociadaNorma}).

  \question Aplicar el proceso de Gram-Schmidt para ortogonalizar
  \[
    \left\{\mqty(1\\ 1\\ i), \mqty(1\\ i\\ 0), \mqty(i\\ 1\\ 1)\right\}\,.
  \]

  \question Sea $W = \text{gen}\left\{\mqty(1\\ i\\ 0), \mqty(i\\ 1\\ 1)\right\} \subset \C^3$.
  \begin{parts}
    \part Encontrar una base ortonormal para $W$.
    \part Calcular $\text{proj}_W(1, 1, 1)$.
    \part Encontrar el complemento ortogonal $W^\perp$.
  \end{parts}

  \question Para las matrices $A = \begin{pmatrix} 2 & 1+i \\ 1-i & 3 \end{pmatrix}$ y $B = \begin{pmatrix} 1 & 2i \\ -2i & 1 \end{pmatrix}$
  \begin{parts}
    \part Comprobar si son hermitianas.
    \part Calcular los valores propios y vectores propios de las hermitianas.
    \part Diagonalizar las matrices hermitianas usando una base ortonormal.
  \end{parts}

  \question Demostrar que las siguientes matrices son unitarias y encontrar su descomposición espectral:
  \begin{parts}
    \part $U_1 = \frac{1}{\sqrt{2}}\begin{pmatrix} 1 & i \\ i & 1 \end{pmatrix}$.
    \part $U_2 = \frac{1}{\sqrt{2}}\begin{pmatrix} 1 & 1 \\ 1 & -1 \end{pmatrix}$ (matriz de Hadamard).
  \end{parts}

  \question Demostrar las siguientes propiedades para operadores en espacios de Hilbert:
  \begin{parts}
    \part Si $T$ es hermitiano, entonces $\langle Tv, v \rangle \in \R$ para todo $v$.
    \part Si $U$ es unitario, entonces $U$ preserva el producto interno.
    \part El producto de dos operadores unitarios es unitario.
  \end{parts}

  \question Para el operador $H = \begin{pmatrix} 1 & 0 & 0 \\ 0 & 0 & 1 \\ 0 & 1 & 0 \end{pmatrix}$.
  \begin{parts}
    \part Verificar que $H$ es hermitiano.
    \part Encontrar todos los valores propios y vectores propios.
    \part Construir la descomposición espectral $H = \sum_i \lambda_i P_i$ donde $P_i$ son proyecciones ortogonales.
  \end{parts}

  \question Para el operador $A = \begin{pmatrix} 2 & 1-i \\ 1+i & 3 \end{pmatrix}$.
  \begin{parts}
    \part Verificar que $A$ es hermitiano.
    \part Encontrar los valores propios y vectores propios.
    \part Escribir la descomposición espectral de $A$.
  \end{parts}

  \question Demostrar que si $A$ y $B$ son operadores hermitianos que conmutan, $[A, B] = 0$, entonces tienen una base común de vectores propios.

  \question Para el operador $H = \frac{1}{\sqrt{2}}(\sigma_x + \sigma_z)$.
  \begin{parts}
    \part Verificar que $H$ es hermitiano.
    \part Encontrar la descomposición espectral de $H$.
  \end{parts}

  \question En $\mathbb{C}^2$, usar el teorema de Riesz-Fréchet para encontrar el vector $y_f \in \mathbb{C}^2$ que representa a $f(x) = (1-i, 2+i) x$.

  \question Considerar el operador $T: \mathbb{C}^2 \to \mathbb{C}^2$ definido por la matriz $T = \begin{pmatrix} 1 & i \\ 0 & 2 \end{pmatrix}$.
  \begin{parts}
    \part Encontrar la representación matricial del operador adjunto $T^\dagger$.
    \part Verificar que $\langle Tx, y \rangle = \langle x, T^\dagger y \rangle$ para $x = \mqty(1\\ i)$ y $y = \mqty(1+i\\ 1-i)$.
    \part Determinar si $T$ es hermitiano, unitario o normal.
  \end{parts}

  \question Sea el operador lineal $A$ actuando sobre $\mathbb{C}^2$, representado por la matriz
  \[
    A = \begin{pmatrix} 0 & -i \\ i & 0 \end{pmatrix}\,.
  \]

  \begin{parts}
    \part Calcule el operador adjunto $A^\dagger$.
    \part Determine si $A$ es un operador \textbf{hermítico}.
    \part Calcule los valores propios de $A$.
  \end{parts}

  \question Considere el operador hermítico $H$ dado por
  $$H = \begin{pmatrix} 3 & 1 \\ 1 & 3 \end{pmatrix}\,.$$

  \begin{parts}
    \part Encuentre los valores propios.
    \part Determine los vectores propios normalizados.
    \part Escriba la matriz diagonal $D$ y la matriz unitaria $U$ que cumple con la descomposición espectral
    \[
      H = U D U^\dagger\,.
    \]
  \end{parts}

\end{questions}