\unirsection{A fondo}

Los siguientes libros pueden servir de material de apoyo y para profundizar más en los contenidos de este tema.

\textbf{Nielsen, M. A., and Chuang, I. L. (2010). Quantum computation and quantum information : 10th anniversary edition. Cambridge University Press.}

El capítulo 4 formaliza los conceptos de operadores lineales y matrices en espacios de Hilbert para computación cuántica. Presenta las puertas unitarias sobre un cúbit y prepara el camino para el estudio de operadores más generales.

\textbf{
  Nakara, M. (2007). Quantum computing : from linear algebra to physical realizations. CRC Press.}

En los capítulos 3 y 4 presenta las bases de la computación cuántica, cúbits, puertas cuánticas y circuitos cuánticos, haciendo uso de los conceptos de álgebra lineal.

\textbf{Yanofsky, N. S., and Mannucci, M. A. (2008). Quantum computing for computer scientists. Cambridge University Press.}

El capítulo 5 nos permitirá amplicar nuestro conocimiento sobre operadores cúbit y puertas cuánticas, así como la representación matricial de los mismos.
