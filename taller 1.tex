\documentclass[]{unirTema}

\newcommand{\tq}{\mid}
\newcommand{\K}{\mathrm{K}}
\newcommand{\V}{\mathrm{V}}
\newcommand{\N}{\mathbb{N}}
\newcommand{\Z}{\mathbb{Z}}
\newcommand{\F}{\mathbb{F}}
\newcommand{\Q}{\mathbb{Q}}
\newcommand{\R}{\mathbb{R}}
\newcommand{\C}{\mathbb{C}}
\renewcommand{\H}{\mathcal{H}}
\newcommand{\Cinf}{\mathcal{C}^\infty}
\newcommand{\Cu}{\mathcal{C}^1}
\newcommand{\Rp}{\mathfrak{Re}}
\newcommand{\Ip}{\mathfrak{Im}}
\renewcommand{\d}{\mathrm{d}}
\newcommand{\dm}{\mathrm{d}\mu}
\newcommand{\conj}[1]{\overline{#1}}
\newcommand{\Tras}[1]{{#1}^{\text{T}}}

\DeclareMathOperator{\Log}{Log}
\DeclareMathOperator{\Arg}{Arg}
\DeclareMathOperator{\Dom}{Dom}
\DeclareMathOperator{\Ima}{Im}
\DeclareMathOperator{\sgn}{sgn}
\DeclareMathOperator{\mcd}{MCD}
\DeclareMathOperator{\mcm}{mcm}
\DeclareMathOperator{\Resi}{Res}
\DeclareMathOperator{\Ker}{Ker}
\DeclareMathOperator{\End}{End}
\DeclareMathOperator{\Mat}{Mat}

\newcommand{\parder}[2]{\frac{\partial #1}{\partial #2}}
\newcommand{\dparder}[2]{\dfrac{\partial #1}{\partial #2}}
\newcommand{\tparder}[2]{\partial #1/\partial #2}
\newcommand{\parderr}[3]{\frac{\partial^2 #1}{\partial #2\partial #3}}
\newcommand{\dparderr}[3]{\dfrac{\partial^2 #1}{\partial #2\partial #3}}
\newcommand{\tparderr}[3]{\partial^2 #1/\partial #2\partial #3}
\newcommand{\intx}[1]{\int #1\,dx}
\newcommand{\intt}[1]{\int #1\,dt}
\newcommand{\intdx}[3]{\int_{#1}^{#2} #3\,dx}
\newcommand{\intdt}[3]{\int_{#1}^{#2} #3\,dt}
\newcommand{\intdz}[2]{\int_{#1} #2\,dz}
\newcommand{\set}[1]{\left\{#1\right\}}
\newcommand{\so}{\Rightarrow}
\newcommand{\sii}{\Leftrightarrow}
\newcommand{\by}[1]{\overset{\fbox{\tiny #1}}{=}}
\newcommand{\byref}[1]{\overset{\fbox{\tiny\ref{#1}}}{=}}
\newcommand{\cardinal}[1]{\left|#1\right|}
\newcommand{\maps}[3]{#1 \colon #2\longrightarrow #3}
\newcommand{\equationmaps}[5]{\begin{aligned}[t]#1 \colon #2 &\longrightarrow #3 \\	#4 &\longmapsto #5\end{aligned}}
\newcommand{\coma}{,\thinspace}
\newcommand{\pari}[2]{(#1,\thinspace #2)}
\newcommand{\where}{\mathrel{}\middle|\mathrel{}}
\newcommand{\no}[1]{{\neg}{#1}}
\newcommand{\dcomilla}[1]{{\guillemotleft}#1{\guillemotright}}
\newcommand{\separa}{\vspace*{.75\baselineskip}}
\newcommand{\semisepara}{\vspace*{.25\baselineskip}}
\newcommand{\restrict}[1]{\raisebox{-.5ex}{$|$}_{#1}}

% ========================================
% COMANDOS PERSONALIZADOS PARA COMPUTACIÓN CUÁNTICA
% ========================================

% Estados comunes
\newcommand{\zero}{\ket{0}}
\newcommand{\one}{\ket{1}}
\newcommand{\plus}{\ket{+}}
\newcommand{\minus}{\ket{-}}

% Operadores especiales
\newcommand{\tensor}{\otimes}         % Producto tensorial
\newcommand{\comp}{\circ}             % Composición

% Estados de Bell
\newcommand{\bellphi}{\ket{\Phi^+}}
\newcommand{\bellpsi}{\ket{\Psi^+}}
\newcommand{\bellphiminus}{\ket{\Phi^-}}
\newcommand{\bellpsiminus}{\ket{\Psi^-}}

\newcommand{\floor}[1]{\left\lfloor #1 \right\rfloor}
\newcommand{\ceil}[1]{\left\lceil #1 \right\rceil}


\printanswers

\begin{document}
\author{Francisco Costa Cano}
\titulacion{Máster en computación cuántica}
\asignatura{Álgebra lineal en computación cuántica}
\bloque{1}{Fundamentos matemáticos}
\tema{3}{Operadores lineales y representación matricial}
Sea $P_2(\C)$ el espacio vectorial de polinomios de grado menor o igual a 2 con coeficientes en $\C$.
Sea $T: P_2(\C) \to \C^2$ definida por
\[
  T(p(x)) = \left(p(1), p(i)\right)
\]
\begin{parts}
  \part Comprobar que $T$ es un homomorfismo.

  Para ello usaremos la caracterización que dice que $T$ es un homomorfismo si y solo si $\forall p,q\in P_2(\C)$ y $\forall \alpha,\beta \in\C$ se cumple
  \[
    T(\alpha p(x) + \beta q(x)) = \alpha T(p(x)) + \beta T(q(x))\,.
  \]
  Para ello llamemos $r(x) = \alpha p(x) + \beta q(x)$ y calculemos:
  \begin{align*}
    T(\alpha p(x) + \beta q(x)) & = T(r(x)) = \left(r(1), r(i)\right)                                           \\
                                & = \left(\alpha p(1) + \beta q(1), \alpha p(i) + \beta q(i)\right)             \\
                                & = \left(\alpha p(1), \alpha p(i)\right) + \left(\beta q(1), \beta q(i)\right) \\
                                & = \alpha \left(p(1), p(i)\right) + \beta \left(q(1), q(i)\right)              \\
                                & = \alpha T(p(x)) + \beta T(q(x))\,.
  \end{align*}
  Por lo tanto $T$ es un homomorfismo.
  \part Calcular el núcleo de $T$.

  Por definición $\ker(T) = \{p(x) \in P_2(\C) \mid T(p(x)) = (0,0)\}$.

  Pero esta condición se cumple si y solo si $p(1) = 0$ y $p(i) = 0$. Si escribimos $p(x) = ax^2 + bx + c$, debemos resolver el sistema
  \[
    \begin{cases}
      p(1) = a + b + c = 0 \\
      p(i) = -a + bi + c = 0
    \end{cases}
  \]
  Sumando ambas ecuaciones obtenemos $b+bi+2c=0\Rightarrow b(1+i)+2c=0$. Podemos obtener $c$ como $c=-\frac{b(1+i)}{2}$.

  Sustituyendo en la primera ecuación obtenemos $a+b-\frac{b(1+i)}{2}=0\Rightarrow a=-\frac{b(i-1)}{2}$.

  Por lo tanto el núcleo de $T$ es
  \[
    \ker(T) = \left\{p(x) = -\frac{b(i-1)}{2}x^2 + bx - \frac{b(1+i)}{2} \mid b \in \C\right\}\,.
  \]
  De este resultado podemos además concluir que $\dim(\ker(T)) = 1$ y que tiene como base $B_k=\left\{-\frac{i-1}{2}x^2 + x - \frac{1+i}{2}\right\}$.

  \part Calcular la antiimagen de $T$.

  Por definición $\operatorname{Im}(T) = \{T(p(x)) \mid p(x) \in P_2(\C)\}$.

  Sea $(w,z) \in \C^2$ y $p(x) = ax^2 + bx + c$ tal que $T(p(x)) = (w,z)$. Entonces

  \begin{align*}
    a+b+c   & = w \\
    -a+bi+c & = z \\
  \end{align*}

  Sumando ambas ecuaciones obtenemos $2c+b(1+i) = w+z\Rightarrow b=\frac{w+z-2c}{1+i}$.

  Multiplicando por $i$ la primera ecuación y sumando $a(i-1)+c(i+1) = wi+z\Rightarrow a=\frac{wi+z-c(i-1)}{i-1}$.

  Tomando $c=0$, por ejemplo, obtenemos $a=\frac{wi+z}{i-1}$ y $b=\frac{w+z}{1+i}$ y así el polinomio $p(x) = \frac{wi+z}{i-1}x^2 + \frac{w+z}{1+i}x$ cumple $T(p(x)) = (w,z)$.

  Como esperábamos, la dimensión de la imagen de $T$ es 2 y tiene como base $B_I=\left\{\frac{i-1}{2}x^2 + x - \frac{1+i}{2}, x\right\}$.
\end{parts}
\end{document}