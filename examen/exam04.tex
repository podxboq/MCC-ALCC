\codigonombre{}{MCC-ALCC-25Q104}
\begin{questions}
	\question[4] Responda a las siguientes cuestiones:
	\begin{parts}
		\part Explique el producto tensorial entre operadores lineales y su aplicación en sistemas cuánticos compuestos.

		\begin{solution}
			El producto tensorial entre operadores lineales es una operación fundamental que permite construir operadores que actúan sobre espacios vectoriales compuestos a partir de operadores que actúan sobre espacios más simples.

			Definición formal:
			Dados dos operadores lineales $A: V \rightarrow V'$ y $B: W \rightarrow W'$, su producto tensorial $A \otimes B: V \otimes W \rightarrow V' \otimes W'$ es un operador lineal definido por:
			$(A \otimes B)(v \otimes w) = (Av) \otimes (Bw)$
			para todo $v \in V$ y $w \in W$, extendido por linealidad al espacio completo.

			Representación matricial:
			Si $A$ es una matriz $m \times n$ y $B$ es una matriz $p \times q$, entonces $A \otimes B$ es una matriz $(mp) \times (nq)$ con estructura de bloques:
			$A \otimes B =
				\begin{pmatrix}
					a_{11}B & a_{12}B & \cdots & a_{1n}B \\
					a_{21}B & a_{22}B & \cdots & a_{2n}B \\
					\vdots  & \vdots  & \ddots & \vdots  \\
					a_{m1}B & a_{m2}B & \cdots & a_{mn}B
				\end{pmatrix}$

			Propiedades principales:
			1) Bilinealidad: $(aA + bC) \otimes B = a(A \otimes B) + b(C \otimes B)$
			2) Asociatividad: $A \otimes (B \otimes C) = (A \otimes B) \otimes C$
			3) Producto con escalares: $(kA) \otimes B = A \otimes (kB) = k(A \otimes B)$
			4) Compatibilidad con el producto de operadores: $(A \otimes B)(C \otimes D) = (AC) \otimes (BD)$
			5) Para operadores hermíticos: $(A \otimes B)^\dagger = A^\dagger \otimes B^\dagger$
			6) Para operadores unitarios: Si $A$ y $B$ son unitarios, entonces $A \otimes B$ es unitario

			Aplicaciones en sistemas cuánticos compuestos:

			1) Operadores sobre múltiples qubits:
			- Si $X, Y, Z$ son matrices de Pauli, $X \otimes Z$ actúa como $X$ en el primer qubit y $Z$ en el segundo
			- Compuertas de dos qubits como CNOT se pueden expresar usando productos tensoriales
			- Para un sistema de $n$ qubits, los operadores tienen forma $A_1 \otimes A_2 \otimes \cdots \otimes A_n$

			2) Operadores de medición:
			- Las mediciones sobre subsistemas individuales se representan como $M \otimes I$
			- Proyectores compuestos: $P = |v\rangle\langle v| \otimes |w\rangle\langle w| = (|v\rangle \otimes |w\rangle)(\langle v| \otimes \langle w|)$

			3) Hamiltoniano de sistemas compuestos:
			- $H_{total} = H_A \otimes I_B + I_A \otimes H_B + H_{int}$, donde $H_{int}$ representa interacciones
			- Permite modelar la evolución de sistemas que interactúan

			4) Operaciones locales:
			- Las operaciones que afectan solo a subsistemas específicos se expresan mediante productos tensoriales
			- Fundamental en protocolos de información cuántica, teleportación y corrección de errores

			5) Matrices de densidad para sistemas compuestos:
			- Estados separables: $\rho = \rho_A \otimes \rho_B$
			- La traza parcial permite obtener matrices de densidad reducidas

			El producto tensorial de operadores es esencial para entender y manipular sistemas cuánticos compuestos, permitiendo construir operaciones complejas a partir de bloques simples y analizar cómo las operaciones locales afectan al sistema global.
		\end{solution}

		\part Describa la matriz de densidad reducida y su utilidad para analizar subsistemas en sistemas cuánticos compuestos.

		\begin{solution}
			La matriz de densidad reducida es una herramienta matemática fundamental que permite describir y analizar el estado de un subsistema dentro de un sistema cuántico compuesto.

			Definición formal:
			Dado un sistema bipartito A+B en un estado $\rho_{AB}$, la matriz de densidad reducida del subsistema A se obtiene mediante la traza parcial sobre el subsistema B:
			$\rho_A = \text{Tr}_B(\rho_{AB}) = \sum_i (I_A \otimes \langle i|_B) \rho_{AB} (I_A \otimes |i\rangle_B)$
			donde $\{|i\rangle_B\}$ es una base ortonormal del espacio de Hilbert del subsistema B.

			En notación de componentes, si $\rho_{AB}$ tiene elementos $\rho_{ab,cd}$ en la base $\{|a\rangle_A \otimes |c\rangle_B\}$, entonces:
			$(\rho_A)_{ab} = \sum_c \rho_{ac,bc}$

			Propiedades fundamentales:
			1) $\rho_A$ es una matriz hermitiana: $\rho_A = \rho_A^\dagger$
			2) $\rho_A$ tiene traza 1: $\text{Tr}(\rho_A) = 1$
			3) $\rho_A$ es positiva semidefinida: $\langle\phi|\rho_A|\phi\rangle \geq 0$ para todo $|\phi\rangle$
			4) El espectro de $\rho_A$ contiene los cuadrados de los coeficientes de Schmidt de la descomposición del estado puro original

			Utilidad para analizar subsistemas:

			1) Descripción de subsistemas:
			- Proporciona la única descripción completa y correcta de un subsistema de un sistema cuántico
			- Contiene toda la información accesible mediante mediciones locales en ese subsistema
			- Los valores esperados de observables locales: $\langle O_A \rangle = \text{Tr}(\rho_A O_A)$

			2) Detección y cuantificación de entrelazamiento:
			- Un estado puro $|\psi\rangle_{AB}$ está entrelazado si y solo si $\rho_A$ (y $\rho_B$) es un estado mixto
			- Para estados puros, la pureza $\text{Tr}(\rho_A^2) < 1$ indica entrelazamiento
			- La entropía de von Neumann $S(\rho_A) = -\text{Tr}(\rho_A \log \rho_A)$ cuantifica el entrelazamiento

			3) Análisis de decoherencia:
			- Permite estudiar cómo la interacción con el entorno afecta a un sistema cuántico
			- La evolución hacia estados mixtos se refleja en la matriz reducida
			- Monitoriza la pérdida de información cuántica

			4) Criptografía y protocolos cuánticos:
			- En teleportación cuántica, la matriz reducida del receptor antes de recibir información clásica es completamente mixta
			- Garantiza la seguridad de ciertos protocolos criptográficos cuánticos
			- Permite verificar la distribución segura de entrelazamiento

			5) Algoritmos cuánticos:
			- Ayuda a analizar el estado de qubits específicos durante la ejecución de algoritmos
			- Permite evaluar el impacto de las operaciones en subsistemas individuales
			- Facilita la depuración de circuitos cuánticos complejos

			6) Tecnologías emergentes:
			- En computación cuántica de estado sólido, permite estudiar la interacción entre el qubit y su entorno
			- En procesadores multicore, permite analizar la coherencia entre diferentes regiones
			- En simulaciones numéricas, permite reducir la complejidad computacional

			La matriz de densidad reducida es, en resumen, la herramienta matemática fundamental que nos permite "abrir la caja" de un sistema cuántico compuesto y examinar sus partes constituyentes, manteniendo toda la información sobre correlaciones cuánticas que afectan al subsistema de interés.
		\end{solution}
	\end{parts}

	\question[3] Considere un sistema de dos qubits en el estado
	$|\psi\rangle = \alpha|00\rangle + \beta|01\rangle + \gamma|10\rangle + \delta|11\rangle$
	donde $\alpha, \beta, \gamma, \delta \in \mathbb{C}$ y $|\alpha|^2 + |\beta|^2 + |\gamma|^2 + |\delta|^2 = 1$.
	\begin{parts}

		\part  Determine la condición necesaria y suficiente sobre los coeficientes para que $|\psi\rangle$ sea un estado separable.
		\part  Calcule la matriz de densidad $\rho = |\psi\rangle\langle\psi|$ para el caso particular $\alpha = \delta = \frac{1}{\sqrt{2}}$ y $\beta = \gamma = 0$.
		\part  Obtenga la matriz de densidad reducida $\rho_A$ del primer qubit en este caso particular.
		\part  Calcule la entropía de von Neumann $S(\rho_A) = -\text{Tr}(\rho_A \log_2 \rho_A)$ y explique su significado físico.
	\end{parts}

	\begin{solution}
		a) Un estado de dos qubits es separable si y solo si puede escribirse como el producto tensorial de dos estados de un qubit:
		$|\psi\rangle = |\phi_A\rangle \otimes |\phi_B\rangle$

		Supongamos que:
		$|\phi_A\rangle = a|0\rangle + b|1\rangle$ y $|\phi_B\rangle = c|0\rangle + d|1\rangle$

		Entonces:
		$|\phi_A\rangle \otimes |\phi_B\rangle = (a|0\rangle + b|1\rangle) \otimes (c|0\rangle + d|1\rangle)$
		$= ac|00\rangle + ad|01\rangle + bc|10\rangle + bd|11\rangle$

		Comparando con $|\psi\rangle = \alpha|00\rangle + \beta|01\rangle + \gamma|10\rangle + \delta|11\rangle$, obtenemos:
		$\alpha = ac$
		$\beta = ad$
		$\gamma = bc$
		$\delta = bd$

		Para que exista una solución, debe cumplirse:
		$\alpha\delta = \beta\gamma$

		Esta es la condición necesaria y suficiente para que $|\psi\rangle$ sea separable. En términos físicos, significa que el determinante de la matriz de coeficientes (cuando se reorganiza el estado como una matriz $2 \times 2$) debe ser cero.

		b) Para el caso particular $\alpha = \delta = \frac{1}{\sqrt{2}}$ y $\beta = \gamma = 0$, el estado es:
		$|\psi\rangle = \frac{1}{\sqrt{2}}(|00\rangle + |11\rangle)$

		Este estado no cumple la condición de separabilidad ya que $\alpha\delta = \frac{1}{2} \neq 0 = \beta\gamma$, por lo que es un estado entrelazado (concretamente, uno de los estados de Bell).

		La matriz de densidad $\rho = |\psi\rangle\langle\psi|$ es:
		$\rho = \frac{1}{2}(|00\rangle + |11\rangle)(\langle00| + \langle11|)$
		$= \frac{1}{2}(|00\rangle\langle00| + |00\rangle\langle11| + |11\rangle\langle00| + |11\rangle\langle11|)$

		En forma matricial (en la base $\{|00\rangle, |01\rangle, |10\rangle, |11\rangle\}$):
		$\rho = \frac{1}{2}\begin{pmatrix}
				1 & 0 & 0 & 1 \\
				0 & 0 & 0 & 0 \\
				0 & 0 & 0 & 0 \\
				1 & 0 & 0 & 1
			\end{pmatrix}$

		c) La matriz de densidad reducida $\rho_A$ del primer qubit se obtiene tomando la traza parcial sobre el segundo qubit:

		$\rho_A = \text{Tr}_B(\rho) = \sum_{i=0}^1 (I \otimes \langle i|)\rho(I \otimes |i\rangle)$

		$\rho_A = (I \otimes \langle 0|)\rho(I \otimes |0\rangle) + (I \otimes \langle 1|)\rho(I \otimes |1\rangle)$

		Calculando cada término:
		$(I \otimes \langle 0|)\rho(I \otimes |0\rangle) = \frac{1}{2}|0\rangle\langle0|$
		$(I \otimes \langle 1|)\rho(I \otimes |1\rangle) = \frac{1}{2}|1\rangle\langle1|$

		Por tanto:
		$\rho_A = \frac{1}{2}|0\rangle\langle0| + \frac{1}{2}|1\rangle\langle1| = \frac{1}{2}I$

		En forma matricial:
		$\rho_A = \frac{1}{2}\begin{pmatrix}
				1 & 0 \\
				0 & 1
			\end{pmatrix}$

		d) La entropía de von Neumann de $\rho_A$ es:
		$S(\rho_A) = -\text{Tr}(\rho_A \log_2 \rho_A)$

		Para calcularla, primero necesitamos $\log_2 \rho_A$. Dado que $\rho_A = \frac{1}{2}I$, tenemos $\log_2 \rho_A = \log_2(\frac{1}{2})I = -I$. Por tanto:

		$S(\rho_A) = -\text{Tr}(\frac{1}{2}I \cdot (-I)) = \text{Tr}(\frac{1}{2}I^2) = \frac{1}{2}\text{Tr}(I) = \frac{1}{2} \cdot 2 = 1$

		Significado físico:
		- La entropía de von Neumann cuantifica la mezcla o incertidumbre en el estado del sistema
		- $S(\rho_A) = 1$ es el valor máximo posible para un sistema de un qubit, indicando un estado completamente mixto
		- Para un estado puro bipartito, $S(\rho_A)$ cuantifica el grado de entrelazamiento entre los subsistemas
		- El valor $S(\rho_A) = 1$ confirma que el estado original $|\psi\rangle$ es máximamente entrelazado
		- Este resultado implica que no podemos predecir el resultado de una medición en el primer qubit independientemente del segundo qubit
		- La información del subsistema A está completamente compartida con el subsistema B

		Este es un ejemplo de cómo el entrelazamiento cuántico causa que un subsistema de un estado global puro aparezca en un estado mixto cuando se considera aisladamente.
	\end{solution}

	\question[3] Considere las siguientes matrices de Pauli:
	$X = \begin{pmatrix} 0 & 1 \\ 1 & 0 \end{pmatrix}$, $Y = \begin{pmatrix} 0 & -i \\ i & 0 \end{pmatrix}$ y $Z = \begin{pmatrix} 1 & 0 \\ 0 & -1 \end{pmatrix}$
	\begin{parts}
		\part  Calcule explícitamente $[X,Y]$, $[Y,Z]$ y $[Z,X]$, donde $[A,B] = AB - BA$ es el conmutador.
		\part  Exprese la matriz $A = \begin{pmatrix} 3 & 2-i \\ 2+i & 5 \end{pmatrix}$ como combinación lineal de la identidad y las matrices de Pauli.
		\part  Verifique si la matriz $A$ es hermitiana, y calcule sus valores y vectores propios.
		\part  Determine la expresión de la matriz exponencial $e^{i\theta X}$ para $\theta \in \mathbb{R}$ y explique su interpretación en la esfera de Bloch.
	\end{parts}

	\begin{solution}
		a) Calculemos los conmutadores:

		$[X,Y] = XY - YX = \begin{pmatrix} 0 & 1 \\ 1 & 0 \end{pmatrix}\begin{pmatrix} 0 & -i \\ i & 0 \end{pmatrix} - \begin{pmatrix} 0 & -i \\ i & 0 \end{pmatrix}\begin{pmatrix} 0 & 1 \\ 1 & 0 \end{pmatrix}$
		$= \begin{pmatrix} i & 0 \\ 0 & i \end{pmatrix} - \begin{pmatrix} -i & 0 \\ 0 & -i \end{pmatrix} = \begin{pmatrix} 2i & 0 \\ 0 & 2i \end{pmatrix} = 2iZ$

		$[Y,Z] = YZ - ZY = \begin{pmatrix} 0 & -i \\ i & 0 \end{pmatrix}\begin{pmatrix} 1 & 0 \\ 0 & -1 \end{pmatrix} - \begin{pmatrix} 1 & 0 \\ 0 & -1 \end{pmatrix}\begin{pmatrix} 0 & -i \\ i & 0 \end{pmatrix}$
		$= \begin{pmatrix} 0 & i \\ i & 0 \end{pmatrix} - \begin{pmatrix} 0 & -i \\ -i & 0 \end{pmatrix} = \begin{pmatrix} 0 & 2i \\ 2i & 0 \end{pmatrix} = 2iX$

		$[Z,X] = ZX - XZ = \begin{pmatrix} 1 & 0 \\ 0 & -1 \end{pmatrix}\begin{pmatrix} 0 & 1 \\ 1 & 0 \end{pmatrix} - \begin{pmatrix} 0 & 1 \\ 1 & 0 \end{pmatrix}\begin{pmatrix} 1 & 0 \\ 0 & -1 \end{pmatrix}$
		$= \begin{pmatrix} 0 & 1 \\ -1 & 0 \end{pmatrix} - \begin{pmatrix} 0 & -1 \\ 1 & 0 \end{pmatrix} = \begin{pmatrix} 0 & 2 \\ -2 & 0 \end{pmatrix} = 2iY$

		Observamos que $[X,Y] = 2iZ$, $[Y,Z] = 2iX$, $[Z,X] = 2iY$, lo que constituye las relaciones de conmutación cíclicas de las matrices de Pauli.

		b) Para expresar $A = \begin{pmatrix} 3 & 2-i \\ 2+i & 5 \end{pmatrix}$ como combinación lineal de la identidad y las matrices de Pauli, usamos:
		$A = aI + bX + cY + dZ$

		Igualando componentes:
		$\begin{pmatrix} 3 & 2-i \\ 2+i & 5 \end{pmatrix} = \begin{pmatrix} a+d & b-ic \\ b+ic & a-d \end{pmatrix}$

		Esto nos da:
		$a+d = 3$
		$b-ic = 2-i$
		$b+ic = 2+i$
		$a-d = 5$

		Resolviendo:
		De $a+d=3$ y $a-d=5$, obtenemos $2a=8 \Rightarrow a=4$ y $d=-1$
		De $b-ic=2-i$ y $b+ic=2+i$, obtenemos $2b=4 \Rightarrow b=2$ y $2ic=2i \Rightarrow c=1$

		Por tanto:
		$A = 4I + 2X + Y - Z$

		c) Verificación de si $A$ es hermitiana:
		$A^\dagger = \begin{pmatrix} 3 & 2+i \\ 2-i & 5 \end{pmatrix} = A$

		Efectivamente, $A$ es hermitiana.

		Para calcular los valores propios, resolvemos la ecuación característica:
		$\det(A-\lambda I) = 0$

		$\det\begin{pmatrix} 3-\lambda & 2-i \\ 2+i & 5-\lambda \end{pmatrix} = (3-\lambda)(5-\lambda) - (2-i)(2+i) = 0$
		$(3-\lambda)(5-\lambda) - (4+1) = 0$
		$(3-\lambda)(5-\lambda) - 5 = 0$
		$15 - 3\lambda - 5\lambda + \lambda^2 - 5 = 0$
		$\lambda^2 - 8\lambda + 10 = 0$

		Usando la fórmula cuadrática:
		$\lambda = \frac{8 \pm \sqrt{64-40}}{2} = \frac{8 \pm \sqrt{24}}{2} = 4 \pm \sqrt{6}$

		Los valores propios son $\lambda_1 = 4 + \sqrt{6} \approx 6.45$ y $\lambda_2 = 4 - \sqrt{6} \approx 1.55$.

		Para los vectores propios, para $\lambda_1 = 4 + \sqrt{6}$:
		$(A - \lambda_1 I)v = 0$
		$\begin{pmatrix} 3-(4+\sqrt{6}) & 2-i \\ 2+i & 5-(4+\sqrt{6}) \end{pmatrix}v = 0$
		$\begin{pmatrix} -1-\sqrt{6} & 2-i \\ 2+i & 1-\sqrt{6} \end{pmatrix}v = 0$

		Resolviendo este sistema, obtenemos el vector propio (no normalizado):
		$v_1 = \begin{pmatrix} 2-i \\ 1+\sqrt{6} \end{pmatrix}$

		Similarmente, para $\lambda_2 = 4 - \sqrt{6}$, el vector propio es:
		$v_2 = \begin{pmatrix} 2-i \\ 1-\sqrt{6} \end{pmatrix}$

		Normalizando estos vectores, obtenemos los vectores propios normalizados.

		d) Para calcular $e^{i\theta X}$, usamos la expansión en serie de la exponencial:
		$e^{i\theta X} = I + i\theta X + \frac{(i\theta X)^2}{2!} + \frac{(i\theta X)^3}{3!} + \ldots$

		Dado que $X^2 = I$, tenemos:
		$e^{i\theta X} = I + i\theta X + \frac{(i\theta)^2}{2!}I + \frac{(i\theta)^3}{3!}X + \ldots$
		$= I(1 - \frac{\theta^2}{2!} + \frac{\theta^4}{4!} - \ldots) + iX(\theta - \frac{\theta^3}{3!} + \frac{\theta^5}{5!} - \ldots)$
		$= I\cos\theta + iX\sin\theta$
		$= \begin{pmatrix} \cos\theta & i\sin\theta \\ i\sin\theta & \cos\theta \end{pmatrix}$

		Interpretación en la esfera de Bloch:
		- $e^{i\theta X}$ representa una rotación en la esfera de Bloch alrededor del eje x por un ángulo $2\theta$
		- Esta operación transforma el estado $|0\rangle$ en $\cos\theta|0\rangle + i\sin\theta|1\rangle$
		- Para $\theta = \pi/2$, obtenemos $e^{i\pi X/2} = iX$, que es equivalente a la compuerta cuántica $X$ salvo fase global
		- Esta compuerta es esencial en circuitos cuánticos para realizar rotaciones controladas de un solo qubit
		- En computación cuántica, se conoce como la compuerta $R_X(\theta)$ y es parte del conjunto universal de compuertas
		- La unitariedad de $e^{i\theta X}$ garantiza que la transformación preserva la norma del estado
	\end{solution}
\end{questions}