\begin{questions}
  \question[4] Responda a las siguientes cuestiones:
  \begin{parts}


    \part Explique cómo se representa un operador lineal mediante una matriz y cómo se calcula la representación matricial de un operador en una base dada.

    \begin{solution}
      Un operador lineal sobre un espacio vectorial puede representarse mediante una matriz respecto a una base determinada, permitiendo realizar cálculos concretos utilizando las propiedades del álgebra matricial.

      Representación matricial de un operador:

      1) Definición formal:
      Sea $T: V \rightarrow W$ un operador lineal entre espacios vectoriales de dimensión finita, y sean $\{v_1, v_2, ..., v_n\}$ y $\{w_1, w_2, ..., w_m\}$ bases ordenadas de $V$ y $W$ respectivamente. La representación matricial de $T$ respecto a estas bases es la matriz $[T]$ de dimensión $m \times n$ cuyos elementos están dados por:

      $T(v_j) = \sum_{i=1}^{m} [T]_{ij} w_i$ para $j = 1, 2, ..., n$

      donde $[T]_{ij}$ es el elemento en la fila $i$, columna $j$ de la matriz $[T]$.

      2) Procedimiento para calcular $[T]$:
      a) Aplicar el operador $T$ a cada vector de la base de $V$
      b) Expresar cada resultado $T(v_j)$ como combinación lineal de los vectores de la base de $W$
      c) Los coeficientes de estas combinaciones lineales constituyen las columnas de la matriz $[T]$

      3) Propiedades clave:
      - Si $T: V \rightarrow V$ y usamos la misma base para representar el dominio y codominio, obtenemos una matriz cuadrada
      - La composición de operadores corresponde al producto de matrices: $[S \circ T] = [S][T]$
      - El operador identidad corresponde a la matriz identidad
      - El cambio de coordenadas transforma la matriz del operador mediante: $[T]_{\mathcal{B}'} = P^{-1}[T]_{\mathcal{B}}P$ donde $P$ es la matriz de cambio de coordenadas

      Cálculo de la representación matricial:

      Sea $T: V \rightarrow V$ un operador lineal en un espacio vectorial $V$ de dimensión $n$ con base $\mathcal{B} = \{v_1, v_2, ..., v_n\}$. Para obtener la matriz $[T]_{\mathcal{B}}$:

      1) Para cada $j = 1, 2, ..., n$, calculamos $T(v_j)$
      2) Expresamos $T(v_j) = \sum_{i=1}^{n} c_{ij} v_i$ (donde $c_{ij}$ son coeficientes escalares)
      3) Los coeficientes $c_{ij}$ forman la $j$-ésima columna de la matriz $[T]_{\mathcal{B}}$

      Ejemplo en contexto cuántico:

      Consideremos el operador de Hadamard $H$ que actúa sobre un cúbit. Determinaremos su representación matricial en la base computacional $\mathcal{B} = \{|0\rangle, |1\rangle\}$.

      Por definición, el operador Hadamard actúa sobre los vectores de la base como:
      $H(|0\rangle) = \frac{1}{\sqrt{2}}(|0\rangle + |1\rangle)$
      $H(|1\rangle) = \frac{1}{\sqrt{2}}(|0\rangle - |1\rangle)$

      Paso 1: Aplicamos $H$ a cada vector de la base.
      $H(|0\rangle) = \frac{1}{\sqrt{2}}|0\rangle + \frac{1}{\sqrt{2}}|1\rangle$
      $H(|1\rangle) = \frac{1}{\sqrt{2}}|0\rangle - \frac{1}{\sqrt{2}}|1\rangle$

      Paso 2: Identificamos los coeficientes en estas expresiones.
      Para $H(|0\rangle)$, los coeficientes son $c_{10} = \frac{1}{\sqrt{2}}$ y $c_{20} = \frac{1}{\sqrt{2}}$
      Para $H(|1\rangle)$, los coeficientes son $c_{11} = \frac{1}{\sqrt{2}}$ y $c_{21} = -\frac{1}{\sqrt{2}}$

      Paso 3: Construimos la matriz $[H]_{\mathcal{B}}$ colocando estos coeficientes como columnas.
      $[H]_{\mathcal{B}} = \begin{pmatrix} \frac{1}{\sqrt{2}} & \frac{1}{\sqrt{2}} \\ \frac{1}{\sqrt{2}} & -\frac{1}{\sqrt{2}} \end{pmatrix}$

      Esta es la conocida matriz de Hadamard:
      $H = \frac{1}{\sqrt{2}}\begin{pmatrix} 1 & 1 \\ 1 & -1 \end{pmatrix}$

      Ejemplo adicional: Representación del operador $Z$ (Pauli-Z) en diferentes bases.

      1) En la base computacional $\mathcal{B} = \{|0\rangle, |1\rangle\}$:
      $Z(|0\rangle) = |0\rangle$ implica coeficientes $c_{10} = 1$, $c_{20} = 0$
      $Z(|1\rangle) = -|1\rangle$ implica coeficientes $c_{11} = 0$, $c_{21} = -1$

      Por tanto:
      $[Z]_{\mathcal{B}} = \begin{pmatrix} 1 & 0 \\ 0 & -1 \end{pmatrix}$

      2) En la base de Hadamard $\mathcal{B}' = \{|+\rangle, |-\rangle\}$ donde $|+\rangle = \frac{1}{\sqrt{2}}(|0\rangle + |1\rangle)$ y $|-\rangle = \frac{1}{\sqrt{2}}(|0\rangle - |1\rangle)$:

      $Z(|+\rangle) = Z\left(\frac{|0\rangle + |1\rangle}{\sqrt{2}}\right) = \frac{|0\rangle - |1\rangle}{\sqrt{2}} = |-\rangle$

      $Z(|-\rangle) = Z\left(\frac{|0\rangle - |1\rangle}{\sqrt{2}}\right) = \frac{|0\rangle + |1\rangle}{\sqrt{2}} = |+\rangle$

      Por tanto:
      $[Z]_{\mathcal{B}'} = \begin{pmatrix} 0 & 1 \\ 1 & 0 \end{pmatrix}$

      Observamos que en la base de Hadamard, ¡el operador $Z$ se representa mediante la misma matriz que representa al operador $X$ en la base computacional!

      Esta representación matricial es fundamental en computación cuántica para:
      - Implementar algoritmos mediante secuencias de matrices
      - Analizar propiedades de observables físicos
      - Calcular evoluciones temporales de sistemas cuánticos
      - Diseñar y optimizar compuertas cuánticas
      - Verificar la unitariedad y otras propiedades de transformaciones cuánticas

      La elección adecuada de la base puede simplificar significativamente el análisis y la implementación de operaciones cuánticas, como demuestra el ejemplo del operador $Z$ en diferentes bases.
    \end{solution}

    \part Explique en qué consiste la descomposición de Schmidt de un estado bipartito y su utilidad para detectar y cuantificar el entrelazamiento cuántico.

    \begin{solution}
      La descomposición de Schmidt es una técnica matemática fundamental en información cuántica que proporciona una forma canónica para representar estados bipartitos, facilitando la detección y cuantificación del entrelazamiento entre los subsistemas.

      Definición formal:

      Sea $|\psi\rangle \in \mathcal{H}_A \otimes \mathcal{H}_B$ un estado puro bipartito (normalizado) en un sistema cuántico compuesto por dos subsistemas $A$ y $B$. La descomposición de Schmidt afirma que existe una base ortonormal $\{|i_A\rangle\}$ para el subsistema $A$ y una base ortonormal $\{|i_B\rangle\}$ para el subsistema $B$ tales que:

      $|\psi\rangle = \sum_{i=1}^{r} \lambda_i |i_A\rangle \otimes |i_B\rangle$

      donde:
      - $\lambda_i > 0$ son números reales positivos llamados coeficientes de Schmidt
      - $\sum_{i=1}^{r} \lambda_i^2 = 1$ (normalización)
      - $r \leq \min(\dim \mathcal{H}_A, \dim \mathcal{H}_B)$ es el rango de Schmidt del estado

      Propiedades clave:

      1) Unicidad:
      - Los coeficientes de Schmidt $\lambda_i$ son únicos (salvo permutaciones)
      - Las bases $\{|i_A\rangle\}$ y $\{|i_B\rangle\}$ son únicas cuando los $\lambda_i$ son todos distintos

      2) Matrices de densidad reducidas:
      - Ambas matrices de densidad reducidas tienen los mismos valores propios no nulos: $\lambda_i^2$
      - $\rho_A = \text{Tr}_B(|\psi\rangle\langle\psi|) = \sum_{i=1}^{r} \lambda_i^2 |i_A\rangle\langle i_A|$
      - $\rho_B = \text{Tr}_A(|\psi\rangle\langle\psi|) = \sum_{i=1}^{r} \lambda_i^2 |i_B\rangle\langle i_B|$

      3) Rango de Schmidt:
      - Coincide con el rango de ambas matrices de densidad reducidas
      - $r = 1$ si y solo si el estado es separable (no entrelazado)
      - $r > 1$ implica que el estado está entrelazado
      - $r = \min(\dim \mathcal{H}_A, \dim \mathcal{H}_B)$ indica entrelazamiento máximo cuando todos los $\lambda_i$ son iguales

      Procedimiento para calcular la descomposición de Schmidt:

      1) Expresar el estado en la forma general:
      $|\psi\rangle = \sum_{i,j} c_{ij} |i_A\rangle \otimes |j_B\rangle$
      donde $\{|i_A\rangle\}$ y $\{|j_B\rangle\}$ son bases ortonormales arbitrarias

      2) Construir la matriz de coeficientes $C = (c_{ij})$

      3) Calcular la descomposición en valores singulares (SVD) de $C$:
      $C = UDV^\dagger$ donde $D$ es diagonal con elementos $\lambda_i$

      4) Construir las nuevas bases:
      $|i_A'\rangle = \sum_j U_{ji} |j_A\rangle$
      $|i_B'\rangle = \sum_j V_{ji}^* |j_B\rangle$

      5) Reescribir el estado en la forma de Schmidt:
      $|\psi\rangle = \sum_i \lambda_i |i_A'\rangle \otimes |i_B'\rangle$

      Ejemplos ilustrativos:

      1) Estado separable:
      $|\psi\rangle = |0_A\rangle \otimes |1_B\rangle$

      Este estado ya está en forma de Schmidt con $r = 1$ y $\lambda_1 = 1$
      Al tener rango de Schmidt 1, es separable (no entrelazado)

      2) Estado de Bell:
      $|\psi\rangle = \frac{1}{\sqrt{2}}(|0_A\rangle \otimes |0_B\rangle + |1_A\rangle \otimes |1_B\rangle)$

      Este estado ya está en forma de Schmidt con $r = 2$ y $\lambda_1 = \lambda_2 = \frac{1}{\sqrt{2}}$
      Al tener rango de Schmidt 2 y coeficientes iguales, está máximamente entrelazado

      3) Estado asimétrico:
      $|\psi\rangle = \frac{\sqrt{3}}{2}|0_A\rangle \otimes |0_B\rangle + \frac{1}{2}|1_A\rangle \otimes |1_B\rangle$

      Este estado ya está en forma de Schmidt con $r = 2$, $\lambda_1 = \frac{\sqrt{3}}{2}$ y $\lambda_2 = \frac{1}{2}$
      Está entrelazado pero no máximamente

      Utilidad para detectar y cuantificar entrelazamiento:

      1) Criterio para detectar entrelazamiento:
      - Un estado bipartito puro está entrelazado si y solo si su rango de Schmidt es mayor que 1
      - Este es un criterio exacto y computacionalmente eficiente para estados puros
      - Análisis: si $r = 1$, entonces $|\psi\rangle = |a\rangle \otimes |b\rangle$ es un producto tensorial (separable)

      2) Cuantificación del entrelazamiento:
      - Entropía de entrelazamiento: $E(|\psi\rangle) = -\sum_i \lambda_i^2 \log_2 \lambda_i^2$
      - Este es un medidor estándar del entrelazamiento en estados puros
      - $E = 0$ para estados separables (un solo coeficiente igual a 1)
      - $E = \log_2 d$ para estados máximamente entrelazados ($d$ coeficientes iguales a $1/\sqrt{d}$)
      - La entropía de entrelazamiento coincide con la entropía de von Neumann de cualquiera de las matrices reducidas

      3) Otras medidas basadas en la descomposición:
      - Entrelazamiento de formación
      - Concurrencia (para dos cúbits)
      - Negatividad
      - Todas estas medidas pueden expresarse en términos de los coeficientes de Schmidt para estados puros

      4) Purificación y extensión a estados mixtos:
      - La descomposición de Schmidt está directamente relacionada con la purificación de estados mixtos
      - Permite desarrollar medidas de entrelazamiento para estados mixtos a través de la "extensión de techo convexo"

      5) Aplicaciones prácticas:
      - Evaluación de recursos cuánticos disponibles para protocolos de información cuántica
      - Análisis de capacidad de canales cuánticos
      - Diseño de códigos correctores de errores cuánticos
      - Caracterización experimental de estados cuánticos

      Limitaciones:

      1) Solo aplicable a estados puros bipartitos
      - Para estados mixtos se requieren otras técnicas como PPT o testigos de entrelazamiento
      - Para sistemas multipartitos se necesitan generalizaciones (descomposición de tensor)

      2) No proporciona clasificación completa del entrelazamiento multipartito
      - No distingue entre diferentes clases de entrelazamiento (como GHZ vs. W)
      - No caracteriza completamente correlaciones cuánticas en sistemas de más de dos partes

      La descomposición de Schmidt es una herramienta matemática poderosa que conecta áreas como el álgebra lineal (valores singulares), la teoría de la información (entropía) y la mecánica cuántica (entrelazamiento), proporcionando una descripción elegante y útil de los estados cuánticos compuestos.
    \end{solution}

  \end{parts}

  \question[3]
  Considere la compuerta Hadamard (H) y la compuerta de fase (S) definidas por:
  $H = \frac{1}{\sqrt{2}}\begin{pmatrix} 1 & 1 \\ 1 & -1 \end{pmatrix}, \quad S = \begin{pmatrix} 1 & 0 \\ 0 & i \end{pmatrix}$.
  \begin{parts}

    \part Verifique que ambas compuertas son unitarias.
    \part Calcule el producto $HSH$ y determine si el resultado es una compuerta conocida.
    \part Represente la acción de las compuertas H, S y HSH en la esfera de Bloch.
  \end{parts}

  \begin{solution}
    a) Para verificar que una compuerta es unitaria, debemos comprobar que $U^\dagger U = UU^\dagger = I$.

    Para la compuerta Hadamard $H$:
    $H^\dagger = \frac{1}{\sqrt{2}}\begin{pmatrix} 1 & 1 \\ 1 & -1 \end{pmatrix}$ (H es real y simétrica, por lo que $H^\dagger = H$)

    $H^\dagger H = H^2 = \frac{1}{2}\begin{pmatrix} 1 & 1 \\ 1 & -1 \end{pmatrix}\begin{pmatrix} 1 & 1 \\ 1 & -1 \end{pmatrix}$
    $= \frac{1}{2}\begin{pmatrix} 1+1 & 1-1 \\ 1-1 & 1+1 \end{pmatrix} = \begin{pmatrix} 1 & 0 \\ 0 & 1 \end{pmatrix} = I$

    Para la compuerta de fase $S$:
    $S^\dagger = \begin{pmatrix} 1 & 0 \\ 0 & -i \end{pmatrix}$ (conjugado transpuesto de $S$)

    $S^\dagger S = \begin{pmatrix} 1 & 0 \\ 0 & -i \end{pmatrix}\begin{pmatrix} 1 & 0 \\ 0 & i \end{pmatrix} = \begin{pmatrix} 1 & 0 \\ 0 & -i \cdot i \end{pmatrix} = \begin{pmatrix} 1 & 0 \\ 0 & 1 \end{pmatrix} = I$

    $SS^\dagger = \begin{pmatrix} 1 & 0 \\ 0 & i \end{pmatrix}\begin{pmatrix} 1 & 0 \\ 0 & -i \end{pmatrix} = \begin{pmatrix} 1 & 0 \\ 0 & i \cdot (-i) \end{pmatrix} = \begin{pmatrix} 1 & 0 \\ 0 & 1 \end{pmatrix} = I$

    Por lo tanto, ambas compuertas H y S son unitarias.

    b) Calculemos el producto $HSH$:

    Primero, calculamos $SH$:
    $SH = \begin{pmatrix} 1 & 0 \\ 0 & i \end{pmatrix} \cdot \frac{1}{\sqrt{2}}\begin{pmatrix} 1 & 1 \\ 1 & -1 \end{pmatrix} = \frac{1}{\sqrt{2}}\begin{pmatrix} 1 & 1 \\ i & -i \end{pmatrix}$

    Ahora, calculamos $HSH$:
    $HSH = \frac{1}{\sqrt{2}}\begin{pmatrix} 1 & 1 \\ 1 & -1 \end{pmatrix} \cdot \frac{1}{\sqrt{2}}\begin{pmatrix} 1 & 1 \\ i & -i \end{pmatrix}$
    $= \frac{1}{2}\begin{pmatrix} 1+i & 1-i \\ 1-i & -1-i \end{pmatrix}$

    Para determinar si esta es una compuerta conocida, podemos compararla con compuertas estándar o buscar propiedades especiales.

    En este caso, podemos factorizar un término de fase global $e^{i\pi/4}$ (que no afecta las mediciones):
    $HSH = e^{i\pi/4} \cdot \frac{1}{\sqrt{2}}\begin{pmatrix} e^{-i\pi/4} & e^{i\pi/4} \\ e^{i\pi/4} & -e^{-i\pi/4} \end{pmatrix}$

    Comparando con compuertas conocidas, observamos que $HSH$ se relaciona con la compuerta $Y = \begin{pmatrix} 0 & -i \\ i & 0 \end{pmatrix}$ de la siguiente manera:
    $HSH = e^{i\pi/4} \cdot \frac{Y+I}{\sqrt{2}}$

    Esto significa que $HSH$ es equivalente (salvo fase global) a la compuerta $\frac{Y+I}{\sqrt{2}}$, que es una rotación específica en la esfera de Bloch. También puede verse como una rotación del tipo $R_y(\pi/2)$, es decir, una rotación de 90° alrededor del eje y.

    c) Representación en la esfera de Bloch:

    1) Compuerta H:
    - Efecto sobre los ejes: $H$ transforma el eje z al eje x y el eje x al eje z (con inversión)
    - Geométricamente: Rotación de 90° alrededor del eje y, seguida de una reflexión en el plano xz
    - Mapeo específico:
    * $|0\rangle$ (polo norte) → $|+\rangle$ (eje x positivo)
    * $|1\rangle$ (polo sur) → $|-\rangle$ (eje x negativo)
    * $|+\rangle$ (eje x positivo) → $|0\rangle$ (polo norte)
    * $|-\rangle$ (eje x negativo) → $|1\rangle$ (polo sur)

    2) Compuerta S:
    - Efecto: Rotación de 90° alrededor del eje z
    - Mapeo específico:
    * $|0\rangle$ (polo norte) → permanece igual
    * $|1\rangle$ (polo sur) → permanece igual
    * $|+\rangle$ (eje x positivo) → $|+i\rangle$ (eje y positivo)
    * $|-\rangle$ (eje x negativo) → $|-i\rangle$ (eje y negativo)

    3) Compuerta HSH:
    - Efecto compuesto:
    * H: Rota el eje z al eje x
    * S: Rota el eje x al eje y
    * H: Rota el eje y al eje z (con inversión)
    * Resultado neto: Rotación de 90° alrededor del eje x
    - Mapeo específico:
    * $|0\rangle$ (polo norte) → eje y positivo
    * $|1\rangle$ (polo sur) → eje y negativo
    * $|+\rangle$ (eje x positivo) → permanece similar
    * $|-\rangle$ (eje x negativo) → permanece similar

    La secuencia HSH efectúa una rotación que lleva el eje z al eje y, realizando una rotación de 90° en la esfera de Bloch alrededor del eje x.

    d) Estado resultante al aplicar HSH al estado $|0\rangle$:

    Calculemos paso a paso:
    $H|0\rangle = \frac{1}{\sqrt{2}}(|0\rangle + |1\rangle) = |+\rangle$

    $SH|0\rangle = S|+\rangle = S\frac{1}{\sqrt{2}}(|0\rangle + |1\rangle) = \frac{1}{\sqrt{2}}(|0\rangle + i|1\rangle)$

    $HSH|0\rangle = H\frac{1}{\sqrt{2}}(|0\rangle + i|1\rangle)$
    $= \frac{1}{\sqrt{2}} \cdot \frac{1}{\sqrt{2}}[(|0\rangle + |1\rangle) + i(|0\rangle - |1\rangle)]$
    $= \frac{1}{2}[(1+i)|0\rangle + (1-i)|1\rangle]$

    Simplificando:
    $HSH|0\rangle = \frac{1+i}{2}|0\rangle + \frac{1-i}{2}|1\rangle$

    Podemos expresar este estado en forma polar:
    $\frac{1+i}{2} = \frac{\sqrt{2}}{2}e^{i\pi/4}$ y $\frac{1-i}{2} = \frac{\sqrt{2}}{2}e^{-i\pi/4}$

    Por lo tanto:
    $HSH|0\rangle = \frac{\sqrt{2}}{2}e^{i\pi/4}|0\rangle + \frac{\sqrt{2}}{2}e^{-i\pi/4}|1\rangle$

    Extrayendo una fase global $e^{i\pi/4}$:
    $HSH|0\rangle = e^{i\pi/4} \cdot \frac{1}{\sqrt{2}}(|0\rangle + e^{-i\pi/2}|1\rangle) = e^{i\pi/4} \cdot \frac{1}{\sqrt{2}}(|0\rangle - i|1\rangle)$

    Ignorando la fase global $e^{i\pi/4}$ (que no es observable):
    $HSH|0\rangle \approx \frac{1}{\sqrt{2}}(|0\rangle - i|1\rangle)$

    Este estado corresponde al estado $|-i\rangle$, que se encuentra en el eje y negativo de la esfera de Bloch, confirmando nuestra descripción geométrica anterior.
  \end{solution}

  \question[3]
  Considere un sistema de dos cúbits y los siguientes estados:
  $|\psi_1\rangle = \frac{1}{\sqrt{2}}(|00\rangle + |11\rangle)$, $|\psi_2\rangle = |0\rangle \otimes \frac{1}{\sqrt{2}}(|0\rangle + |1\rangle)$.
  \begin{parts}

    \part ¿Cuál de estos estados está entrelazado? Justifique su respuesta.
    \part Calcule las matrices de densidad reducidas para el primer cúbit en ambos estados.
    \part Supongamos que aplicamos la compuerta CNOT al estado $|\psi_2\rangle$. ¿Cuál será el estado resultante? ¿Está entrelazado?
  \end{parts}

  \begin{solution}
    a) Para determinar si un estado está entrelazado, debemos verificar si puede expresarse como un producto tensorial de estados individuales.

    Análisis del estado $|\psi_1\rangle = \frac{1}{\sqrt{2}}(|00\rangle + |11\rangle)$:

    Supongamos que $|\psi_1\rangle$ es separable. Entonces existirían estados $|\phi_A\rangle = a|0\rangle + b|1\rangle$ y $|\phi_B\rangle = c|0\rangle + d|1\rangle$ tales que:
    $|\psi_1\rangle = |\phi_A\rangle \otimes |\phi_B\rangle = (a|0\rangle + b|1\rangle) \otimes (c|0\rangle + d|1\rangle)$
    $= ac|00\rangle + ad|01\rangle + bc|10\rangle + bd|11\rangle$

    Comparando con $|\psi_1\rangle = \frac{1}{\sqrt{2}}(|00\rangle + |11\rangle)$, tenemos:
    $ac = \frac{1}{\sqrt{2}}$
    $ad = 0$
    $bc = 0$
    $bd = \frac{1}{\sqrt{2}}$

    De $ad = 0$ y $bc = 0$, dado que $ac \neq 0$ y $bd \neq 0$, deducimos que $a \neq 0$, $b \neq 0$, $c \neq 0$, $d \neq 0$. Pero esto contradice las ecuaciones $ad = 0$ y $bc = 0$.

    Por lo tanto, $|\psi_1\rangle$ no puede expresarse como un producto tensorial, lo que significa que está entrelazado. Este es un ejemplo de un estado de Bell, que es un estado máximamente entrelazado de dos cúbits.

    Análisis del estado $|\psi_2\rangle = |0\rangle \otimes \frac{1}{\sqrt{2}}(|0\rangle + |1\rangle)$:

    Este estado ya está escrito explícitamente como un producto tensorial:
    $|\psi_2\rangle = |0\rangle \otimes \frac{1}{\sqrt{2}}(|0\rangle + |1\rangle)$

    donde $|\phi_A\rangle = |0\rangle$ y $|\phi_B\rangle = \frac{1}{\sqrt{2}}(|0\rangle + |1\rangle)$.

    Por lo tanto, $|\psi_2\rangle$ es un estado separable (no entrelazado).

    Conclusión: $|\psi_1\rangle$ está entrelazado, mientras que $|\psi_2\rangle$ no lo está.

    b) Para calcular las matrices de densidad reducidas del primer cúbit, debemos calcular primero las matrices de densidad completas y luego tomar la traza parcial sobre el segundo cúbit.

    Para $|\psi_1\rangle = \frac{1}{\sqrt{2}}(|00\rangle + |11\rangle)$:

    La matriz de densidad completa es:
    $\rho_1 = |\psi_1\rangle\langle\psi_1| = \frac{1}{2}(|00\rangle + |11\rangle)(\langle00| + \langle11|)$
    $= \frac{1}{2}(|00\rangle\langle00| + |00\rangle\langle11| + |11\rangle\langle00| + |11\rangle\langle11|)$

    La matriz de densidad reducida del primer cúbit se obtiene tomando la traza parcial sobre el segundo cúbit:
    $\rho_{1A} = \text{Tr}_B(\rho_1) = \sum_i (I \otimes \langle i|)\rho_1(I \otimes |i\rangle)$

    $\rho_{1A} = (I \otimes \langle 0|)\rho_1(I \otimes |0\rangle) + (I \otimes \langle 1|)\rho_1(I \otimes |1\rangle)$

    Calculando cada término:
    $(I \otimes \langle 0|)\rho_1(I \otimes |0\rangle) = \frac{1}{2}|0\rangle\langle0|$
    $(I \otimes \langle 1|)\rho_1(I \otimes |1\rangle) = \frac{1}{2}|1\rangle\langle1|$

    Por tanto:
    $\rho_{1A} = \frac{1}{2}|0\rangle\langle0| + \frac{1}{2}|1\rangle\langle1| = \frac{1}{2}I$

    En forma matricial:
    $\rho_{1A} = \frac{1}{2}\begin{pmatrix} 1 & 0 \\ 0 & 1 \end{pmatrix}$

    Para $|\psi_2\rangle = |0\rangle \otimes \frac{1}{\sqrt{2}}(|0\rangle + |1\rangle) = \frac{1}{\sqrt{2}}(|00\rangle + |01\rangle)$:

    La matriz de densidad completa es:
    $\rho_2 = |\psi_2\rangle\langle\psi_2| = \frac{1}{2}(|00\rangle + |01\rangle)(\langle00| + \langle01|)$
    $= \frac{1}{2}(|00\rangle\langle00| + |00\rangle\langle01| + |01\rangle\langle00| + |01\rangle\langle01|)$

    La matriz de densidad reducida del primer cúbit:
    $\rho_{2A} = \text{Tr}_B(\rho_2)$

    Calculando:
    $(I \otimes \langle 0|)\rho_2(I \otimes |0\rangle) = \frac{1}{2}|0\rangle\langle0|$
    $(I \otimes \langle 1|)\rho_2(I \otimes |1\rangle) = \frac{1}{2}|0\rangle\langle0|$

    Por tanto:
    $\rho_{2A} = \frac{1}{2}|0\rangle\langle0| + \frac{1}{2}|0\rangle\langle0| = |0\rangle\langle0|$

    En forma matricial:
    $\rho_{2A} = \begin{pmatrix} 1 & 0 \\ 0 & 0 \end{pmatrix}$

    Observación: Mientras que para el estado entrelazado $|\psi_1\rangle$ la matriz reducida es máximamente mixta, para el estado separable $|\psi_2\rangle$ la matriz reducida representa un estado puro. Esto es consistente con la teoría del entrelazamiento.

    c) Aplicación de la compuerta CNOT al estado $|\psi_2\rangle$:

    Recordemos cómo actúa la compuerta CNOT en la base computacional:
    $\text{CNOT}|00\rangle = |00\rangle$
    $\text{CNOT}|01\rangle = |01\rangle$
    $\text{CNOT}|10\rangle = |11\rangle$
    $\text{CNOT}|11\rangle = |10\rangle$

    Ahora apliquemos CNOT a $|\psi_2\rangle$:
    $\text{CNOT}|\psi_2\rangle = \text{CNOT}\frac{1}{\sqrt{2}}(|00\rangle + |01\rangle)$
    $= \frac{1}{\sqrt{2}}(\text{CNOT}|00\rangle + \text{CNOT}|01\rangle)$
    $= \frac{1}{\sqrt{2}}(|00\rangle + |01\rangle)$

    Por lo tanto, $\text{CNOT}|\psi_2\rangle = |\psi_2\rangle$

    El estado resultante sigue siendo $|\psi_2\rangle = |0\rangle \otimes \frac{1}{\sqrt{2}}(|0\rangle + |1\rangle)$, que está en forma de producto tensorial. Por lo tanto, permanece como un estado no entrelazado.

    Esto tiene sentido porque la compuerta CNOT solo entrelaza estados cuando el cúbit de control (el primer cúbit en este caso) está en superposición. Como el primer cúbit de $|\psi_2\rangle$ está en el estado determinado $|0\rangle$, CNOT no crea entrelazamiento.

  \end{solution}
\end{questions}