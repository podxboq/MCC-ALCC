\codigonombre{}{MCC-ALCC-25Q110}

\begin{questions}

  \question[4] Responda a las siguientes cuestiones sobre espacios vectoriales:
  \begin{parts}
    \part Sea $H$ un espacio de Hilbert y $E$ un sistema ortonormal en $H$. Enuncie y demuestre la desigualdad de Bessel: para todo $x \in H$, $\sum_{z \in E} |\langle x, z\rangle|^2 \leq \|x\|^2$.

    \part Demuestre que si $A$ es una matriz hermitiana entonces todos sus valores propios son números reales.


  \end{parts}

  \question[3] Responda a las siguientes cuestiones sobre espacios de Hilbert:
  \begin{parts}
    \part Demuestre que el producto tensorial de dos operadores hermitianos es hermitiano.

    \part Considere los operadores $X = \begin{pmatrix} 0 & 1 \\ 1 & 0 \end{pmatrix}$, $Y = \begin{pmatrix} 0 & -i \\ i & 0 \end{pmatrix}$ y $Z = \begin{pmatrix} 1 & 0 \\ 0 & -1 \end{pmatrix}$. Verifique que forman una base para el espacio vectorial $\mathbb{C}^{2 \times 2}$ junto con la identidad $I$.

    \part Calcule el producto tensorial $X \otimes H$ donde $H = \frac{1}{\sqrt{2}}\begin{pmatrix} 1 & 1 \\ 1 & -1 \end{pmatrix}$ es la matriz de Hadamard. Justifique que el resultado es una matriz unitaria.
  \end{parts}

  \question[3] Sea $V$ un espacio vectorial complejo con producto interno, y sean $|v_1\rangle, |v_2\rangle, \ldots, |v_n\rangle$ vectores en $V$.
  \begin{parts}
    \part Sea $\{|v_1\rangle, |v_2\rangle, \ldots, |v_n\rangle\}$ un conjunto de vectores ortonormales en $V$. Demuestre la identidad de Parseval: para todo $|x\rangle, |y\rangle \in V$ se tiene que
    $\langle x|y \rangle = \sum_{k=1}^{n} \langle x|v_k\rangle \langle v_k|y\rangle$

    \part Aplique el proceso de ortonormalización de Gram-Schmidt al conjunto $\{(1,i), (1+i,2)\} \subseteq \mathbb{C}^2$ para obtener una base ortonormal.

    \part Sea $U$ un operador unitario que actúa sobre un qubit. Demuestre que los valores propios de $U$ tienen módulo 1. ¿Qué forma tienen los valores propios de un operador hermitiano?
  \end{parts}

\end{questions}