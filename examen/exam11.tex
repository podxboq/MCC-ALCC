\begin{questions}

  \question[4] Responda a las siguientes cuestiones sobre espacios de Hilbert:
  \begin{parts}
    \part Explique qué significa que un estado cuántico de dos qubits esté entrelazado. Proporcione un ejemplo de un estado entrelazado y uno separable, y demuestre esta propiedad para ambos casos.

    \part Describe como se calculan las raices unitarias de la unidad y explica su representación geométrica.
  \end{parts}

  \question[3] Considere la notación de Dirac para vectores en espacios de Hilbert.
  \begin{parts}
    \part Para el vector $|\psi\rangle = (-3-i, -2i, i, 2) \in \mathbb{C}^4$, calcule la norma $\||\psi\rangle\|$ y obtenga la expresión de $|\psi\rangle$ como combinación lineal de los vectores de la base canónica.

    \part Considere el producto tensorial $|0\rangle \otimes |+\rangle$, donde $|+\rangle = \frac{1}{\sqrt{2}}(|0\rangle + |1\rangle)$. Calcule este producto y exprese el resultado como un vector en $\mathbb{C}^4$ usando la base computacional.
  \end{parts}

  \question[3] Responda a las siguientes cuestiones sobre operadores lineales:
  \begin{parts}
    \part Sea $T: H \rightarrow H$ un operador lineal en un espacio de Hilbert $H$. Demuestre que $T$ es hermitiano si y solo si $\langle T(x), x\rangle \in \mathbb{R}$ para todo $x \in H$.

    \part Defina qué es un operador unitario y demuestre que un operador $U: H \rightarrow H$ es unitario si y solo si $U$ conserva la norma en $H$.

  \end{parts}

\end{questions}