\begin{questions}
  \question[4] Responda a las siguientes cuestiones:
  \begin{parts}
    \part Explique el concepto de vector propio y valor propio de un operador lineal, y su importancia en la mecánica cuántica.

    \begin{solution}
      Los vectores propios y valores propios son conceptos fundamentales del álgebra lineal que adquieren una interpretación física profunda en mecánica cuántica, constituyendo la base matemática para entender observables, mediciones y evolución de sistemas cuánticos.

      Definición formal:
      Sea $A$ un operador lineal que actúa sobre un espacio vectorial $V$. Un vector no nulo $|\psi\rangle \in V$ es un vector propio de $A$ si existe un escalar $\lambda$ tal que:
      $A|\psi\rangle = \lambda|\psi\rangle$

      El escalar $\lambda$ se denomina valor propio asociado al vector propio $|\psi\rangle$. En otras palabras, un vector propio es aquel que, al ser transformado por el operador, resulta en un múltiplo escalar de sí mismo.

      Propiedades matemáticas fundamentales:

      1) Para un operador lineal $A$ en un espacio de dimensión $n$:
      - Puede tener hasta $n$ valores propios distintos
      - Si $A$ es diagonalizable, existe una base formada por sus vectores propios
      - Los valores propios son las raíces del polinomio característico $\det(A - \lambda I) = 0$
      - Vectores propios correspondientes a valores propios distintos son linealmente independientes

      2) Para operadores hermitianos (caso crucial en mecánica cuántica):
      - Todos los valores propios son reales
      - Existe una base ortonormal completa de vectores propios
      - Vectores propios correspondientes a valores propios distintos son ortogonales
      - Descomposición espectral: $A = \sum_i \lambda_i |i\rangle\langle i|$, donde $|i\rangle$ son vectores propios ortonormales

      3) Para operadores unitarios:
      - Todos los valores propios tienen módulo 1 (forma $e^{i\theta}$)
      - Existe una base ortonormal completa de vectores propios
      - Descomposición: $U = \sum_i e^{i\theta_i} |i\rangle\langle i|$

      Importancia en mecánica cuántica:

      1) Interpretación de observables:
      - Los observables físicos están representados por operadores hermitianos
      - Los valores propios corresponden a los posibles resultados de medición
      - Los vectores propios corresponden a los estados donde el resultado de la medición es determinista
      - Ejemplo: El operador energía (hamiltoniano) tiene como valores propios las energías posibles del sistema y como vectores propios los estados estacionarios

      2) Postulado de medición:
      - Al medir un observable $A$ en un estado $|\psi\rangle$, el resultado será alguno de los valores propios $\lambda_i$ de $A$
      - La probabilidad de obtener el valor $\lambda_i$ es $P(\lambda_i) = |\langle \psi_i|\psi\rangle|^2$, donde $|\psi_i\rangle$ es el vector propio asociado
      - Tras la medición, el estado colapsa al vector propio correspondiente al valor medido
      - Ejemplo: La medición de espín de un electrón en la dirección z da como resultado $+\hbar/2$ o $-\hbar/2$ (valores propios de $S_z$)

      3) Resolución de la identidad:
      - Si $\{|\psi_i\rangle\}$ es una base ortonormal de vectores propios de un operador, entonces $\sum_i |\psi_i\rangle\langle\psi_i| = I$
      - Esta propiedad permite descomponer cualquier estado en términos de los vectores propios
      - Facilita cálculos como valores esperados: $\langle A \rangle = \sum_i \lambda_i |\langle \psi_i|\psi\rangle|^2$

      4) Estados estacionarios:
      - Los vectores propios del hamiltoniano son estados de energía definida
      - Su evolución temporal es particularmente simple: $|\psi(t)\rangle = e^{-iEt/\hbar}|\psi(0)\rangle$
      - Son fundamentales para entender sistemas cuánticos estables
      - Ejemplo: Estados electrónicos en átomos, niveles de energía en osciladores armónicos

      5) Principio de incertidumbre:
      - Dos observables son simultáneamente medibles si y solo si conmutan
      - Operadores que conmutan tienen una base común de vectores propios
      - Operadores que no conmutan no pueden tener una base común completa de vectores propios
      - Ejemplo: Posición y momento no conmutan, por lo que no pueden medirse simultáneamente con precisión arbitraria

      6) En computación cuántica:
      - Las compuertas cuánticas se diseñan considerando sus valores y vectores propios
      - Los algoritmos cuánticos a menudo explotan la estructura espectral de operadores específicos
      - La transformada cuántica de Fourier diagonaliza ciertos operadores
      - El algoritmo de estimación de fase encuentra valores propios de operadores unitarios

      7) Estados entrelazados:
      - Los valores propios de la matriz de densidad reducida cuantifican el entrelazamiento
      - La multiplicidad de valores propios no nulos indica el grado de entrelazamiento
      - La entropía de entrelazamiento se calcula a partir de estos valores propios

      8) Decoherencia:
      - Los "punteros" o estados preferidos que sobreviven a la decoherencia son aproximadamente vectores propios del hamiltoniano de interacción con el entorno
      - La base preferida de la decoherencia define la transición cuántico-clásica

      Ejemplos concretos:

      1) Matrices de Pauli:
      - $\sigma_z = \begin{pmatrix} 1 & 0 \\ 0 & -1 \end{pmatrix}$ tiene valores propios $\lambda_1 = 1, \lambda_2 = -1$
      - Vectores propios: $|0\rangle = \begin{pmatrix} 1 \\ 0 \end{pmatrix}$ y $|1\rangle = \begin{pmatrix} 0 \\ 1 \end{pmatrix}$
      - Representan estados de espín "arriba" y "abajo" en la dirección z

      2) Compuerta Hadamard:
      - $H = \frac{1}{\sqrt{2}}\begin{pmatrix} 1 & 1 \\ 1 & -1 \end{pmatrix}$ tiene valores propios $\lambda_1 = 1, \lambda_2 = -1$
      - Vectores propios: $|+\rangle = \frac{1}{\sqrt{2}}\begin{pmatrix} 1 \\ 1 \end{pmatrix}$ y $|-\rangle = \frac{1}{\sqrt{2}}\begin{pmatrix} 1 \\ -1 \end{pmatrix}$
      - Son estados de superposición igual de $|0\rangle$ y $|1\rangle$ con diferentes fases relativas

      Los valores y vectores propios proporcionan así una caracterización completa de las propiedades físicas de los sistemas cuánticos, conectando el formalismo matemático con observaciones experimentales y constituyendo el lenguaje fundamental para describir y manipular estados cuánticos.
    \end{solution}

    \part Describe como se calculan las raices unitarias de la unidad y explica su representación geométrica.

    \begin{solution}
      El teorema de no clonación cuántica establece una limitación fundamental en la manipulación de información cuántica: es imposible crear una copia exacta de un estado cuántico arbitrario desconocido sin perturbar el original. Este resultado, descubierto independientemente por Wootters, Zurek y Dieks en 1982, marca una diferencia esencial entre la información clásica y cuántica.

      Enunciado formal:
      No existe ningún mecanismo físico (operador unitario) $U$ que pueda realizar la siguiente transformación para estados cuánticos arbitrarios:
      $U(|\psi\rangle \otimes |s\rangle) = |\psi\rangle \otimes |\psi\rangle$
      donde $|\psi\rangle$ es un estado cuántico arbitrario desconocido y $|s\rangle$ es un estado "en blanco" fijo.

      Demostración:
      La demostración es sorprendentemente simple y elegante, basándose únicamente en la linealidad de la mecánica cuántica.

      1) Supongamos que existiera un operador unitario $U$ capaz de clonar estados cuánticos arbitrarios. Entonces para dos estados diferentes $|\phi\rangle$ y $|\psi\rangle$:

      $U(|\phi\rangle \otimes |s\rangle) = |\phi\rangle \otimes |\phi\rangle$
      $U(|\psi\rangle \otimes |s\rangle) = |\psi\rangle \otimes |\psi\rangle$

      2) Por la unitariedad, se preserva el producto interno entre estados:

      $\langle\phi| \otimes \langle s|U^\dagger U|\psi\rangle \otimes |s\rangle = \langle\phi|\psi\rangle \langle s|s\rangle = \langle\phi|\psi\rangle$

      Por otro lado:

      $\langle\phi| \otimes \langle\phi||\psi\rangle \otimes |\psi\rangle = \langle\phi|\psi\rangle \langle\phi|\psi\rangle = \langle\phi|\psi\rangle^2$

      3) Igualando ambas expresiones:

      $\langle\phi|\psi\rangle = \langle\phi|\psi\rangle^2$

      4) Esta igualdad solo tiene dos soluciones:
      - $\langle\phi|\psi\rangle = 0$ (estados ortogonales)
      - $\langle\phi|\psi\rangle = 1$ (estados idénticos)

      5) Concluimos que solo se pueden clonar estados que son ortogonales o idénticos, no estados arbitrarios con solapamiento parcial $(0 < |\langle\phi|\psi\rangle| < 1)$.

      Casos especiales y excepciones:

      1) Clonación de conjuntos ortogonales:
      - Es posible clonar un conjunto de estados mutuamente ortogonales, como $\{|0\rangle, |1\rangle\}$
      - Ejemplo: La puerta CNOT realiza $|0\rangle|0\rangle \to |0\rangle|0\rangle$ y $|1\rangle|0\rangle \to |1\rangle|1\rangle$

      2) Clonación aproximada:
      - El límite de Bužek-Hillery establece la máxima fidelidad posible para clonación universal aproximada
      - Fidelidad máxima: $F = \frac{2}{3}$ para clonación universal simétrica

      3) Clonación probabilística:
      - Se puede clonar perfectamente con cierta probabilidad de éxito < 1
      - La probabilidad depende del conjunto de estados a clonar

      4) Clonación de información clásica:
      - Los bits clásicos pueden copiarse perfectamente
      - La medición colapsa los cúbits a estados clásicos que luego pueden copiarse

      Implicaciones para la información cuántica:

      1) Seguridad criptográfica:
      - Fundamento de la distribución cuántica de claves (QKD)
      - Un espía no puede copiar cúbits en tránsito sin perturbarlos, revelando su presencia
      - Protocolos como BB84 basan su seguridad en este principio
      - La seguridad tiene garantía teórica, no solo computacional como en criptografía clásica

      2) Privacidad cuántica:
      - La información cuántica no puede copiarse sin autorización
      - Proporciona un principio físico fundamental para la privacidad
      - Permite protocolos como el compromiso de bit cuántico y firma digital cuántica

      3) Corrección de errores:
      - Complica la corrección de errores cuánticos, ya que no se pueden crear copias redundantes directamente
      - Requiere técnicas sofisticadas como códigos estabilizadores
      - Necesita entrelazamiento para distribuir información en múltiples cúbits sin clonarla

      4) Borrado cuántico:
      - Complementa el teorema de no clonación: es imposible borrar un estado cuántico desconocido
      - Juntos forman el principio de conservación de información cuántica
      - Importante para la reversibilidad de operaciones cuánticas

      5) Irreversibilidad de mediciones:
      - Explica por qué las mediciones cuánticas son destructivas
      - Impide determinar completamente un estado desconocido con una sola copia
      - Justifica la tomografía de estado cuántico que requiere múltiples copias idénticas

      6) Amplificación cuántica:
      - Imposibilita la amplificación perfecta de señales cuánticas
      - Limita el diseño de repetidores cuánticos para comunicaciones de larga distancia
      - Necesidad de repetidores basados en entrelazamiento y teleportación

      7) Computación cuántica:
      - Restringe ciertos algoritmos que en el mundo clásico dependen de crear copias
      - Impide el "fanout" (distribución de datos) clásico en circuitos
      - Limita el diseño de arquitecturas de computadores cuánticos

      8) Propiedades fundamentales:
      - Marca una diferencia esencial entre información clásica y cuántica
      - Refleja la naturaleza probabilística de la mecánica cuántica
      - Conecta con otros principios como la complementariedad y el teorema de no-telefonía

      9) Ventajas prácticas:
      - Base para el dinero cuántico: billetes que no pueden falsificarse
      - Permite verificación de posición cuántica
      - Posibilita testigos cuánticos no falsificables
      - Habilita esquemas de firma ciega cuántica

      El teorema de no clonación representa uno de los resultados más fundamentales en información cuántica, resaltando tanto las limitaciones como las capacidades únicas que ofrece la mecánica cuántica para el procesamiento y transmisión de información. Sus implicaciones técnicas y conceptuales continúan siendo exploradas y aplicadas en nuevos protocolos y algoritmos, consolidando su estatus como piedra angular de la teoría de información cuántica.
    \end{solution}

  \end{parts}

  \question[3]
  Considere un sistema de dos cúbits y los operadores de Pauli $X$, $Y$ y $Z$ que actúan sobre cúbits individuales.
  \begin{parts}

    \part Escriba las expresiones matriciales de $X \otimes I$, $I \otimes Z$ y $X \otimes Z$ en la base computacional $\{|00\rangle, |01\rangle, |10\rangle, |11\rangle\}$.
    \part Verifique que $[X \otimes I, I \otimes Z] = 0$ y calcule $[X \otimes I, Z \otimes I]$.
    \part ¿Es el estado resultante entrelazado? Justifique su respuesta.
  \end{parts}

  \begin{solution}
    a) Para escribir las expresiones matriciales de los operadores en la base computacional, recordemos primero las matrices de Pauli y la identidad:

    $X = \begin{pmatrix} 0 & 1 \\ 1 & 0 \end{pmatrix}$, $Y = \begin{pmatrix} 0 & -i \\ i & 0 \end{pmatrix}$, $Z = \begin{pmatrix} 1 & 0 \\ 0 & -1 \end{pmatrix}$, $I = \begin{pmatrix} 1 & 0 \\ 0 & 1 \end{pmatrix}$

    Ahora calculamos los productos tensoriales:

    $X \otimes I = \begin{pmatrix} 0 & 1 \\ 1 & 0 \end{pmatrix} \otimes \begin{pmatrix} 1 & 0 \\ 0 & 1 \end{pmatrix} = \begin{pmatrix} 0 \cdot I & 1 \cdot I \\ 1 \cdot I & 0 \cdot I \end{pmatrix} = \begin{pmatrix} 0 & 0 & 1 & 0 \\ 0 & 0 & 0 & 1 \\ 1 & 0 & 0 & 0 \\ 0 & 1 & 0 & 0 \end{pmatrix}$

    $I \otimes Z = \begin{pmatrix} 1 & 0 \\ 0 & 1 \end{pmatrix} \otimes \begin{pmatrix} 1 & 0 \\ 0 & -1 \end{pmatrix} = \begin{pmatrix} 1 \cdot Z & 0 \cdot Z \\ 0 \cdot Z & 1 \cdot Z \end{pmatrix} = \begin{pmatrix} 1 & 0 & 0 & 0 \\ 0 & -1 & 0 & 0 \\ 0 & 0 & 1 & 0 \\ 0 & 0 & 0 & -1 \end{pmatrix}$

    $X \otimes Z = \begin{pmatrix} 0 & 1 \\ 1 & 0 \end{pmatrix} \otimes \begin{pmatrix} 1 & 0 \\ 0 & -1 \end{pmatrix} = \begin{pmatrix} 0 \cdot Z & 1 \cdot Z \\ 1 \cdot Z & 0 \cdot Z \end{pmatrix} = \begin{pmatrix} 0 & 0 & 1 & 0 \\ 0 & 0 & 0 & -1 \\ 1 & 0 & 0 & 0 \\ 0 & -1 & 0 & 0 \end{pmatrix}$

    b) Para verificar que $[X \otimes I, I \otimes Z] = 0$, calculamos el conmutador:

    $[X \otimes I, I \otimes Z] = (X \otimes I)(I \otimes Z) - (I \otimes Z)(X \otimes I)$

    Primero calculamos el producto $(X \otimes I)(I \otimes Z)$:
    $(X \otimes I)(I \otimes Z) = X \otimes Z$

    Esto se debe a que $(A \otimes B)(C \otimes D) = (AC) \otimes (BD)$, por lo que:
    $(X \otimes I)(I \otimes Z) = (XI) \otimes (IZ) = X \otimes Z$

    De manera similar, calculamos $(I \otimes Z)(X \otimes I)$:
    $(I \otimes Z)(X \otimes I) = (IX) \otimes (ZI) = X \otimes Z$

    Por lo tanto:
    $[X \otimes I, I \otimes Z] = X \otimes Z - X \otimes Z = 0$

    Esto confirma que $X \otimes I$ y $I \otimes Z$ conmutan, lo cual es esperado ya que actúan en cúbits diferentes.

    Para calcular $[X \otimes I, Z \otimes I]$:

    $(X \otimes I)(Z \otimes I) = (XZ) \otimes (II) = XZ \otimes I$
    $(Z \otimes I)(X \otimes I) = (ZX) \otimes (II) = ZX \otimes I$

    Sabemos que $[X, Z] = XZ - ZX = 2iY$, por lo tanto:
    $[X \otimes I, Z \otimes I] = XZ \otimes I - ZX \otimes I = [X, Z] \otimes I = 2iY \otimes I$

    En forma matricial:
    $2iY \otimes I = 2i \begin{pmatrix} 0 & -i \\ i & 0 \end{pmatrix} \otimes \begin{pmatrix} 1 & 0 \\ 0 & 1 \end{pmatrix} = 2i \begin{pmatrix} 0 & -i & 0 & 0 \\ i & 0 & 0 & 0 \\ 0 & 0 & 0 & -i \\ 0 & 0 & i & 0 \end{pmatrix} = \begin{pmatrix} 0 & 2 & 0 & 0 \\ -2 & 0 & 0 & 0 \\ 0 & 0 & 0 & 2 \\ 0 & 0 & -2 & 0 \end{pmatrix}$

    c) Para determinar el resultado de aplicar $X \otimes Z$ al estado $|\psi\rangle = \frac{1}{\sqrt{2}}(|00\rangle + |11\rangle)$, calculamos:

    $(X \otimes Z)|\psi\rangle = \frac{1}{\sqrt{2}}(X \otimes Z)|00\rangle + \frac{1}{\sqrt{2}}(X \otimes Z)|11\rangle$

    Calculamos la acción de $X \otimes Z$ sobre los estados base:
    $(X \otimes Z)|00\rangle = X|0\rangle \otimes Z|0\rangle = |1\rangle \otimes |0\rangle = |10\rangle$
    $(X \otimes Z)|11\rangle = X|1\rangle \otimes Z|1\rangle = |0\rangle \otimes (-|1\rangle) = -|01\rangle$

    Por lo tanto:
    $(X \otimes Z)|\psi\rangle = \frac{1}{\sqrt{2}}(|10\rangle - |01\rangle)$

    d) Para determinar si el estado resultante $|\psi'\rangle = \frac{1}{\sqrt{2}}(|10\rangle - |01\rangle)$ está entrelazado, verificamos si puede escribirse como un producto tensorial de dos estados de un cúbit.

    Supongamos que $|\psi'\rangle$ es separable. Entonces debe existir $|\phi_A\rangle = a|0\rangle + b|1\rangle$ y $|\phi_B\rangle = c|0\rangle + d|1\rangle$ tales que:
    $|\psi'\rangle = |\phi_A\rangle \otimes |\phi_B\rangle = (a|0\rangle + b|1\rangle) \otimes (c|0\rangle + d|1\rangle) = ac|00\rangle + ad|01\rangle + bc|10\rangle + bd|11\rangle$

    Comparando con $|\psi'\rangle = \frac{1}{\sqrt{2}}(|10\rangle - |01\rangle)$, obtenemos:
    $ac = 0$
    $ad = -\frac{1}{\sqrt{2}}$
    $bc = \frac{1}{\sqrt{2}}$
    $bd = 0$

    De $ac = 0$ y $bd = 0$, tenemos que $a = 0$ o $c = 0$, y $b = 0$ o $d = 0$.

    Si $a = 0$, entonces de $bc = \frac{1}{\sqrt{2}}$ necesitamos $b \neq 0$ y $c \neq 0$. Pero esto implica que $bd = 0$ requiere $d = 0$. Pero entonces $ad = -\frac{1}{\sqrt{2}}$ no puede satisfacerse ya que $a = 0$.

    De manera similar, podemos verificar que ninguna otra combinación de $a = 0$ o $c = 0$ junto con $b = 0$ o $d = 0$ puede satisfacer todas las ecuaciones simultáneamente.

    Por lo tanto, $|\psi'\rangle$ no puede expresarse como un producto tensorial, lo que significa que está entrelazado.

    De hecho, $|\psi'\rangle = \frac{1}{\sqrt{2}}(|10\rangle - |01\rangle)$ es uno de los estados de Bell (específicamente $|\Psi^-\rangle$), conocido por ser un estado máximamente entrelazado de dos cúbits.

    Otra forma de verificar el entrelazamiento es calcular la matriz de densidad reducida de uno de los cúbits. Si el estado es puro y entrelazado, la matriz reducida será mixta.

    $\rho = |\psi'\rangle\langle\psi'| = \frac{1}{2}(|10\rangle - |01\rangle)(\langle10| - \langle01|) = \frac{1}{2}(|10\rangle\langle10| + |01\rangle\langle01| - |10\rangle\langle01| - |01\rangle\langle10|)$

    La matriz de densidad reducida del primer cúbit es:
    $\rho_A = \text{Tr}_B(\rho) = \frac{1}{2}(|1\rangle\langle1| + |0\rangle\langle0|) = \frac{1}{2}I$

    Dado que $\rho_A$ representa un estado máximamente mixto (sus valores propios son ambos $\frac{1}{2}$), podemos confirmar que $|\psi'\rangle$ está máximamente entrelazado.
  \end{solution}

  \question[3]
  Considere un sistema cuántico con espacio de Hilbert $\mathbb{C}^4$ y el operador hermitiano
  \[
    H = \begin{pmatrix}
      2 & 0 & 1 & 0 \\
      0 & 3 & 0 & 0 \\
      1 & 0 & 2 & 0 \\
      0 & 0 & 0 & 1
    \end{pmatrix}
  \]
  \begin{parts}

    \part Determine los valores propios y vectores propios de $H$.
    \part Escriba la descomposición espectral de $H$.
    \part Si el sistema está inicialmente en el estado $|\psi(0)\rangle = \frac{1}{2}(|0\rangle + |1\rangle + |2\rangle + |3\rangle)$, determine el estado $|\psi(t)\rangle = e^{-iHt}|\psi(0)\rangle$.
  \end{parts}

  \begin{solution}
    a) Para determinar los valores propios y vectores propios del operador hermitiano $H$, primero observamos que $H$ tiene una estructura especial que nos permite simplificar el cálculo.

    La matriz $H$ puede verse como una suma directa de dos bloques:
    - Un bloque $2 \times 2$ en la esquina superior izquierda: $\begin{pmatrix} 2 & 1 \\ 1 & 2 \end{pmatrix}$
    - Dos valores en la diagonal: $3$ y $1$

    Calculemos primero los valores propios y vectores propios del bloque $2 \times 2$:

    $\begin{pmatrix} 2 & 1 \\ 1 & 2 \end{pmatrix}$

    El polinomio característico es:
    $\det\begin{pmatrix} 2-\lambda & 1 \\ 1 & 2-\lambda \end{pmatrix} = (2-\lambda)^2 - 1 = \lambda^2 - 4\lambda + 3 = (\lambda-3)(\lambda-1)$

    Por lo tanto, los valores propios son $\lambda_1 = 3$ y $\lambda_2 = 1$.

    Para $\lambda_1 = 3$, encontramos el vector propio:
    $\begin{pmatrix} 2-3 & 1 \\ 1 & 2-3 \end{pmatrix}\begin{pmatrix} v_1 \\ v_2 \end{pmatrix} = \begin{pmatrix} 0 \\ 0 \end{pmatrix}$
    $\begin{pmatrix} -1 & 1 \\ 1 & -1 \end{pmatrix}\begin{pmatrix} v_1 \\ v_2 \end{pmatrix} = \begin{pmatrix} 0 \\ 0 \end{pmatrix}$

    Esto nos da $v_1 = v_2$. Normalizando, obtenemos el vector propio:
    $|v_1\rangle = \frac{1}{\sqrt{2}}\begin{pmatrix} 1 \\ 1 \end{pmatrix}$

    Para $\lambda_2 = 1$, encontramos el vector propio:
    $\begin{pmatrix} 2-1 & 1 \\ 1 & 2-1 \end{pmatrix}\begin{pmatrix} v_1 \\ v_2 \end{pmatrix} = \begin{pmatrix} 0 \\ 0 \end{pmatrix}$
    $\begin{pmatrix} 1 & 1 \\ 1 & 1 \end{pmatrix}\begin{pmatrix} v_1 \\ v_2 \end{pmatrix} = \begin{pmatrix} 0 \\ 0 \end{pmatrix}$

    Esto nos da $v_1 = -v_2$. Normalizando, obtenemos el vector propio:
    $|v_2\rangle = \frac{1}{\sqrt{2}}\begin{pmatrix} 1 \\ -1 \end{pmatrix}$

    Ahora, necesitamos extender estos vectores al espacio completo $\mathbb{C}^4$.

    Los valores propios y vectores propios de $H$ son:

    $\lambda_1 = 3$ con vector propio $|u_1\rangle = \frac{1}{\sqrt{2}}\begin{pmatrix} 1 \\ 0 \\ 1 \\ 0 \end{pmatrix}$

    $\lambda_2 = 1$ con vector propio $|u_2\rangle = \frac{1}{\sqrt{2}}\begin{pmatrix} 1 \\ 0 \\ -1 \\ 0 \end{pmatrix}$

    $\lambda_3 = 3$ con vector propio $|u_3\rangle = \begin{pmatrix} 0 \\ 1 \\ 0 \\ 0 \end{pmatrix}$

    $\lambda_4 = 1$ con vector propio $|u_4\rangle = \begin{pmatrix} 0 \\ 0 \\ 0 \\ 1 \end{pmatrix}$

    Nota: Hay una degeneración en los valores propios, ya que $\lambda_1 = \lambda_3 = 3$ y $\lambda_2 = \lambda_4 = 1$.

    b) La descomposición espectral de $H$ es:

    $H = \sum_{i=1}^{4} \lambda_i |u_i\rangle\langle u_i|$

    $H = 3 |u_1\rangle\langle u_1| + 1 |u_2\rangle\langle u_2| + 3 |u_3\rangle\langle u_3| + 1 |u_4\rangle\langle u_4|$

    Sustituyendo los vectores propios:

    $H = 3 \cdot \frac{1}{2}\begin{pmatrix} 1 \\ 0 \\ 1 \\ 0 \end{pmatrix}\begin{pmatrix} 1 & 0 & 1 & 0 \end{pmatrix} + 1 \cdot \frac{1}{2}\begin{pmatrix} 1 \\ 0 \\ -1 \\ 0 \end{pmatrix}\begin{pmatrix} 1 & 0 & -1 & 0 \end{pmatrix} + 3 \begin{pmatrix} 0 \\ 1 \\ 0 \\ 0 \end{pmatrix}\begin{pmatrix} 0 & 1 & 0 & 0 \end{pmatrix} + 1 \begin{pmatrix} 0 \\ 0 \\ 0 \\ 1 \end{pmatrix}\begin{pmatrix} 0 & 0 & 0 & 1 \end{pmatrix}$

    $H = \frac{3}{2}\begin{pmatrix} 1 & 0 & 1 & 0 \\ 0 & 0 & 0 & 0 \\ 1 & 0 & 1 & 0 \\ 0 & 0 & 0 & 0 \end{pmatrix} + \frac{1}{2}\begin{pmatrix} 1 & 0 & -1 & 0 \\ 0 & 0 & 0 & 0 \\ -1 & 0 & 1 & 0 \\ 0 & 0 & 0 & 0 \end{pmatrix} + 3\begin{pmatrix} 0 & 0 & 0 & 0 \\ 0 & 1 & 0 & 0 \\ 0 & 0 & 0 & 0 \\ 0 & 0 & 0 & 0 \end{pmatrix} + \begin{pmatrix} 0 & 0 & 0 & 0 \\ 0 & 0 & 0 & 0 \\ 0 & 0 & 0 & 0 \\ 0 & 0 & 0 & 1 \end{pmatrix}$

    Sumando estas matrices, obtenemos:

    $H = \begin{pmatrix} 2 & 0 & 1 & 0 \\ 0 & 3 & 0 & 0 \\ 1 & 0 & 2 & 0 \\ 0 & 0 & 0 & 1 \end{pmatrix}$

    Lo cual verifica nuestra descomposición espectral.

    c) Para calcular $e^{iH}$ utilizando la descomposición espectral, aplicamos la fórmula:

    $e^{iH} = \sum_{i=1}^{4} e^{i\lambda_i} |u_i\rangle\langle u_i|$

    $e^{iH} = e^{3i} |u_1\rangle\langle u_1| + e^i |u_2\rangle\langle u_2| + e^{3i} |u_3\rangle\langle u_3| + e^i |u_4\rangle\langle u_4|$

    Sustituyendo los vectores propios:

    $e^{iH} = e^{3i} \cdot \frac{1}{2}\begin{pmatrix} 1 \\ 0 \\ 1 \\ 0 \end{pmatrix}\begin{pmatrix} 1 & 0 & 1 & 0 \end{pmatrix} + e^i \cdot \frac{1}{2}\begin{pmatrix} 1 \\ 0 \\ -1 \\ 0 \end{pmatrix}\begin{pmatrix} 1 & 0 & -1 & 0 \end{pmatrix} + e^{3i} \begin{pmatrix} 0 \\ 1 \\ 0 \\ 0 \end{pmatrix}\begin{pmatrix} 0 & 1 & 0 & 0 \end{pmatrix} + e^i \begin{pmatrix} 0 \\ 0 \\ 0 \\ 1 \end{pmatrix}\begin{pmatrix} 0 & 0 & 0 & 1 \end{pmatrix}$

    $e^{iH} = \frac{e^{3i}}{2}\begin{pmatrix} 1 & 0 & 1 & 0 \\ 0 & 0 & 0 & 0 \\ 1 & 0 & 1 & 0 \\ 0 & 0 & 0 & 0 \end{pmatrix} + \frac{e^i}{2}\begin{pmatrix} 1 & 0 & -1 & 0 \\ 0 & 0 & 0 & 0 \\ -1 & 0 & 1 & 0 \\ 0 & 0 & 0 & 0 \end{pmatrix} + e^{3i}\begin{pmatrix} 0 & 0 & 0 & 0 \\ 0 & 1 & 0 & 0 \\ 0 & 0 & 0 & 0 \\ 0 & 0 & 0 & 0 \end{pmatrix} + e^i\begin{pmatrix} 0 & 0 & 0 & 0 \\ 0 & 0 & 0 & 0 \\ 0 & 0 & 0 & 0 \\ 0 & 0 & 0 & 1 \end{pmatrix}$

    Simplificando:

    $e^{iH} = \begin{pmatrix} \frac{e^{3i}+e^i}{2} & 0      & \frac{e^{3i}-e^i}{2} & 0   \\
                0                    & e^{3i} & 0                    & 0   \\
                \frac{e^{3i}-e^i}{2} & 0      & \frac{e^{3i}+e^i}{2} & 0   \\
                0                    & 0      & 0                    & e^i\end{pmatrix}$

    d) Para determinar el estado $|\psi(t)\rangle = e^{-iHt}|\psi(0)\rangle$, primero necesitamos calcular $e^{-iHt}$ utilizando la descomposición espectral:

    $e^{-iHt} = \sum_{i=1}^{4} e^{-i\lambda_i t} |u_i\rangle\langle u_i|$

    $e^{-iHt} = e^{-3it} |u_1\rangle\langle u_1| + e^{-it} |u_2\rangle\langle u_2| + e^{-3it} |u_3\rangle\langle u_3| + e^{-it} |u_4\rangle\langle u_4|$

    Ahora, aplicamos este operador al estado inicial:

    $|\psi(0)\rangle = \frac{1}{2}(|0\rangle + |1\rangle + |2\rangle + |3\rangle) = \frac{1}{2}\begin{pmatrix} 1 \\ 1 \\ 1 \\ 1 \end{pmatrix}$

    Para facilitar el cálculo, expresamos $|\psi(0)\rangle$ en términos de los vectores propios de $H$. Para esto, calculamos los coeficientes $c_i = \langle u_i|\psi(0)\rangle$:

    $c_1 = \langle u_1|\psi(0)\rangle = \frac{1}{\sqrt{2}} \cdot \frac{1}{2}(1 + 0 + 1 + 0) = \frac{1}{\sqrt{2}} \cdot \frac{2}{2} = \frac{1}{\sqrt{2}}$

    $c_2 = \langle u_2|\psi(0)\rangle = \frac{1}{\sqrt{2}} \cdot \frac{1}{2}(1 + 0 - 1 + 0) = \frac{1}{\sqrt{2}} \cdot \frac{0}{2} = 0$

    $c_3 = \langle u_3|\psi(0)\rangle = 1 \cdot \frac{1}{2}(0 + 1 + 0 + 0) = \frac{1}{2}$

    $c_4 = \langle u_4|\psi(0)\rangle = 1 \cdot \frac{1}{2}(0 + 0 + 0 + 1) = \frac{1}{2}$

    Entonces:
    $|\psi(0)\rangle = c_1|u_1\rangle + c_2|u_2\rangle + c_3|u_3\rangle + c_4|u_4\rangle = \frac{1}{\sqrt{2}}|u_1\rangle + 0|u_2\rangle + \frac{1}{2}|u_3\rangle + \frac{1}{2}|u_4\rangle$

    Ahora, calculamos el estado evolucionado:
    $|\psi(t)\rangle = e^{-iHt}|\psi(0)\rangle = \sum_{i=1}^{4} e^{-i\lambda_i t} \langle u_i|\psi(0)\rangle |u_i\rangle$

    $|\psi(t)\rangle = e^{-3it} \cdot \frac{1}{\sqrt{2}} \cdot |u_1\rangle + e^{-it} \cdot 0 \cdot |u_2\rangle + e^{-3it} \cdot \frac{1}{2} \cdot |u_3\rangle + e^{-it} \cdot \frac{1}{2} \cdot |u_4\rangle$

    $|\psi(t)\rangle = \frac{e^{-3it}}{\sqrt{2}}|u_1\rangle + \frac{e^{-3it}}{2}|u_3\rangle + \frac{e^{-it}}{2}|u_4\rangle$

    Sustituyendo los vectores propios:

    $|\psi(t)\rangle = \frac{e^{-3it}}{\sqrt{2}} \cdot \frac{1}{\sqrt{2}}\begin{pmatrix} 1 \\ 0 \\ 1 \\ 0 \end{pmatrix} + \frac{e^{-3it}}{2}\begin{pmatrix} 0 \\ 1 \\ 0 \\ 0 \end{pmatrix} + \frac{e^{-it}}{2}\begin{pmatrix} 0 \\ 0 \\ 0 \\ 1 \end{pmatrix}$

    $|\psi(t)\rangle = \frac{e^{-3it}}{2}\begin{pmatrix} 1 \\ 0 \\ 1 \\ 0 \end{pmatrix} + \frac{e^{-3it}}{2}\begin{pmatrix} 0 \\ 1 \\ 0 \\ 0 \end{pmatrix} + \frac{e^{-it}}{2}\begin{pmatrix} 0 \\ 0 \\ 0 \\ 1 \end{pmatrix}$

    $|\psi(t)\rangle = \frac{1}{2}\begin{pmatrix} e^{-3it} \\ e^{-3it} \\ e^{-3it} \\ e^{-it} \end{pmatrix}$

    Esto muestra que el estado evoluciona con diferentes fases para sus componentes, de acuerdo con los valores propios de $H$. Las componentes correspondientes a los valores propios $\lambda = 3$ oscilan con frecuencia $3$ veces mayor que las componentes correspondientes a $\lambda = 1$.

    En la base computacional, el estado evolucionado es:

    $|\psi(t)\rangle = \frac{1}{2}(e^{-3it}|0\rangle + e^{-3it}|1\rangle + e^{-3it}|2\rangle + e^{-it}|3\rangle)$

    $|\psi(t)\rangle = \frac{e^{-3it}}{2}(|0\rangle + |1\rangle + |2\rangle) + \frac{e^{-it}}{2}|3\rangle$

    Este resultado muestra cómo el hamiltoniano $H$ genera una evolución temporal que afecta de manera diferente a las distintas componentes del estado, creando una dinámica de fases relativas que es característica de los sistemas cuánticos.
  \end{solution}
\end{questions}