\codigonombre{}{MCC-ALCC-25Q103}

Conteste a las siguientes preguntas en el espacio anteriormente indicado.

\begin{questions}
	\question[4] Responda a las siguientes cuestiones:
	\begin{parts}

		\part Explique la notación de Dirac (bra-ket) y cómo se utiliza para representar estados cuánticos y operaciones sobre ellos.

		\begin{solution}
			La notación de Dirac o notación bra-ket es un formalismo matemático introducido por Paul Dirac para representar estados y operaciones en mecánica cuántica de manera compacta y elegante.

			Elementos básicos:
			1) Ket $|v\rangle$: Representa un vector columna en un espacio de Hilbert (estado cuántico)
			2) Bra $\langle v|$: Representa el dual de un ket, es decir, un vector fila que es el conjugado transpuesto del ket: $\langle v| = (|v\rangle)^\dagger$
			3) Braket $\langle u|v\rangle$: Producto interno entre dos vectores (amplitud de transición)
			4) Producto exterior $|u\rangle\langle v|$: Operador que proyecta en la dirección de $|u\rangle$ y desde la dirección de $|v\rangle$

			Usos principales:
			- Representación de estados: $|\psi\rangle = \alpha|0\rangle + \beta|1\rangle$ representa un qubit en superposición
			- Producto interno: $\langle\phi|\psi\rangle$ calcula la amplitud de probabilidad de transición del estado $|\psi\rangle$ al estado $|\phi\rangle$
			- Proyección: $P = |\psi\rangle\langle\psi|$ para $\langle\psi|\psi\rangle = 1$ representa un operador de proyección
			- Operadores: $A = \sum_{i,j} a_{ij}|i\rangle\langle j|$ representa un operador lineal en la base $\{|i\rangle\}$
			- Cálculo de valores esperados: $\langle\psi|A|\psi\rangle$ representa el valor esperado del observable $A$ en el estado $|\psi\rangle$
			- Expansión en base ortonormal: $|\psi\rangle = \sum_i |i\rangle\langle i|\psi\rangle$ donde $\{|i\rangle\}$ es una base ortonormal

			Ventajas:
			- Simplifica notablemente las expresiones matemáticas
			- Enfatiza la dualidad entre estados y mediciones
			- Facilita los cálculos en espacios de Hilbert de dimensión infinita
			- Permite manipular operadores de forma intuitiva
			- Hace evidentes las relaciones entre estados y operadores
		\end{solution}

		\part Explique el concepto de matriz de densidad y su utilidad para describir estados cuánticos mixtos.

		\begin{solution}
			La matriz de densidad (u operador de densidad) es una generalización de la descripción de estados cuánticos que permite representar tanto estados puros como mixtos.

			Definición formal:
			- Para un estado puro $|\psi\rangle$, la matriz de densidad es $\rho = |\psi\rangle\langle\psi|$
			- Para un estado mixto (ensamble estadístico de estados puros), la matriz de densidad es $\rho = \sum_i p_i |\psi_i\rangle\langle\psi_i|$, donde $p_i$ son probabilidades ($p_i \geq 0$, $\sum_i p_i = 1$) y $|\psi_i\rangle$ son estados puros

			Propiedades fundamentales:
			1) $\rho$ es hermitiana: $\rho = \rho^\dagger$
			2) $\rho$ es positiva semidefinida: $\langle\phi|\rho|\phi\rangle \geq 0$ para todo $|\phi\rangle$
			3) $\rho$ tiene traza 1: $Tr(\rho) = 1$
			4) $Tr(\rho^2) \leq 1$, con igualdad si y solo si $\rho$ representa un estado puro

			Utilidad y ventajas:
			- Permite representar estados con incertidumbre clásica (mixtos) además de superposiciones cuánticas
			- Es necesaria para describir subsistemas de sistemas entrelazados (matrices de densidad reducidas)
			- Simplifica el cálculo de valores esperados: $\langle A \rangle = Tr(\rho A)$
			- Es invariante bajo cambios de fase global en los estados constituyentes
			- Permite caracterizar completamente la información físicamente accesible de un sistema cuántico
			- Facilita la descripción de procesos con decoherencia y disipación
			- Es esencial en teoría de información cuántica para cuantificar entrelazamiento y entropía

			Matrices de densidad reducidas:
			- Para sistemas compuestos, si tenemos $\rho_{AB}$ para un sistema bipartito, podemos obtener la matriz reducida del subsistema A mediante la traza parcial: $\rho_A = Tr_B(\rho_{AB})$
			- Permite estudiar las propiedades de un subsistema cuando no se puede acceder al sistema completo
			- Para estados entrelazados, las matrices reducidas siempre representan estados mixtos, aunque el estado global sea puro

			El formalismo de la matriz de densidad es indispensable en campos como la computación cuántica, información cuántica, óptica cuántica y decoherencia, donde frecuentemente se trabaja con estados no ideales o subsistemas de sistemas más grandes.
		\end{solution}
	\end{parts}

	\question[3] Considere un operador hermitiano $A$ con valores propios $\lambda_1 = 2$ y $\lambda_2 = -1$, cuyos vectores propios normalizados son $|v_1\rangle = \frac{1}{\sqrt{5}}(2|0\rangle + i|1\rangle)$ y $|v_2\rangle = \frac{1}{\sqrt{5}}(i|0\rangle + 2|1\rangle)$ respectivamente.
	\begin{parts}
		\part Verifique que los vectores propios son ortogonales.
		\part Escriba el operador $A$ en notación de Dirac utilizando su descomposición espectral.
		\part Determine la representación matricial de $A$ en la base computacional $\{|0\rangle, |1\rangle\}$.
	\end{parts}

	\begin{solution}
		a) Para verificar que los vectores propios son ortogonales, calculamos su producto interno:

		$\langle v_1|v_2\rangle = \frac{1}{\sqrt{5}}\frac{1}{\sqrt{5}}(2\langle 0| - i\langle 1|)(i|0\rangle + 2|1\rangle)$
		$= \frac{1}{5}(2i + (-i)2)$
		$= \frac{1}{5}(2i - 2i)$
		$= 0$

		Por tanto, los vectores propios son ortogonales, como era de esperar para vectores propios de un operador hermitiano correspondientes a valores propios distintos.

		b) La descomposición espectral de $A$ en notación de Dirac es:

		$A = \sum_i \lambda_i |v_i\rangle\langle v_i| = \lambda_1 |v_1\rangle\langle v_1| + \lambda_2 |v_2\rangle\langle v_2|$
		$= 2|v_1\rangle\langle v_1| + (-1)|v_2\rangle\langle v_2|$
		$= 2|v_1\rangle\langle v_1| - |v_2\rangle\langle v_2|$

		c) Para determinar la representación matricial de $A$ en la base computacional, sustituimos las expresiones de los vectores propios:

		$A = 2\frac{1}{5}(2|0\rangle + i|1\rangle)(2\langle 0| - i\langle 1|) - \frac{1}{5}(i|0\rangle + 2|1\rangle)(i^*\langle 0| + 2\langle 1|)$
		$= \frac{2}{5}(4|0\rangle\langle 0| + 2i|0\rangle\langle 1| - 2i|1\rangle\langle 0| + |1\rangle\langle 1|) - \frac{1}{5}(-|0\rangle\langle 0| - 2i|0\rangle\langle 1| + 2i|1\rangle\langle 0| + 4|1\rangle\langle 1|)$
		$= \frac{1}{5}(8|0\rangle\langle 0| + 4i|0\rangle\langle 1| - 4i|1\rangle\langle 0| + 2|1\rangle\langle 1| + |0\rangle\langle 0| + 2i|0\rangle\langle 1| - 2i|1\rangle\langle 0| - 4|1\rangle\langle 1|)$
		$= \frac{1}{5}(9|0\rangle\langle 0| + 6i|0\rangle\langle 1| - 6i|1\rangle\langle 0| - 2|1\rangle\langle 1|)$

		En forma matricial:
		$A = \frac{1}{5}\begin{pmatrix} 9 & 6i \\ -6i & -2 \end{pmatrix}$

	\end{solution}

	\question[3] Considere las siguientes compuertas cuánticas: la compuerta Hadamard ($H$), la compuerta de fase ($S$), y la compuerta $T$.
	\begin{parts}

		\part Escriba las representaciones matriciales de estas compuertas en la base computacional.
		\part Verifique que estas compuertas son unitarias.
		\part Calcule el resultado de aplicar la secuencia $HSH$ al estado $|0\rangle$.
		\part Represente en la esfera de Bloch el efecto de la compuerta $H$ sobre los estados $|0\rangle$ y $|1\rangle$.
	\end{parts}

	\begin{solution}
		a) Las representaciones matriciales de las compuertas son:

		Compuerta Hadamard:
		$H = \frac{1}{\sqrt{2}}\begin{pmatrix} 1 & 1 \\ 1 & -1 \end{pmatrix}$

		Compuerta de fase $S$ (también conocida como $Z^{1/2}$):
		$S = \begin{pmatrix} 1 & 0 \\ 0 & i \end{pmatrix}$

		Compuerta $T$ (también conocida como $Z^{1/4}$):
		$T = \begin{pmatrix} 1 & 0 \\ 0 & e^{i\pi/4} \end{pmatrix}$

		b) Para verificar que son unitarias, calculamos $UU^\dagger$ para cada compuerta:

		Para $H$:
		$HH^\dagger = \frac{1}{\sqrt{2}}\begin{pmatrix} 1 & 1 \\ 1 & -1 \end{pmatrix} \cdot \frac{1}{\sqrt{2}}\begin{pmatrix} 1 & 1 \\ 1 & -1 \end{pmatrix}^\dagger = \frac{1}{2}\begin{pmatrix} 1 & 1 \\ 1 & -1 \end{pmatrix}\begin{pmatrix} 1 & 1 \\ 1 & -1 \end{pmatrix} = \frac{1}{2}\begin{pmatrix} 2 & 0 \\ 0 & 2 \end{pmatrix} = \begin{pmatrix} 1 & 0 \\ 0 & 1 \end{pmatrix} = I$

		Para $S$:
		$SS^\dagger = \begin{pmatrix} 1 & 0 \\ 0 & i \end{pmatrix}\begin{pmatrix} 1 & 0 \\ 0 & -i \end{pmatrix} = \begin{pmatrix} 1 & 0 \\ 0 & 1 \end{pmatrix} = I$

		Para $T$:
		$TT^\dagger = \begin{pmatrix} 1 & 0 \\ 0 & e^{i\pi/4} \end{pmatrix}\begin{pmatrix} 1 & 0 \\ 0 & e^{-i\pi/4} \end{pmatrix} = \begin{pmatrix} 1 & 0 \\ 0 & 1 \end{pmatrix} = I$

		Todas las compuertas son unitarias, como era de esperar para compuertas cuánticas válidas.

		c) Calculamos el resultado de aplicar $HSH$ al estado $|0\rangle$:

		Paso 1: $H|0\rangle = \frac{1}{\sqrt{2}}(|0\rangle + |1\rangle)$

		Paso 2: $S(H|0\rangle) = S\frac{1}{\sqrt{2}}(|0\rangle + |1\rangle) = \frac{1}{\sqrt{2}}(|0\rangle + i|1\rangle)$

		Paso 3: $H(S(H|0\rangle)) = H\frac{1}{\sqrt{2}}(|0\rangle + i|1\rangle) = \frac{1}{\sqrt{2}}[H|0\rangle + iH|1\rangle]$
		$= \frac{1}{\sqrt{2}}[\frac{1}{\sqrt{2}}(|0\rangle + |1\rangle) + i\frac{1}{\sqrt{2}}(|0\rangle - |1\rangle)]$
		$= \frac{1}{2}[(1+i)|0\rangle + (1-i)|1\rangle]$
		$= \frac{1+i}{2}|0\rangle + \frac{1-i}{2}|1\rangle$

		Por tanto, $HSH|0\rangle = \frac{1+i}{2}|0\rangle + \frac{1-i}{2}|1\rangle$

		d) Representación en la esfera de Bloch del efecto de $H$ sobre $|0\rangle$ y $|1\rangle$:

		$H|0\rangle = \frac{1}{\sqrt{2}}(|0\rangle + |1\rangle) = |+\rangle$
		- El estado $|0\rangle$ está en el polo norte de la esfera de Bloch (coordenadas $(0,0,1)$)
		- El estado $|+\rangle$ está en el eje x positivo de la esfera de Bloch (coordenadas $(1,0,0)$)
		- Por tanto, $H$ rota $|0\rangle$ desde el polo norte hacia el eje x positivo

		$H|1\rangle = \frac{1}{\sqrt{2}}(|0\rangle - |1\rangle) = |-\rangle$
		- El estado $|1\rangle$ está en el polo sur de la esfera de Bloch (coordenadas $(0,0,-1)$)
		- El estado $|-\rangle$ está en el eje x negativo de la esfera de Bloch (coordenadas $(-1,0,0)$)
		- Por tanto, $H$ rota $|1\rangle$ desde el polo sur hacia el eje x negativo

		En general, la compuerta Hadamard realiza una rotación de 90° alrededor del eje y, seguida de una reflexión en el plano xz. Geométricamente, $H$ mapea el eje z al eje x y viceversa en la esfera de Bloch.
	\end{solution}
\end{questions}