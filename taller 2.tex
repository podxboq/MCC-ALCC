\documentclass[]{unirTema}

\newcommand{\tq}{\mid}
\newcommand{\K}{\mathrm{K}}
\newcommand{\V}{\mathrm{V}}
\newcommand{\N}{\mathbb{N}}
\newcommand{\Z}{\mathbb{Z}}
\newcommand{\F}{\mathbb{F}}
\newcommand{\Q}{\mathbb{Q}}
\newcommand{\R}{\mathbb{R}}
\newcommand{\C}{\mathbb{C}}
\renewcommand{\H}{\mathcal{H}}
\newcommand{\Cinf}{\mathcal{C}^\infty}
\newcommand{\Cu}{\mathcal{C}^1}
\newcommand{\Rp}{\mathfrak{Re}}
\newcommand{\Ip}{\mathfrak{Im}}
\renewcommand{\d}{\mathrm{d}}
\newcommand{\dm}{\mathrm{d}\mu}
\newcommand{\conj}[1]{\overline{#1}}
\newcommand{\Tras}[1]{{#1}^{\text{T}}}

\DeclareMathOperator{\Log}{Log}
\DeclareMathOperator{\Arg}{Arg}
\DeclareMathOperator{\Dom}{Dom}
\DeclareMathOperator{\Ima}{Im}
\DeclareMathOperator{\sgn}{sgn}
\DeclareMathOperator{\mcd}{MCD}
\DeclareMathOperator{\mcm}{mcm}
\DeclareMathOperator{\Resi}{Res}
\DeclareMathOperator{\Ker}{Ker}
\DeclareMathOperator{\End}{End}
\DeclareMathOperator{\Mat}{Mat}

\newcommand{\parder}[2]{\frac{\partial #1}{\partial #2}}
\newcommand{\dparder}[2]{\dfrac{\partial #1}{\partial #2}}
\newcommand{\tparder}[2]{\partial #1/\partial #2}
\newcommand{\parderr}[3]{\frac{\partial^2 #1}{\partial #2\partial #3}}
\newcommand{\dparderr}[3]{\dfrac{\partial^2 #1}{\partial #2\partial #3}}
\newcommand{\tparderr}[3]{\partial^2 #1/\partial #2\partial #3}
\newcommand{\intx}[1]{\int #1\,dx}
\newcommand{\intt}[1]{\int #1\,dt}
\newcommand{\intdx}[3]{\int_{#1}^{#2} #3\,dx}
\newcommand{\intdt}[3]{\int_{#1}^{#2} #3\,dt}
\newcommand{\intdz}[2]{\int_{#1} #2\,dz}
\newcommand{\set}[1]{\left\{#1\right\}}
\newcommand{\so}{\Rightarrow}
\newcommand{\sii}{\Leftrightarrow}
\newcommand{\by}[1]{\overset{\fbox{\tiny #1}}{=}}
\newcommand{\byref}[1]{\overset{\fbox{\tiny\ref{#1}}}{=}}
\newcommand{\cardinal}[1]{\left|#1\right|}
\newcommand{\maps}[3]{#1 \colon #2\longrightarrow #3}
\newcommand{\equationmaps}[5]{\begin{aligned}[t]#1 \colon #2 &\longrightarrow #3 \\	#4 &\longmapsto #5\end{aligned}}
\newcommand{\coma}{,\thinspace}
\newcommand{\pari}[2]{(#1,\thinspace #2)}
\newcommand{\where}{\mathrel{}\middle|\mathrel{}}
\newcommand{\no}[1]{{\neg}{#1}}
\newcommand{\dcomilla}[1]{{\guillemotleft}#1{\guillemotright}}
\newcommand{\separa}{\vspace*{.75\baselineskip}}
\newcommand{\semisepara}{\vspace*{.25\baselineskip}}
\newcommand{\restrict}[1]{\raisebox{-.5ex}{$|$}_{#1}}

% ========================================
% COMANDOS PERSONALIZADOS PARA COMPUTACIÓN CUÁNTICA
% ========================================

% Estados comunes
\newcommand{\zero}{\ket{0}}
\newcommand{\one}{\ket{1}}
\newcommand{\plus}{\ket{+}}
\newcommand{\minus}{\ket{-}}

% Operadores especiales
\newcommand{\tensor}{\otimes}         % Producto tensorial
\newcommand{\comp}{\circ}             % Composición

% Estados de Bell
\newcommand{\bellphi}{\ket{\Phi^+}}
\newcommand{\bellpsi}{\ket{\Psi^+}}
\newcommand{\bellphiminus}{\ket{\Phi^-}}
\newcommand{\bellpsiminus}{\ket{\Psi^-}}

\newcommand{\floor}[1]{\left\lfloor #1 \right\rfloor}
\newcommand{\ceil}[1]{\left\lceil #1 \right\rceil}


\printanswers

\author{Francisco Costa Cano}
\titulacion{Máster en computación cuántica}
\asignatura{Álgebra lineal en computación cuántica}
\bloque{1}{Fundamentos matemáticos}
\tema{5}{Espacios de Hilbert}

\begin{document}

\caratula

\section{Taller previo a la notación de Dirac}

\begin{questions}

  \question Demuestra que $\mathcal{H} = \mathbb{C}^4$ con las operaciones usuales es un espacio vectorial sobre $\mathbb{C}$.

  \begin{solution}

    Debemos verificar los axiomas de espacio vectorial. Para vectores $v, w, u \in \mathbb{C}^4$ y escalares $\alpha, \beta \in \mathbb{C}$:

    \begin{enumerate}
      \item \textbf{Cerradura bajo suma:} Si $v = (v_1, v_2, v_3, v_4)$ y $w = (w_1, w_2, w_3, w_4)$, entonces:
            \[
              v + w = (v_1+w_1, v_2+w_2, v_3+w_3, v_4+w_4) \in \mathbb{C}^4
            \]

      \item \textbf{Cerradura bajo producto por escalar:}
            \[
              \alpha v = (\alpha v_1, \alpha v_2, \alpha v_3, \alpha v_4) \in \mathbb{C}^4
            \]

      \item \textbf{Elemento neutro:} $0 = (0,0,0,0)$ satisface $v + 0 = v$.

      \item \textbf{Elemento opuesto:} Para cada $v$ existe $-v = (-v_1, -v_2, -v_3, -v_4)$ tal que $v + (-v) = 0$.

      \item \textbf{Asociatividad de la suma:} $(u + v) + w = u + (v + w)$.

      \item \textbf{Conmutatividad de la suma:} $v + w = w + v$.

      \item \textbf{Distributividad:} $\alpha(v + w) = \alpha v + \alpha w$ y $(\alpha + \beta)v = \alpha v + \beta w$.

      \item \textbf{Asociatividad del producto:} $(\alpha\beta)v = \alpha(\beta w)$.

      \item \textbf{Elemento neutro del producto:} $1\cdot v = v$.
    \end{enumerate}

  \end{solution}

  \question \textbf{Base de Bell}

  Consideramos la base de Bell $\mathcal{B}_B=\{B_1,B_2,B_3,B_4\}$, fundamental en computación cuántica:
  \begin{align*}
    B_1 & = \frac{1}{\sqrt{2}}(1, 0, 0, 1)\,,  \\
    B_2 & = \frac{1}{\sqrt{2}}(1, 0, 0, -1)\,, \\
    B_3 & = \frac{1}{\sqrt{2}}(0, 1, 1, 0)\,,  \\
    B_4 & = \frac{1}{\sqrt{2}}(0, 1, -1, 0)\,.
  \end{align*}
  Verifica que $\mathcal{B}_B$ forma una base ortonormal.

  \begin{solution}
    Recordamos que la norma de un vector se define como
    \[
      \|(v_1, v_2, v_3, v_4)\| = \sqrt{|v_1|^2 + |v_2|^2 + |v_3|^2 + |v_4|^2}\,.
    \]
    \textbf{Normalización:}
    \begin{align*}
      \|B_1\| & = \frac{1}{\sqrt{2}}\sqrt{1^2 + 0^2 + 0^2 + 1^2} = \frac{1}{\sqrt{2}}\sqrt{2} = 1 \,.    \\
      \|B_2\| & = \frac{1}{\sqrt{2}}\sqrt{1^2 + 0^2 + 0^2 + (-1)^2} = \frac{1}{\sqrt{2}}\sqrt{2} = 1 \,. \\
      \|B_3\| & = \frac{1}{\sqrt{2}}\sqrt{0^2 + 1^2 + 1^2 + 0^2} = \frac{1}{\sqrt{2}}\sqrt{2} = 1 \,.    \\
      \|B_4\| & = \frac{1}{\sqrt{2}}\sqrt{0^2 + 1^2 + (-1)^2 + 0^2} = \frac{1}{\sqrt{2}}\sqrt{2} = 1 \,. \\
    \end{align*}

    \textbf{Ortogonalidad:}
    Recordamos que dos vectores $v,w$ son ortogonales si
    \[
      \langle v, w \rangle = 0\,.
    \]
    Por la propiedad del producto interno $\langle v, w \rangle = \overline{\langle w, v \rangle}$, para la ortogonalidad solo será necesario por parejas sin importar el orden.
    \begin{align*}
      \langle B_1, B_2 \rangle & = \frac{1}{\sqrt{2}}\frac{1}{\sqrt{2}}(1\bar{1}+ 0\bar{0} + 0\bar{0}+ 1(\overline{-1})= \frac{1}{2}(1 - 1) = 0        \\
      \langle B_1, B_3 \rangle & = \frac{1}{\sqrt{2}}\frac{1}{\sqrt{2}}(1\bar{0}+ 0\bar{1} + 0\bar{1}+ 1\overline{0}= \frac{1}{2}(0 + 0) = 0           \\
      \langle B_1, B_4 \rangle & = \frac{1}{\sqrt{2}}\frac{1}{\sqrt{2}}(1\bar{0}+ 0\bar{1} + 0(\overline{-1})+ 1\overline{0}= \frac{1}{2}(0 - 0) = 0   \\
      \langle B_2, B_3 \rangle & = \frac{1}{\sqrt{2}}\frac{1}{\sqrt{2}}(1\bar{0}+ 0\bar{1} + 0\bar{1}- 1\overline{0}= \frac{1}{2}(0 - 0) = 0           \\
      \langle B_2, B_4 \rangle & = \frac{1}{\sqrt{2}}\frac{1}{\sqrt{2}}(1\bar{0}+ 0(\overline{-1}) + 0\bar{1}- 1\overline{0}= \frac{1}{2}(0 - 0) = 0   \\
      \langle B_3, B_4 \rangle & = \frac{1}{\sqrt{2}}\frac{1}{\sqrt{2}}(0\bar{0}+ 1\bar{1} + 1(\overline{-1})+ 0(\overline{0})= \frac{1}{2}(1 - 1) = 0 \\
    \end{align*}

    Todos los productos internos entre estados distintos son cero, verificando la ortogonalidad.

    \textbf{Base:}
    Como sabemos que $\C^4$ es un espacio vectorial de dimensión 4, y tenemos 4 vectores, solo debemos o ver que son linealmente independientes o un sistema generador.

    Veamos que son linealmente independientes. Por definición $\{v_1,v_2,v_3,v_4\}$ son linealmente independientes si y solo si $\forall \alpha_1,\alpha_2,\alpha_3,\alpha_4 \in \C$ se cumple
    \[
      \alpha_1 v_1 + \alpha_2 v_2 + \alpha_3 v_3 + \alpha_4 v_4 = 0 \Rightarrow \alpha_1 = \alpha_2 = \alpha_3 = \alpha_4 = 0\,.
    \]
    En nuestro caso
    \[
      \alpha_1 B_1 + \alpha_2 B_2 + \alpha_3 B_3 + \alpha_4 B_4 = 0 \Rightarrow \begin{cases}
        \alpha_1 + \alpha_2 = 0 \\
        \alpha_3 + \alpha_4 = 0 \\
        \alpha_3 - \alpha_4 = 0 \\
        \alpha_1 - \alpha_2 = 0
      \end{cases} \Rightarrow \alpha_i = 0\,.
    \]
  \end{solution}

  \question Encuentra la matriz $P$ que permite cambiar de la base estándar a la base de Bell.
  \begin{solution}

    Llamando $\mathcal{B}=\{e_1, e_2, e_3, e_4\}$ la base estándar, la matriz de cambio de base $\mathcal{B}$ a $\mathcal{B_B}$ tiene como filas las coordenadas de los vectores de la nueva base expresados en la base estándar. Por lo tanto

    \[
      M_{\mathcal{B}}^{\mathcal{B}_B} = \frac{1}{\sqrt{2}}\begin{pmatrix}
        1 & 0 & 0  & 1  \\
        1 & 0 & 0  & -1 \\
        0 & 1 & 1  & 0  \\
        0 & 1 & -1 & 0
      \end{pmatrix}\,.
    \]

    Podemos observar como
    \[
      e_i M_{\mathcal{B}}^{\mathcal{B}_B} = B_i\Rightarrow e_i = B_i (M_{\mathcal{B}}^{\mathcal{B}_B})^{-1}= B_i M_{\mathcal{B}_B}^{\mathcal{B}}\,,
    \]
    y por tanto si $v\in\C^4$ tiene coordenadas $(v_1,v_2,v_3,v_4)$ en la base estándar, y $(b_1,b_2,b_3,b_4)$ en la base de Bell, entonces
    \begin{align*}
      \sum_{i=1}^4 b_i B_i & = v = \sum_{i=1}^4 v_i e_i = \sum_{i=1}^4 v_i B_i (M_{\mathcal{B}}^{\mathcal{B}_B})^{-1} = \sum_{i=1}^4 v_i M_{\mathcal{B}_B}^{\mathcal{B}}B_i \\
                           & \Rightarrow b_i = v_i M_{\mathcal{B}_B}^{\mathcal{B}} \,.
    \end{align*}

    Por lo tanto la matriz $P$ que permite cambiar de la base estándar a la base de Bell es
    \[
      P = M_{\mathcal{B}_B}^{\mathcal{B}}\,.
    \]
  \end{solution}

  \question Expresa el vector
  \[
    u = \frac{1}{2}(1, 1, 1, 1)\,,
  \]
  en la base de Bell.

  \begin{solution}

    Como las coordenadas de $u$ están dadas en la base estándar, su expresión desarrollada en esta base es
    \[
      u = \frac{1}{2}\sum_{i=1}^4 e_i\,.
    \]

    Para obtener las coordenadas en la base de Bell, usamos $M_{\mathcal{B}_B}^{\mathcal{B}} = (M_{\mathcal{B}}^{\mathcal{B}_B})^{-1}$. Como $M_{\mathcal{B}}^{\mathcal{B}_B}$ es unitaria (producto de operaciones unitarias), $(M_{\mathcal{B}}^{\mathcal{B}_B})^{-1} = (M_{\mathcal{B}}^{\mathcal{B}_B})^\dagger$:

    \begin{align*}
      u_{\mathcal{B}_B} = u_{\mathcal{B}} M_{\mathcal{B}_B}^{\mathcal{B}} & = \frac{1}{2}(1, 1, 1, 1) \frac{1}{\sqrt{2}}\begin{pmatrix}
                                                                                                                          1 & 1  & 0 & 0  \\
                                                                                                                          0 & 0  & 1 & 1  \\
                                                                                                                          0 & 0  & 1 & -1 \\
                                                                                                                          1 & -1 & 0 & 0
                                                                                                                        \end{pmatrix}                                        \\
                                                                          & = \frac{1}{2\sqrt{2}}(2, 0, 2, 0)_{\mathcal{B}_B} = \frac{1}{\sqrt{2}}(1, 0, 1, 0)_{\mathcal{B}_B}
    \end{align*}

    Por lo tanto:
    \[
      u = \frac{1}{\sqrt{2}}B_1 + \frac{1}{\sqrt{2}}B_3\,.
    \]
  \end{solution}

  \question
  Sea el subespacio $\mathcal{S} \subset \mathbb{C}^4$ definido por
  \[
    \mathcal{S} = \text{span}\{B_1, B_3\}\,.
  \]

  Encuentra una base ortonormal para el subespacio ortogonal $\mathcal{S}^\perp$.

  \begin{solution}

    El subespacio ortogonal $\mathcal{S}^\perp$ contiene todos los vectores ortogonales a $\mathcal{S}$. De la base de Bell, sabemos que $B_2$ y $B_4$ son ortogonales a $B_1$ y $B_3$, que son linealmente independientes y que
    \[
      \mathbb{C}^4 = \mathcal{S} \oplus \mathcal{S}^\perp \Rightarrow \dim(\C^4) = \dim(\mathcal{S}) + \dim(\mathcal{S}^\perp)\Rightarrow \dim(\mathcal{S}^\perp) = 2\,.
    \]

    Por lo tanto, una base ortonormal para $\mathcal{S}^\perp$ es
    \[
      \mathcal{S}^\perp = \text{span}\{B_2, B_4\}\,.
    \]
  \end{solution}

  \question Construye el proyector ortogonal $P_{\mathcal{S}}$ sobre el subespacio simétrico.

  \begin{solution}

    El proyector sobre un subespacio generado por la base ortonormal $\{B_1, B_3\}$ se construye usando el producto externo como
    \[
      P_{\mathcal{S}} = B_1\wedge B_1 + B_3\wedge B_3\,.
    \]

    Calculamos explícitamente
    \begin{align*}
      B_1\wedge B_1 & = \frac{1}{\sqrt{2}}(1, 0, 0, 1)\otimes \frac{1}{\sqrt{2}}\begin{pmatrix} 1 \\ 0 \\ 0 \\ 1 \end{pmatrix} = \frac{1}{2}\begin{pmatrix}
                                                                                                                                              1 & 0 & 0 & 1 \\
                                                                                                                                              0 & 0 & 0 & 0 \\
                                                                                                                                              0 & 0 & 0 & 0 \\
                                                                                                                                              1 & 0 & 0 & 1
                                                                                                                                            \end{pmatrix}\,. \\
      B_3\wedge B_3 & = \frac{1}{\sqrt{2}}(0, 1, 1, 0)\otimes \frac{1}{\sqrt{2}}\begin{pmatrix} 0 \\ 1 \\ 1 \\ 0 \end{pmatrix} = \frac{1}{2}\begin{pmatrix}
                                                                                                                                              0 & 0 & 0 & 0 \\
                                                                                                                                              0 & 1 & 1 & 0 \\
                                                                                                                                              0 & 1 & 1 & 0 \\
                                                                                                                                              0 & 0 & 0 & 0
                                                                                                                                            \end{pmatrix}\,.
    \end{align*}


    Por lo tanto:
    \[
      P_{\mathcal{S}} = \frac{1}{2}\begin{pmatrix}
        1 & 0 & 0 & 1 \\
        0 & 1 & 1 & 0 \\
        0 & 1 & 1 & 0 \\
        1 & 0 & 0 & 1
      \end{pmatrix}
    \]

    Verificamos que es un proyector: $P_{\mathcal{S}}^2 = P_{\mathcal{S}}$ y $P_{\mathcal{S}}^\dagger = P_{\mathcal{S}}$.
  \end{solution}

  \question Calcula la proyección del vector $(1, 0 , 0, 0)$ sobre el subespacio $\mathcal{S}$.
  \begin{solution}

    La proyección se calcula aplicando el proyector, o lo que es lo mismo, multiplicando por la matriz del proyector:
    \[
      P_{\mathcal{S}}((1, 0, 0,0)) = (1, 0, 0,0)\frac{1}{2}\begin{pmatrix}
        1 & 0 & 0 & 1 \\
        0 & 1 & 1 & 0 \\
        0 & 1 & 1 & 0 \\
        1 & 0 & 0 & 1
      \end{pmatrix} = \frac{1}{2}(1, 0, 0, 1)\,.
    \]
  \end{solution}

  \question Considera los siguientes vectores linealmente independientes en $\mathbb{C}^4$:
  \begin{align*}
    v_1 & = (1, 0, 0, 1) \\
    v_2 & = (1, 1, 0, 0) \\
    v_3 & = (1, 1, 1, 0)
  \end{align*}
  Aplica el proceso de Gram-Schmidt para obtener un conjunto ortonormal.

  \begin{solution}

    \textbf{Paso 1:} Normalizamos
    \[
      u_1 = \frac{v_1}{\|v_1\|} = \frac{1}{\sqrt{2}}(1, 0, 0, 1)\,.
    \]

    \textbf{Paso 2:} Ortogonalizamos $v_2$ respecto a $u_1$
    \[
      w_2 = v_2 - \langle u_1, v_2\rangle u_1 = (1, 1, 0, 0) - \frac{1}{\sqrt{2}} \cdot \frac{1}{\sqrt{2}}(1, 0, 0, 1) = \frac{1}{2}(1, 2, 0, -1)\,.
    \]

    Normalizamos
    \[
      u_2 = \frac{1}{\sqrt{3/2}}\frac{1}{2}(1, 2, 0, -1) = \frac{1}{\sqrt{6}}(1, 2, 0, -1)\,.
    \]

    \textbf{Paso 3:} Ortogonalizamos $v_3$ respecto a $u_1$ y $u_2$
    \begin{align*}
      w_3 & = v_3 - \langle u_1, v_3\rangle u_1 - \langle u_2, v_3\rangle u_2                                                                   \\
          & = (1, 1, 1, 0) - \frac{1}{\sqrt{2}} \cdot \frac{1}{\sqrt{2}}(1, 0, 0, 1) - \sqrt{\frac{3}{2}} \cdot \frac{1}{\sqrt{6}}(1, 2, 0, -1) \\
          & = (1, 1, 1, 0) - \frac{1}{2}(1, 0, 0, 1) - \frac{1}{2}(1, 2, 0, -1)                                                                 \\
          & = (0, 0, 1, 0)\,.
    \end{align*}

    Como la proyección ya está normalizada
    \[
      u_3 = (0, 0, 1, 0)\,.
    \]

    \textbf{Conjunto ortonormal resultante}
    \[
      \left\{\frac{1}{\sqrt{2}}(1, 0, 0, 1), \frac{1}{\sqrt{6}}(1, 2, 0, -1), (0, 0, 1, 0)\right\}\,.
    \]
  \end{solution}

  \question Considera el operador de intercambio (SWAP) que intercambia dos coordenadas de posición
  \begin{align*}
    \text{SWAP} : \C^4   & \longrightarrow \C^4            \\
    (v_1, v_2, v_3, v_4) & \mapsto (v_1, v_3, v_2, v_4)\,.
  \end{align*}

  Encuentra la matriz del operador SWAP en la base canónica.
  \begin{solution}

    Para encontrar la matriz, aplicamos el operador a cada vector de la base canónica
    \begin{align*}
      \text{SWAP}(e_1) & = e_1 \\
      \text{SWAP}(e_2) & = e_3 \\
      \text{SWAP}(e_3) & = e_2 \\
      \text{SWAP}(e_4) & = e_4
    \end{align*}

    Las filas de la matriz son las imágenes de los vectores de la base canónica:
    \[
      [\text{SWAP}] = \begin{pmatrix}
        1 & 0 & 0 & 0 \\
        0 & 0 & 1 & 0 \\
        0 & 1 & 0 & 0 \\
        0 & 0 & 0 & 1
      \end{pmatrix}\,.
    \]
  \end{solution}

  \question Encuentra la matriz del operador SWAP en la base de Bell.

  \begin{solution}

    Aplicamos el operador a cada vector de la base de Bell:
    \begin{align*}
      \text{SWAP}(B_1) & = B_1  \\
      \text{SWAP}(B_2) & = B_2  \\
      \text{SWAP}(B_3) & = B_3  \\
      \text{SWAP}(B_4) & = -B_4
    \end{align*}

    Por lo tanto, en la base de Bell
    \[
      [\text{SWAP}]_{\mathcal{B}_B} = \begin{pmatrix}
        1 & 0 & 0 & 0  \\
        0 & 1 & 0 & 0  \\
        0 & 0 & 1 & 0  \\
        0 & 0 & 0 & -1
      \end{pmatrix}\,.
    \]

    Observamos que en la base de Bell, el operador SWAP es diagonal.
  \end{solution}

  \question Verifica que las dos representaciones están relacionadas por la matric cambio de base.

  \begin{solution}
    Tenemos que verificar que
    \[
      [\text{SWAP}]_{\mathcal{B}_B} = M_{\mathcal{B}}^{\mathcal{B}_B} [\text{SWAP}]_{\mathcal{B}_c} M_{\mathcal{B}_B}^{\mathcal{B}}\,.
    \]

    Calculamos $M_{\mathcal{B}}^{\mathcal{B}_B} [\text{SWAP}] M_{\mathcal{B}_B}^{\mathcal{B}}$
    \begin{align*}
       & \frac{1}{\sqrt{2}}\begin{pmatrix}
                             1 & 0 & 0  & 1  \\
                             1 & 0 & 0  & -1 \\
                             0 & 1 & 1  & 0  \\
                             0 & 1 & -1 & 0
                           \end{pmatrix} \begin{pmatrix}
                                           1 & 0 & 0 & 0 \\
                                           0 & 0 & 1 & 0 \\
                                           0 & 1 & 0 & 0 \\
                                           0 & 0 & 0 & 1
                                         \end{pmatrix} \frac{1}{\sqrt{2}}\begin{pmatrix}
                                                                           1 & 1  & 0 & 0  \\
                                                                           0 & 0  & 1 & 1  \\
                                                                           0 & 0  & 1 & -1 \\
                                                                           1 & -1 & 0 & 0
                                                                         \end{pmatrix} \\
       & = \frac{1}{2}\begin{pmatrix}
                        1 & 0  & 0 & 1  \\
                        1 & 0  & 0 & -1 \\
                        0 & 1  & 1 & 0  \\
                        0 & -1 & 1 & 0
                      \end{pmatrix}\begin{pmatrix}
                                     1 & 1  & 0 & 0  \\
                                     0 & 0  & 1 & 1  \\
                                     0 & 0  & 1 & -1 \\
                                     1 & -1 & 0 & 0
                                   \end{pmatrix}
      = \frac{1}{2}\begin{pmatrix}
                     2 & 0 & 0 & 0  \\
                     0 & 2 & 0 & 0  \\
                     0 & 0 & 2 & 0  \\
                     0 & 0 & 0 & -2
                   \end{pmatrix}\,,
    \end{align*}
    que coincide con $[\text{SWAP}]_{\mathcal{B}_B}$.
  \end{solution}

  \question Calcula el espectro del operador SWAP.

  \begin{solution}
    Para buscar los valores propios y vectores propios del operador SWAP, podemos usar cualquier representación matricial del operador.

    De la representación en la base de Bell, vemos directamente que:
    \begin{enumerate}
      \item Valor propio $\lambda_1 = 1$, con multiplicidad 3.

            Vectores propios $B_1, B_2, B_3$.

            El subespacio propio es $S_1 = \text{span}\{B_1, B_2, B_3\}$.

      \item Valor propio $\lambda_2 = -1$, con multiplicidad 1.

            Vector propio $B_4$.

            El subespacio propio es $S_{-1} = \text{span}\{B_4\}$.
    \end{enumerate}

    Como esperamos y ya hemos verificado se cumple que
    \[
      S_1 \perp S_{-1}\,,
    \]
    y
    \[
      \dim(S_1) + \dim(S_{-1}) = 4\,.
    \]
  \end{solution}

  \question Escribe el operador SWAP como suma de proyectores sobre sus subespacios propios.

  \begin{solution}

    La descomposición espectral es:
    \[
      \text{SWAP} = 1 \cdot P_{E_1} + (-1) \cdot P_{E_{-1}}
    \]

    donde
    \begin{align*}
      P_{E_1}    & = B_1\wedge B_1 + B_2\wedge B_2 + B_3\wedge B_3\,, \\
      P_{E_{-1}} & = B_4\wedge B_4\,.
    \end{align*}

    Calculamos explícitamente
    \begin{align*}
      P_{E_{-1}} & = \frac{1}{2}\begin{pmatrix} 0 & 1 & -1 & 0 \end{pmatrix} \otimes \begin{pmatrix} 0 \\ 1 \\ -1 \\ 0 \end{pmatrix} = \frac{1}{2}\begin{pmatrix}
                                                                                                                                                    0 & 0  & 0  & 0 \\
                                                                                                                                                    0 & 1  & -1 & 0 \\
                                                                                                                                                    0 & -1 & 1  & 0 \\
                                                                                                                                                    0 & 0  & 0  & 0
                                                                                                                                                  \end{pmatrix}\,, \\
      P_{E_1}    & = \mathbb{I} - P_{E_{-1}} = \begin{pmatrix}
                                                 1 & 0   & 0   & 0 \\
                                                 0 & 1/2 & 1/2 & 0 \\
                                                 0 & 1/2 & 1/2 & 0 \\
                                                 0 & 0   & 0   & 1
                                               \end{pmatrix}\,.
    \end{align*}
    Por lo tanto:
    \[
      \text{SWAP} = P_{E_1} - P_{E_{-1}} = \begin{pmatrix}
        1 & 0   & 0   & 0 \\
        0 & 1/2 & 1/2 & 0 \\
        0 & 1/2 & 1/2 & 0 \\
        0 & 0   & 0   & 1
      \end{pmatrix} - \frac{1}{2}\begin{pmatrix}
        0 & 0  & 0  & 0 \\
        0 & 1  & -1 & 0 \\
        0 & -1 & 1  & 0 \\
        0 & 0  & 0  & 0
      \end{pmatrix}
    \]

    \[
      = \begin{pmatrix}
        1 & 0 & 0 & 0 \\
        0 & 0 & 1 & 0 \\
        0 & 1 & 0 & 0 \\
        0 & 0 & 0 & 1
      \end{pmatrix}\,.
    \]
  \end{solution}

  \question Verificar que el operador SWAP es hermitiano.

  \begin{solution}

    Un operador es hermitiano si $A^\dagger = A$, tanto como operador como para su representación como matriz. Es fácil observar que
    \[
      [\text{SWAP}]^\dagger = [\text{SWAP}]\,.
    \]

    Por lo tanto, SWAP es hermitiano.

    \textbf{Consecuencias:}
    \begin{itemize}
      \item Los valores propios son reales: $\lambda \in \{1, -1\} \subset \mathbb{R}$.
      \item Los vectores propios correspondientes a valores propios distintos son ortogonales.
      \item El operador es diagonalizable en una base ortonormal.
    \end{itemize}
  \end{solution}

  \question Verificar que el operador SWAP es unitario.

  \begin{solution}

    Un operador es unitario si $U^\dagger U = U U^\dagger = \mathbb{I}$. Como SWAP es hermitiano, debemos verificar que $\text{SWAP}^2 = \mathbb{I}$

    \[
      [\text{SWAP}]_{\mathcal{B}_c}^2 = \begin{pmatrix}
        1 & 0 & 0 & 0 \\
        0 & 0 & 1 & 0 \\
        0 & 1 & 0 & 0 \\
        0 & 0 & 0 & 1
      \end{pmatrix}^2 = \begin{pmatrix}
        1 & 0 & 0 & 0 \\
        0 & 1 & 0 & 0 \\
        0 & 0 & 1 & 0 \\
        0 & 0 & 0 & 1
      \end{pmatrix} = \mathbb{I}
    \]

    Por lo tanto, SWAP es unitario.

    \textbf{Consecuencias:}
    \begin{itemize}
      \item SWAP preserva el producto interno.
      \item Los valores propios tienen módulo 1: $|\lambda| = 1$.
    \end{itemize}
  \end{solution}

  \question  Verifica que $\mathbb{C}^4 \cong \mathbb{C}^2 \otimes \mathbb{C}^2$ construyendo explícitamente el isomorfismo.

  \begin{solution}

    La manera estándar de construir isomorfismo entre espacio vectoriales es indicar una base de cada espacio y mapearla entre sí. Si llamamos $f_1=(1, 0)$ y $f_2=(0, 1)$ a las bases de $\mathbb{C}^2$, entonces el isomorfismo es
    \[
      f_1 \otimes f_1 \mapsto e_1
    \]
    \[
      f_1 \otimes f_2 \mapsto e_2
    \]
    \[
      f_2 \otimes f_1 \mapsto e_3
    \]
    \[
      f_2 \otimes f_2 \mapsto e_4\,.
    \]

    Explícitamente para $v=\alpha_{11}f_1 \otimes f_1 + \alpha_{12}f_1 \otimes f_2 + \alpha_{21}f_2 \otimes f_1 + \alpha_{22}f_2 \otimes f_2$
    \[
      v \mapsto \alpha_{11}e_1 + \alpha_{12}e_2 + \alpha_{21}e_3 + \alpha_{22}e_4\,.
    \]

    Este mapeo es lineal, biyectivo y preserva el producto interno, por lo que es un isomorfismo de espacios de Hilbert.
  \end{solution}

  \question Determina cuáles de los siguientes vectores son separables.

  \begin{enumerate}
    \item $u_1 = \frac{1}{2}(f_1\otimes f_1 + f_1\otimes f_2 + f_2\otimes f_1 + f_2\otimes f_2)$
    \item $u_2 = \frac{1}{\sqrt{2}}(f_1\otimes f_1 + f_2\otimes f_2)$
  \end{enumerate}

  \begin{solution}

    \begin{enumerate}
      \item
            \begin{align*}
              u_1 & = \frac{1}{2}(f_1\otimes f_1 + f_1\otimes f_2 + f_2\otimes f_1 + f_2\otimes f_2) \\
                  & = \frac{1}{2}(f_1\otimes (f_1 + f_2) + f_2\otimes (f_1 + f_2))                   \\
                  & = \frac{1}{2}(f_1+f_2) \otimes (f_1 + f_2)\,.
            \end{align*}
            Este estado es \textbf{separable}.

      \item
            Para que $u_2$ sea separable, debería existir
            \begin{align*}
              u_2 & = (\alpha_1 f_1+\beta_1 f_2)\otimes (\alpha_2 f_1+\beta_2 f_2)                                                                         \\
                  & = \alpha_1\alpha_2 f_1\otimes f_1 + \alpha_1\beta_2 f_1\otimes f_2 + \beta_1\alpha_2 f_2\otimes f_1 + \beta_1\beta_2 f_2\otimes f_2\,.
            \end{align*}

            Comparando con $u_2 = \frac{1}{\sqrt{2}}(f_1\otimes f_1 + f_2\otimes f_2)$, nos da el sistema de ecuaciones
            \begin{align*}
              \alpha_1 \alpha_2 & = \frac{1}{2}    \\
              \alpha_1 \beta_2  & = 0              \\
              \beta_1 \alpha_2  & = 0              \\
              \beta_1 \beta_2   & = \frac{1}{2}\,.
            \end{align*}

            El sistema anterior no tiene solución. Este estado no es separable.
    \end{enumerate}
  \end{solution}

  \question Para el vector $v = \frac{1}{\sqrt{2}}(B_1 + B_2)$, encuentra el funcional lineal $v^* \in (\mathbb{C}^4)^*$ mediante el isomorfismo de Riesz.

  \begin{solution}
    \textbf{Solución:}

    El isomorfismo de Riesz asocia a cada vector $v$ el funcional lineal
    \[
      v^*: \mathbb{C}^4 \to \mathbb{C}, \quad w \mapsto \langle v, w \rangle\,.
    \]

    En coordenadas, sabemos que $v^*$ es la transpuesta conjugada de $v$
    \[
      [v*] = \frac{1}{\sqrt{2}}\begin{pmatrix} 1 \\ 0 \\ 0 \\ 1 \end{pmatrix}\,.
    \]

    Entonces, para cualquier $w = (w_1, w_2, w_3, w_4)$ se tiene
    \[
      v^*(w) = (w_1, w_2, w_3, w_4)\frac{1}{\sqrt{2}}\begin{pmatrix} 1 \\ 0 \\ 0 \\ 1 \end{pmatrix} = \frac{1}{\sqrt{2}}(w_1 + w_4)\,.
    \]
  \end{solution}

\end{questions}
\end{document}