\documentclass[]{unirTema}

\newcommand{\tq}{\mid}
\newcommand{\K}{\mathrm{K}}
\newcommand{\V}{\mathrm{V}}
\newcommand{\N}{\mathbb{N}}
\newcommand{\Z}{\mathbb{Z}}
\newcommand{\F}{\mathbb{F}}
\newcommand{\Q}{\mathbb{Q}}
\newcommand{\R}{\mathbb{R}}
\newcommand{\C}{\mathbb{C}}
\renewcommand{\H}{\mathcal{H}}
\newcommand{\Cinf}{\mathcal{C}^\infty}
\newcommand{\Cu}{\mathcal{C}^1}
\newcommand{\Rp}{\mathfrak{Re}}
\newcommand{\Ip}{\mathfrak{Im}}
\renewcommand{\d}{\mathrm{d}}
\newcommand{\dm}{\mathrm{d}\mu}
\newcommand{\conj}[1]{\overline{#1}}
\newcommand{\Tras}[1]{{#1}^{\text{T}}}

\DeclareMathOperator{\Log}{Log}
\DeclareMathOperator{\Arg}{Arg}
\DeclareMathOperator{\Dom}{Dom}
\DeclareMathOperator{\Ima}{Im}
\DeclareMathOperator{\sgn}{sgn}
\DeclareMathOperator{\mcd}{MCD}
\DeclareMathOperator{\mcm}{mcm}
\DeclareMathOperator{\Resi}{Res}
\DeclareMathOperator{\Ker}{Ker}
\DeclareMathOperator{\End}{End}
\DeclareMathOperator{\Mat}{Mat}

\newcommand{\parder}[2]{\frac{\partial #1}{\partial #2}}
\newcommand{\dparder}[2]{\dfrac{\partial #1}{\partial #2}}
\newcommand{\tparder}[2]{\partial #1/\partial #2}
\newcommand{\parderr}[3]{\frac{\partial^2 #1}{\partial #2\partial #3}}
\newcommand{\dparderr}[3]{\dfrac{\partial^2 #1}{\partial #2\partial #3}}
\newcommand{\tparderr}[3]{\partial^2 #1/\partial #2\partial #3}
\newcommand{\intx}[1]{\int #1\,dx}
\newcommand{\intt}[1]{\int #1\,dt}
\newcommand{\intdx}[3]{\int_{#1}^{#2} #3\,dx}
\newcommand{\intdt}[3]{\int_{#1}^{#2} #3\,dt}
\newcommand{\intdz}[2]{\int_{#1} #2\,dz}
\newcommand{\set}[1]{\left\{#1\right\}}
\newcommand{\so}{\Rightarrow}
\newcommand{\sii}{\Leftrightarrow}
\newcommand{\by}[1]{\overset{\fbox{\tiny #1}}{=}}
\newcommand{\byref}[1]{\overset{\fbox{\tiny\ref{#1}}}{=}}
\newcommand{\cardinal}[1]{\left|#1\right|}
\newcommand{\maps}[3]{#1 \colon #2\longrightarrow #3}
\newcommand{\equationmaps}[5]{\begin{aligned}[t]#1 \colon #2 &\longrightarrow #3 \\	#4 &\longmapsto #5\end{aligned}}
\newcommand{\coma}{,\thinspace}
\newcommand{\pari}[2]{(#1,\thinspace #2)}
\newcommand{\where}{\mathrel{}\middle|\mathrel{}}
\newcommand{\no}[1]{{\neg}{#1}}
\newcommand{\dcomilla}[1]{{\guillemotleft}#1{\guillemotright}}
\newcommand{\separa}{\vspace*{.75\baselineskip}}
\newcommand{\semisepara}{\vspace*{.25\baselineskip}}
\newcommand{\restrict}[1]{\raisebox{-.5ex}{$|$}_{#1}}

% ========================================
% COMANDOS PERSONALIZADOS PARA COMPUTACIÓN CUÁNTICA
% ========================================

% Estados comunes
\newcommand{\zero}{\ket{0}}
\newcommand{\one}{\ket{1}}
\newcommand{\plus}{\ket{+}}
\newcommand{\minus}{\ket{-}}

% Operadores especiales
\newcommand{\tensor}{\otimes}         % Producto tensorial
\newcommand{\comp}{\circ}             % Composición

% Estados de Bell
\newcommand{\bellphi}{\ket{\Phi^+}}
\newcommand{\bellpsi}{\ket{\Psi^+}}
\newcommand{\bellphiminus}{\ket{\Phi^-}}
\newcommand{\bellpsiminus}{\ket{\Psi^-}}

\newcommand{\floor}[1]{\left\lfloor #1 \right\rfloor}
\newcommand{\ceil}[1]{\left\lceil #1 \right\rceil}


\printanswers

\author{Francisco Costa Cano}
\titulacion{Máster en computación cuántica}
\asignatura{Álgebra lineal en computación cuántica}
\bloque{1}{Fundamentos matemáticos}
\tema{5}{Espacios de Hilbert}

\begin{document}

\portada

\section{Introducción}

Este taller integra los conceptos fundamentales de espacios de Hilbert a través de ejercicios progresivos. El objetivo es consolidar la comprensión de cada concepto y su interrelación mediante ejemplos concretos que preparan para aplicaciones en computación cuántica.

\begin{info}
  \textbf{Estructura del taller:}
  \begin{itemize}
    \item Duración: 2 horas.
    \item Enfoque: Resolución guiada de ejercicios interconectados.
    \item Metodología: Construcción progresiva desde conceptos básicos hasta descomposiciones avanzadas.
  \end{itemize}
\end{info}

\section{Ejercicio Integrador: Sistema de dos cúbits}

Consideramos el espacio de Hilbert $\mathcal{H} = \mathbb{C}^4$, que representa el espacio de estados de un sistema de dos cúbits. Trabajaremos con este espacio para explorar todos los conceptos del curso.

\subsection{Parte 1: Espacio vectorial y bases (15 min)}

\begin{questions}

  \question Demuestra que $\mathcal{H} = \mathbb{C}^4$ con las operaciones usuales es un espacio vectorial sobre $\mathbb{C}$.

  \begin{solution}

    Debemos verificar los axiomas de espacio vectorial. Para vectores $v, w, u \in \mathbb{C}^4$ y escalares $\alpha, \beta \in \mathbb{C}$:

    \begin{enumerate}
      \item \textbf{Cerradura bajo suma:} Si $v = (v_1, v_2, v_3, v_4)$ y $w = (w_1, w_2, w_3, w_4)$, entonces:
            \[
              v + w = (v_1+w_1, v_2+w_2, v_3+w_3, v_4+w_4) \in \mathbb{C}^4
            \]

      \item \textbf{Cerradura bajo producto por escalar:}
            \[
              \alpha v = (\alpha v_1, \alpha v_2, \alpha v_3, \alpha v_4) \in \mathbb{C}^4
            \]

      \item \textbf{Elemento neutro:} $0 = (0,0,0,0)$ satisface $v + 0 = v$.

      \item \textbf{Elemento opuesto:} Para cada $v$ existe $-v = (-v_1, -v_2, -v_3, -v_4)$ tal que $v + (-v) = 0$.

      \item \textbf{Asociatividad de la suma:} $(u + v) + w = u + (v + w)$.

      \item \textbf{Conmutatividad de la suma:} $v + w = w + v$.

      \item \textbf{Distributividad:} $\alpha(v + w) = \alpha v + \alpha w$ y $(\alpha + \beta)v = \alpha v + \beta w$.

      \item \textbf{Asociatividad del producto:} $(\alpha\beta)v = \alpha(\beta w)$.

      \item \textbf{Elemento neutro del producto:} $1\cdot v = v$.
    \end{enumerate}

  \end{solution}

  \question \textbf{Base de Bell}

  Consideramos la base de Bell $\mathcal{B}_B=\{B_1,B_2,B_3,B_4\}$, fundamental en computación cuántica:
  \begin{align*}
    B_1 & = \frac{1}{\sqrt{2}}(1, 0, 0, 1)\,,  \\
    B_2 & = \frac{1}{\sqrt{2}}(1, 0, 0, -1)\,, \\
    B_3 & = \frac{1}{\sqrt{2}}(0, 1, 1, 0)\,,  \\
    B_4 & = \frac{1}{\sqrt{2}}(0, 1, -1, 0)\,.
  \end{align*}
  Verifica que $\mathcal{B}_B$ forma una base ortonormal.

  \begin{solution}
    Recordamos que la norma de un vector se define como
    \[
      \|(v_1, v_2, v_3, v_4)\| = \sqrt{|v_1|^2 + |v_2|^2 + |v_3|^2 + |v_4|^2}\,.
    \]
    \textbf{Normalización:}
    \begin{align*}
      \|B_1\| & = \frac{1}{\sqrt{2}}\sqrt{1^2 + 0^2 + 0^2 + 1^2} = \frac{1}{\sqrt{2}}\sqrt{2} = 1 \,.    \\
      \|B_2\| & = \frac{1}{\sqrt{2}}\sqrt{1^2 + 0^2 + 0^2 + (-1)^2} = \frac{1}{\sqrt{2}}\sqrt{2} = 1 \,. \\
      \|B_3\| & = \frac{1}{\sqrt{2}}\sqrt{0^2 + 1^2 + 1^2 + 0^2} = \frac{1}{\sqrt{2}}\sqrt{2} = 1 \,.    \\
      \|B_4\| & = \frac{1}{\sqrt{2}}\sqrt{0^2 + 1^2 + (-1)^2 + 0^2} = \frac{1}{\sqrt{2}}\sqrt{2} = 1 \,. \\
    \end{align*}

    \textbf{Ortogonalidad:}
    Recordamos que dos vectores $v,w$ son ortogonales si
    \[
      \langle v, w \rangle = 0\,.
    \]
    Por la propiedad del producto interno $\langle v, w \rangle = \overline{\langle w, v \rangle}$, para la ortogonalidad solo será necesario por parejas sin importar el orden.
    \begin{align*}
      \langle B_1, B_2 \rangle & = \frac{1}{\sqrt{2}}\frac{1}{\sqrt{2}}(1\bar{1}+ 0\bar{0} + 0\bar{0}+ 1(\overline{-1})= \frac{1}{2}(1 - 1) = 0        \\
      \langle B_1, B_3 \rangle & = \frac{1}{\sqrt{2}}\frac{1}{\sqrt{2}}(1\bar{0}+ 0\bar{1} + 0\bar{1}+ 1\overline{0}= \frac{1}{2}(0 + 0) = 0           \\
      \langle B_1, B_4 \rangle & = \frac{1}{\sqrt{2}}\frac{1}{\sqrt{2}}(1\bar{0}+ 0\bar{1} + 0(\overline{-1})+ 1\overline{0}= \frac{1}{2}(0 - 0) = 0   \\
      \langle B_2, B_3 \rangle & = \frac{1}{\sqrt{2}}\frac{1}{\sqrt{2}}(1\bar{0}+ 0\bar{1} + 0\bar{1}- 1\overline{0}= \frac{1}{2}(0 - 0) = 0           \\
      \langle B_2, B_4 \rangle & = \frac{1}{\sqrt{2}}\frac{1}{\sqrt{2}}(1\bar{0}+ 0(\overline{-1}) + 0\bar{1}- 1\overline{0}= \frac{1}{2}(0 - 0) = 0   \\
      \langle B_3, B_4 \rangle & = \frac{1}{\sqrt{2}}\frac{1}{\sqrt{2}}(0\bar{0}+ 1\bar{1} + 1(\overline{-1})+ 0(\overline{0})= \frac{1}{2}(1 - 1) = 0 \\
    \end{align*}

    Todos los productos internos entre estados distintos son cero, verificando la ortogonalidad.

    \textbf{Base:}
    Como sabemos que $\C^4$ es un espacio vectorial de dimensión 4, y tenemos 4 vectores, solo debemos o ver que son linealmente independientes o un sistema generador.

    Veamos que son linealmente independientes. Por definición $\{v_1,v_2,v_3,v_4\}$ son linealmente independientes si y solo si $\forall \alpha_1,\alpha_2,\alpha_3,\alpha_4 \in \C$ se cumple
    \[
      \alpha_1 v_1 + \alpha_2 v_2 + \alpha_3 v_3 + \alpha_4 v_4 = 0 \Rightarrow \alpha_1 = \alpha_2 = \alpha_3 = \alpha_4 = 0\,.
    \]
    En nuestro caso
    \[
      \alpha_1 B_1 + \alpha_2 B_2 + \alpha_3 B_3 + \alpha_4 B_4 = 0 \Rightarrow \begin{cases}
        \alpha_1 + \alpha_2 = 0 \\
        \alpha_3 + \alpha_4 = 0 \\
        \alpha_3 - \alpha_4 = 0 \\
        \alpha_1 - \alpha_2 = 0
      \end{cases} \Rightarrow \alpha_i = 0\,.
    \]
  \end{solution}

  \question Encuentra la matriz $P$ que permite cambiar de la base estándar a la base de Bell.
  \begin{solution}

    Llamando $\mathcal{B}=\{e_1, e_2, e_3, e_4\}$ la base estándar, la matriz de cambio de base $\mathcal{B_B}$ a $\mathcal{B}$ tiene como filas las coordenadas de los vectores de la nueva base expresados en la base antigua. Por lo tanto

    \[
      M_{\mathcal{B}_B}^{\mathcal{B}} = \frac{1}{\sqrt{2}}\begin{pmatrix}
        1 & 0 & 0  & 1  \\
        1 & 0 & 0  & -1 \\
        0 & 1 & 1  & 0  \\
        0 & 1 & -1 & 0
      \end{pmatrix}\,.
    \]
  \end{solution}

  \question Expresa el vector
  \[
    u = \frac{1}{2}(1, 1, 1, 1)\,,
  \]
  en la base de Bell.

  \begin{solution}

    Como las coordenadas de $u$ están dadas en la base estándar, su expresión desarrollada en esta base es
    \[
      u = \frac{1}{2}\sum_{i=1}^4 e_i\,.
    \]

    Para obtener las coordenadas en la base de Bell, usamos $M_{\mathcal{B}}^{\mathcal{B}_B} = (M_{\mathcal{B}_B}^{\mathcal{B}})^{-1}$. Como $M_{\mathcal{B}_B}^{\mathcal{B}}$ es unitaria (producto de operaciones unitarias), $(M_{\mathcal{B}_B}^{\mathcal{B}})^{-1} = (M_{\mathcal{B}_B}^{\mathcal{B}})^\dagger$:

    \begin{align*}
      u_{\mathcal{B}_B} = u_{\mathcal{B}} M_{\mathcal{B}_B}^{\mathcal{B}} & = \frac{1}{2}(1, 1, 1, 1) \frac{1}{\sqrt{2}}\begin{pmatrix}
                                                                                                                          1 & 1  & 0 & 0  \\
                                                                                                                          0 & 0  & 1 & 1  \\
                                                                                                                          0 & 0  & 1 & -1 \\
                                                                                                                          1 & -1 & 0 & 0
                                                                                                                        \end{pmatrix}        \\
                                                                          & = \frac{1}{2\sqrt{2}}(2, 0, 2, 0) = \frac{1}{\sqrt{2}}(1, 0, 1, 0)
    \end{align*}

    Por lo tanto:
    \[
      u = \frac{1}{\sqrt{2}}B_1 + \frac{1}{\sqrt{2}}B_3\,.
    \]
  \end{solution}

  \question
  Sea el subespacio $\mathcal{S} \subset \mathbb{C}^4$ definido por
  \[
    \mathcal{S} = \text{span}\{B_1, B_3\}
  \]

  Encuentra una base ortonormal para el subespacio ortogonal $\mathcal{S}^\perp$.

  \begin{solution}

    El subespacio ortogonal $\mathcal{S}^\perp$ contiene todos los vectores ortogonales a $\mathcal{S}$. De la base de Bell, sabemos que $B_2$ y $B_4$ son ortogonales a $B_1$ y $B_3$, que son linealmente independientes y que
    \[
      \mathbb{C}^4 = \mathcal{S} \oplus \mathcal{S}^\perp \Rightarrow \dim(\C^4) = \dim(\mathcal{S}) + \dim(\mathcal{S}^\perp)\Rightarrow \dim(\mathcal{S}^\perp) = 2
    \]

    Por lo tanto, una base ortonormal para $\mathcal{S}^\perp$ es
    \[
      \mathcal{S}^\perp = \text{span}\{B_2, B_4\}
    \]
  \end{solution}

  \question Construye el proyector ortogonal $P_{\mathcal{S}}$ sobre el subespacio simétrico.

  \begin{solution}

    El proyector sobre un subespacio generado por la base ortonormal $\{B_1, B_3\}$ se construye usando el producto externo como
    \[
      P_{\mathcal{S}} = B_1\wedge B_1 + B_3\wedge B_3\,.
    \]

    Calculamos explícitamente
    \begin{align*}
      B_1\wedge B_1 & = \frac{1}{\sqrt{2}}(1, 0, 0, 1)\otimes \frac{1}{\sqrt{2}}\begin{pmatrix} 1 \\ 0 \\ 0 \\ 1 \end{pmatrix} = \frac{1}{2}\begin{pmatrix}
                                                                                                                                              1 & 0 & 0 & 1 \\
                                                                                                                                              0 & 0 & 0 & 0 \\
                                                                                                                                              0 & 0 & 0 & 0 \\
                                                                                                                                              1 & 0 & 0 & 1
                                                                                                                                            \end{pmatrix}\,. \\
      B_3\wedge B_3 & = \frac{1}{\sqrt{2}}(0, 1, 1, 0)\otimes \frac{1}{\sqrt{2}}\begin{pmatrix} 0 \\ 1 \\ 1 \\ 0 \end{pmatrix} = \frac{1}{2}\begin{pmatrix}
                                                                                                                                              0 & 0 & 0 & 0 \\
                                                                                                                                              0 & 1 & 1 & 0 \\
                                                                                                                                              0 & 1 & 1 & 0 \\
                                                                                                                                              0 & 0 & 0 & 0
                                                                                                                                            \end{pmatrix}\,.
    \end{align*}


    Por lo tanto:
    \[
      P_{\mathcal{S}} = \frac{1}{2}\begin{pmatrix}
        1 & 0 & 0 & 1 \\
        0 & 1 & 1 & 0 \\
        0 & 1 & 1 & 0 \\
        1 & 0 & 0 & 1
      \end{pmatrix}
    \]

    Verificamos que es un proyector: $P_{\mathcal{S}}^2 = P_{\mathcal{S}}$ y $P_{\mathcal{S}}^\dagger = P_{\mathcal{S}}$.
  \end{solution}

  \question Calcula la proyección del vector $(1, 0 , 0, 0)$ sobre el subespacio $\mathcal{S}$.
  \begin{solution}

    La proyección se calcula aplicando el proyector, o lo que es lo mismo, multiplicando por la matriz del proyector:
    \[
      P_{\mathcal{S}}((1, 0, 0,0)) = (1, 0, 0,0)\frac{1}{2}\begin{pmatrix}
        1 & 0 & 0 & 1 \\
        0 & 1 & 1 & 0 \\
        0 & 1 & 1 & 0 \\
        1 & 0 & 0 & 1
      \end{pmatrix} = \frac{1}{2}(1, 0, 0, 1)\,.
    \]
  \end{solution}

  \subsection{Parte 4: Proceso de Gram-Schmidt (15 min)}

  \begin{solution}
    Considera los siguientes vectores linealmente independientes en $\mathbb{C}^4$:
    \begin{align*}
      \ket{v_1} & = \ket{00} + \ket{11} = \begin{pmatrix} 1 \\ 0 \\ 0 \\ 1 \end{pmatrix}            \\
      \ket{v_2} & = \ket{00} + \ket{01} = \begin{pmatrix} 1 \\ 1 \\ 0 \\ 0 \end{pmatrix}            \\
      \ket{v_3} & = \ket{00} + \ket{01} + \ket{10} = \begin{pmatrix} 1 \\ 1 \\ 1 \\ 0 \end{pmatrix}
    \end{align*}

    Aplica el proceso de Gram-Schmidt para obtener una base ortonormal.

    \textbf{Solución:}

    \textbf{Paso 1:} Normalizamos $\ket{v_1}$:
    \[
      \ket{u_1} = \frac{\ket{v_1}}{\|\ket{v_1}\|} = \frac{1}{\sqrt{2}}\begin{pmatrix} 1 \\ 0 \\ 0 \\ 1 \end{pmatrix}
    \]

    donde $\|\ket{v_1}\| = \sqrt{1^2 + 0^2 + 0^2 + 1^2} = \sqrt{2}$.

    \textbf{Paso 2:} Ortogonalizamos $\ket{v_2}$ respecto a $\ket{u_1}$:
    \[
      \ket{w_2} = \ket{v_2} - \braket{u_1|v_2}\ket{u_1}
    \]

    Calculamos:
    \[
      \braket{u_1|v_2} = \frac{1}{\sqrt{2}}(1, 0, 0, 1)\begin{pmatrix} 1 \\ 1 \\ 0 \\ 0 \end{pmatrix} = \frac{1}{\sqrt{2}}
    \]

    Por lo tanto:
    \[
      \ket{w_2} = \begin{pmatrix} 1 \\ 1 \\ 0 \\ 0 \end{pmatrix} - \frac{1}{\sqrt{2}} \cdot \frac{1}{\sqrt{2}}\begin{pmatrix} 1 \\ 0 \\ 0 \\ 1 \end{pmatrix} = \begin{pmatrix} 1 \\ 1 \\ 0 \\ 0 \end{pmatrix} - \begin{pmatrix} 1/2 \\ 0 \\ 0 \\ 1/2 \end{pmatrix} = \begin{pmatrix} 1/2 \\ 1 \\ 0 \\ -1/2 \end{pmatrix}
    \]

    Normalizamos:
    \[
      \|\ket{w_2}\| = \sqrt{(1/2)^2 + 1^2 + 0^2 + (-1/2)^2} = \sqrt{1/4 + 1 + 1/4} = \sqrt{3/2}
    \]

    \[
      \ket{u_2} = \frac{1}{\sqrt{3/2}}\begin{pmatrix} 1/2 \\ 1 \\ 0 \\ -1/2 \end{pmatrix} = \sqrt{\frac{2}{3}}\begin{pmatrix} 1/2 \\ 1 \\ 0 \\ -1/2 \end{pmatrix} = \frac{1}{\sqrt{6}}\begin{pmatrix} 1 \\ 2 \\ 0 \\ -1 \end{pmatrix}
    \]

    \textbf{Paso 3:} Ortogonalizamos $\ket{v_3}$ respecto a $\ket{u_1}$ y $\ket{u_2}$:
    \[
      \ket{w_3} = \ket{v_3} - \braket{u_1|v_3}\ket{u_1} - \braket{u_2|v_3}\ket{u_2}
    \]

    Calculamos:
    \[
      \braket{u_1|v_3} = \frac{1}{\sqrt{2}}(1, 0, 0, 1)\begin{pmatrix} 1 \\ 1 \\ 1 \\ 0 \end{pmatrix} = \frac{1}{\sqrt{2}}
    \]

    \[
      \braket{u_2|v_3} = \frac{1}{\sqrt{6}}(1, 2, 0, -1)\begin{pmatrix} 1 \\ 1 \\ 1 \\ 0 \end{pmatrix} = \frac{1}{\sqrt{6}}(1 + 2) = \frac{3}{\sqrt{6}} = \sqrt{\frac{3}{2}}
    \]

    Por lo tanto:
    \[
      \ket{w_3} = \begin{pmatrix} 1 \\ 1 \\ 1 \\ 0 \end{pmatrix} - \frac{1}{\sqrt{2}} \cdot \frac{1}{\sqrt{2}}\begin{pmatrix} 1 \\ 0 \\ 0 \\ 1 \end{pmatrix} - \sqrt{\frac{3}{2}} \cdot \frac{1}{\sqrt{6}}\begin{pmatrix} 1 \\ 2 \\ 0 \\ -1 \end{pmatrix}
    \]

    \[
      = \begin{pmatrix} 1 \\ 1 \\ 1 \\ 0 \end{pmatrix} - \begin{pmatrix} 1/2 \\ 0 \\ 0 \\ 1/2 \end{pmatrix} - \frac{1}{2}\begin{pmatrix} 1 \\ 2 \\ 0 \\ -1 \end{pmatrix} = \begin{pmatrix} 1 \\ 1 \\ 1 \\ 0 \end{pmatrix} - \begin{pmatrix} 1/2 \\ 0 \\ 0 \\ 1/2 \end{pmatrix} - \begin{pmatrix} 1/2 \\ 1 \\ 0 \\ -1/2 \end{pmatrix}
    \]

    \[
      = \begin{pmatrix} 0 \\ 0 \\ 1 \\ 0 \end{pmatrix}
    \]

    Este vector ya está normalizado:
    \[
      \ket{u_3} = \begin{pmatrix} 0 \\ 0 \\ 1 \\ 0 \end{pmatrix} = \ket{10}
    \]

    \textbf{Base ortonormal resultante:}
    \[
      \left\{\frac{1}{\sqrt{2}}\begin{pmatrix} 1 \\ 0 \\ 0 \\ 1 \end{pmatrix}, \frac{1}{\sqrt{6}}\begin{pmatrix} 1 \\ 2 \\ 0 \\ -1 \end{pmatrix}, \begin{pmatrix} 0 \\ 0 \\ 1 \\ 0 \end{pmatrix}\right\}
    \]
  \end{solution}

  \subsection{Parte 5: Endomorfismos y representación matricial (15 min)}

  \begin{solution}
    Considera el operador de intercambio (SWAP) que intercambia los dos cúbits:
    \[
      \text{SWAP}: \ket{ij} \mapsto \ket{ji}
    \]

    \textbf{a) Representación en la base computacional:}

    Encuentra la matriz del operador SWAP en la base computacional.

    \textbf{Solución:}

    Para encontrar la matriz, aplicamos el operador a cada vector de la base:
    \begin{align*}
      \text{SWAP}(\ket{00}) & = \ket{00} \\
      \text{SWAP}(\ket{01}) & = \ket{10} \\
      \text{SWAP}(\ket{10}) & = \ket{01} \\
      \text{SWAP}(\ket{11}) & = \ket{11}
    \end{align*}

    Las columnas de la matriz son las imágenes de los vectores de la base:
    \[
      [\text{SWAP}]_{\mathcal{B}_c} = \begin{pmatrix}
        1 & 0 & 0 & 0 \\
        0 & 0 & 1 & 0 \\
        0 & 1 & 0 & 0 \\
        0 & 0 & 0 & 1
      \end{pmatrix}
    \]
  \end{solution}

  \begin{solution}
    \textbf{b) Representación en la base de Bell:}

    Encuentra la matriz del operador SWAP en la base de Bell.

    \textbf{Solución:}

    Aplicamos el operador a cada vector de la base de Bell:
    \begin{align*}
      \text{SWAP}(\ket{\Phi^+}) & = \text{SWAP}\left(\frac{1}{\sqrt{2}}(\ket{00} + \ket{11})\right) = \frac{1}{\sqrt{2}}(\ket{00} + \ket{11}) = \ket{\Phi^+}  \\
      \text{SWAP}(\ket{\Phi^-}) & = \text{SWAP}\left(\frac{1}{\sqrt{2}}(\ket{00} - \ket{11})\right) = \frac{1}{\sqrt{2}}(\ket{00} - \ket{11}) = \ket{\Phi^-}  \\
      \text{SWAP}(\ket{\Psi^+}) & = \text{SWAP}\left(\frac{1}{\sqrt{2}}(\ket{01} + \ket{10})\right) = \frac{1}{\sqrt{2}}(\ket{10} + \ket{01}) = \ket{\Psi^+}  \\
      \text{SWAP}(\ket{\Psi^-}) & = \text{SWAP}\left(\frac{1}{\sqrt{2}}(\ket{01} - \ket{10})\right) = \frac{1}{\sqrt{2}}(\ket{10} - \ket{01}) = -\ket{\Psi^-}
    \end{align*}

    Por lo tanto, en la base de Bell:
    \[
      [\text{SWAP}]_{\mathcal{B}_B} = \begin{pmatrix}
        1 & 0 & 0 & 0  \\
        0 & 1 & 0 & 0  \\
        0 & 0 & 1 & 0  \\
        0 & 0 & 0 & -1
      \end{pmatrix}
    \]

    Observamos que en la base de Bell, el operador SWAP es diagonal.
  \end{solution}

  \begin{solution}
    \textbf{c) Cambio de representación:}

    Verifica que las dos representaciones están relacionadas por:
    \[
      [\text{SWAP}]_{\mathcal{B}_B} = P [\text{SWAP}]_{\mathcal{B}_c} P
    \]

    \textbf{Solución:}

    Calculamos el lado derecho:
    \[
      P [\text{SWAP}]_{\mathcal{B}_c} P = P \begin{pmatrix}
        1 & 0 & 0 & 0 \\
        0 & 0 & 1 & 0 \\
        0 & 1 & 0 & 0 \\
        0 & 0 & 0 & 1
      \end{pmatrix} P
    \]

    Primero calculamos $[\text{SWAP}]_{\mathcal{B}_c} P$:
    \[
      \begin{pmatrix}
        1 & 0 & 0 & 0 \\
        0 & 0 & 1 & 0 \\
        0 & 1 & 0 & 0 \\
        0 & 0 & 0 & 1
      \end{pmatrix} \frac{1}{\sqrt{2}}\begin{pmatrix}
        1 & 1  & 0 & 0  \\
        0 & 0  & 1 & 1  \\
        0 & 0  & 1 & -1 \\
        1 & -1 & 0 & 0
      \end{pmatrix} = \frac{1}{\sqrt{2}}\begin{pmatrix}
        1 & 1  & 0 & 0  \\
        0 & 0  & 1 & -1 \\
        0 & 0  & 1 & 1  \\
        1 & -1 & 0 & 0
      \end{pmatrix}
    \]

    Luego multiplicamos por $P$:
    \[
      \frac{1}{\sqrt{2}}\begin{pmatrix}
        1 & 0 & 0  & 1  \\
        1 & 0 & 0  & -1 \\
        0 & 1 & 1  & 0  \\
        0 & 1 & -1 & 0
      \end{pmatrix} \frac{1}{\sqrt{2}}\begin{pmatrix}
        1 & 1  & 0 & 0  \\
        0 & 0  & 1 & -1 \\
        0 & 0  & 1 & 1  \\
        1 & -1 & 0 & 0
      \end{pmatrix} = \begin{pmatrix}
        1 & 0 & 0 & 0  \\
        0 & 1 & 0 & 0  \\
        0 & 0 & 1 & 0  \\
        0 & 0 & 0 & -1
      \end{pmatrix}
    \]

    que coincide con $[\text{SWAP}]_{\mathcal{B}_B}$.
  \end{solution}

  \subsection{Parte 6: Valores propios y subespacios propios (15 min)}

  \begin{solution}
    \textbf{a) Espectro del operador SWAP:}

    Encuentra los valores propios y vectores propios del operador SWAP.

    \textbf{Solución:}

    De la representación en la base de Bell, vemos directamente que:

    \textbf{Valor propio $\lambda_1 = 1$:} Con multiplicidad 3.

    Vectores propios:
    \[
      \ket{\Phi^+}, \quad \ket{\Phi^-}, \quad \ket{\Psi^+}
    \]

    El subespacio propio es:
    \[
      E_1 = \text{span}\{\ket{\Phi^+}, \ket{\Phi^-}, \ket{\Psi^+}\} = \text{span}\left\{\ket{00}, \ket{11}, \frac{1}{\sqrt{2}}(\ket{01} + \ket{10})\right\}
    \]

    \textbf{Valor propio $\lambda_2 = -1$:} Con multiplicidad 1.

    Vector propio:
    \[
      \ket{\Psi^-} = \frac{1}{\sqrt{2}}(\ket{01} - \ket{10})
    \]

    El subespacio propio es:
    \[
      E_{-1} = \text{span}\{\ket{\Psi^-}\}
    \]

    Verificamos para $\ket{\Psi^-}$:
    \[
      \text{SWAP}(\ket{\Psi^-}) = \text{SWAP}\left(\frac{1}{\sqrt{2}}(\ket{01} - \ket{10})\right) = \frac{1}{\sqrt{2}}(\ket{10} - \ket{01}) = -\frac{1}{\sqrt{2}}(\ket{01} - \ket{10}) = -\ket{\Psi^-}
    \]
  \end{solution}

  \begin{solution}
    \textbf{b) Descomposición espectral:}

    Escribe el operador SWAP como suma de proyectores sobre sus subespacios propios.

    \textbf{Solución:}

    La descomposición espectral es:
    \[
      \text{SWAP} = 1 \cdot P_{E_1} + (-1) \cdot P_{E_{-1}}
    \]

    donde:
    \[
      P_{E_1} = \ket{\Phi^+}\bra{\Phi^+} + \ket{\Phi^-}\bra{\Phi^-} + \ket{\Psi^+}\bra{\Psi^+}
    \]

    \[
      P_{E_{-1}} = \ket{\Psi^-}\bra{\Psi^-}
    \]

    Calculamos explícitamente:
    \[
      P_{E_{-1}} = \frac{1}{2}\begin{pmatrix} 0 \\ 1 \\ -1 \\ 0 \end{pmatrix}\begin{pmatrix} 0 & 1 & -1 & 0 \end{pmatrix} = \frac{1}{2}\begin{pmatrix}
        0 & 0  & 0  & 0 \\
        0 & 1  & -1 & 0 \\
        0 & -1 & 1  & 0 \\
        0 & 0  & 0  & 0
      \end{pmatrix}
    \]

    Y podemos verificar:
    \[
      P_{E_1} = \mathbb{I} - P_{E_{-1}} = \begin{pmatrix}
        1 & 0 & 0 & 0 \\
        0 & 1 & 0 & 0 \\
        0 & 0 & 1 & 0 \\
        0 & 0 & 0 & 1
      \end{pmatrix} - \frac{1}{2}\begin{pmatrix}
        0 & 0  & 0  & 0 \\
        0 & 1  & -1 & 0 \\
        0 & -1 & 1  & 0 \\
        0 & 0  & 0  & 0
      \end{pmatrix}
    \]

    \[
      = \begin{pmatrix}
        1 & 0   & 0   & 0 \\
        0 & 1/2 & 1/2 & 0 \\
        0 & 1/2 & 1/2 & 0 \\
        0 & 0   & 0   & 1
      \end{pmatrix}
    \]

    Por lo tanto:
    \[
      \text{SWAP} = P_{E_1} - P_{E_{-1}} = \begin{pmatrix}
        1 & 0   & 0   & 0 \\
        0 & 1/2 & 1/2 & 0 \\
        0 & 1/2 & 1/2 & 0 \\
        0 & 0   & 0   & 1
      \end{pmatrix} - \frac{1}{2}\begin{pmatrix}
        0 & 0  & 0  & 0 \\
        0 & 1  & -1 & 0 \\
        0 & -1 & 1  & 0 \\
        0 & 0  & 0  & 0
      \end{pmatrix}
    \]

    \[
      = \begin{pmatrix}
        1 & 0 & 0 & 0 \\
        0 & 0 & 1 & 0 \\
        0 & 1 & 0 & 0 \\
        0 & 0 & 0 & 1
      \end{pmatrix}
    \]
  \end{solution}

  \subsection{Parte 7: Operadores hermitianos y unitarios (10 min)}

  \begin{solution}
    \textbf{a) Hermiticidad del operador SWAP:}

    Verifica que el operador SWAP es hermitiano.

    \textbf{Solución:}

    Un operador es hermitiano si $A^\dagger = A$. Para la matriz en la base computacional:
    \[
      [\text{SWAP}]_{\mathcal{B}_c}^\dagger = \begin{pmatrix}
        1 & 0 & 0 & 0 \\
        0 & 0 & 1 & 0 \\
        0 & 1 & 0 & 0 \\
        0 & 0 & 0 & 1
      \end{pmatrix}^* = \begin{pmatrix}
        1 & 0 & 0 & 0 \\
        0 & 0 & 1 & 0 \\
        0 & 1 & 0 & 0 \\
        0 & 0 & 0 & 1
      \end{pmatrix} = [\text{SWAP}]_{\mathcal{B}_c}
    \]

    Por lo tanto, SWAP es hermitiano.

    \textbf{Consecuencias:}
    \begin{itemize}
      \item Los valores propios son reales: $\lambda \in \{1, -1\} \subset \mathbb{R}$.
      \item Los vectores propios correspondientes a valores propios distintos son ortogonales.
      \item El operador es diagonalizable en una base ortonormal.
    \end{itemize}
  \end{solution}

  \begin{solution}
    \textbf{b) Unitariedad del operador SWAP:}

    Verifica que el operador SWAP es unitario.

    \textbf{Solución:}

    Un operador es unitario si $U^\dagger U = U U^\dagger = \mathbb{I}$. Como SWAP es hermitiano, debemos verificar que $\text{SWAP}^2 = \mathbb{I}$:

    \[
      [\text{SWAP}]_{\mathcal{B}_c}^2 = \begin{pmatrix}
        1 & 0 & 0 & 0 \\
        0 & 0 & 1 & 0 \\
        0 & 1 & 0 & 0 \\
        0 & 0 & 0 & 1
      \end{pmatrix}^2 = \begin{pmatrix}
        1 & 0 & 0 & 0 \\
        0 & 1 & 0 & 0 \\
        0 & 0 & 1 & 0 \\
        0 & 0 & 0 & 1
      \end{pmatrix} = \mathbb{I}
    \]

    Por lo tanto, SWAP es unitario.

    \textbf{Consecuencias:}
    \begin{itemize}
      \item SWAP preserva el producto interno: $\braket{\text{SWAP}(\psi)|\text{SWAP}(\phi)} = \braket{\psi|\phi}$.
      \item Los valores propios tienen módulo 1: $|\lambda| = 1$.
      \item SWAP es reversible: $\text{SWAP}^{-1} = \text{SWAP}^\dagger = \text{SWAP}$.
    \end{itemize}

    En este caso particular, SWAP es tanto hermitiano como unitario, lo que implica que sus valores propios son reales y de módulo 1, es decir, $\lambda \in \{-1, 1\}$.
  \end{solution}

  \subsection{Parte 8: Descomposición en valores singulares (10 min)}

  \begin{solution}
    Considera el operador de medida parcial en el primer cúbit definido por:
    \[
      M_0 = \ket{0}\bra{0} \otimes \mathbb{I}_2 = \begin{pmatrix}
        1 & 0 \\
        0 & 0
      \end{pmatrix} \otimes \begin{pmatrix}
        1 & 0 \\
        0 & 1
      \end{pmatrix}
    \]

    \textbf{a) Representación matricial:}

    Calcula la representación matricial de $M_0$ en la base computacional de dos cúbits.

    \textbf{Solución:}

    Usando la definición del producto tensorial:
    \[
      M_0 = \begin{pmatrix}
        1 \cdot \mathbb{I}_2 & 0 \cdot \mathbb{I}_2 \\
        0 \cdot \mathbb{I}_2 & 0 \cdot \mathbb{I}_2
      \end{pmatrix} = \begin{pmatrix}
        1 & 0 & 0 & 0 \\
        0 & 1 & 0 & 0 \\
        0 & 0 & 0 & 0 \\
        0 & 0 & 0 & 0
      \end{pmatrix}
    \]

    Verificamos que $M_0$ proyecta sobre el subespacio donde el primer cúbit está en $\ket{0}$:
    \begin{align*}
      M_0 \ket{00} & = \ket{00}, \quad M_0 \ket{01} = \ket{01} \\
      M_0 \ket{10} & = 0, \quad M_0 \ket{11} = 0
    \end{align*}
  \end{solution}

  \begin{solution}
    \textbf{b) Descomposición en valores singulares:}

    Calcula la SVD de $M_0$.

    \textbf{Solución:}

    Primero calculamos $M_0^\dagger M_0$:
    \[
      M_0^\dagger M_0 = M_0^2 = M_0 = \begin{pmatrix}
        1 & 0 & 0 & 0 \\
        0 & 1 & 0 & 0 \\
        0 & 0 & 0 & 0 \\
        0 & 0 & 0 & 0
      \end{pmatrix}
    \]

    Los valores propios de $M_0^\dagger M_0$ son los cuadrados de los valores singulares:
    \[
      \sigma_1^2 = 1, \quad \sigma_2^2 = 1, \quad \sigma_3^2 = 0, \quad \sigma_4^2 = 0
    \]

    Por lo tanto, los valores singulares son:
    \[
      \sigma_1 = 1, \quad \sigma_2 = 1, \quad \sigma_3 = 0, \quad \sigma_4 = 0
    \]

    Los vectores propios de $M_0^\dagger M_0$ forman la matriz $V$:
    \[
      V = \begin{pmatrix}
        1 & 0 & 0 & 0 \\
        0 & 1 & 0 & 0 \\
        0 & 0 & 1 & 0 \\
        0 & 0 & 0 & 1
      \end{pmatrix}
    \]

    Para $M_0 M_0^\dagger = M_0$, los vectores propios forman la matriz $U = V$.

    La descomposición SVD es:
    \[
      M_0 = U \Sigma V^\dagger = \begin{pmatrix}
        1 & 0 & 0 & 0 \\
        0 & 1 & 0 & 0 \\
        0 & 0 & 1 & 0 \\
        0 & 0 & 0 & 1
      \end{pmatrix} \begin{pmatrix}
        1 & 0 & 0 & 0 \\
        0 & 1 & 0 & 0 \\
        0 & 0 & 0 & 0 \\
        0 & 0 & 0 & 0
      \end{pmatrix} \begin{pmatrix}
        1 & 0 & 0 & 0 \\
        0 & 1 & 0 & 0 \\
        0 & 0 & 1 & 0 \\
        0 & 0 & 0 & 1
      \end{pmatrix}
    \]

    \textbf{Interpretación:} Los dos primeros valores singulares no nulos corresponden a las direcciones $\ket{00}$ y $\ket{01}$ que se preservan bajo la proyección.
  \end{solution}

  \subsection{Parte 9: Producto tensorial (10 min)}

  \begin{solution}
    \textbf{a) Construcción del espacio producto:}

    Verifica que $\mathbb{C}^4 \cong \mathbb{C}^2 \otimes \mathbb{C}^2$ construyendo explícitamente el isomorfismo.

    \textbf{Solución:}

    El isomorfismo natural mapea:
    \[
      \ket{i} \otimes \ket{j} \mapsto \ket{ij}
    \]

    donde $\ket{i}, \ket{j} \in \{\ket{0}, \ket{1}\}$ son las bases de $\mathbb{C}^2$.

    Explícitamente:
    \begin{align*}
      \ket{0} \otimes \ket{0} & = \begin{pmatrix} 1 \\ 0 \end{pmatrix} \otimes \begin{pmatrix} 1 \\ 0 \end{pmatrix} = \begin{pmatrix} 1 \\ 0 \\ 0 \\ 0 \end{pmatrix} = \ket{00} \\
      \ket{0} \otimes \ket{1} & = \begin{pmatrix} 1 \\ 0 \end{pmatrix} \otimes \begin{pmatrix} 0 \\ 1 \end{pmatrix} = \begin{pmatrix} 0 \\ 1 \\ 0 \\ 0 \end{pmatrix} = \ket{01} \\
      \ket{1} \otimes \ket{0} & = \begin{pmatrix} 0 \\ 1 \end{pmatrix} \otimes \begin{pmatrix} 1 \\ 0 \end{pmatrix} = \begin{pmatrix} 0 \\ 0 \\ 1 \\ 0 \end{pmatrix} = \ket{10} \\
      \ket{1} \otimes \ket{1} & = \begin{pmatrix} 0 \\ 1 \end{pmatrix} \otimes \begin{pmatrix} 0 \\ 1 \end{pmatrix} = \begin{pmatrix} 0 \\ 0 \\ 0 \\ 1 \end{pmatrix} = \ket{11}
    \end{align*}

    Este mapeo es lineal, biyectivo y preserva el producto interno, por lo que es un isomorfismo de espacios de Hilbert.
  \end{solution}

  \begin{solution}
    \textbf{b) Estados separables vs. entrelazados:}

    Determina cuáles de los siguientes estados son separables (pueden escribirse como $\ket{\psi} = \ket{\phi} \otimes \ket{\chi}$) y cuáles están entrelazados:

    \begin{enumerate}
      \item $\ket{\psi_1} = \frac{1}{2}(\ket{00} + \ket{01} + \ket{10} + \ket{11})$
      \item $\ket{\psi_2} = \frac{1}{\sqrt{2}}(\ket{00} + \ket{11})$
      \item $\ket{\psi_3} = \frac{1}{\sqrt{2}}(\ket{01} + \ket{10})$
    \end{enumerate}

    \textbf{Solución:}

    \textbf{Estado 1:}
    \[
      \ket{\psi_1} = \frac{1}{2}(\ket{00} + \ket{01} + \ket{10} + \ket{11}) = \frac{1}{\sqrt{2}}(\ket{0} + \ket{1}) \otimes \frac{1}{\sqrt{2}}(\ket{0} + \ket{1})
    \]

    Este estado es \textbf{separable}. Cada cúbit está en el estado $\ket{+} = \frac{1}{\sqrt{2}}(\ket{0} + \ket{1})$.

    \textbf{Estado 2:}
    Para que $\ket{\psi_2}$ sea separable, debería existir:
    \[
      \ket{\psi_2} = (\alpha\ket{0} + \beta\ket{1}) \otimes (\gamma\ket{0} + \delta\ket{1}) = \alpha\gamma\ket{00} + \alpha\delta\ket{01} + \beta\gamma\ket{10} + \beta\delta\ket{11}
    \]

    Comparando con $\ket{\psi_2} = \frac{1}{\sqrt{2}}(\ket{00} + \ket{11})$:
    \[
      \alpha\gamma = \frac{1}{\sqrt{2}}, \quad \alpha\delta = 0, \quad \beta\gamma = 0, \quad \beta\delta = \frac{1}{\sqrt{2}}
    \]

    De $\alpha\delta = 0$, tenemos $\alpha = 0$ o $\delta = 0$. Si $\alpha = 0$, entonces $\alpha\gamma = 0 \neq \frac{1}{\sqrt{2}}$. Si $\delta = 0$, entonces $\beta\delta = 0 \neq \frac{1}{\sqrt{2}}$.

    Este estado es \textbf{entrelazado}. Es el estado de Bell $\ket{\Phi^+}$.

    \textbf{Estado 3:}
    Similarmente, para $\ket{\psi_3} = \frac{1}{\sqrt{2}}(\ket{01} + \ket{10})$:
    \[
      \alpha\gamma = 0, \quad \alpha\delta = \frac{1}{\sqrt{2}}, \quad \beta\gamma = \frac{1}{\sqrt{2}}, \quad \beta\delta = 0
    \]

    De $\alpha\gamma = 0$, tenemos $\alpha = 0$ o $\gamma = 0$. Si $\alpha = 0$, entonces $\alpha\delta = 0 \neq \frac{1}{\sqrt{2}}$. Si $\gamma = 0$, entonces $\beta\gamma = 0 \neq \frac{1}{\sqrt{2}}$.

    Este estado es \textbf{entrelazado}. Es el estado de Bell $\ket{\Psi^+}$.
  \end{solution}

  \subsection{Parte 10: Espacio dual y producto externo (15 min)}

  \begin{solution}
    \textbf{a) Isomorfismo de Riesz:}

    Para el vector $\ket{\psi} = \frac{1}{\sqrt{2}}(\ket{00} + \ket{11})$, encuentra el funcional lineal correspondiente $\bra{\psi} \in (\mathbb{C}^4)^*$ mediante el isomorfismo de Riesz.

    \textbf{Solución:}

    El isomorfismo de Riesz asocia a cada vector $\ket{\psi}$ el funcional lineal:
    \[
      \bra{\psi}: \mathbb{C}^4 \to \mathbb{C}, \quad \ket{\phi} \mapsto \braket{\psi|\phi}
    \]

    En coordenadas, si:
    \[
      \ket{\psi} = \frac{1}{\sqrt{2}}\begin{pmatrix} 1 \\ 0 \\ 0 \\ 1 \end{pmatrix}
    \]

    entonces:
    \[
      \bra{\psi} = \frac{1}{\sqrt{2}}\begin{pmatrix} 1 & 0 & 0 & 1 \end{pmatrix}
    \]

    Para cualquier $\ket{\phi} = \begin{pmatrix} a \\ b \\ c \\ d \end{pmatrix}$:
    \[
      \braket{\psi|\phi} = \frac{1}{\sqrt{2}}(1, 0, 0, 1)\begin{pmatrix} a \\ b \\ c \\ d \end{pmatrix} = \frac{1}{\sqrt{2}}(a + d)
    \]

    Este funcional es antilineal en el sentido de que $\bra{\psi}(\alpha\ket{\phi}) = \bar{\alpha}\braket{\psi|\phi}$.
  \end{solution}

  \begin{solution}
    \textbf{b) Producto externo:}

    Calcula el operador $\ket{\psi}\bra{\phi}$ donde:
    \[
      \ket{\psi} = \frac{1}{\sqrt{2}}(\ket{00} + \ket{11}), \quad \ket{\phi} = \frac{1}{\sqrt{2}}(\ket{01} + \ket{10})
    \]

    \textbf{Solución:}

    El producto externo es:
    \[
      \ket{\psi}\bra{\phi} = \frac{1}{2}\begin{pmatrix} 1 \\ 0 \\ 0 \\ 1 \end{pmatrix}\begin{pmatrix} 0 & 1 & 1 & 0 \end{pmatrix} = \frac{1}{2}\begin{pmatrix}
        0 & 1 & 1 & 0 \\
        0 & 0 & 0 & 0 \\
        0 & 0 & 0 & 0 \\
        0 & 1 & 1 & 0
      \end{pmatrix}
    \]

    Este operador mapea:
    \begin{align*}
      \ket{\psi}\bra{\phi}(\ket{00}) & = 0                                                                \\
      \ket{\psi}\bra{\phi}(\ket{01}) & = \frac{1}{2}\ket{\psi} = \frac{1}{2\sqrt{2}}(\ket{00} + \ket{11}) \\
      \ket{\psi}\bra{\phi}(\ket{10}) & = \frac{1}{2}\ket{\psi} = \frac{1}{2\sqrt{2}}(\ket{00} + \ket{11}) \\
      \ket{\psi}\bra{\phi}(\ket{11}) & = 0
    \end{align*}

    Verificamos:
    \[
      \ket{\psi}\bra{\phi}(\ket{\chi}) = \ket{\psi} \braket{\phi|\chi}
    \]

    Por ejemplo, para $\ket{\chi} = \ket{01}$:
    \[
      \braket{\phi|01} = \frac{1}{\sqrt{2}}(0, 1, 1, 0)\begin{pmatrix} 0 \\ 1 \\ 0 \\ 0 \end{pmatrix} = \frac{1}{\sqrt{2}}
    \]

    \[
      \ket{\psi}\braket{\phi|01} = \frac{1}{\sqrt{2}}\ket{\psi} = \frac{1}{2}(\ket{00} + \ket{11}) = \ket{\psi}\bra{\phi}(\ket{01})
    \]
  \end{solution}

  \begin{solution}
    \textbf{c) Propiedades del producto externo:}

    Verifica las siguientes propiedades:
    \begin{enumerate}
      \item $(\ket{\psi}\bra{\phi})^\dagger = \ket{\phi}\bra{\psi}$
      \item $(\ket{\psi}\bra{\phi})(\ket{\chi}\bra{\omega}) = \braket{\phi|\chi}\ket{\psi}\bra{\omega}$
      \item $\text{Tr}(\ket{\psi}\bra{\phi}) = \braket{\phi|\psi}$
    \end{enumerate}

    \textbf{Solución:}

    \textbf{Propiedad 1:} Adjunto del producto externo.
    \[
      (\ket{\psi}\bra{\phi})^\dagger = \left(\begin{pmatrix} 1 \\ 0 \\ 0 \\ 1 \end{pmatrix}\begin{pmatrix} 0 & 1 & 1 & 0 \end{pmatrix}\right)^\dagger = \begin{pmatrix} 0 & 1 & 1 & 0 \end{pmatrix}^\dagger \begin{pmatrix} 1 \\ 0 \\ 0 \\ 1 \end{pmatrix}^\dagger
    \]

    \[
      = \begin{pmatrix} 0 \\ 1 \\ 1 \\ 0 \end{pmatrix}\begin{pmatrix} 1 & 0 & 0 & 1 \end{pmatrix} = \ket{\phi}\bra{\psi}
    \]

    \textbf{Propiedad 2:} Producto de operadores de rango 1.

    Tomemos $\ket{\chi} = \ket{00}$ y $\ket{\omega} = \ket{11}$:
    \[
      (\ket{\psi}\bra{\phi})(\ket{\chi}\bra{\omega}) = \ket{\psi}\bra{\phi}(\ket{\chi})\bra{\omega} = \ket{\psi}\braket{\phi|\chi}\bra{\omega}
    \]

    Calculamos:
    \[
      \braket{\phi|\chi} = \frac{1}{\sqrt{2}}(0, 1, 1, 0)\begin{pmatrix} 1 \\ 0 \\ 0 \\ 0 \end{pmatrix} = 0
    \]

    Por lo tanto:
    \[
      (\ket{\psi}\bra{\phi})(\ket{\chi}\bra{\omega}) = 0 = 0 \cdot \ket{\psi}\bra{\omega}
    \]

    \textbf{Propiedad 3:} Traza del producto externo.
    \[
      \text{Tr}(\ket{\psi}\bra{\phi}) = \text{Tr}\left(\frac{1}{2}\begin{pmatrix}
        0 & 1 & 1 & 0 \\
        0 & 0 & 0 & 0 \\
        0 & 0 & 0 & 0 \\
        0 & 1 & 1 & 0
      \end{pmatrix}\right) = \frac{1}{2}(0 + 0 + 0 + 0) = 0
    \]

    Verificamos:
    \[
      \braket{\phi|\psi} = \frac{1}{2}(0, 1, 1, 0)\begin{pmatrix} 1 \\ 0 \\ 0 \\ 1 \end{pmatrix} = 0
    \]
  \end{solution}

  \subsection{Parte 11: Descomposición como suma de proyectores (10 min)}

  \begin{solution}
    \textbf{a) Matriz densidad como suma de proyectores:}

    La matriz densidad de un estado puro $\ket{\psi}$ es $\rho = \ket{\psi}\bra{\psi}$. Para el estado de Bell:
    \[
      \ket{\Phi^+} = \frac{1}{\sqrt{2}}(\ket{00} + \ket{11})
    \]

    Expresa $\rho = \ket{\Phi^+}\bra{\Phi^+}$ como suma de proyectores sobre los estados de la base computacional.

    \textbf{Solución:}

    Calculamos:
    \[
      \rho = \ket{\Phi^+}\bra{\Phi^+} = \frac{1}{2}(\ket{00} + \ket{11})(\bra{00} + \bra{11})
    \]

    \[
      = \frac{1}{2}(\ket{00}\bra{00} + \ket{00}\bra{11} + \ket{11}\bra{00} + \ket{11}\bra{11})
    \]

    En forma matricial:
    \[
      \rho = \frac{1}{2}\begin{pmatrix}
        1 & 0 & 0 & 1 \\
        0 & 0 & 0 & 0 \\
        0 & 0 & 0 & 0 \\
        1 & 0 & 0 & 1
      \end{pmatrix}
    \]

    Esta no es una suma de proyectores ortogonales (que requeriría $P_i P_j = 0$ para $i \neq j$), sino la descomposición en productos externos:
    \[
      \rho = \frac{1}{2}P_{00} + \frac{1}{2}\ket{00}\bra{11} + \frac{1}{2}\ket{11}\bra{00} + \frac{1}{2}P_{11}
    \]

    donde $P_{ij} = \ket{ij}\bra{ij}$.
  \end{solution}

  \begin{solution}
    \textbf{b) Operador general como suma de proyectores:}

    Considera el operador:
    \[
      A = \begin{pmatrix}
        2 & 0 & 0 & 0  \\
        0 & 1 & 0 & 0  \\
        0 & 0 & 1 & 0  \\
        0 & 0 & 0 & -1
      \end{pmatrix}
    \]

    Exprésalo como suma de proyectores sobre sus subespacios propios.

    \textbf{Solución:}

    Los valores propios son:
    \[
      \lambda_1 = 2, \quad \lambda_2 = 1, \quad \lambda_3 = -1
    \]

    Los vectores propios correspondientes son:
    \begin{align*}
      \ket{e_1} & = \ket{00} \text{ para } \lambda_1 = 2  \\
      \ket{e_2} & = \ket{01} \text{ para } \lambda_2 = 1  \\
      \ket{e_3} & = \ket{10} \text{ para } \lambda_2 = 1  \\
      \ket{e_4} & = \ket{11} \text{ para } \lambda_3 = -1
    \end{align*}

    La descomposición espectral es:
    \[
      A = 2P_1 + 1P_2 + (-1)P_3
    \]

    donde:
    \begin{align*}
      P_1 & = \ket{00}\bra{00}                    \\
      P_2 & = \ket{01}\bra{01} + \ket{10}\bra{10} \\
      P_3 & = \ket{11}\bra{11}
    \end{align*}

    Explícitamente:
    \[
      A = 2\begin{pmatrix}
        1 & 0 & 0 & 0 \\
        0 & 0 & 0 & 0 \\
        0 & 0 & 0 & 0 \\
        0 & 0 & 0 & 0
      \end{pmatrix} + \begin{pmatrix}
        0 & 0 & 0 & 0 \\
        0 & 1 & 0 & 0 \\
        0 & 0 & 1 & 0 \\
        0 & 0 & 0 & 0
      \end{pmatrix} - \begin{pmatrix}
        0 & 0 & 0 & 0 \\
        0 & 0 & 0 & 0 \\
        0 & 0 & 0 & 0 \\
        0 & 0 & 0 & 1
      \end{pmatrix}
    \]
  \end{solution}

  \section{Conclusiones y conexión con computación cuántica}

  \begin{resaltado}
    Este taller ha integrado los conceptos fundamentales del álgebra lineal en espacios de Hilbert aplicados a sistemas cuánticos de dos cúbits. Hemos visto cómo:

    \begin{itemize}
      \item Las diferentes bases (computacional, Bell) proporcionan distintas perspectivas del mismo espacio de estados.

      \item Los subespacios ortogonales permiten clasificar estados según simetría (estados simétricos vs. antisimétricos).

      \item Los operadores hermitianos representan observables físicos con valores propios reales.

      \item Los operadores unitarios representan evoluciones cuánticas reversibles.

      \item El entrelazamiento cuántico se caracteriza mediante el producto tensorial y estados no separables.

      \item Las descomposiciones espectrales y en valores singulares revelan la estructura de operadores cuánticos.
    \end{itemize}

    Estos conceptos son fundamentales para:
    \begin{itemize}
      \item Algoritmos cuánticos (teleportación, superdensa codificación).
      \item Corrección de errores cuánticos.
      \item Teoría de la información cuántica.
      \item Simulación de sistemas cuánticos.
    \end{itemize}
  \end{resaltado}

\end{questions}
\end{document}