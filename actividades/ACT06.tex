\actividad{3}{matriz de densidad}
\maketitulo

\section{Objetivos}
Con esta actividad reforzaremos el concepto de matriz de densidad aplicada a sistemas cuánticos y su relación con los estados de Bell. Analizaremos cómo estos estados se representan y cómo se pueden utilizar en la computación cuántica.

\section{Ejercicios}

\begin{questions}
  \question Reformula los cuatros postulados de la mecánica cuántica en términos del operador densidad.

  \question Formula el teorema de la libertad unitaria en el conjunto para matrices de densidad y demuéstralo (Schrödinger).

  \question ¿Qué se puede afirmar si tienes dos estados normalizados tales que sus operadores densidad son iguales? Demuestra tu afirmación.

  \question Considera los siguientes conjuntos de estados mezcla, ordenados como pares estado y probabilidad asociada:
  \begin{align*}
    A & = \left\{\left(\frac{3\ket{0}+\ket{1}}{\sqrt{10}}, \frac{2}{5}\right),\left(\frac{2\ket{0}-3i\ket{1}}{\sqrt{13}}, \frac{3}{10}\right),\left(\frac{i\ket{0}+\sqrt{3}\ket{1}}{2}, \frac{3}{10}\right)\right\} \\
    B & = \left\{\left(\ket{0}, \frac{1}{2}\right),\left(\frac{\ket{0}+\ket{1}}{\sqrt{2}}, \frac{1}{3}\right),\left(\frac{\ket{0}-i\ket{1}}{\sqrt{2}}, \frac{1}{6}\right)\right\}                                   \\
  \end{align*}

  \begin{parts}
    \part Encuentra la matriz de densidad de cada conjunto y compáralas.
    \part ¿Qué podrías decir sobre los operadores obtenidos? ¿Son iguales o diferentes? Justifica tu respuesta.
  \end{parts}

  \question Encuentra el operador de densidad y la matriz de densidad para los cuatro estados de Bell, así como el operador de densidad reducida para cada cúbit.

  \question Demuestra que la matriz de densidad es un operador hermitiano. En caso de que no lo sea, proporciona un contraejemplo o menciona las condiciones en las que se cumple esta propiedad.

\end{questions}

\subsection*{Extensión y formato}

\begin{itemize}
  \item Longitud estimada: de 8 a 10 páginas.
  \item	Formato obligatorio: estructura formal que siga la plantilla LATEX que se suministrará (introducción, desarrollo, resultados, discusión, conclusiones, bibliografía). Este formato es muy similar a una publicación científica y al formato que usaremos en el Trabajo de Fin de Máster (TFM).
  \item	Se debe dar una introducción del operador densidad, mencionar las propiedades y su importancia.
  \item	Debe tener el mismo orden en el que se dieron las preguntas.
  \item	En cuanto al teorema de la libertad unitaria, se debe mencionar la referencia en la que te has guiado para su demostración y cada paso debe estar justificado, lo mismo para el punto 3.
  \item	Los cálculos deben contener todas las operaciones para encontrar las matrices de densidad y los operadores de densidad reducidos.

  \item	Penalización: la utilización de plantillas diferentes, la no utilización de \LaTeX u otros defectos generalizados, así como no seguir el formato obligatorio, supondrá que la memoria no se corregirá y, por tanto, se considerará como una calificación de cero puntos.

\end{itemize}

\subsection*{Criterios de evaluación}

\begin{rubrica}
  \setrubrica{Exactitud del cálculo}{Los resultados de los cálculos son correctos y se han realizado los pasos adecuados para llegar al resultado}{3}{30}
  \setrubrica{Razonamiento y explicación formal}{Se han razonado los pasos realizados y las explicaciones son coherentes y certeras}{3}{30}
  \setrubrica{Claridad de la exposición}{La exposición se ha realizado de forma clara y secuencial y es cómodamente legible, tanto en la parte matemática como en la de las explicaciones}{2}{20}
  \setrubrica{Bibliografía y citas APA}{Descripción de las librerías y referencias utilizadas para la resolución de la actividad}{2}{20}
\end{rubrica}