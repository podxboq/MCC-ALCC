\actividad{1}{Operaciones}
\maketitulo

\section{Objetivos}
Con esta actividad reforzarás los conceptos aprendidos hasta el momento en el curso y comenzarás a explorar herramientas fundamentales para la computación cuántica.

\section{Pautas de elaboración}

La actividad consiste en familiarizarse con el entorno de Python y con librerías como NumPy. Además, usaremos Qiskit, una de las herramientas clave para la computación cuántica. La actividad se dividirá en dos partes:

\begin{itemize}
  \item Uso de NumPy y Cmath para operaciones en álgebra lineal.

  \item Uso de Qiskit para representar y manipular estados cuánticos.
\end{itemize}

Para ello, realizaremos las siguientes operaciones aprendidas en la primera parte del curso:

\begin{itemize}
  \item Identificar matrices normales, hermitianas y unitarias.
  \item Calcular el producto tensorial entre matrices y vectores.
\end{itemize}

Para el desarrollo de esta actividad necesitar acceso a un entorno de Python con las librerías NumPy y Qiskit instaladas. Algunas opciones son:

\begin{itemize}
  \item \textbf{Google Colab}: Desde el navegador web en la url \url{https://colab.research.google.com/} o como extensión de VSCODE.
  \item \textbf{Jupyter Notebook}: Puedes instalarlo siguiendo las instrucciones de la url \url{https://jupyter.org/install} o ejecutarlo en la nube siguiendo las instrucciones de \url{https://jupyter.org/try}.
\end{itemize}

En ambos casos, has de instalar la librería Qiskit con el comando \texttt{!pip install qiskit} si lo ejecutas desde la nube y \texttt{pip install qiskit} si lo ejecutas en local.

\section{Ejercicios}
Considera las siguientes matrices:

\begin{equation*}
  A=\begin{pmatrix}
    3+2i  & 1-i   & 4-3i \\
    2+i   & -5+4i & 6    \\
    -3+2i & 7-i   & 8+3i
  \end{pmatrix},
  B=\begin{pmatrix}
    2+3i & 4   & 1+2i \\
    3-2i & 5+i & 7-3i \\
    6+4i & -2  & 9+5i
  \end{pmatrix},
\end{equation*}
\begin{equation*}
  C=\begin{pmatrix}
    1+4i  & 2-2i  & 3+i   & -5  \\
    -2+3i & 6     & 4+7i  & 2+i \\
    -1+2i & -3-4i & -6+5i & 3-i \\
    -8i   & 7+4i  & -5-3i & 5i
  \end{pmatrix},
  D=\begin{pmatrix}
    i    & 2    \\
    2-3i & 4+5i
  \end{pmatrix}
\end{equation*}

\begin{questions}
  \question Utiliza la librería de Numpy, junto con los módulos Cmath y math, y obtén:
  \begin{parts}
    \part La forma polar de cada uno de los determinantes y de la traza de cada matriz.
    \part $2i(|D|+\tr(C))AB-(1+i) |C|\tr(D)BA$.
    \part La inversa de cada matriz, si existe.
    \part Además, usando la definición y, al mismo tiempo, las propiedades de los valores propios, clasifica cada una de las matrices en normales, hermitianas o unitarias.
  \end{parts}


  \begin{solution}
    Primero escribimos $z = -8i$ en forma exponencial. Como $|z| = 8$ y $\arg(z) = -\frac{\pi}{2}$ (o $\frac{3\pi}{2}$), tenemos:
    $$z = 8e^{-i\pi/2}$$

    Las raíces cúbicas son:
    $$z_k = \sqrt[3]{8}e^{i\frac{-\pi/2 + 2\pi k}{3}} = 2e^{i\frac{-\pi + 4\pi k}{6}}, \quad k = 0, 1, 2$$

    Calculando cada raíz:
    \begin{align*}
      z_0 & = 2e^{-i\pi/6} = 2\left(\cos\frac{\pi}{6} - i\sin\frac{\pi}{6}\right) = \sqrt{3} - i    \\
      z_1 & = 2e^{i\pi/2} = 2i                                                                      \\
      z_2 & = 2e^{i7\pi/6} = 2\left(\cos\frac{7\pi}{6} + i\sin\frac{7\pi}{6}\right) = -\sqrt{3} - i
    \end{align*}

    Geométricamente, forman un triángulo equilátero inscrito en una circunferencia de radio 2 centrada en el origen.
  \end{solution}

  \question Estos resultados deben quedar en formato \LaTeX. Utiliza NumPy y Qiskit y obtén:
  \begin{parts}
    \part Crear los vectores estados $\C^2$:  $\ket{0}, \ket{1}$.
    \part A partir de los vectores estado creados, halla los estados $\ket{+}, \ket{-}, \ket{i+}, \ket{i-}$ y los estados de Bell.
    \part A partir de los vectores estado creados, encuentra la base computacional de $\C^4$, es decir, por ejemplo, $\ket{10}$.
    \part A partir de los vectores estados, halla los vectores $\frac{1}{\sqrt{2}}\ket{000}+\frac{1}{\sqrt{2}}\ket{111}$, $\frac{i}{\sqrt{2}}\ket{000}-\frac{1}{\sqrt{2}}\ket{111}$, $\frac{1}{\sqrt{2}}\ket{000}+\frac{i}{\sqrt{2}}\ket{111}$.
  \end{parts}
\end{questions}

\subsection*{Extensión y formato}

Calibri 12, interlineado 1,5.

\subsection*{Criterios de evaluación}

\begin{rubrica}
  \setrubrica{Exactitud del cálculo}{Los resultados de los cálculos son correctos y se han realizado los pasos adecuados para llegar al resultado}{3}{30}
  \setrubrica{Razonamiento y explicación formal}{Se han razonado los pasos realizados y las explicaciones son coherentes y certeras}{3}{30}
  \setrubrica{Claridad de la exposición}{La exposición se ha realizado de forma clara y secuencial y es cómodamente legible, tanto en la parte matemática como en la de las explicaciones}{2}{20}
  \setrubrica{Bibliografía y citas APA}{Descripción de las librerías y referencias utilizadas para la resolución de la actividad}{2}{20}
\end{rubrica}