\actividad{3}{Análisis de decoherencia en un cúbit}
\maketitulo

\section{Objetivos y pautas de elaboración}

El principal objetivo de esta actividad es aplicar integralmente la teoría de matrices de densidad al análisis de un problema real: la \textbf{decoherencia cuántica}, que es el principal obstáculo en la construcción de computadores cuánticos prácticos.

A través de un caso de estudio completo, los estudiantes deberán:
\begin{itemize}
  \item Modelar matemáticamente la degradación de un estado cuántico.
  \item Cuantificar la pérdida de información cuántica usando múltiples métricas.
  \item Interpretar físicamente los resultados en el contexto de la computación cuántica.
  \item Analizar la viabilidad temporal de operaciones cuánticas.
\end{itemize}

Este ejercicio integrador requiere el uso de todos los conceptos del tema de estados probabilísticos.

\section{Problema: Decoherencia de fase en un cúbit}

\subsection{Contexto}

Un laboratorio de computación cuántica ha preparado un cúbit en el estado de superposición:
$$\ket{\psi_0} = \frac{1}{\sqrt{2}}(\ket{0} + \ket{1})$$

Este cúbit forma parte de un procesador cuántico y será utilizado en un algoritmo que requiere mantener la coherencia cuántica durante al menos 100 microsegundos ($\mu s$).

Sin embargo, el cúbit no está perfectamente aislado del entorno. La interacción incontrolada con campos electromagnéticos externos, vibraciones térmicas y otros factores ambientales provocan un proceso de \textbf{decoherencia de fase} (dephasing).

Este proceso se modela matemáticamente mediante la evolución:
$$\rho(t) = (1-p(t))\rho_{\text{puro}} + p(t)\rho_{\text{mixto}}$$

donde:
\begin{itemize}
  \item $\rho_{\text{puro}} = \ketbra{\psi_0}{\psi_0}$ es la matriz de densidad del estado inicial puro.
  \item $\rho_{\text{mixto}} = \frac{1}{2}\ketbra{0}{0} + \frac{1}{2}\ketbra{1}{1}$ es el estado completamente mixto.
  \item $p(t) = 1 - e^{-t/T_2}$ es la probabilidad de decoherencia.
  \item $T_2 = 150\ \mu s$ es el \textbf{tiempo de coherencia} característico del cúbit.
\end{itemize}

\subsection{Apartados}

\begin{questions}
  \question \textbf{Modelado inicial}

  Escribir explícitamente en forma matricial $4 \times 4$ la matriz de densidad del estado inicial puro $\rho_{\text{puro}} = \ketbra{\psi_0}{\psi_0}$ en la base computacional $\{\ket{0}, \ket{1}\}$.

  \begin{solution}
    El estado inicial es:
    $$\ket{\psi_0} = \frac{1}{\sqrt{2}}(\ket{0} + \ket{1}) = \frac{1}{\sqrt{2}}\begin{pmatrix} 1 \\ 1 \end{pmatrix}$$

    El bra correspondiente:
    $$\bra{\psi_0} = \frac{1}{\sqrt{2}}\begin{pmatrix} 1 & 1 \end{pmatrix}$$

    La matriz de densidad:
    $$\rho_{\text{puro}} = \ketbra{\psi_0}{\psi_0} = \frac{1}{2}\begin{pmatrix} 1 \\ 1 \end{pmatrix}\begin{pmatrix} 1 & 1 \end{pmatrix} = \frac{1}{2}\begin{pmatrix} 1 & 1 \\ 1 & 1 \end{pmatrix}$$

    \textbf{Verificación:}
    \begin{itemize}
      \item Hermitiana: $\rho^\dagger = \rho$ ✓
      \item Traza unitaria: $\text{Tr}(\rho) = \frac{1}{2}(1 + 1) = 1$ ✓
      \item Pureza: $\text{Tr}(\rho^2) = \text{Tr}\left(\frac{1}{4}\begin{pmatrix} 2 & 2 \\ 2 & 2 \end{pmatrix}\right) = 1$ ✓ (estado puro)
    \end{itemize}
  \end{solution}

  \question \textbf{Evolución temporal}

  Escribir la expresión matricial explícita de $\rho(t)$ sustituyendo $\rho_{\text{puro}}$, $\rho_{\text{mixto}}$ y $p(t)$.

  \begin{solution}
    El estado completamente mixto es:
    $$\rho_{\text{mixto}} = \frac{1}{2}\ketbra{0}{0} + \frac{1}{2}\ketbra{1}{1} = \frac{1}{2}\begin{pmatrix} 1 & 0 \\ 0 & 1 \end{pmatrix} = \frac{I}{2}$$

    Sustituyendo en la fórmula de evolución con $p(t) = 1 - e^{-t/T_2}$:
    \begin{align*}
      \rho(t) & = (1-p(t))\rho_{\text{puro}} + p(t)\rho_{\text{mixto}}                                                                                                                        \\
              & = e^{-t/T_2} \cdot \frac{1}{2}\begin{pmatrix} 1 & 1 \\ 1 & 1 \end{pmatrix} + (1-e^{-t/T_2}) \cdot \frac{1}{2}\begin{pmatrix} 1 & 0 \\ 0 & 1 \end{pmatrix}                     \\
              & = \frac{1}{2}\begin{pmatrix} e^{-t/T_2} & e^{-t/T_2} \\ e^{-t/T_2} & e^{-t/T_2} \end{pmatrix} + \frac{1}{2}\begin{pmatrix} 1-e^{-t/T_2} & 0 \\ 0 & 1-e^{-t/T_2} \end{pmatrix} \\
              & = \frac{1}{2}\begin{pmatrix} 1 & e^{-t/T_2} \\ e^{-t/T_2} & 1 \end{pmatrix}
    \end{align*}

    \textbf{Observación:} Los elementos diagonales permanecen constantes en $1/2$ (las poblaciones no cambian), pero los elementos fuera de la diagonal decaen exponencialmente (pérdida de coherencia).
  \end{solution}

  \question \textbf{Criterio de pureza}

  Calcular $\text{Tr}(\rho^2(t))$ como función del tiempo y analizar:
  \begin{parts}
    \part ¿En qué instante el estado deja de ser puro?
    \part ¿Cuándo alcanza el 50\% de pureza?
    \part ¿Hacia qué valor tiende cuando $t \to \infty$?
  \end{parts}

  \begin{solution}
    Calculamos $\rho^2(t)$:
    $$\rho^2(t) = \frac{1}{4}\begin{pmatrix} 1 & e^{-t/T_2} \\ e^{-t/T_2} & 1 \end{pmatrix}\begin{pmatrix} 1 & e^{-t/T_2} \\ e^{-t/T_2} & 1 \end{pmatrix}$$
    $$= \frac{1}{4}\begin{pmatrix} 1 + e^{-2t/T_2} & 2e^{-t/T_2} \\ 2e^{-t/T_2} & 1 + e^{-2t/T_2} \end{pmatrix}$$

    La traza:
    $$\text{Tr}(\rho^2(t)) = \frac{1}{4}\left(2(1 + e^{-2t/T_2})\right) = \frac{1 + e^{-2t/T_2}}{2}$$

    \textbf{Apartado (a):} El estado es puro cuando $\text{Tr}(\rho^2) = 1$:
    $$\frac{1 + e^{-2t/T_2}}{2} = 1 \Rightarrow e^{-2t/T_2} = 1 \Rightarrow t = 0$$

    Para cualquier $t > 0$, el estado es mixto. La decoherencia comienza instantáneamente.

    \textbf{Apartado (b):} Pureza del 50\%:
    $$\frac{1 + e^{-2t/T_2}}{2} = 0.5 \Rightarrow 1 + e^{-2t/T_2} = 1 \Rightarrow e^{-2t/T_2} = 0 \Rightarrow t \to \infty$$

    Mejor reformular: ¿cuándo es $\text{Tr}(\rho^2) = 0.75$ (punto medio entre 1 y 0.5)?
    $$\frac{1 + e^{-2t/T_2}}{2} = 0.75 \Rightarrow e^{-2t/T_2} = 0.5 \Rightarrow t = \frac{T_2 \ln 2}{2} \approx 0.347 T_2 = 52\ \mu s$$

    \textbf{Apartado (c):} Límite asintótico:
    $$\lim_{t \to \infty} \text{Tr}(\rho^2(t)) = \frac{1 + 0}{2} = \frac{1}{2}$$

    Esto corresponde al estado completamente mixto (máxima mezcla estadística).
  \end{solution}

  \question \textbf{Entropía de von Neumann}

  Calcular la entropía $S(\rho(t)) = -\text{Tr}(\rho \log_2 \rho)$ como función del tiempo.

  \textit{Pista: Primero encuentre los valores propios de $\rho(t)$.}

  \begin{solution}
    El polinomio característico de $\rho(t)$:
    $$\det(\rho(t) - \lambda I) = \det\begin{pmatrix} \frac{1}{2}-\lambda & \frac{e^{-t/T_2}}{2} \\ \frac{e^{-t/T_2}}{2} & \frac{1}{2}-\lambda \end{pmatrix}$$
    $$= \left(\frac{1}{2}-\lambda\right)^2 - \frac{e^{-2t/T_2}}{4} = \lambda^2 - \lambda + \frac{1 - e^{-2t/T_2}}{4}$$

    Resolviendo:
    $$\lambda = \frac{1 \pm \sqrt{1 - 4 \cdot \frac{1-e^{-2t/T_2}}{4}}}{2} = \frac{1 \pm \sqrt{e^{-2t/T_2}}}{2} = \frac{1 \pm e^{-t/T_2}}{2}$$

    Valores propios:
    $$\lambda_1(t) = \frac{1 + e^{-t/T_2}}{2}, \quad \lambda_2(t) = \frac{1 - e^{-t/T_2}}{2}$$

    Entropía de von Neumann:
    $$S(\rho(t)) = -\lambda_1 \log_2 \lambda_1 - \lambda_2 \log_2 \lambda_2$$
    $$= -\frac{1 + e^{-t/T_2}}{2}\log_2\frac{1 + e^{-t/T_2}}{2} - \frac{1 - e^{-t/T_2}}{2}\log_2\frac{1 - e^{-t/T_2}}{2}$$

    \textbf{Casos límite:}
    \begin{itemize}
      \item $t = 0$: $\lambda_1 = 1, \lambda_2 = 0 \Rightarrow S = 0$ (estado puro, información máxima).
      \item $t \to \infty$: $\lambda_1 = \lambda_2 = 1/2 \Rightarrow S = 1$ bit (máxima incertidumbre).
    \end{itemize}
  \end{solution}

  \question \textbf{Fidelidad cuántica}

  Calcular la fidelidad $F(t) = \text{Tr}(\sqrt{\sqrt{\rho(0)}\rho(t)\sqrt{\rho(0)}})$ entre el estado inicial y el estado en tiempo $t$.

  Para estados de cúbit, se puede usar la fórmula simplificada:
  $$F(\rho, \sigma) = \text{Tr}(\sqrt{\rho}\sqrt{\sigma})^2$$

  \begin{solution}
    Para el caso particular de nuestro estado, donde $\rho(0) = \rho_{\text{puro}}$ es un proyector ($\rho^2 = \rho$), la fórmula se simplifica:

    $$F(t) = \text{Tr}(\rho(0)\rho(t))$$

    Calculamos:
    $$F(t) = \text{Tr}\left(\frac{1}{2}\begin{pmatrix} 1 & 1 \\ 1 & 1 \end{pmatrix} \cdot \frac{1}{2}\begin{pmatrix} 1 & e^{-t/T_2} \\ e^{-t/T_2} & 1 \end{pmatrix}\right)$$
    $$= \frac{1}{4}\text{Tr}\begin{pmatrix} 1+e^{-t/T_2} & 1+e^{-t/T_2} \\ 1+e^{-t/T_2} & 1+e^{-t/T_2} \end{pmatrix} = \frac{1}{4} \cdot 2(1+e^{-t/T_2}) = \frac{1+e^{-t/T_2}}{2}$$

    \textbf{Interpretación:}
    \begin{itemize}
      \item $F(0) = 1$: fidelidad perfecta (estados idénticos).
      \item $F(\infty) = 1/2$: fidelidad mínima para un cúbit.
      \item $F(T_2) = \frac{1+e^{-1}}{2} \approx 0.684$: la fidelidad cae al 68.4\% en un tiempo característico.
    \end{itemize}
  \end{solution}

  \question \textbf{Distancia de traza}

  Calcular la distancia de traza $D(t) = \frac{1}{2}\text{Tr}|\rho(0) - \rho(t)|$ entre el estado inicial y el estado en tiempo $t$.

  \begin{solution}
    Calculamos la diferencia:
    $$\rho(0) - \rho(t) = \frac{1}{2}\begin{pmatrix} 1 & 1 \\ 1 & 1 \end{pmatrix} - \frac{1}{2}\begin{pmatrix} 1 & e^{-t/T_2} \\ e^{-t/T_2} & 1 \end{pmatrix}$$
    $$= \frac{1}{2}\begin{pmatrix} 0 & 1-e^{-t/T_2} \\ 1-e^{-t/T_2} & 0 \end{pmatrix}$$

    Esta matriz es hermitiana, así que sus valores propios son reales. Calculamos:
    $$\det\begin{pmatrix} -\lambda & \frac{1-e^{-t/T_2}}{2} \\ \frac{1-e^{-t/T_2}}{2} & -\lambda \end{pmatrix} = \lambda^2 - \frac{(1-e^{-t/T_2})^2}{4} = 0$$
    $$\lambda = \pm \frac{1-e^{-t/T_2}}{2}$$

    La traza del valor absoluto (suma de valores propios absolutos):
    $$\text{Tr}|\rho(0) - \rho(t)| = \frac{1-e^{-t/T_2}}{2} + \frac{1-e^{-t/T_2}}{2} = 1-e^{-t/T_2}$$

    Distancia de traza:
    $$D(t) = \frac{1}{2}(1-e^{-t/T_2}) = \frac{p(t)}{2}$$

    \textbf{Casos límite:}
    \begin{itemize}
      \item $D(0) = 0$: sin distancia (estados idénticos).
      \item $D(\infty) = 1/2$: máxima distancia distinguible para cúbits.
      \item $D(T_2) = \frac{1-e^{-1}}{2} \approx 0.316$.
    \end{itemize}
  \end{solution}

  \question \textbf{Vector de Bloch}

  El estado de un cúbit puede representarse mediante el vector de Bloch $\vec{r} = (r_x, r_y, r_z)$ donde:
  $$r_i = \text{Tr}(\rho \sigma_i), \quad i = x, y, z$$

  Calcular $\vec{r}(t)$ y describir geométricamente la trayectoria del estado en la esfera de Bloch.

  \begin{solution}
    Calculamos cada componente:

    \textbf{Componente $r_x$:}
    $$r_x(t) = \text{Tr}(\rho(t)\sigma_x) = \text{Tr}\left(\frac{1}{2}\begin{pmatrix} 1 & e^{-t/T_2} \\ e^{-t/T_2} & 1 \end{pmatrix}\begin{pmatrix} 0 & 1 \\ 1 & 0 \end{pmatrix}\right)$$
    $$= \text{Tr}\left(\frac{1}{2}\begin{pmatrix} e^{-t/T_2} & 1 \\ 1 & e^{-t/T_2} \end{pmatrix}\right) = e^{-t/T_2}$$

    \textbf{Componente $r_y$:}
    $$r_y(t) = \text{Tr}(\rho(t)\sigma_y) = \text{Tr}\left(\frac{1}{2}\begin{pmatrix} 1 & e^{-t/T_2} \\ e^{-t/T_2} & 1 \end{pmatrix}\begin{pmatrix} 0 & -i \\ i & 0 \end{pmatrix}\right)$$
    $$= \text{Tr}\left(\frac{1}{2}\begin{pmatrix} ie^{-t/T_2} & -i \\ i & -ie^{-t/T_2} \end{pmatrix}\right) = 0$$

    \textbf{Componente $r_z$:}
    $$r_z(t) = \text{Tr}(\rho(t)\sigma_z) = \text{Tr}\left(\frac{1}{2}\begin{pmatrix} 1 & e^{-t/T_2} \\ e^{-t/T_2} & 1 \end{pmatrix}\begin{pmatrix} 1 & 0 \\ 0 & -1 \end{pmatrix}\right)$$
    $$= \text{Tr}\left(\frac{1}{2}\begin{pmatrix} 1 & 0 \\ 0 & -1 \end{pmatrix}\right) = 0$$

    Vector de Bloch:
    $$\vec{r}(t) = (e^{-t/T_2}, 0, 0)$$

    \textbf{Interpretación geométrica:}
    \begin{itemize}
      \item En $t=0$: $\vec{r} = (1, 0, 0)$, el estado está en la superficie de la esfera (estado puro $\ket{+}$).
      \item Para $t > 0$: el vector se mueve hacia el origen a lo largo del eje $x$.
      \item Cuando $t \to \infty$: $\vec{r} \to (0, 0, 0)$, el estado alcanza el centro (completamente mixto).
      \item La longitud $|\vec{r}(t)| = e^{-t/T_2}$ decae exponencialmente.
    \end{itemize}

    Este movimiento hacia el centro representa la pérdida de pureza del estado cuántico.
  \end{solution}

  \question \textbf{Análisis de viabilidad}

  Dado que el algoritmo cuántico requiere mantener coherencia durante 100 $\mu s$ y el tiempo característico es $T_2 = 150\ \mu s$:

  \begin{parts}
    \part Calcular la fidelidad $F(100\ \mu s)$.
    \part Calcular la pureza $\text{Tr}(\rho^2(100\ \mu s))$.
    \part Si el algoritmo requiere fidelidad mínima del 80\%, ¿es viable ejecutarlo?
    \part ¿Cuál es el tiempo máximo disponible para mantener $F \geq 0.8$?
  \end{parts}

  \begin{solution}
    \textbf{Apartado (a):} Fidelidad en $t = 100\ \mu s$:
    $$F(100) = \frac{1 + e^{-100/150}}{2} = \frac{1 + e^{-2/3}}{2} \approx \frac{1 + 0.5134}{2} \approx 0.757 = 75.7\%$$

    \textbf{Apartado (b):} Pureza en $t = 100\ \mu s$:
    $$\text{Tr}(\rho^2(100)) = \frac{1 + e^{-200/150}}{2} = \frac{1 + e^{-4/3}}{2} \approx \frac{1 + 0.2636}{2} \approx 0.632 = 63.2\%$$

    \textbf{Apartado (c):} Como $F(100\ \mu s) = 75.7\% < 80\%$, el algoritmo \textbf{NO es viable} con los parámetros actuales.

    \textbf{Apartado (d):} Tiempo máximo para $F \geq 0.8$:
    $$\frac{1 + e^{-t/T_2}}{2} \geq 0.8 \Rightarrow e^{-t/150} \geq 0.6$$
    $$-\frac{t}{150} \geq \ln(0.6) \Rightarrow t \leq -150\ln(0.6) \approx 76.7\ \mu s$$

    \textbf{Conclusión:} El algoritmo solo es viable si su ejecución completa dura menos de 77 microsegundos, o si se mejora el tiempo de coherencia $T_2$ del cúbit (por ejemplo, mediante mejor aislamiento, temperaturas más bajas, o corrección de errores cuánticos).

    \textbf{Implicaciones prácticas:}
    \begin{itemize}
      \item Se necesita optimizar el algoritmo para reducir su duración.
      \item Considerar técnicas de corrección de errores cuánticos (quantum error correction).
      \item Mejorar el hardware para aumentar $T_2$ (refrigeración, blindaje electromagnético, etc.).
      \item Implementar pulsos de refocalización (spin echo) para extender el tiempo de coherencia efectivo.
    \end{itemize}
  \end{solution}
\end{questions}

\section{Requisitos de la actividad}

\subsection*{Modalidad de trabajo}
Esta actividad se realizará de manera \textbf{grupal}.  Cada grupo debe indicar claramente los nombres de todos los miembros en la portada. Aunque se trabaje en grupo, cada miembro debe entregar su propia versión de la actividad.

\subsection*{Formato y presentación}
El trabajo deberá cumplir los siguientes requisitos formales:

\begin{itemize}
  \item \textbf{Estructura}: El documento seguirá la estructura de un artículo académico, incluyendo introducción, desarrollo, conclusiones y referencias bibliográficas.
  \item \textbf{Herramientas}: La redacción se realizará íntegramente en \LaTeX.
  \item \textbf{Entrega}: Se entregará un único archivo en formato PDF compilado.
  \item \textbf{Citación}: Todas las referencias bibliográficas seguirán estrictamente la normativa APA 7.ª edición.
\end{itemize}

\subsection*{Criterios de evaluación}
Se valorarán especialmente los siguientes aspectos:

\begin{itemize}
  \item Claridad y rigor en la exposición de los cálculos intermedios.
  \item Justificación matemática y razonamiento lógico de los pasos realizados.
  \item Interpretación crítica de los resultados obtenidos y su contextualización.
  \item Corrección en el uso del lenguaje técnico y la notación matemática.
\end{itemize}

\subsection*{Extensión recomendada}
Se recomienda un trabajo entre 8 y 12 páginas incluyendo cálculos, gráficas (opcionales pero recomendadas) e interpretaciones.

\begin{center}
  \fcolorbox{red}{yellow!20}{%
    \begin{minipage}{0.9\textwidth}
      \textbf{Nota importante:} La rúbrica de evaluación únicamente se aplicará a aquellos trabajos que cumplan todos los requisitos de formato especificados. Los trabajos que no satisfagan estos requisitos podrán ser devueltos sin evaluación.
    \end{minipage}
  }
\end{center}

\section{Rúbrica}

\begin{rubrica}
  \setrubrica{Normativa}{Entregar la actividad en plazo y cumpliendo las indicaciones.}{0,5}{5}
  \setrubrica{Biliografía y citas APA}{Cumplir con las normas de citación y bibliografía según la normativa APA 7.}{0,5}{5}
  \setrubrica{Apartado 1}{Matriz de densidad inicial correcta con verificaciones}{1}{10}
  \setrubrica{Apartado 2}{Evolución temporal explícita y correcta}{1}{10}
  \setrubrica{Apartado 3}{Pureza: cálculo correcto y análisis temporal completo}{1.5}{15}
  \setrubrica{Apartado 4}{Entropía: valores propios y límites correctos}{1.5}{15}
  \setrubrica{Apartado 5}{Fidelidad: cálculo e interpretación}{1}{10}
  \setrubrica{Apartado 6}{Distancia de traza correcta}{1}{10}
  \setrubrica{Apartado 7}{Vector de Bloch y descripción geométrica}{1}{10}
  \setrubrica{Apartado 8}{Análisis de viabilidad completo y propuestas justificadas}{1}{10}
\end{rubrica}