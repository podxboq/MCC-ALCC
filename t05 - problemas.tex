\unirsection{Problemas}

\begin{questions}

  \question Sea $V = \mathbb{C}^2$ con la base $\mathcal{B} = \left\{\begin{pmatrix} 1 \\ i \end{pmatrix}, \begin{pmatrix} 1 \\ -i \end{pmatrix}\right\}$.
  \begin{parts}
    \part Verificar que $\mathcal{B}$ es efectivamente una base de $V$.
    \part Encontrar la base dual $\mathcal{B}^* = \{e_1^*, e_2^*\}$ correspondiente.
    \part Expresar el funcional $f(x,y) = x + 2y$ como combinación lineal de elementos de $\mathcal{B}^*$.
    \part Calcular $e_1^*\begin{pmatrix} 2+i \\ 1-i \end{pmatrix}$ y $e_2^*\begin{pmatrix} 2+i \\ 1-i \end{pmatrix}$.
  \end{parts}

  \question En un espacio de Hilbert $\mathcal{H} = \mathbb{C}^2$ con el producto interno estándar:
  \begin{parts}
    \part Usar el teorema de Riesz-Fréchet para encontrar el vector $y_f \in \mathcal{H}$ que representa el funcional $f(x) = \begin{pmatrix} 1-i & 2+i \end{pmatrix} x$
    \part Verificar que $f(x) = \langle x, y_f \rangle$ para cualquier $x \in \mathcal{H}$.
  \end{parts}

  \question Considerar el operador $T: \mathbb{C}^2 \to \mathbb{C}^2$ definido por la matriz $T = \begin{pmatrix} 1 & i \\ 0 & 2 \end{pmatrix}$.
  \begin{parts}
    \part Encontrar la representación matricial del operador adjunto $T^*$.
    \part Verificar que $\langle Tx, y \rangle = \langle x, T^*y \rangle$ para $x = \begin{pmatrix} 1 \\ i \end{pmatrix}$ y $y = \begin{pmatrix} 1+i \\ 1-i \end{pmatrix}$.
    \part Determinar si $T$ es hermitiano, unitario o normal.
  \end{parts}

  \question Considere el estado cuántico normalizado en el espacio de Hilbert $\mathcal{H} \cong \mathbb{C}^2$:
  $$ |\psi\rangle = \frac{1}{\sqrt{5}} \begin{pmatrix} 1+i \\ 2 \end{pmatrix} $$

  \question Sea el operador lineal $A$ actuando sobre $\mathbb{C}^2$, representado por la matriz:
  $$A = \begin{pmatrix} 0 & -i \\ i & 0 \end{pmatrix}$$

  \begin{parts}
    \part Calcule el operador adjunto $A^\dagger$ (transpuesta conjugada de $A$).
    \part Determine si $A$ es un operador \textbf{hermítico} (autoadjunto). ¿Qué implicación física tiene esta propiedad en la Mecánica Cuántica?
    \part Calcule los valores propios de $A$ y demuestre que son reales.
  \end{parts}

  \question Considere el operador hermítico (observable) $H$ dado por:
  $$H = \begin{pmatrix} 3 & 1 \\ 1 & 3 \end{pmatrix}$$

  \begin{parts}
    \part Encuentre los valores propios $\lambda_1$ y $\lambda_2$.
    \part Determine los vectores propios normalizados $v_1$ y $v_2$ asociados a estos valores propios.
    \part Escriba la matriz diagonal $D$ y la matriz unitaria de cambio de base $U$ (la matriz de los estados propios). Confirme la descomposición espectral: $H = U D U^\dagger$.
  \end{parts}

  \question Considere el sistema compuesto de dos qubits. El espacio de Hilbert es $\mathcal{H}_{AB} = \mathcal{H}_A \otimes \mathcal{H}_B$, donde $\mathcal{H}_A \cong \mathbb{C}^2$ y $\mathcal{H}_B \cong \mathbb{C}^2$.

  \begin{parts}
    \part ¿Cuál es la dimensión de $\mathcal{H}_{AB}$? Enumere todos los elementos de la base canónica tensorial.
    \part Considere el tensor $t = \frac{1}{\sqrt{2}}(e_0\otimes e_1 - e_1\otimes e_0)$. Intente encontrar dos vectores $a \in \mathcal{H}_A$ y $b \in \mathcal{H}_B$ tal que $t = a \otimes b$. ¿Es $t$ un tensor simple (estado separable)?
  \end{parts}

  \question Sea el operador $X$ (la matriz NOT cuántica o Pauli $X$) y el operador $Z$ (Pauli $Z$):
  $$ X = \begin{pmatrix} 0 & 1 \\ 1 & 0 \end{pmatrix}, \quad Z = \begin{pmatrix} 1 & 0 \\ 0 & -1 \end{pmatrix} $$

  \begin{parts}
    \part Calcule el operador compuesto $C = X \otimes Z$ utilizando el producto de Kronecker. Escriba la matriz resultante en $\mathbb{C}^4$.
    \part El estado $|01\rangle$ se representa como el vector:
    $$ |01\rangle = |0\rangle \otimes |1\rangle = \begin{pmatrix} 1 \\ 0 \end{pmatrix} \otimes \begin{pmatrix} 0 \\ 1 \end{pmatrix} = \begin{pmatrix} 0 \\ 1 \\ 0 \\ 0 \end{pmatrix} $$
    Calcule el resultado de aplicar el operador $C$ a este estado, es decir, $C|01\rangle$.
    \part Verifique el resultado anterior aplicando la propiedad del producto tensorial de operadores directamente sobre el tensor simple:
    $$ C|01\rangle = (X \otimes Z) (|0\rangle \otimes |1\rangle) = (X|0\rangle) \otimes (Z|1\rangle)\,. $$
  \end{parts}

  \question Para el sistema de dos cúbits en el estado $\ket{\psi} = \frac{1}{\sqrt{3}}(\ket{00} + \ket{01} + \ket{11})$:
  \begin{parts}
    \part Verificar que el estado está normalizado
    \part Calcular las probabilidades de medir cada estado de la base computacional
    \part Determinar si el estado es separable o entrelazado
    \part Calcular la probabilidad de medir el primer cúbit en $\ket{0}$
  \end{parts}

  \question Para los cuatro estados de Bell:
  \begin{parts}
    \part Verificar que forman una base ortonormal
    \part Expresar cada estado de Bell como combinación lineal de la base computacional
    \part Demostrar que todos son entrelazados usando la descomposición de Schmidt
    \part Calcular las probabilidades de medir cada cúbit individualmente
  \end{parts}

  \question Determine si el estado $\ket{\psi} = \frac{1}{\sqrt{3}}(\ket{00} + \ket{01} + \ket{10})$ es entrelazado.

  \question Para $A = \sigma_x$ y $B = \sigma_y$, calcule explícitamente $(A \otimes B)\ket{01}$.

\end{questions}