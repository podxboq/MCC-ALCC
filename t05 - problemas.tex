\unirsection{Problemas}

\begin{questions}

  \question Expresar los siguientes estados en notación de Dirac y verificar su normalización:
  \begin{parts}
    \part $\begin{pmatrix} \frac{1}{\sqrt{3}} \\ \frac{\sqrt{2}}{\sqrt{3}} \end{pmatrix}$
    \part $\begin{pmatrix} \frac{1+i}{2} \\ \frac{1-i}{2} \end{pmatrix}$
    \part $\frac{1}{2}\begin{pmatrix} 1 \\ 1 \\ 1 \\ 1 \end{pmatrix}$ (estado de dos qubits)
  \end{parts}

  \question Para el estado $\ket{\psi} = \frac{1}{\sqrt{5}}(2\ket{0} + i\ket{1})$:
  \begin{parts}
    \part Calcular las probabilidades de medir $\ket{0}$ y $\ket{1}$
    \part Calcular las probabilidades de medir $\ket{+}$ y $\ket{-}$
    \part Determinar el estado después de medir $\ket{+}$ y obtener resultado positivo
  \end{parts}

  \question Verificar las siguientes identidades usando notación de Dirac:
  \begin{parts}
    \part $\sigma_x = \ketbra{0}{1} + \ketbra{1}{0}$
    \part $\sigma_y = -i\ketbra{0}{1} + i\ketbra{1}{0}$
    \part $\sigma_z = \ketbra{0}{0} - \ketbra{1}{1}$
    \part $I = \ketbra{0}{0} + \ketbra{1}{1}$
  \end{parts}

  \question Calcular los valores esperados $\langle\sigma_x\rangle$, $\langle\sigma_y\rangle$ y $\langle\sigma_z\rangle$ para:
  \begin{parts}
    \part $\ket{\psi_1} = \ket{0}$
    \part $\ket{\psi_2} = \ket{+} = \frac{\ket{0} + \ket{1}}{\sqrt{2}}$
    \part $\ket{\psi_3} = \frac{\ket{0} + i\ket{1}}{\sqrt{2}}$
  \end{parts}

  \question Para el sistema de dos qubits en el estado $\ket{\psi} = \frac{1}{\sqrt{3}}(\ket{00} + \ket{01} + \ket{11})$:
  \begin{parts}
    \part Verificar que el estado está normalizado
    \part Calcular las probabilidades de medir cada estado de la base computacional
    \part Determinar si el estado es separable o entrelazado
    \part Calcular la probabilidad de medir el primer qubit en $\ket{0}$
  \end{parts}

  \question Para los cuatro estados de Bell:
  \begin{parts}
    \part Verificar que forman una base ortonormal
    \part Expresar cada estado de Bell como combinación lineal de la base computacional
    \part Demostrar que todos son maximalmente entrelazados
    \part Calcular las probabilidades de medir cada qubit individualmente
  \end{parts}




  \question Determine si el estado $\ket{\psi} = \frac{1}{\sqrt{3}}(\ket{00} + \ket{01} + \ket{10})$ es entrelazado.


  \question Para $A = \sigma_x$ y $B = \sigma_y$, donde $\sigma_x$ y $\sigma_y$ son las matrices de Pauli, calcule explícitamente $(A \otimes B)\ket{01}$.

  \question Demuestre que la puerta CNOT puede crear entrelazamiento aplicándola a estados separables apropiados. Proporcione al menos dos ejemplos específicos.

  \question Consideraciones sobre los recursos para la computación cuántica. Responda a las siguientes preguntas:
  \begin{parts}
    \part Calcule cuántos parámetros reales se necesitan para especificar completamente el estado de un sistema de 5 cúbits.
    \part Si cada parámetro requiere 64 bits de almacenamiento, ¿cuánta memoria se necesitaría para almacenar el estado de 30 cúbits?
    \part Compare con la memoria total de todas las computadoras del mundo (estimada en $\sim 10^{21}$ bits).
  \end{parts}

  \question Para el estado $\ket{\psi} = \cos\frac{\pi}{8}\ket{0} + e^{i\pi/4}\sin\frac{\pi}{8}\ket{1}$:
  \begin{parts}
    \part Calcular las probabilidades de medir $\sigma_z$ en los valores $\pm 1$
    \part Calcular las probabilidades de medir $\sigma_x$ en los valores $\pm 1$
    \part Si se mide primero $\sigma_z$ y se obtiene $+1$, ¿cuáles son las probabilidades para una medición posterior de $\sigma_x$?
  \end{parts}

  \question Un cúbit evoluciona bajo el hamiltoniano $\hat{H} = \frac{\pi}{4}\sigma_y$:
  \begin{parts}
    \part Calcular el operador de evolución $U(t) = e^{-i\hat{H}t}$
    \part Si el estado inicial es $\ket{0}$, determinar $\ket{\psi(t)}$
    \part ¿En qué instante $t$ el estado se convierte en $\ket{1}$?
  \end{parts}

  \question Para la rotación $R_z(\theta) = e^{-i\theta\sigma_z/2}$:
  \begin{parts}
    \part Escribir la forma matricial explícita de $R_z(\theta)$
    \part Demostrar que $R_z(\theta) = \cos\frac{\theta}{2}I - i\sin\frac{\theta}{2}\sigma_z$
    \part Aplicar $R_z(\pi/2)$ al estado $\ket{+}$ y expresar el resultado
  \end{parts}

  \question Demostrar que la composición de rotaciones alrededor del mismo eje se suma: $$R_z(\alpha)R_z(\beta) = R_z(\alpha + \beta)\,.$$

\end{questions}
