\title{Álgebra lineal en computación cuántica}
\subtitle{Máster en computación cuántica}
\portada


\section{Calendario}
\begin{frame}
  \frametitle{\insertsection}
  \begin{itemize}
    \item Periodo formativo: 03/11/2025 - 27/02/2026.\pause
    \item Periodo repaso: 02/03/2026 - 05/03/2026.\pause
    \item Exámenes: 06/03/2026 - 08/03/2026.\pause
    \item Exámenes Extraordinaria	: 04/09/2026 - 06/09/2026.\pause
    \item Periodo vacacional: 22/12/2025 - 04/01/2026.
  \end{itemize}
\end{frame}


\section{Contenido formativo}
\begin{frame}
  \frametitle{\insertsection}
  \begin{itemize}
    \item 8 Temas.\pause
    \item 21 sesiones formativas.\pause
          \begin{itemize}
            \item 1 Sesión de presentación.\pause
            \item 15 sesiones de temario.\pause
            \item 3 Sesiones de refuerzo.\pause
            \item 1 Sesión de laboratorio.\pause
            \item 1 Sesión de examen.\pause
          \end{itemize}
    \item 3 actividades (2 individual, 1 en grupo).\pause
          \begin{itemize}
            \item Operaciones (28/11/2025).
            \item Producto tensorial y esfera de Bloch (09/01/2026).
            \item Matriz densidad (06/02/2026).
          \end{itemize}\pause
    \item 8 test.
  \end{itemize}
\end{frame}


\section{Temario}
\begin{frame}
  \frametitle{Temario}
  \begin{enumerate}
    \item Número complejos
    \item Geometría de números complejos
    \item Matrices complejas
    \item Espacios vectoriales
    \item Espacios de Hilbert
    \item Notación de Dirac
    \item Operadores lineales
    \item Compuertas cuánticas y la esfera de Bloch
  \end{enumerate}
\end{frame}

\section{Recursos}
\begin{frame}
  \begin{itemize}
    \item El temario en PDF.
    \item Biblioteca UNIR.
    \item Vídeos con ampliación del contenido.
    \item Foro.
    \item Internet y Youtube.
    \item IA generativas.\pause\ (\textcolor{red}{OJO:} Sistema antiplagio y contenido inventado).
  \end{itemize}
\end{frame}

\section{Evaluación}

\begin{frame}
  \frametitle{Evaluación}
  \begin{itemize}
    \item Evaluación continua (40\%): Se pueden conseguir un máximo de 10 puntos de hasta 15 puntos disponibles. Solo suman las actividades con calificación mayor o igual a 5.\pause
          \begin{itemize}
            \item Asistencia a clase, 0.5 punto: 0.25 por clase.
            \item Tests, 1.2 puntos: 0.15 punto cada test.
            \item Actividades, 13 puntos: Actividad 1: 5 puntos, actividad 2: 5 puntos (grupal) y actividad 3: 3.3 puntos.\pause
          \end{itemize}
    \item Examen final (60\%): es necesario su aprobado (puntuación mayor o igual a 5)
  \end{itemize}
\end{frame}

\begin{frame}
  \frametitle{Ejemplos de evaluación}

  10 puntos en evaluación continua son 4 puntos para la nota.\pause
  \[
    \begin{cases}
      \text{5 puntos en examen son 3 puntos para la nota.}\pause  & \text{\textcolor{azul}{Aprobado (7)}}\pause     \\
      \text{4 puntos en examen son 2.4 puntos para la nota}\pause & \text{\textcolor{marrón}{Suspenso (2.4)}}\pause
    \end{cases}
  \]

  0 puntos en evaluación continua son 0 puntos para la nota.\pause
  \[
    \begin{cases}
      \text{8 puntos en examen son 4.8 puntos para la nota.}\pause & \text{\textcolor{marrón}{Suspenso (4.8)}}\pause \\
      \text{10 puntos en examen son 6 puntos para la nota}\pause   & \text{\textcolor{azul}{Aprobado (6)}}
    \end{cases}
  \]
\end{frame}