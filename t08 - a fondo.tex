\unirsection{A fondo}

Los siguientes libros pueden servir de material de apoyo y para profundizar más en los contenidos de este tema.

\textbf{Yanofsky, N. y Mannucci, M. A. (2008). Quantum computing for computer scientists. Cambridge University Press.}

El tema 5 trata los conceptos básicos sobre un cúbit.

\textbf{
  Kholevo, A. S. (2012). Quantum systems, channels, information : a mathematical introduction. (1st ed.). Walter de Gruyter GmbH \& Co. KG.}

Los espacios de Hilbert se presentan en el capítulo 1, sección 1.1.

\textbf{Marinescu, D. and Marinescu, G. (2012). Classical and quantum information. Academic Press.}

Se recomienda la lectura del capítulo 3, las secciones 3.1, 3.2  y 3.3.


\textbf{Nalajara, N., Ohmi, T. (2008). Quantum computing. From linear algebra to physical realizations. CRC Press.}

Se recomienda la lectura del capítulo 1, donde se presentan los conceptos de aplicaciones lineales y representación matricial.

