\unirsection{Problemas}

\begin{questions}
  \question Calcular las siguientes operaciones, expresando el resultado en forma binomial:
  \begin{parts}
    \part $(3 + 2i) + (1 - 4i)$
    \part $(2 + i)(3 - 2i)$
    \part $\frac{1 + 2i}{2 - i}$
    \part $|3 + 4i|$
    \part $\conj{(2 - 3i)(1 + i)}$
  \end{parts}

  \question Convertir a forma polar:
  \begin{parts}
    \part $z_1 = 1 + i$
    \part $z_2 = -2 + 2i\sqrt{3}$
    \part $z_3 = -4i$
  \end{parts}

  \question Calcular todas las raíces cuartas de $16i$.

  \question Demostrar que para cualquier $z \in \C$: $|z|^2 = z \cdot \conj{z}$.

  \question Calcular la magnitud y fase de los siguientes números complejos:
  \begin{parts}
    \part $z_1 = 3 + 4i$
    \part $z_2 = -2 + 2i\sqrt{3}$
    \part $z_3 = -5i$
    \part $z_4 = 7$
  \end{parts}

  \question Expresar en forma exponencial y realizar las operaciones:
  \begin{parts}
    \part $(1 + i)^8$
    \part $\frac{2e^{i\pi/3}}{1 + i\sqrt{3}}$
    \part $\sqrt[3]{-8i}$ (todas las raíces)
  \end{parts}

  \question Dadas las amplitudes cuánticas $\alpha = \frac{2}{3}$ y $\beta = \frac{\sqrt{5}i}{3}$:
  \begin{parts}
    \part Verificar que $|\alpha|^2 + |\beta|^2 = 1$
    \part Expresar cada amplitud en forma $re^{i\theta}$
    \part Calcular las probabilidades asociadas a cada amplitud
  \end{parts}

  \question Un sistema cuántico tiene amplitudes $\alpha_1 = \frac{1}{\sqrt{2}}e^{i\pi/4}$ y $\alpha_2 = \frac{1}{\sqrt{2}}e^{i3\pi/4}$:
  \begin{parts}
    \part Calcular la amplitud total $\alpha_1 + \alpha_2$
    \part Determinar si hay interferencia constructiva, destructiva o parcial
    \part Calcular la probabilidad total $|\alpha_1 + \alpha_2|^2$
  \end{parts}

  \question Encontrar todos los números complejos $z$ que satisfacen:
  \begin{parts}
    \part $|z - i| = |z + 1|$
    \part $|z|^2 + z + \conj{z} = 3$
    \part $z^4 = -16$
  \end{parts}

  \question Tres amplitudes cuánticas tienen la misma magnitud $\frac{1}{\sqrt{3}}$ pero fases diferentes: $0$, $\frac{3\pi}{4}$ y $\frac{3\pi}{2}$:
  \begin{parts}
    \part Escribir las tres amplitudes en forma binomial.
    \part Calcular la suma total de las tres amplitudes expresada en forma binomial.
    \part Explicar por qué el resultado tiene sentido físicamente
  \end{parts}

  \begin{solution}
    \begin{parts}
      \part

      \begin{itemize}
        \item $\frac{1}{\sqrt{3}} \cdot e^{i \cdot 0} = \frac{1}{\sqrt{3}}$.
        \item $\frac{1}{\sqrt{3}} \cdot e^{i \cdot \frac{3\pi}{4}} = \frac{1}{\sqrt{3}} \left(\cos\left(\frac{3\pi}{4}\right) + i\sin\left(\frac{3\pi}{4}\right)\right) = \frac{1}{\sqrt{3}} \left(-\frac{\sqrt{2}}{2} + i\frac{\sqrt{2}}{2}\right)$.
        \item $\frac{1}{\sqrt{3}} \cdot e^{i \cdot \frac{3\pi}{2}} = \frac{1}{\sqrt{3}} \left(\cos\left(\frac{3\pi}{2}\right) + i\sin\left(\frac{3\pi}{2}\right)\right) = \frac{-i}{\sqrt{3}}$.
      \end{itemize}

      \part
      \begin{align*}
        \frac{1}{\sqrt{3}} \left(1 + \left(-\frac{\sqrt{2}}{2} - i\frac{\sqrt{2}}{2}\right) - i\right) = \frac{1}{\sqrt{3}} \left(\left(1 - \frac{\sqrt{2}}{2}\right) - \left(1 + \frac{\sqrt{2}}{2}\right) i\right)
      \end{align*}

      \part La suma de las tres amplitudes resulta en una amplitud con magnitud menor que la de cada una de las individuales, lo que indica interferencia destructiva parcial. Esto tiene sentido físicamente porque las fases diferentes causan que las ondas asociadas a las amplitudes se cancelen parcialmente entre sí.
    \end{parts}

  \end{solution}

  \question Para las amplitudes $\alpha = ae^{i\phi}$ y $\beta = be^{i\psi}$ con $a, b \in \mathbb{R}^+$:
  \begin{parts}
    \part Demostrar que $|\alpha + \beta|^2 = a^2 + b^2 + 2ab\cos(\psi - \phi)$
    \part ¿Para qué diferencia de fases se obtiene interferencia máxima?
    \part ¿Para qué diferencia de fases se obtiene interferencia mínima?
  \end{parts}

  \question Un número complejo $z$ satisface $z^3 = 2 + 2i\sqrt{3}$:
  \begin{parts}
    \part Expresar $2 + 2i\sqrt{3}$ en forma exponencial
    \part Encontrar las tres raíces cúbicas
    \part Representar las raíces en el plano complejo
    \part ¿Cuál es la interpretación geométrica de estas raíces?
  \end{parts}

  \question Dos amplitudes cuánticas $\alpha_1$ y $\alpha_2$ tienen magnitudes $\frac{1}{\sqrt{5}}$ y $\frac{2}{\sqrt{5}}$ respectivamente:
  \begin{parts}
    \part Si sus fases son $\theta_1 = 0$ y $\theta_2 = \pi/3$, calcular $|\alpha_1 + \alpha_2|^2$
    \part Encontrar el valor de $\theta_2$ que maximiza $|\alpha_1 + \alpha_2|^2$
    \part Encontrar el valor de $\theta_2$ que minimiza $|\alpha_1 + \alpha_2|^2$
  \end{parts}

  \question Demostrar las siguientes propiedades de la conjugación compleja aplicadas a amplitudes cuánticas:
  \begin{parts}
    \part Si $\alpha = re^{i\theta}$, entonces $\conj{\alpha} = re^{-i\theta}$
    \part $\alpha \cdot \conj{\alpha} = |\alpha|^2$
    \part $\conj{(\alpha + \beta)} = \conj{\alpha} + \conj{\beta}$
    \part Para amplitudes normalizadas: $|\alpha|^2 + |\beta|^2 = 1 \Rightarrow |\conj{\alpha}|^2 + |\conj{\beta}|^2 = 1$
  \end{parts}

\end{questions}
