\unirsection{A fondo}

Los siguientes libros pueden servir de material de apoyo y para profundizar más en los contenidos de este tema.

\textbf{
  Horn, R. A. y Johnson, C. R. (2012). Matrix Analysis (2.ª ed.). Cambridge University Press.}

Excelente libro sobre representación y descomposición de matrices, estudio de los valores propios y normas de matrices.

\textbf{
  Alexander S. Holevo (2012). Quantum Systems, Channels, Information: A Mathematical Introduction. De Gruyter.}

La exposición de los espacios tensoriales se realiza en la sección 3.1.1 del capítulo 3.

\textbf{Worfgang, S. (2019). Mathematics of Quantum Computation. An Introduction. Springer.}

El autor le dedica todo el capítulo 3 al producto tensorial y sus propiedades.


\textbf{Nielsen, M. A. y Chuang, I. L. (2010). Quantum Computation and Quantum Information (10.ª ed.). Cambridge University Press.}

Primero introducen el producto tensorial en la sección 2.1.7 del capítulo 2 y luego lo utilizan en múltiples ocasiones a lo largo del libro.

