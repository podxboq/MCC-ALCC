\begin{questions}

  \question La matriz de densidad de un estado puro $\ket{\psi}$ es:

  \begin{choices}
    \choice $\bra{\psi}$
    \CorrectChoice $\ket{\psi}\bra{\psi}$
    \choice $\braket{\psi}{\psi}$
    \choice $I$
  \end{choices}
  \begin{solution}
    Para estados puros, la matriz de densidad es el proyector sobre el estado: $\rho = \ket{\psi}\bra{\psi}$.
  \end{solution}

  \question Un estado es puro si y solo si:

  \begin{choices}
    \choice $\text{Tr}(\rho) = 0$
    \CorrectChoice $\text{Tr}(\rho^2) = 1$
    \choice $\text{Tr}(\rho^2) = 0$
    \choice $\rho = I$
  \end{choices}
  \begin{solution}
    El criterio de pureza establece que $\text{Tr}(\rho^2) = 1$ para estados puros y $\text{Tr}(\rho^2) < 1$ para estados mixtos.
  \end{solution}

  \question La matriz de densidad del estado completamente mixto de un qubit es:

  \begin{choices}
    \choice $\ketbra{0}{0}$
    \choice $\ketbra{1}{1}$
    \CorrectChoice $\frac{1}{2}I$
    \choice $\ketbra{+}{+}$
  \end{choices}
  \begin{solution}
    El estado completamente mixto es equiprobable en todos los estados: $\rho = \frac{1}{2}\ketbra{0}{0} + \frac{1}{2}\ketbra{1}{1} = \frac{1}{2}I$.
  \end{solution}

  \question Para el estado de Bell $\ket{\Phi^+} = \frac{1}{\sqrt{2}}(\ket{00} + \ket{11})$, la traza parcial sobre el subsistema B da:

  \begin{choices}
    \choice $\ketbra{0}{0}$
    \choice $\ketbra{1}{1}$
    \CorrectChoice $\frac{1}{2}I$
    \choice $\ketbra{+}{+}$
  \end{choices}
  \begin{solution}
    $\rho_A = \text{Tr}_B(\ketbra{\Phi^+}{\Phi^+}) = \frac{1}{2}\ketbra{0}{0} + \frac{1}{2}\ketbra{1}{1} = \frac{1}{2}I$. A pesar de que el sistema total está en estado puro, cada subsistema está en estado mixto.
  \end{solution}

  \question La entropía de von Neumann de un estado puro es:

  \begin{choices}
    \CorrectChoice $0$
    \choice $1$
    \choice $\log_2 d$ donde $d$ es la dimensión
    \choice Depende del estado
  \end{choices}
  \begin{solution}
    Para estados puros, $S(\rho) = -\text{Tr}(\rho \log_2 \rho) = 0$ ya que $\rho$ tiene un único valor propio igual a 1.
  \end{solution}

  \question La entropía de von Neumann del estado completamente mixto $\rho = \frac{I}{2}$ de un qubit es:

  \begin{choices}
    \choice $0$
    \CorrectChoice $1$ bit
    \choice $2$ bits
    \choice $\frac{1}{2}$ bit
  \end{choices}
  \begin{solution}
    $\rho = \frac{I}{2}$ tiene valores propios $\lambda_1 = \lambda_2 = \frac{1}{2}$. Por tanto, $S(\rho) = -2 \cdot \frac{1}{2}\log_2\frac{1}{2} = 1$ bit.
  \end{solution}

  \question La fidelidad entre dos estados puros ortogonales $\ket{\psi}$ y $\ket{\phi}$ con $\braket{\psi}{\phi} = 0$ es:

  \begin{choices}
    \CorrectChoice $0$
    \choice $\frac{1}{2}$
    \choice $1$
    \choice $-1$
  \end{choices}
  \begin{solution}
    Para estados puros, $F(\ket{\psi}, \ket{\phi}) = |\braket{\psi}{\phi}| = |0| = 0$.
  \end{solution}

  \question El teorema de no-clonación establece que:

  \begin{choices}
    \choice Todo estado cuántico puede ser copiado
    \CorrectChoice No existe operación que pueda clonar un estado cuántico arbitrario desconocido
    \choice Solo los estados de la base computacional pueden ser clonados
    \choice Solo los estados entrelazados no pueden ser clonados
  \end{choices}
  \begin{solution}
    El teorema de no-clonación es fundamental en teoría cuántica de información: no existe canal cuántico que pueda copiar un estado desconocido arbitrario.
  \end{solution}

  \question Los operadores de Kraus $\{K_i\}$ de un canal cuántico satisfacen:

  \begin{choices}
    \choice $\sum_i K_i = I$
    \CorrectChoice $\sum_i K_i^\dagger K_i = I$
    \choice $\sum_i K_i K_i = I$
    \choice $K_i K_i^\dagger = I$ para todo $i$
  \end{choices}
  \begin{solution}
    La condición de completitud para operadores de Kraus es $\sum_i K_i^\dagger K_i = I$, lo que asegura que la traza se preserve.
  \end{solution}

  \question Un canal de despolarización con probabilidad $p = 1$ transforma cualquier estado en:

  \begin{choices}
    \choice $\ketbra{0}{0}$
    \choice $\ketbra{1}{1}$
    \CorrectChoice $\frac{I}{2}$
    \choice No cambia el estado
  \end{choices}
  \begin{solution}
    Para $p = 1$, el canal de despolarización $\mathcal{E}(\rho) = (1-p)\rho + \frac{p}{3}(\sigma_x\rho\sigma_x + \sigma_y\rho\sigma_y + \sigma_z\rho\sigma_z)$ se reduce a aplicar todas las matrices de Pauli con igual probabilidad, resultando en $\frac{I}{2}$.
  \end{solution}

\end{questions}