\portada

\begin{esquemaExplorador}
  \temaEsquema{Fundamentos algebraicos}{
    \conceptoEsquema{Definición y operaciones}{}
    \conceptoEsquema{Forma cartesiana y polar}{}
    \conceptoEsquema{Fórmula de Euler}{$e^{i\theta} = \cos\theta + i\sin\theta$}
    \conceptoEsquema{Fórmula de De Moivre}{$z^n = r^n(\cos(n\theta) + i\sin(n\theta))$}
  }
  \temaEsquema{Representación geométrica}{
    \conceptoEsquema{Plano complejo}{}
    \conceptoEsquema{Módulo y argumento}{}
    \conceptoEsquema{Distancias y ángulos}{}
  }
  \temaEsquema{Funciones complejas}{
    \conceptoEsquema{Función exponencial}{$e^z = \sum_{n=0}^{\infty} \frac{z^n}{n!}$}
    \conceptoEsquema{Función seno}{$\sin(z) = \frac{e^{iz} - e^{-iz}}{2i}$}
    \conceptoEsquema{Función coseno}{$\cos(z) = \frac{e^{iz} + e^{-iz}}{2}$}
  }
  \temaEsquema{Correspondencia cuántica}{
    \conceptoEsquema{Amplitud cuántica}{$\alpha\in\C$}
    \conceptoEsquema{Magnitud y probabilidad}{$P = |\alpha|^2$}
    \conceptoEsquema{Fase e interferencia}{$\theta = arg(\alpha)$}
  }
\end{esquemaExplorador}

\unirsection{Ideas clave}

\subsection{Introducción y objetivos}

Los números complejos constituyen el fundamento matemático esencial de la mecánica cuántica y, por tanto, de la computación cuántica. Mientras que en la física clásica las magnitudes se describen mediante números reales, en el mundo cuántico necesitamos la riqueza matemática de los números complejos para describir fenómenos como la superposición, la interferencia y el entrelazamiento.

La estrecha relación entre los números complejos y la mecánica cuántica se refleja en:

\begin{itemize}
  \item Los \textbf{estados cuánticos} son vectores en un espacio vectorial complejo.
  \item Las \textbf{amplitudes cuánticas} son números complejos que determinan las probabilidades de los estados cuánticos.
  \item La \textbf{evolución temporal} de los sistemas cuánticos se describe mediante operadores unitarios con entradas complejas.
  \item La \textbf{interferencia cuántica}, base de muchos algoritmos cuánticos, requiere la aritmética compleja para su comprensión.
\end{itemize}

En este primer capítulo estudiaremos el plano complejo desde el punto de vista algebraico, geométrico y topológico. Empezaremos motivando la existencia de estos números y definiendo el cuerpo $\C$ de los números complejos. También veremos que podemos considerar $\R$ como un subcuerpo de $\C$.

Repasaremos las distintas formas de escribir los números complejos: binomial, polar y exponencial. Veremos las ventajas de trabajar con los números complejos en forma exponencial, en particular en el cálculo de las raíces $n$-ésimas de un número complejo. Finalmente repasaremos algunas propiedades de las raíces de polinomios y presentaremos el teorema fundamental del álgebra.

\subsection{El cuerpo de los números complejos}

En esta primera sección recordaremos brevemente algunas de las propiedades algebraicas y geométricas de los números complejos. Empezaremos con el origen de los números complejos. El propósito de los números complejos es proporcionar soluciones a ecuaciones polinómicas que no tienen solución real. Por ejemplo, las ecuaciones
\[
  x^2 + 1 = 0\,,\qquad x^2 + 2x + 5 = 0\,,
\]
no tienen solución en el cuerpo $\R$ de los números reales. Para resolver estas ecuaciones, introducimos el concepto de \textbf{unidad imaginaria}, que denotaremos por $i$, que cumple la relación
\[ i^2 = -1\,. \]
Extendiendo los números reales con esta unidad imaginaria $i$, veremos que podemos resolver todas las ecuaciones polinómicas con coeficientes reales (y, de hecho, complejos).

\begin{defi}[El plano complejo]
  Llamaremos $\C$, \textbf{plano complejo} o \textbf{plano de Argand} a la terna $(\R^2,+,\cdot)$, donde las operaciones son:
  \begin{itemize}
    \item suma: $(a,b) + (c,d) = (a+c,b+d)$,
    \item producto: $(a,b)\cdot(c,d) = (ac-bd,ad+bc)$.
  \end{itemize}
\end{defi}
\separa
\begin{prop}
  El plano complejo $\C$ con las operaciones de suma y producto es un cuerpo conmutativo. Es decir, satisface las siguientes propiedades:
  \begin{enumerate}
    \item \textbf{Clausura:} dados $z,w\in\C$, $z+w,zw\in\C$.
    \item \textbf{Asociativa:} dados $z,w,x\in\C$, $z+(w+x) = (z+w)+x$, y $z(wx) = (zw)x$.
    \item \textbf{Conmutativa:} dados $z,w\in\C$, $z+w = w+z$, y $zw = wz$.
    \item \textbf{Distributiva:} dados $z,w,z\in\C$, $z(w+x) = zw + zx$.
    \item \textbf{Neutro para la suma:} $\exists!\ 0\in\C$ tal que $\forall z\in\C$, $z+0 = z$.
    \item \textbf{Neutro para el producto:} $\exists!\ 1\in\C$ tal que $\forall z\in\C$, $1z = z$.
    \item \textbf{Opuesto para la suma:} $\forall z\in\C$, existe $-z\in\C$ tal que $z+(-z) = 0$.
    \item \textbf{Inverso para el producto:} $\forall z\in\C\setminus\{0\}$, existe $z^{-1}\in\C$ tal que $zz^{-1} = 1$.
  \end{enumerate}
\end{prop}

Tal como hemos definido las operaciones entre números complejos podemos identificar el número real $a\in\R$ con el complejo $(a,0)\in\C$. Mediante esta identificación $\R\hookrightarrow\C$ dada por $a\mapsto(a,0)$, podemos entender los números reales como un subcuerpo de los números complejos.

Teniendo en cuenta esta identificación y el hecho de que
\[(0,1)\cdot(0,1) = (-1,0)\,, \]
le daremos el nombre de unidad imaginaria $i$ al complejo $(0,1)$. De este modo, podemos escribir el complejo $z = (a,b)$ como $z = a+bi$. Esta forma de escribir los números complejos recibe el nombre de \textbf{forma binomial}.

\begin{defi}[Parte real e imaginaria de un complejo]
  Sea $z = a+bi\in\C$ un número complejo. Los números reales
  \[ a = \Rp(z)\ ,\qquad b = \Ip(z) \]
  que lo componen reciben el nombre de \textbf{parte real} y \textbf{parte imaginaria} de $z$ respectivamente. Los números complejos con parte real nula se llaman \textbf{imaginarios puros}.
\end{defi}

Ya hemos visto que podemos identificar a los números reales con los complejos con parte imaginaria nula.

A diferencia del cuerpo de los números reales, el cuerpo de los números complejos no es un cuerpo ordenado. Dicho de forma más precisa, no se puede definir un orden en $\C$ que sea compatible con las operaciones del cuerpo.

\begin{defi}[Conjugación]
  El \textbf{conjugado} del número complejo $z = a+bi$ es el complejo $\bar z = a-bi$.
\end{defi}
\separa
\begin{nota}
  En algunos libros de texto al número conjugado de $z$ se le denota por $z^*$.
\end{nota}

Notamos que un número complejo $z$ es real si y solo si $z = \bar z$. Por otro lado, $z$ es imaginario si y solo si $\bar z = -z$.

La conjugación satisface las siguientes propiedades:

\begin{prop}
  Si $z$ y $w$ son números complejos, se cumplen las siguientes igualdades:
  \begin{itemize}
    \item Parte real: $\Rp(z) = \frac{z + \conj{z}}{2}$.
    \item Parte imaginaria: $\Ip(z) = \frac{z - \conj{z}}{2i}$.
    \item Conjugación de la suma: $\conj{z + w} = \conj{z} + \conj{w}$.
    \item Conjugación del producto: $\conj{zw} = \conj{z} \cdot \conj{w}$.
    \item Conjugación de la inversa: $\conj{z^{-1}} = (\conj{z})^{-1}$.
    \item Doble conjugación: $\conj{\conj{z}} = z$.
  \end{itemize}
\end{prop}

\begin{defi}[Módulo]
  El \textbf{módulo} de un número complejo $z$ es el real no negativo definido por:
  \[ \abs{z} = \sqrt{z\bar z}\,, \]
  o, expresado en forma binomial
  \[ \abs{a + bi} = \sqrt{a^2 + b^2}\,. \]
\end{defi}

\begin{prop}
  Si $z$ y $w$ son números complejos, se cumplen las siguientes propiedades:
  \begin{itemize}
    \item $\abs{z}\geq 0$.
    \item $\abs{z} = 0$ si y solo si $z = 0$.
    \item \textbf{Desigualdad triangular}. $\abs{z+w}\leq\abs{z}+\abs{w}$.
    \item $\abs{zw} = \abs{z}\abs{w}$.
  \end{itemize}
\end{prop}

La conjugación y el módulo permiten calcular fácilmente el inverso multiplicativo de cualquier complejo $z\in\C^\ast = \C\setminus\{0\}$:
\[ z^{-1} = \frac{\bar z}{\abs{z}^2}\,. \]

\subsection{Forma polar}

Recordamos que un punto $(x,y)\in\R^2$, diferente del origen de coordenadas, puede expresarse en \textbf{coordenadas polares} $(r,\theta)$, de forma que
\[ x = r\cos\theta\ ,\quad y = r\sin\theta\,. \]

\begin{defi}
  \label{def:argumento}
  Sea $z=a+bi$ un número complejo no nulo. Diremos que el ángulo $\theta$ es un \textbf{argumento} de $z$ si:
  \begin{align*}
    \cos(\theta) & = \frac{a}{|z|} \\
    \sin(\theta) & = \frac{b}{|z|}
  \end{align*}
  Si $z=0$ diremos que su argumento es nulo.
\end{defi}

Observaremos que el argumento de un número complejo no está completamente definido, pues si $\theta$ cumple las propiedades del argumento~\eqref{def:argumento}, entonces $\theta + 2\pi$ también las cumple.

Si restringimos el valor del argumento al intervalo $[0, 2\pi)$, tendremos unicidad en la definición y por este motivo le damos el nombre de \textbf{argumento principal} y lo denotamos por $\Arg z$.

Observamos por último que de las coordenadas polares, podemos deducir $r = \sqrt{x^2+y^2}$, y que este valor es exactamente el módulo del número complejo asociado.

\begin{defi}[Forma polar]
  Sea $z$ un número  complejo. Llamaremos \textbf{forma polar} a la expresión:
  \[ z = |z|(\cos(\Arg z) + i\sin(\Arg z))\,. \]
\end{defi}


\subsection{Forma exponencial}

El interés de la forma polar radica en las siguientes fórmulas, que relacionan los números complejos con las funciones trigonométricas. Estas fórmulas son fundamentales en muchas aplicaciones de los números complejos, incluyendo la computación cuántica.

\begin{theo}[Fórmula de Euler]
  Sea $\theta\in\R$, se cumple:
  \begin{equation}
    \label{eq:formula-euler}
    e^{i\theta} = \cos\theta + i\sin\theta\,.
  \end{equation}
\end{theo}

\begin{theo}[Fórmula de De Moivre]
  Para $z$ un número complejo y $n \in \Z$:
  \begin{equation}
    \label{theo:formula-moivre}
    z^n = |z|^n(\cos(n\Arg z) + i\sin(n\Arg z))
  \end{equation}
\end{theo}

La fórmula de Euler es muy importante porque motiva la siguiente notación, que será de extrema utilidad a partir de ahora.

\begin{defi}[Forma exponencial]
  \label{def:forma-exponencial}
  Un número complejo $z$ puede escribirse en su \textbf{forma exponencial} como
  \[ z = |z|e^{i\Arg z}\,. \]
\end{defi}

\begin{eje}
  Consideremos el número complejo $z=-1-\sqrt{3}i$, vamos a calcular el conjugado, así como las distintas formas de representación.

  Por definición, el conjugado es el número $\bar{z} = -1 + \sqrt{3}i$ y si lo multiplicamos por $z$ tendremos $z\bar{z}=(-1)^2+(\sqrt{3})^2= 4$.

  El módulo de $z$ es $|z|=\sqrt{z\bar{z}}=\sqrt{4}=2$.

  Para calcular el argumento, como $\Rp(z)\neq 0$ tenemos que $\theta = \arctan(\frac{-\sqrt{3}}{-1}) = \arctan(\sqrt{3}) = \frac{\pi}{3}$. Pero este cálculo no tiene en cuenta el verdadero cuadrante del ángulo. En este ejemplo, el argumento principal de $z$ está en tercer cuadrante, por lo tanto $\Arg z = \frac{\pi}{3}+\pi = \frac{4\pi}{3}$.

  Expresado $z$ de forma polar sería $z=2(\cos(\frac{4\pi}{3})+\sin(\frac{4\pi}{3})i)$, mientras que expresado de forma exponencial sería $z=2e^{\frac{4\pi i}{3}}$.
\end{eje}

\subsection{Geometría compleja}

Notamos que un complejo de módulo 1 es de la forma $e^{i\theta}$. Geometricamente, la multiplicación $z\mapsto e^{i\theta}z$ es una rotación en el plano complejo de ángulo $\theta$ alrededor del origen.

La forma exponencial permite hacer algunos cálculos más fácilmente, especialmente los que involucran productos y cocientes.

\begin{prop}
  Dados $re^{i\theta}$ y $se^{i\varphi}$ dos números complejos expresados en forma exponencial, se tiene:
  \begin{itemize}
    \item $\conj{re^{i\theta}} = re^{-i\theta}$.
    \item $re^{i\theta}\cdot se^{i\varphi} = (rs)e^{i(\theta+\varphi)}$.
    \item Para $r$ no nulo, $\left(re^{i\theta}\right)^{-1} = r^{-1}e^{-i\theta}$.
  \end{itemize}
\end{prop}

Otra ventaja de la notación exponencial es que facilita enormemente el cálculo de las raíces $n$-ésimas.

\begin{defi}[Raíces $n$-ésimas de un complejo]
  Sea $z = re^{i\theta}$ un número complejo no nulo y $n\in\N$. Las \textbf{raíces $n$-ésimas} de $z$ son los complejos $w = \rho e^{i\psi}$ tales que $w^n = z$. Por lo tanto,
  \[
    \rho = \sqrt[n]{r}\ ,\qquad \psi = \frac{\theta + 2\pi k}{n}\,,\quad k = 0,1,\dotsc,n-1\,.
  \]
  En particular, las \textbf{raíces $n$-ésimas de la unidad} son
  \[
    \sqrt[n]{1}=\set{e^{\frac{2\pi i k}{n}}\ ,\quad k = 0,1,\dotsc,n-1}\,.
  \]
\end{defi}

Geometricamente, las raíces $n$-ésimas de la unidad de un complejo son los vértices de un polígono regular de $n$ lados centrado en el origen y circunscrito en una esfera de radio 1.

\begin{eje}
  Vamos a calcular las raíces cúbicas del número complejo $z = -1+i$. Si escribimos el complejo $z$ en forma exponencial, tenemos
  \[ z = \sqrt{2}e^{\frac{3\pi}{4}i}\,. \]

  Si $w = \rho e^{i\psi}$ es una raíz cúbica de $z$, entonces
  \[
    w^3 = \rho^3 e^{3i\psi} = \sqrt{2}e^{\frac{3\pi}{4}i}\Rightarrow \begin{dcases}
      \rho^3 = \sqrt{2} \Rightarrow \rho = \sqrt[6]{2} \\
      3\psi = \frac{3\pi}{4} + 2\pi k\Rightarrow \psi = \frac{(3+8k)\pi}{12}\ ,\quad k\in\Z
    \end{dcases}\,.
  \]

  Por lo tanto, las raíces cúbicas son:
  \begin{align*}
     & k=0:\quad w_0 = \sqrt[6]{2}e^{\frac{\pi}{4}i}\,,    \\
     & k=1:\quad w_1 = \sqrt[6]{2}e^{\frac{11\pi}{12}i}\,, \\
     & k=2:\quad w_2 = \sqrt[6]{2}e^{\frac{19\pi}{12}i}\,.
  \end{align*}
  Como podemos ver en la figura~\ref{fig:raices_cubicas}, estas raíces son los vértices de un triángulo equilátero inscrito en una circunferencia de radio $\sqrt[6]{2}$.
\end{eje}

\begin{figure}[h]
  \centering
  \includegraphics[width=\textwidth]{imgs/raices_cubicas.png}
  \caption{Raíces cúbicas del complejo $z = -1+i$. Fuente: https://complex-analysis.com}.
  \label{fig:raices_cubicas}
\end{figure}

\subsubsection{Geometría euclídea}

Como elementos geométricos heredados de la geometría euclídea de $\R^2$ tenemos:
\begin{itemize}
  \item \textbf{Rectas}: En particular, las rectas paralelas a los ejes $\Rp(z) = a$, o $\Ip(z) = b$.
  \item \textbf{Semiplanos} (abiertos o cerrados). En particular, $\Rp(z) > a$, $\Ip(z) > b$, etc.
  \item \textbf{Bandas}: regiones comprendidas entre dos rectas paralelas.
  \item \textbf{Discos}: (abiertos o cerrados):
        \[ D(z_0,r) = \{ z\in\C\ \vert\ d(z_0,z)<r \}\ ,\qquad \bar D(z_0,r) = \{ z\in\C\ \vert\ d(z_0,z)\leq r \}\,. \]
  \item \textbf{Circunferencias}:
        \[ S(z_0,r) = \{ z\in\C\ \vert\ d(z_0,z) = r \} = \partial D(z_0,r)\,. \]
        En particular, la circunferencia unidad $S_1 = \partial D(0, 1)= \{z\in\C\ \vert\ \abs{z}=1\}$.
  \item \textbf{Coronas circulares}:
        \[ C(z_0,r_1,r_2) = \{ z\in\C\ \vert\ r_1 < d(z_0,z) < r_2 \}\,. \]
  \item El \textbf{módulo} es la distancia del origen al punto, $|z|=d(0, z)$.
  \item El \textbf{conjugado} es la reflexión de respecto al eje real.
  \item La \textbf{suma} corresponde a la suma vectorial en el plano.
\end{itemize}

Es conveniente observar, como se aprecia en la figura~\ref{fig:producto_complejo} que la multiplicación de números complejos tiene dos efectos geométricos:
\begin{itemize}
  \item Una rotación de ángulo la suma de los argumentos.
  \item Una multiplicación de los módulos.
\end{itemize}

\begin{figure}[h]
  \centering
  \includegraphics[width=0.5\textwidth]{imgs/producto_complejo.png}
  \caption{Producto de dos números complejos $2e^{i\frac{\pi}{6}}$ y $3e^{i\frac{\pi}{4}}$. Fuente: Elaboración propia con Geogebra.}
  \label{fig:producto_complejo}
\end{figure}

\subsection{Funciones complejas elementales}

\begin{defi}[Función compleja]
  Sea $A$ un subconjunto de $\C$. Una \textbf{función compleja} es una función $f\colon A\to\C$ que toma valores complejos. Como $\C = \R\oplus i\R$, toda función compleja puede escribirse de forma única como $f = u+iv$, donde $u,v\colon A\to\R$ son funciones reales, llamadas respectivamente \textbf{parte real}, $u = \Rp(f)$, y \textbf{parte imaginaria}, $v = \Ip(f)$, de $f$.
\end{defi}

Ya hemos usado la función exponencial para expresar su valor con exponente un número complejo, pero podemos obtener la formulación analítica de la función exponencial, extendiendo la correspondiente función analítica de la exponencial real.

\begin{defi}[Función exponencial compleja]
  Para $z \in \C$, la función exponencial se define como:
  $$e^z = \sum_{n=0}^{\infty} \frac{z^n}{n!}$$
\end{defi}

La función exponencial compleja, cumple las mismas propiedades que la exponencial real en cuanto a las manipulaciones algebraicas:

\begin{prop}
  Sean $z$ y $w$ dos números complejos, se cumple:
  \begin{itemize}
    \item $e^{z+w} = e^z e^w$.
    \item $(e^z)^n = e^{nz}\ \forall n\in\N$.
  \end{itemize}
\end{prop}

A partir de la definición de la función exponencial y usando la fórmula de Euler~\eqref{eq:formula-euler}, podemos definir analíticamente también las funciones seno y coseno para números complejos.

\begin{defi}[Funciones trigonométricas complejas]
  Para $z \in \C$, las funciones seno y coseno se definen como:
  \begin{align*}
    \cos (z) & = \frac{e^{iz} + e^{-iz}}{2}\,.  \\
    \sin (z) & = \frac{e^{iz} - e^{-iz}}{2i}\,.
  \end{align*}
\end{defi}

\subsection{Matemáticas cuánticas}

En el contexto de la mecánica cuántica, un número complejo se describe como una amplitud cuántica que está formado por un par de números llamados amplitud y fase, que no son más que el módulo y un argumento.

\begin{defi}
  Se denota por \textbf{amplitud cuántica} a un número complejo $\alpha \in \C$ que está caracterizado por dos valores:
  \begin{itemize}
    \item \textbf{Magnitud}: Igual al módulo $|\alpha|$.
    \item \textbf{Fase}: Pertenece al argumento $\arg(\alpha)$.
  \end{itemize}
\end{defi}

Por lo tanto, dos amplitudes cuánticas son iguales si tienen la misma magnitud y su diferencia de fases es un múltiplo de $2\pi$.

\begin{eje}
  Una amplitud cuántica de magnitud 2 con fase $\pi$ es igual a otra amplitud cuántica de magnitud 2 con fase $-\pi$.
\end{eje}

\begin{eje}[Interferencia cuántica]
  Cuando dos amplitudes cuánticas interfieren, la amplitud cuántica resultante puede tener una fase menor, diremos entonces que las amplitudes interfieren de forma destructiva, si la fase resultante es mayor diremos que interfieren de forma constructiva:

  \textbf{Interferencia destructiva}: Una amplitud cuántica de magnitud 2 con fase $\pi$ y otra amplitud cuántica de magnitud 2 con fase $\pi$ interfieren dando una amplitud cuántica de fase $0$.

\end{eje}

\begin{info}
  La interferencia cuántica es el principio fundamental detrás de muchos algoritmos cuánticos: amplificamos las amplitudes de respuestas correctas e interferimos destructivamente con las incorrectas.
\end{info}

Al final una amplitud cuántica es la representación exponencial de un número complejo, y podemos trabajar con ellas usando las reglas vistas en este tema.
