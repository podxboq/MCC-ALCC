\unirsection{Problemas}

\begin{questions}

  \question \label{ejer:propiedades-lineal-independencia} Demostrar las propiedades sobre conjuntos generadores y linealmente independientes enunciadas en el resultado~\ref{prop:propiedades-lineal-independencia}.
  \begin{solution}
    \begin{parts}
      \part El cero es combinación lineal de cualquier conjunto de vectores.
      \part La misma relación de dependencia lineal se mantiene al extender el conjunto.
      \part Si el subconjunto fuera linealmente dependiente, la extensión al conjunto original también lo será.
      \part Si fuera linealmente independiente, se podría extender a una base, y por tanto la dimensión de $\C^n$ sería mayor que $n$.
      \part Si tuviera más de $n$ vectores, por la propiedad anterior, tendría que ser linealmente dependiente.
    \end{parts}
  \end{solution}

  \question Determinar si los siguientes conjuntos, definidos por las coordenadas de sus vectores son subespacios de $\C^3$:
  \begin{parts}
    \part $W_1 = \{(x, y, z)\in\C^3 \mid x + 2y - z = 1\}$.
    \part $W_2 = \{(x, y, z)\in\C^3 \mid x = \conj{z}\}$.
    \part $W_3 = \{(x, y, z)\in\C^3 \mid |x|^2 + |y|^2 + |z|^2 = 1\}$.
  \end{parts}
  \begin{solution}
    Ninguno de los tres casos son subespacios vectoriales.
    Sean $v_i = (x_i, y_i, z_i)$ vectores de $W_i$. Entonces
    \begin{parts}
      \part $2 v_1$ no cumple con las condiciones de $W_1$, pues $2x_1 + 4y_1 - 2z_1 = 2(x_1 + 2y_1 - z_1) = 2 \neq 1$.
      \part Del mismo modo $i v_2$ no cumple con las condiciones de $W_2$, ya que $\conj{i y_2} = -i\conj{y_2} = -i x_2 \neq ix_2$.
      \part Por último, $2 v_3$ no cumple con las condiciones de $W_3$, ya que $|2 x_3|^2 + |2 y_3|^2 + |2 z_3|^2 = 4(|x_3|^2 + |y_3|^2 + |z_3|^2) = 4 \neq 1$.
    \end{parts}
  \end{solution}


  \question
  Encontrar una base y la dimensión del subespacio $W$ de $\C^4$ generado por los vectores
  \[
    v_1=\mqty(1\\ i\\ 0\\ 1), \quad v_2=\mqty(i\\ 1\\ 1\\ 0), \quad v_3=\mqty(1+i\\ 1+i\\ 1\\ 1)\,.
  \]
  \begin{solution}
    Si los tres vectores son linealmente independientes, entonces forman una base de $W$. Se observa que $v_3 = v_1 + v_2$, por lo que los vectores son linealmente dependientes.

    Sin embargo, $v_1$ y $v_2$ sí son linealmente independientes. Por lo tanto, la dimensión de $W$ es 2 y $\mathcal{B} = \{v_1, v_2\}$ es una base de $W$.
  \end{solution}

  \question
  Verificar que el conjunto $\{v_1, v_2, v_3\}$ es linealmente independiente en $\C^4$ y extenderlo a una base de $\C^4$ para
  \[
    v_1=\mqty(1\\ 1\\ 0\\ 1), \quad v_2=\mqty(1\\ i\\ 0\\ 1), \quad v_3=\mqty(0\\ 0\\ 1\\ 1)\,.
  \]
  \begin{solution}
    Sean $\alpha, \beta, \gamma \in \C$ tales que $\alpha v_1 + \beta v_2 + \gamma v_3 = 0$. Tenemos el siguiente sistema de ecuaciones
    \begin{align*}
      \alpha + \beta          & = 0 \\
      \alpha + i\beta         & = 0 \\
      \gamma                  & = 0 \\
      \alpha + \beta + \gamma & = 0
    \end{align*}
    De la tercera ecuación obtenemos que $\gamma = 0$. Sustituyendo en la primera ecuación obtenemos que $\alpha = -\beta$. Sustituyendo en la segunda ecuación obtenemos que $\alpha = -i\beta$. Por lo tanto, $\beta = i\beta$, lo que implica que $\beta = 0$. De la primera ecuación obtenemos que $\alpha = 0$. Por lo tanto, los vectores son linealmente independientes.
    Para extender a base de $\C^4$ se puede agregar el vector $v_4 = (0,0,0,1)$.
  \end{solution}

  \question
  Demostrar que si $v_1, v_2, \ldots, v_k$ son linealmente independientes en un espacio vectorial $V$, entonces $v_1, v_1 + v_2, v_1 + v_2 + v_3, \ldots, v_1 + v_2 + \cdots + v_k$ también son linealmente independientes.
  \begin{solution}
    Sean $\alpha_1, \alpha_2, \ldots, \alpha_k \in \C$ tales que
    \begin{align*}
      \alpha_1 v_1 + \alpha_2 (v_1 + v_2) + \cdots + \alpha_k (v_1 + v_2 + \cdots + v_k)                                    & = 0 \Rightarrow \\
      (\alpha_1 + \alpha_2 + \cdots + \alpha_k) v_1 + (\alpha_2 + \alpha_3 + \cdots + \alpha_k) v_2 + \cdots + \alpha_k v_k & = 0\,.
    \end{align*}
    Como los vectores son linealmente independientes, tenemos que
    \begin{align*}
      \alpha_1 + \alpha_2 + \cdots + \alpha_k & = 0      \\
      \alpha_2 + \cdots + \alpha_k            & = 0      \\
      \vdots                                  & = \vdots \\
      \alpha_{k-1} + \alpha_k                 & = 0      \\
      \alpha_k                                & = 0
    \end{align*}
    Resolviendo el sistema de ecuaciones de forma ascendente empezando por la última ecuación obtenemos que $\alpha_i = 0$ para todo $i \in \{1, 2, \ldots, k\}$.
  \end{solution}

  \question
  Sea $V = \mathcal{P}_2(\C)$ el espacio de polinomios complejos de grado menor o igual que 2.
  \begin{parts}
    \part Demostrar que $\{1, z, z^2\}$ es una base de $V$.
    \part Encontrar las coordenadas del polinomio $p(z) = (1+i) + 2iz - z^2$ en esta base.
    \part Encontrar la base de $V$ para la cual $p(z)$ tiene como coordenadas $(1,1,1)$.
  \end{parts}
  \begin{solution}
    \begin{parts}
      \part Para cualquier combinación lineal igualada a cero, tiene que cumplir $\alpha_0 + \alpha_1 z + \alpha_2 z^2 = 0$, lo que implica que $\alpha_0 = \alpha_1 = \alpha_2 = 0$. Por lo tanto, los vectores son linealmente independientes. Además, cualquier polinomio de grado menor o igual que 2 se puede escribir como combinación lineal de estos vectores.
      \part Por la propia definición de coordenadas, tenemos que $p(z) = (1+i) \cdot 1 + 2i \cdot z + (-1) \cdot z^2$, por lo que las coordenadas son $(1+i, 2i, -1)$.
      \part Sea $\mathcal{B}=\{v_1 = 1+i$, $v_2 = 2iz$, $v_3 = -z^2\}$. Entonces $p(z) = v_1 + v_2 + v_3$, por lo que las coordenadas son $(1,1,1)_\mathcal{B}$.
    \end{parts}
  \end{solution}

  \question
  En el espacio $\mathcal{M}_2(\C)$ de matrices complejas $2 \times 2$:
  \begin{parts}
    \part Verificar que las siguientes matrices (llamadas \textbf{matrices de Pauli}) junto con la identidad forman una base
    \[
      \mathcal{B} = \left\{I = \begin{pmatrix} 1 & 0 \\ 0 & 1 \end{pmatrix}, \sigma_x = \begin{pmatrix} 0 & 1 \\ 1 & 0 \end{pmatrix}, \sigma_y = \begin{pmatrix} 0 & -i \\ i & 0 \end{pmatrix}, \sigma_z = \begin{pmatrix} 1 & 0 \\ 0 & -1 \end{pmatrix}\right\}\,.
    \]
    \part Expresar la matriz
    \[
      A = \begin{pmatrix} 2+i & 1-i \\ 1+i & 2-i \end{pmatrix}
    \]
    como combinación lineal de esta base.
  \end{parts}
  \begin{solution}
    \begin{parts}
      \part Dada una combinación lineal de las matrices de Pauli y la identidad igualada a cero, tenemos que
      \begin{align*}
        \alpha_0 \mqty(1          & 0                    \\ 0 & 1) + \alpha_1 \mqty(0 & 1 \\ 1 & 0) + \alpha_2 \mqty(0 & -i \\ i & 0) + \alpha_3 \mqty(1 & 0 \\ 0 & -1) & = \mqty(0            & 0 \\ 0 & 0) \\
        \mqty(\alpha_0 + \alpha_3 & \alpha_1 -\alpha_2 i \\ \alpha_1 + \alpha_2 i & \alpha_0 -\alpha_3) & = \mqty(0 & 0 \\ 0 & 0) \,.
      \end{align*}
      De esta igualdad salen 4 ecuaciones
      \begin{align*}
        \alpha_0 + \alpha_3   & = 0 \\
        \alpha_1 -\alpha_2 i  & = 0 \\
        \alpha_1 + \alpha_2 i & = 0 \\
        \alpha_0 -\alpha_3    & = 0
      \end{align*}
      De la primera y la cuarta obtenemos $\alpha_0=\alpha_3=-\alpha_3$ y por tanto $\alpha_0=\alpha_3=0$.
      De la segunda y la tercera obtenemos $\alpha_1=\alpha_2 i=-\alpha_2 i$ y por tanto $\alpha_1=\alpha_2=0$.
      \part Si substituimos la matriz cero por la matriz $A$ en la ecuación del apartado a), obtenemos el sistema de ecuaciones
      \begin{align*}
        \alpha_0 + \alpha_3   & = 2+i \\
        \alpha_1 -\alpha_2 i  & = 1-i \\
        \alpha_1 + \alpha_2 i & = 1+i \\
        \alpha_0 -\alpha_3    & = 2-i
      \end{align*}
      Sumando la segunda y la tercera ecuación obtenemos $2\alpha_1 = 2$, por lo que $\alpha_1 = 1$.
      Restando la segunda de la tercera ecuación obtenemos $2\alpha_2 i = 2i$, por lo que $\alpha_2 = 1$.
      Sumando la primera y la cuarta ecuación obtenemos $2\alpha_0 = 4$, por lo que $\alpha_0 = 2$.
      Restando la primera de la cuarta ecuación obtenemos $2\alpha_3 = 0$, por lo que $\alpha_3 = 0$.
      Por lo tanto, las coordenadas de la matriz $A$ en la base $\mathcal{B}$ son $(2, 1, 1, 0)$.
    \end{parts}
  \end{solution}

  \question
  Sean $V$ y $W$ espacios vectoriales.
  Demostrar que
  \[
    \dim(V \times W) = \dim(V) \dim(W)\,.
  \]
  \begin{solution}
    Sean $\mathcal{B} = \{v_1, \ldots, v_n\}$ una base de $V$ y $\mathcal{B}' = \{w_1, \ldots, w_m\}$ una base de $W$. Vamos a demostrar que
    \[
      \mathcal{B} \times \mathcal{B}' = \{(v_i, w_j) \mid v_i \in \mathcal{B}, w_j \in \mathcal{B}'\}\,,
    \]
    es una base de $V \times W$.
    Para demostrar que es un conjunto generador, sea un elemento $(v,w) \in V \times W$. Entonces $v \in V$ y $w \in W$. Como $\mathcal{B}$ es una base de $V$, existen $\alpha_1, \ldots, \alpha_n \in \mathbb{C}$ tales que
    \[
      v = \alpha_1 v_1 + \cdots + \alpha_n v_n\,.
    \]
    Como $\mathcal{B}'$ es una base de $W$, existen $\beta_1, \ldots, \beta_m \in \mathbb{C}$ tales que
    \[
      w = \beta_1 w_1 + \cdots + \beta_m w_m\,.
    \]
    Por lo tanto,
    \begin{align*}
      (v,w) & = (\alpha_1 v_1 + \cdots + \alpha_n v_n, \beta_1 w_1 + \cdots + \beta_m w_m)  \\
            & = \alpha_1 \beta_1 (v_1, w_1) + \cdots + \alpha_1 \beta_m (v_1, w_m) + \cdots \\
            & + \alpha_n \beta_1 (v_n, w_1) + \cdots + \alpha_n \beta_m (v_n, w_m)\,.
    \end{align*}
    Por lo tanto, $\mathcal{B} \times \mathcal{B}'$ es un conjunto generador de $V \times W$.
    Para demostrar que es linealmente independiente, sea una combinación lineal de los elementos de $\mathcal{B} \times \mathcal{B}'$ igualada a cero
    \begin{align*}
      (0,0) = \sum_{i=1}^n \sum_{j=1}^m \alpha_{ij} (v_i, w_j) = \sum_{i=1}^n (v_i, \sum_{j=1}^m \alpha_{ij} w_j) = (\sum_{i=1}^n v_i, \sum_{i=1}^n \sum_{j=1}^m \alpha_{ij} w_j)\,.
    \end{align*}
    La primera coordenada nos dice que $\sum_{i=1}^n v_i = 0$ y como $\mathcal{B}$ es una base de $V$, esta expresión es imposible.
    Por lo tanto, $\mathcal{B} \times \mathcal{B}'$ es linealmente independiente.


  \end{solution}

  \question Demostrar que si $V$ y $W$ son subespacios de $\C^3$ y $\dim(V) + \dim(W) = 4$, entonces $V \cap W \neq \{0\}$.
  \begin{solution}
    Como $V$ y $W$ son subespacios de $\C^3$, tenemos que $V \cup W$ es un subespacio de $\C^3$, por lo que $\dim(V \cup W) \leq 3$.
    Usando el teorema de la dimensión, tenemos
    \[
      \dim(V \cap W) = \dim(V) + \dim(W) - \dim(V \cup W) \geq 4-3 = 1\,.
    \]
  \end{solution}

  \question Sean $\mathcal{B}$ y $\mathcal{B}^\prime$ bases de $\C^2$ formadas por
  \[
    \mathcal{B} = \left\{\mqty(1\\ i), \mqty(i\\ 1)\right\},\quad
    \mathcal{B}^\prime = \left\{\mqty(i\\ 0), \mqty(0\\ i)\right\}
  \]
  \begin{parts}
    \part Hallar la matriz cambio de coordenadas de $\mathcal{B}$ a $\mathcal{B}^\prime$.
    \part Hallar la matriz cambio de coordenadas de $\mathcal{B}^\prime$ a $\mathcal{B}$.
    \part Compruebe que la matriz cambio de coordenadas de $\mathcal{B}$ a $\mathcal{B}^\prime$ y la matriz cambio de coordenadas de $\mathcal{B}^\prime$ a $\mathcal{B}$ son inversas.
    \part Hallar la matriz cambio de base de $\mathcal{B}$ a $\mathcal{B}^\prime$.
  \end{parts}
  \begin{solution}
    \begin{parts}
      \part
      \begin{align*}
        M_\mathcal{B}^{\mathcal{B}^\prime} & = M_\mathcal{B}^\mathcal{C} M_\mathcal{C}^{\mathcal{B}^\prime} = M_\mathcal{B}^\mathcal{C} (M_\mathcal{B}^\mathcal{C})^{-1}      \\
                                           & = \mqty(1                                                                                                                   & i  \\ i & 1) \mqty(i & 0 \\ 0 & i)^{-1}\\
                                           & = -\mqty(1                                                                                                                  & i  \\ i & 1) \mqty(i & 0 \\ 0 & i)\\
                                           & = -\mqty(i                                                                                                                  & -1 \\ -1 & i)\,.
      \end{align*}
      \part
      \begin{align*}
        M_{\mathcal{B}^\prime}^{\mathcal{B}} & = M_{\mathcal{B}^\prime}^\mathcal{C} M_\mathcal{C}^{\mathcal{B}} = M_{\mathcal{B}^\prime}^\mathcal{C} (M_\mathcal{B}^\mathcal{C})^{-1}     \\
                                             & = \mqty(i                                                                                                                              & 0 \\ 0 & i) \mqty(1 & i \\ i & 1)^{-1}\\
                                             & = \frac{1}{2}\mqty(i                                                                                                                   & 0 \\ 0 & i) \mqty(1 & -i \\ -i & 1)\\
                                             & = \frac{1}{2}\mqty(i                                                                                                                   & 1 \\ 1 & i)\,.
      \end{align*}
      \part
      \begin{align*}
        M_\mathcal{B}^{\mathcal{B}^\prime} M_{\mathcal{B}^\prime}^{\mathcal{B}} & = -\mqty(i             & -1 \\ -1 & i)\frac{1}{2}\mqty(i & 1 \\ 1 & i)\\
                                                                                & = -\frac{1}{2}\mqty(-2 & 0  \\ 0 & -2)\\
                                                                                & = \mqty(1              & 0  \\ 0 & 1)\,.
      \end{align*}
      \part
      \begin{align*}
        M_{\mathcal{B}\mathcal{B^\prime}} & = \left(M_{\mathcal{C}}^{\mathcal{B}} M_{\mathcal{B^\prime}}^{\mathcal{C}}\right)^t = \left((M_{\mathcal{B}}^{\mathcal{C}})^{-1} M_{\mathcal{B^\prime}}^{\mathcal{C}}\right)^t      \\
                                          & = \left(\mqty(1                                                                                                                                                                & i  \\ i & 1)^{-1} \mqty(i & 0 \\ 0 & i)\right)^t\\
                                          & = \left(\frac{1}{2}\mqty(1                                                                                                                                                     & -i \\ -i & 1) \mqty(i & 0 \\ 0 & i)\right)^t\\
                                          & = \left(\frac{1}{2}\mqty(i                                                                                                                                                     & -1 \\ -1 & i)\right)^t\\
                                          & = \frac{1}{2}\mqty(i                                                                                                                                                           & -1 \\ -1 & i)\,.
      \end{align*}
    \end{parts}
  \end{solution}

\end{questions}