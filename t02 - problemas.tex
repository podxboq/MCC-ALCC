\unirsection{Problemas}

\begin{questions}


  \question Determinar si los siguientes conjuntos son subespacios de $\C^3$:
  \begin{parts}
    \part $W_1 = \left\{\begin{pmatrix} x \\ y \\ z \end{pmatrix} : x + 2y - z = 1\right\}$
    \part $W_2 = \left\{\begin{pmatrix} x \\ y \\ z \end{pmatrix} : x = \conj{z}\right\}$
    \part $W_3 = \left\{\begin{pmatrix} x \\ y \\ z \end{pmatrix} : |x|^2 + |y|^2 + |z|^2 = 1\right\}$
  \end{parts}


  \question
  Encontrar una base y la dimensión del subespacio de $\C^4$ generado por:
  $$\left\{\begin{pmatrix} 1 \\ i \\ 0 \\ 1 \end{pmatrix}, \begin{pmatrix} i \\ 1 \\ 1 \\ 0 \end{pmatrix}, \begin{pmatrix} 1+i \\ 1+i \\ 1 \\ 1 \end{pmatrix}\right\}$$


  \question
  Verificar que el conjunto $\left\{\begin{pmatrix} 1 \\ 1 \\ 0 \end{pmatrix}, \begin{pmatrix} 1 \\ i \\ 0 \end{pmatrix}, \begin{pmatrix} 0 \\ 0 \\ 1 \end{pmatrix}\right\}$ es linealmente independiente en $\C^3$ y extenderlo a una base de $\C^3$.


  \question
  Calcular el producto escalar hermítico y las normas de:
  $\mathbf{u} = \begin{pmatrix} 1+i \\ 2 \\ 3-i \end{pmatrix}, \quad \mathbf{v} = \begin{pmatrix} 2-i \\ 1+i \\ 1 \end{pmatrix}$


  \question
  Demostrar que si $\mathbf{v}_1, \mathbf{v}_2, \ldots, \mathbf{v}_k$ son linealmente independientes en un espacio vectorial $V$, entonces $\mathbf{v}_1, \mathbf{v}_1 + \mathbf{v}_2, \mathbf{v}_1 + \mathbf{v}_2 + \mathbf{v}_3, \ldots, \mathbf{v}_1 + \mathbf{v}_2 + \cdots + \mathbf{v}_k$ también son linealmente independientes.

  \question
  Sea $V = \mathcal{P}_2(\C)$ el espacio de polinomios complejos de grado menor o igual que 2.
  \begin{parts}
    \part Demostrar que $\{1, z, z^2\}$ es una base de $V$
    \part Encontrar las coordenadas del polinomio $p(z) = (1+i) + 2iz - z^2$ en esta base
    \part Proponer otra base para $V$ y expresar $p(z)$ en ella
  \end{parts}


  \question
  En el espacio $\C^{2 \times 2}$ de matrices complejas $2 \times 2$:
  \begin{parts}
    \part Verificar que las matrices de Pauli junto con la identidad forman una base:
    $\mathcal{B} = \left\{I = \begin{pmatrix} 1 & 0 \\ 0 & 1 \end{pmatrix}, \sigma_x = \begin{pmatrix} 0 & 1 \\ 1 & 0 \end{pmatrix}, \sigma_y = \begin{pmatrix} 0 & -i \\ i & 0 \end{pmatrix}, \sigma_z = \begin{pmatrix} 1 & 0 \\ 0 & -1 \end{pmatrix}\right\}$
    \part Expresar la matriz $A = \begin{pmatrix} 2+i & 1-i \\ 1+i & 2-i \end{pmatrix}$ como combinación lineal de esta base
  \end{parts}


  \question
  Sean $V$ y $W$ espacios vectoriales complejos de dimensiones finitas $m$ y $n$ respectivamente.
  \begin{parts}
    \part Demostrar que $\dim(V \times W) = \dim(V) + \dim(W)$
    \part Si $f: V \to W$ es una transformación lineal, demostrar que $\dim(V) = \dim(\Ker(f)) + \dim(\Ima(f))$
  \end{parts}

  \question Verificar que los siguientes vectores representan estados cuánticos válidos:
  \begin{parts}
    \part $\ket{\psi_1} = \frac{3}{5}\ket{0} + \frac{4i}{5}\ket{1}$
    \part $\ket{\psi_2} = \frac{1}{\sqrt{3}}\ket{0} + \frac{\sqrt{2}}{\sqrt{3}}\ket{1}$
    \part $\ket{\psi_3} = \frac{1+i}{2}\ket{0} + \frac{1-i}{2}\ket{1}$
  \end{parts}

  \question Para el estado cuántico $\ket{\psi} = \cos\frac{\pi}{6}\ket{0} + e^{i\pi/3}\sin\frac{\pi}{6}\ket{1}$:
  \begin{parts}
    \part Escribir el estado en forma matricial
    \part Calcular las probabilidades de medir $\ket{0}$ y $\ket{1}$
    \part Determinar los valores de $\theta$ y $\varphi$ en la parametrización estándar
  \end{parts}
\end{questions}