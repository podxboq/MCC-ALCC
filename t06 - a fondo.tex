\unirsection{A fondo}

Los siguientes libros pueden servir de material de apoyo y para profundizar más en los contenidos de este tema.

\textbf{Sakurai, J. J. y Napolitano, J. (2017). Modern Quantum Mechanics (2.ª ed.). Cambridge University Press.}

En la sección 1.2 presenta la notación de Dirac desde el principio, desarrollando cada concepto con la notación como única forma de representar el álgebra, lo que permite conocer todas las propiedades con esta forma de escritura.

\textbf{Nielsen, M. A. y Chuang, I. L. (2010). Quantum Computation and Quantum Information (10.ª ed.). Cambridge University Press.}

En el capítulo 2 introduce la notación de Dirac y presenta una tabla resumen con las equivalencias entre la notación matricial y la notación de Dirac.


