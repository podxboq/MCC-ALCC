\documentclass[aspectratio=169]{beamer}
\usetheme{Unir}
\usepackage{float}
\usepackage{array}
\usepackage{multirow}
\usepackage{physics}
\newcommand{\tq}{\mid}
\newcommand{\K}{\mathrm{K}}
\newcommand{\V}{\mathrm{V}}
\newcommand{\N}{\mathbb{N}}
\newcommand{\Z}{\mathbb{Z}}
\newcommand{\F}{\mathbb{F}}
\newcommand{\Q}{\mathbb{Q}}
\newcommand{\R}{\mathbb{R}}
\newcommand{\C}{\mathbb{C}}
\renewcommand{\H}{\mathcal{H}}
\newcommand{\Cinf}{\mathcal{C}^\infty}
\newcommand{\Cu}{\mathcal{C}^1}
\newcommand{\Rp}{\mathfrak{Re}}
\newcommand{\Ip}{\mathfrak{Im}}
\renewcommand{\d}{\mathrm{d}}
\newcommand{\dm}{\mathrm{d}\mu}
\newcommand{\conj}[1]{\overline{#1}}
\newcommand{\Tras}[1]{{#1}^{\text{T}}}

\DeclareMathOperator{\Log}{Log}
\DeclareMathOperator{\Arg}{Arg}
\DeclareMathOperator{\Dom}{Dom}
\DeclareMathOperator{\Ima}{Im}
\DeclareMathOperator{\sgn}{sgn}
\DeclareMathOperator{\mcd}{MCD}
\DeclareMathOperator{\mcm}{mcm}
\DeclareMathOperator{\Resi}{Res}
\DeclareMathOperator{\Ker}{Ker}
\DeclareMathOperator{\End}{End}
\DeclareMathOperator{\Mat}{Mat}

\newcommand{\parder}[2]{\frac{\partial #1}{\partial #2}}
\newcommand{\dparder}[2]{\dfrac{\partial #1}{\partial #2}}
\newcommand{\tparder}[2]{\partial #1/\partial #2}
\newcommand{\parderr}[3]{\frac{\partial^2 #1}{\partial #2\partial #3}}
\newcommand{\dparderr}[3]{\dfrac{\partial^2 #1}{\partial #2\partial #3}}
\newcommand{\tparderr}[3]{\partial^2 #1/\partial #2\partial #3}
\newcommand{\intx}[1]{\int #1\,dx}
\newcommand{\intt}[1]{\int #1\,dt}
\newcommand{\intdx}[3]{\int_{#1}^{#2} #3\,dx}
\newcommand{\intdt}[3]{\int_{#1}^{#2} #3\,dt}
\newcommand{\intdz}[2]{\int_{#1} #2\,dz}
\newcommand{\set}[1]{\left\{#1\right\}}
\newcommand{\so}{\Rightarrow}
\newcommand{\sii}{\Leftrightarrow}
\newcommand{\by}[1]{\overset{\fbox{\tiny #1}}{=}}
\newcommand{\byref}[1]{\overset{\fbox{\tiny\ref{#1}}}{=}}
\newcommand{\cardinal}[1]{\left|#1\right|}
\newcommand{\maps}[3]{#1 \colon #2\longrightarrow #3}
\newcommand{\equationmaps}[5]{\begin{aligned}[t]#1 \colon #2 &\longrightarrow #3 \\	#4 &\longmapsto #5\end{aligned}}
\newcommand{\coma}{,\thinspace}
\newcommand{\pari}[2]{(#1,\thinspace #2)}
\newcommand{\where}{\mathrel{}\middle|\mathrel{}}
\newcommand{\no}[1]{{\neg}{#1}}
\newcommand{\dcomilla}[1]{{\guillemotleft}#1{\guillemotright}}
\newcommand{\separa}{\vspace*{.75\baselineskip}}
\newcommand{\semisepara}{\vspace*{.25\baselineskip}}
\newcommand{\restrict}[1]{\raisebox{-.5ex}{$|$}_{#1}}

% ========================================
% COMANDOS PERSONALIZADOS PARA COMPUTACIÓN CUÁNTICA
% ========================================

% Estados comunes
\newcommand{\zero}{\ket{0}}
\newcommand{\one}{\ket{1}}
\newcommand{\plus}{\ket{+}}
\newcommand{\minus}{\ket{-}}

% Operadores especiales
\newcommand{\tensor}{\otimes}         % Producto tensorial
\newcommand{\comp}{\circ}             % Composición

% Estados de Bell
\newcommand{\bellphi}{\ket{\Phi^+}}
\newcommand{\bellpsi}{\ket{\Psi^+}}
\newcommand{\bellphiminus}{\ket{\Phi^-}}
\newcommand{\bellpsiminus}{\ket{\Psi^-}}

\newcommand{\floor}[1]{\left\lfloor #1 \right\rfloor}
\newcommand{\ceil}[1]{\left\lceil #1 \right\rceil}


\title[5 Postulados de la Mecánica Cuántica]{Notación de Dirac: Los 5 postulados}
\subtitle{Conexión entre matemáticas y física cuántica}
\author[F. Costa Cano]{Dr. Francisco Costa Cano}
\institute[UNIR]{Universidad Internacional de La Rioja}
\date{Enero 2026}

\begin{document}
\maketitle

\section{Postulado I: Sistemas Cuánticos}
\begin{frame}{Postulado I: Sistemas Cuánticos}
  \begin{columns}[T]
    \begin{column}{0.5\textwidth}
      \textbf{Representación Matemática:}
      \begin{itemize}
        \item Espacio de Hilbert $\H$.
        \item Vector $u \in \H$.
        \item $u,v \in \H\Rightarrow \langle u,v \rangle\in\C$.\pause
      \end{itemize}
    \end{column}
    \begin{column}{0.5\textwidth}
      \textbf{Representación Dirac:}
      \begin{itemize}
        \item Espacio de Hilbert $\H$.
        \item Vector $\ket{u} \in \H$ (Ket).
        \item Unitario $\|\ket{u}\| = 1$.
        \item $\ket{u},\ket{v} \in \H\Rightarrow \braket{u}{v}\in\C$.\pause
      \end{itemize}
    \end{column}
  \end{columns}

  \separa
  \textbf{Computación cuántica:}

  \separa
  \textbf{Cúbit}: Espacio de Hilbert de dimensión 2. $\mathcal{C} \subset \C^2$, con base canónica $\{\ket{0}, \ket{1}\}$.

  Un ket $\ket{u} = \alpha \ket{0} + \beta \ket{1}$, con $\alpha, \beta \in \C$.

  $\|\ket{u}\| = 1 \Rightarrow |\alpha|^2 + |\beta|^2 = 1$.

\end{frame}

\section{Postulado II: Observables}
\begin{frame}{Postulado II: Observables}
  \begin{columns}[T]
    \begin{column}{0.5\textwidth}
      \textbf{Representación Matemática:}
      \begin{itemize}
        \item Operador hermitiano $O$.
        \item $O^\dagger = O$.
        \item Valores propios reales.
        \item Autovectores ortonormales.
        \item $v\in\H\Rightarrow v^*\in\H^*$. $v^* = \langle v,\cdot\rangle$.
        \item $[v^*] = (\bar{v}_1, \dots, \bar{v}_n)^t$
        \item producto externo $v\wedge w$.\pause
      \end{itemize}
    \end{column}
    \begin{column}{0.5\textwidth}
      \textbf{Representación Dirac:}
      \begin{itemize}
        \item Autovector normalizado $\ket{\lambda_i}$.
        \item $\ket{v}\in\H\Rightarrow \bra{v}\in\H^*$ (Bra). $\bra{v} = \braket{v}{\cdot}$.
        \item $\bra{v}(\ket{w}) = \braket{v}{w}$.
        \item $\bra{v} = \ket{v}^\dagger$.
        \item producto externo $\ketbra{v}{w}$.
        \item $O = \sum_{i}\lambda_i \ket{\lambda_i}\bra{\lambda_i}$.\pause
      \end{itemize}
    \end{column}
  \end{columns}
  \separa
  \textbf{Computación cuántica:}

  \separa
  \textbf{Observables}: Matrices $2\times 2$ hermitianos.

  Las matrices de Pauli son observables.


\end{frame}

\section{Postulado III: Medición}
\begin{frame}{Postulado III: Medición}
  \begin{columns}[T]
    \begin{column}{0.5\textwidth}
      \textbf{Representación Matemática:}
      \begin{itemize}
        \item proyección sobre autovectores:
              \[w = \frac{\text{proy}_{V_i}v}{\|\text{proy}_{V_i}v\|}\,.\]\pause
      \end{itemize}
    \end{column}
    \begin{column}{0.5\textwidth}
      \textbf{Representación Dirac:}
      \begin{itemize}
        \item Probabilidad: $P(\lambda_i) = \|P_i\ket{v}\|^2$.
        \item Estado post-medición:
              \[
                \ket{w} = \frac{P_i\ket{v}}{\sqrt{P(\lambda_i)}}\,.
              \]\pause
      \end{itemize}
    \end{column}
  \end{columns}

  \separa
  \textbf{Computación cuántica:}

  \separa
  \textbf{Mediciones sobre X}: posibles valores $-1$ quedando el estado en $\ket{0}$ y $1$ quedando el estado en $\ket{1}$.

\end{frame}

\section{Postulado IV: Evolución Temporal}
\begin{frame}{Postulado IV: Evolución Temporal}
  \begin{columns}[T]
    \begin{column}{0.5\textwidth}
      \textbf{Representación Matemática:}
      \begin{itemize}
        \item Ecuación de Schrödinger
              \[i\hbar \frac{\partial \psi}{\partial t} = H \psi\,.\]
        \item Solución: $\psi(t) = e^{-iHt/\hbar}\psi(0)$.
        \item Operador unitario $U(t) = e^{-iHt/\hbar}$.
        \item Conservación de norma.\pause
      \end{itemize}
    \end{column}
    \begin{column}{0.5\textwidth}
      \textbf{Representación de Dirac:}
      \begin{itemize}
        \item Ecuación de Schrödinger
              \[i\hbar \frac{\partial \ket{\psi}}{\partial t} = H \ket{\psi}\,.\]
        \item Solución: $\ket{\psi(t)} = e^{-iHt/\hbar}\ket{\psi(0)}$.
        \item Operador unitario $U(t) = e^{-iHt/\hbar}$.
        \item Conservación de norma\pause
      \end{itemize}
    \end{column}
  \end{columns}

  \separa
  \textbf{Computación cuántica:}

  \separa
  \textbf{Puertas cuánticas:}
  Puerta de Hadamard: $H = \frac{1}{\sqrt{2}}\begin{pmatrix} 1 & 1 \\ 1 & -1 \end{pmatrix}$

\end{frame}

\section{Postulado V: Sistemas Compuestos}
\begin{frame}{Postulado V: Sistemas Compuestos}
  \begin{columns}[T]
    \begin{column}{0.5\textwidth}
      \textbf{Representación Matemática:}
      \begin{itemize}
        \item Producto tensorial $\otimes$.
        \item Estado compuesto: $v\otimes w$.
        \item Estados separables vs no separables.\pause
      \end{itemize}
    \end{column}
    \begin{column}{0.5\textwidth}
      \textbf{Interpretación Cuántica:}
      \begin{itemize}
        \item Producto tensorial.
        \item Estado compuesto: $\ket{vw}$.
        \item Estados separables vs entrelazados.\pause
      \end{itemize}
    \end{column}
  \end{columns}

  \separa
  \textbf{Computación cuántica:}

  \separa
  en $\mathcal{C}^2$ base canónica: $\{\ket{00}, \ket{01}, \ket{10}, \ket{11}\}$.
  Estado entrelazado: $\frac{\ket{00} + \ket{11}}{\sqrt{2}}$.

\end{frame}

\section{Conclusiones}
\begin{frame}{Conclusiones}
  Los 5 postulados describen completamente sistemas cuánticos:
  \begin{itemize}
    \item Notación de Dirac unifica representación matemática y física
    \item Cada postulado tiene interpretación profunda
    \item La mecánica cuántica desafía intuiciones clásicas
  \end{itemize}

  La mecánica cuántica no es solo una teoría, ¡es un nuevo paradigma de comprensión de la realidad!
\end{frame}

\end{document}