\maketitulo

\section{Objetivos y pautas de elaboración}

El principal objetivo de esta actividad es desarrollar la capacidad de análisis de sistemas cuánticos compuestos, entrelazamiento y mediciones en espacios de Hilbert.

Con esta actividad se pretende que el estudiante:
\begin{itemize}
  \item Trabaje con la complejidad conceptual del entrelazamiento cuántico.
  \item Desarrolle práctica en los cálculos con productos tensoriales.
  \item Entienda que la interpretación física de resultados matemáticos es crucial.
  \item Es fundamental distinguir estados separables de estados entrelazados mediante argumentos rigurosos.
\end{itemize}

\section{Problemas}

\begin{questions}
  \question \textbf{Ortogonalización y proyecciones}

  \begin{parts}
    \part Aplicar el proceso de Gram-Schmidt para ortogonalizar el conjunto de vectores en $\C^3$:
    $$\left\{\begin{pmatrix} 1 \\ 0 \\ i \end{pmatrix}, \begin{pmatrix} i \\ 1 \\ 0 \end{pmatrix}, \begin{pmatrix} 0 \\ i \\ 1 \end{pmatrix}\right\}$$

    \begin{solution}
      \textbf{Paso 1:} $\vec{u}_1 = \vec{v}_1 = \begin{pmatrix} 1 \\ 0 \\ i \end{pmatrix}$

      Normalizamos: $\|\vec{u}_1\| = \sqrt{1 + 0 + 1} = \sqrt{2}$, entonces $\vec{e}_1 = \frac{1}{\sqrt{2}}\begin{pmatrix} 1 \\ 0 \\ i \end{pmatrix}$

      \textbf{Paso 2:}
      $$\langle \vec{v}_2, \vec{u}_1 \rangle = \begin{pmatrix} -i & 1 & 0 \end{pmatrix}\begin{pmatrix} 1 \\ 0 \\ i \end{pmatrix} = -i$$
      $$\langle \vec{u}_1, \vec{u}_1 \rangle = 2$$
      $$\vec{u}_2 = \vec{v}_2 - \frac{\langle \vec{v}_2, \vec{u}_1 \rangle}{\langle \vec{u}_1, \vec{u}_1 \rangle}\vec{u}_1 = \begin{pmatrix} i \\ 1 \\ 0 \end{pmatrix} - \frac{-i}{2}\begin{pmatrix} 1 \\ 0 \\ i \end{pmatrix} = \begin{pmatrix} i \\ 1 \\ 0 \end{pmatrix} + \frac{i}{2}\begin{pmatrix} 1 \\ 0 \\ i \end{pmatrix}$$
      $$= \begin{pmatrix} i + \frac{i}{2} \\ 1 \\ -\frac{1}{2} \end{pmatrix} = \begin{pmatrix} \frac{3i}{2} \\ 1 \\ -\frac{1}{2} \end{pmatrix}$$

      Normalizamos: $\|\vec{u}_2\| = \sqrt{\frac{9}{4} + 1 + \frac{1}{4}} = \sqrt{\frac{14}{4}} = \frac{\sqrt{14}}{2}$

      $$\vec{e}_2 = \frac{2}{\sqrt{14}}\begin{pmatrix} \frac{3i}{2} \\ 1 \\ -\frac{1}{2} \end{pmatrix} = \frac{1}{\sqrt{14}}\begin{pmatrix} 3i \\ 2 \\ -1 \end{pmatrix}$$

      \textbf{Paso 3:} Se calcula $\vec{u}_3$ ortogonalizando $\vec{v}_3$ respecto a $\vec{u}_1$ y $\vec{u}_2$ de manera similar.
    \end{solution}

    \part Sea $W = \text{gen}\left\{\begin{pmatrix} 1 \\ i \\ 0 \end{pmatrix}, \begin{pmatrix} 0 \\ 1 \\ i \end{pmatrix}\right\} \subset \C^3$. Calcular la proyección ortogonal de $\vec{v} = \begin{pmatrix} 1 \\ 1 \\ 1 \end{pmatrix}$ sobre $W$.

    \begin{solution}
      Primero ortogonalizamos la base de $W$ usando Gram-Schmidt:

      $\vec{w}_1 = \begin{pmatrix} 1 \\ i \\ 0 \end{pmatrix}$, $\|\vec{w}_1\| = \sqrt{2}$, $\vec{e}_1 = \frac{1}{\sqrt{2}}\begin{pmatrix} 1 \\ i \\ 0 \end{pmatrix}$

      $$\langle \vec{v}_2, \vec{w}_1 \rangle = \begin{pmatrix} 0 & -i & -i \end{pmatrix}\begin{pmatrix} 1 \\ i \\ 0 \end{pmatrix} = 1$$
      $$\vec{w}_2 = \begin{pmatrix} 0 \\ 1 \\ i \end{pmatrix} - \frac{1}{2}\begin{pmatrix} 1 \\ i \\ 0 \end{pmatrix} = \begin{pmatrix} -\frac{1}{2} \\ 1 - \frac{i}{2} \\ i \end{pmatrix}$$

      (Se normaliza $\vec{w}_2$ para obtener $\vec{e}_2$)

      La proyección es:
      $$\text{proj}_W(\vec{v}) = \langle \vec{v}, \vec{e}_1 \rangle\vec{e}_1 + \langle \vec{v}, \vec{e}_2 \rangle\vec{e}_2$$
    \end{solution}

    \part Para el observable $A = \begin{pmatrix} 2 & 1-i \\ 1+i & 3 \end{pmatrix}$, verificar que es hermitiano y encontrar su descomposición espectral.

    \begin{solution}
      Verificamos hermiticidad:
      $$A^\dagger = \begin{pmatrix} 2 & 1+i \\ 1-i & 3 \end{pmatrix}^* = \begin{pmatrix} 2 & 1-i \\ 1+i & 3 \end{pmatrix} = A$$ ✓

      Polinomio característico:
      $$\det(A - \lambda I) = (2-\lambda)(3-\lambda) - |1-i|^2 = \lambda^2 - 5\lambda + 6 - 2 = \lambda^2 - 5\lambda + 4$$

      Valores propios: $\lambda_1 = 4$, $\lambda_2 = 1$

      Para $\lambda_1 = 4$: $(A - 4I)\vec{v} = 0$ da $\vec{v}_1 = \begin{pmatrix} 1-i \\ 2 \end{pmatrix}$ (normalizado: $\ket{e_1} = \frac{1}{\sqrt{6}}\begin{pmatrix} 1-i \\ 2 \end{pmatrix}$)

      Para $\lambda_2 = 1$: $\vec{v}_2 = \begin{pmatrix} -1-i \\ 1 \end{pmatrix}$ (normalizado: $\ket{e_2} = \frac{1}{\sqrt{3}}\begin{pmatrix} -1-i \\ 1 \end{pmatrix}$)

      Descomposición espectral:
      $$A = 4\ketbra{e_1}{e_1} + 1\ketbra{e_2}{e_2}$$
    \end{solution}

    \part Calcular el valor esperado $\langle A \rangle$ para el estado $\ket{\psi} = \frac{1}{\sqrt{3}}(\ket{0} + \sqrt{2}\ket{1})$ si el observable es $A$ del apartado anterior.

    \begin{solution}
      El valor esperado es:
      $$\langle A \rangle = \bra{\psi}A\ket{\psi}$$

      En la base computacional $\{\ket{0}, \ket{1}\}$:
      $$\ket{\psi} = \frac{1}{\sqrt{3}}\begin{pmatrix} 1 \\ \sqrt{2} \end{pmatrix}$$

      $$A\ket{\psi} = \frac{1}{\sqrt{3}}\begin{pmatrix} 2 & 1-i \\ 1+i & 3 \end{pmatrix}\begin{pmatrix} 1 \\ \sqrt{2} \end{pmatrix} = \frac{1}{\sqrt{3}}\begin{pmatrix} 2 + \sqrt{2}(1-i) \\ 1+i + 3\sqrt{2} \end{pmatrix}$$

      $$\langle A \rangle = \frac{1}{3}\begin{pmatrix} 1 & \sqrt{2} \end{pmatrix}\begin{pmatrix} 2 + \sqrt{2}(1-i) \\ 1+i + 3\sqrt{2} \end{pmatrix}$$
      $$= \frac{1}{3}\left[2 + \sqrt{2}(1-i) + \sqrt{2}(1+i) + 6\right] = \frac{1}{3}\left[8 + 2\sqrt{2}\right] = \frac{8 + 2\sqrt{2}}{3}$$
    \end{solution}
  \end{parts}

  \question \textbf{Producto tensorial y sistemas compuestos}

  \begin{parts}
    \part Calcular explícitamente el producto tensorial $(X \otimes Z)\ket{01}$ donde $X = \begin{pmatrix} 0 & 1 \\ 1 & 0 \end{pmatrix}$ y $Z = \begin{pmatrix} 1 & 0 \\ 0 & -1 \end{pmatrix}$.

    \begin{solution}
      Usando la propiedad del producto tensorial:
      $$(X \otimes Z)\ket{01} = (X \otimes Z)(\ket{0} \otimes \ket{1}) = X\ket{0} \otimes Z\ket{1}$$

      Calculamos cada parte:
      $$X\ket{0} = \begin{pmatrix} 0 & 1 \\ 1 & 0 \end{pmatrix}\begin{pmatrix} 1 \\ 0 \end{pmatrix} = \begin{pmatrix} 0 \\ 1 \end{pmatrix} = \ket{1}$$
      $$Z\ket{1} = \begin{pmatrix} 1 & 0 \\ 0 & -1 \end{pmatrix}\begin{pmatrix} 0 \\ 1 \end{pmatrix} = \begin{pmatrix} 0 \\ -1 \end{pmatrix} = -\ket{1}$$

      Por tanto:
      $$(X \otimes Z)\ket{01} = \ket{1} \otimes (-\ket{1}) = -\ket{11} = -\begin{pmatrix} 0 \\ 0 \\ 0 \\ 1 \end{pmatrix}$$

      \textbf{Verificación usando la matriz completa:}
      $$X \otimes Z = \begin{pmatrix} 0 \cdot Z & 1 \cdot Z \\ 1 \cdot Z & 0 \cdot Z \end{pmatrix} = \begin{pmatrix} 0 & 0 & 1 & 0 \\ 0 & 0 & 0 & -1 \\ 1 & 0 & 0 & 0 \\ 0 & -1 & 0 & 0 \end{pmatrix}$$
      $$\ket{01} = \begin{pmatrix} 0 \\ 1 \\ 0 \\ 0 \end{pmatrix} \Rightarrow (X \otimes Z)\ket{01} = \begin{pmatrix} 0 \\ 0 \\ 0 \\ -1 \end{pmatrix} = -\ket{11}$$ ✓
    \end{solution}

    \part Para el sistema de dos qubits en el estado $\ket{\psi} = \frac{1}{\sqrt{3}}(\ket{00} + \ket{01} + \ket{11})$, verificar normalización y calcular las probabilidades de medir cada estado de la base computacional.

    \begin{solution}
      Verificación de normalización:
      $$\braket{\psi}{\psi} = \frac{1}{3}(\braket{00}{00} + \braket{01}{01} + \braket{11}{11} + \text{términos cruzados})$$

      Como los estados de la base son ortogonales, los términos cruzados son cero:
      $$\braket{\psi}{\psi} = \frac{1}{3}(1 + 1 + 1) = 1$$ ✓

      Probabilidades:
      \begin{align*}
        P(\ket{00}) & = |\braket{\psi}{00}|^2 = \left|\frac{1}{\sqrt{3}}\right|^2 = \frac{1}{3} \approx 33.3\% \\
        P(\ket{01}) & = |\braket{\psi}{01}|^2 = \frac{1}{3} \approx 33.3\%                                     \\
        P(\ket{10}) & = |\braket{\psi}{10}|^2 = 0                                                              \\
        P(\ket{11}) & = |\braket{\psi}{11}|^2 = \frac{1}{3} \approx 33.3\%
      \end{align*}

      Suma: $\frac{1}{3} + \frac{1}{3} + 0 + \frac{1}{3} = 1$ ✓
    \end{solution}

    \part Aplicar la puerta CNOT al estado del apartado anterior y determinar el estado resultante.

    \begin{solution}
      La puerta CNOT actúa como:
      $$\text{CNOT}\ket{ab} = \ket{a, b \oplus a}$$

      Aplicando a cada término:
      \begin{align*}
        \text{CNOT}\ket{00} & = \ket{0, 0 \oplus 0} = \ket{00} \\
        \text{CNOT}\ket{01} & = \ket{0, 1 \oplus 0} = \ket{01} \\
        \text{CNOT}\ket{11} & = \ket{1, 1 \oplus 1} = \ket{10}
      \end{align*}

      Estado resultante:
      $$\ket{\psi'} = \text{CNOT}\ket{\psi} = \frac{1}{\sqrt{3}}(\ket{00} + \ket{01} + \ket{10})$$

      \textbf{Observación:} El primer qubit (control) no cambia, el segundo cambia solo cuando el primero es $\ket{1}$.
    \end{solution}

    \part Calcular la probabilidad de medir el primer qubit en $\ket{0}$ para el estado $\ket{\psi}$ original.

    \begin{solution}
      Para obtener la probabilidad de medir el primer qubit en $\ket{0}$, sumamos las probabilidades de todos los estados que comienzan con $\ket{0}$:

      $$P(\text{primer qubit} = \ket{0}) = P(\ket{00}) + P(\ket{01}) = \frac{1}{3} + \frac{1}{3} = \frac{2}{3}$$

      \textbf{Alternativamente}, usando la traza parcial sobre el segundo qubit:
      $$\rho_A = \text{Tr}_B(\ketbra{\psi}{\psi})$$

      Entonces $P(\ket{0}) = \bra{0}\rho_A\ket{0}$, que da el mismo resultado.
    \end{solution}
  \end{parts}

  \question \textbf{Entrelazamiento cuántico}

  \begin{parts}
    \part Determinar si el estado $\ket{\phi} = \frac{1}{\sqrt{3}}(\ket{00} + \ket{01} + \ket{10})$ es separable o entrelazado.

    \begin{solution}
      Un estado es separable si puede escribirse como $\ket{\phi} = \ket{a} \otimes \ket{b}$.

      Supongamos que existe tal descomposición:
      $$\ket{a} = \alpha_0\ket{0} + \alpha_1\ket{1}, \quad \ket{b} = \beta_0\ket{0} + \beta_1\ket{1}$$

      Entonces:
      $$\ket{a} \otimes \ket{b} = \alpha_0\beta_0\ket{00} + \alpha_0\beta_1\ket{01} + \alpha_1\beta_0\ket{10} + \alpha_1\beta_1\ket{11}$$

      Comparando con $\ket{\phi} = \frac{1}{\sqrt{3}}(\ket{00} + \ket{01} + \ket{10})$:
      \begin{align*}
        \alpha_0\beta_0 & = \frac{1}{\sqrt{3}} \\
        \alpha_0\beta_1 & = \frac{1}{\sqrt{3}} \\
        \alpha_1\beta_0 & = \frac{1}{\sqrt{3}} \\
        \alpha_1\beta_1 & = 0
      \end{align*}

      De las dos últimas ecuaciones: si $\alpha_1\beta_0 = \frac{1}{\sqrt{3}}$, entonces $\alpha_1 \neq 0$ y $\beta_0 \neq 0$.

      Pero entonces $\alpha_1\beta_1 = 0$ implica $\beta_1 = 0$.

      Esto contradice $\alpha_0\beta_1 = \frac{1}{\sqrt{3}}$ (que requiere $\beta_1 \neq 0$).

      \textbf{Conclusión:} El estado es \textbf{entrelazado}.
    \end{solution}

    \part Verificar que los cuatro estados de Bell forman una base ortonormal de $\C^4$.

    \begin{solution}
      Los estados de Bell son:
      \begin{align*}
        \ket{\Phi^+} & = \frac{1}{\sqrt{2}}(\ket{00} + \ket{11}) \\
        \ket{\Phi^-} & = \frac{1}{\sqrt{2}}(\ket{00} - \ket{11}) \\
        \ket{\Psi^+} & = \frac{1}{\sqrt{2}}(\ket{01} + \ket{10}) \\
        \ket{\Psi^-} & = \frac{1}{\sqrt{2}}(\ket{01} - \ket{10})
      \end{align*}

      \textbf{Normalización:} Cada estado tiene norma 1 (verificar uno):
      $$\braket{\Phi^+}{\Phi^+} = \frac{1}{2}(\braket{00}{00} + \braket{11}{11}) = \frac{1}{2}(1 + 1) = 1$$ ✓

      \textbf{Ortogonalidad:} Verificamos algunos productos:
      $$\braket{\Phi^+}{\Phi^-} = \frac{1}{2}(\braket{00}{00} - \braket{11}{11}) = \frac{1}{2}(1 - 1) = 0$$ ✓
      $$\braket{\Phi^+}{\Psi^+} = \frac{1}{2}(\braket{00}{01} + \braket{11}{10}) = 0$$ ✓

      (Se verifican los demás pares de manera similar)

      Como son 4 vectores ortonormales en $\C^4$, forman una base.
    \end{solution}

    \part Para el estado de Bell $\ket{\Phi^+} = \frac{1}{\sqrt{2}}(\ket{00} + \ket{11})$, calcular las probabilidades de medir cada qubit individualmente.

    \begin{solution}
      \textbf{Primer qubit:}
      \begin{align*}
        P(\ket{0}) & = P(\ket{00}) + P(\ket{01}) = \frac{1}{2} + 0 = \frac{1}{2} \\
        P(\ket{1}) & = P(\ket{10}) + P(\ket{11}) = 0 + \frac{1}{2} = \frac{1}{2}
      \end{align*}

      \textbf{Segundo qubit:}
      \begin{align*}
        P(\ket{0}) & = P(\ket{00}) + P(\ket{10}) = \frac{1}{2} + 0 = \frac{1}{2} \\
        P(\ket{1}) & = P(\ket{01}) + P(\ket{11}) = 0 + \frac{1}{2} = \frac{1}{2}
      \end{align*}

      \textbf{Interpretación:} Cada qubit individualmente está completamente aleatorio (50\%-50\%), pero las mediciones están perfectamente correlacionadas: si el primero da $\ket{0}$, el segundo necesariamente da $\ket{0}$ (y lo mismo para $\ket{1}$).

      Esto es la esencia del entrelazamiento cuántico.
    \end{solution}

    \part Demostrar que el estado de Bell $\ket{\Phi^+}$ es maximalmente entrelazado calculando la entropía de entrelazamiento $S(\rho_A)$ donde $\rho_A$ es la matriz de densidad reducida del primer qubit.

    \begin{solution}
      La matriz de densidad del sistema completo es:
      $$\rho_{AB} = \ketbra{\Phi^+}{\Phi^+} = \frac{1}{2}(\ket{00} + \ket{11})(\bra{00} + \bra{11})$$
      $$= \frac{1}{2}(\ketbra{00}{00} + \ketbra{00}{11} + \ketbra{11}{00} + \ketbra{11}{11})$$

      Traza parcial sobre el segundo qubit:
      $$\rho_A = \text{Tr}_B(\rho_{AB}) = \bra{0}_B\rho_{AB}\ket{0}_B + \bra{1}_B\rho_{AB}\ket{1}_B$$
      $$= \frac{1}{2}(\ketbra{0}{0} + \ketbra{1}{1}) = \frac{I}{2}$$

      Los valores propios de $\rho_A$ son $\lambda_1 = \lambda_2 = \frac{1}{2}$.

      Entropía de von Neumann:
      $$S(\rho_A) = -\sum_i \lambda_i \log_2 \lambda_i = -2 \cdot \frac{1}{2}\log_2\frac{1}{2} = -\log_2\frac{1}{2} = 1 \text{ bit}$$

      Esta es la entropía máxima para un qubit, confirmando que el estado es maximalmente entrelazado.
    \end{solution}
  \end{parts}
\end{questions}

\section{Requisitos de la actividad}

\subsection*{Modalidad de trabajo}
Esta actividad se realizará de manera \textbf{individual}.

\subsection*{Formato y presentación}
El trabajo deberá cumplir los siguientes requisitos formales:

\begin{itemize}
  \item \textbf{Estructura}: El documento seguirá la estructura de un artículo académico, incluyendo introducción, desarrollo, conclusiones y referencias bibliográficas.
  \item \textbf{Herramientas}: La redacción se realizará íntegramente en \LaTeX.
  \item \textbf{Entrega}: Se entregará un único archivo en formato PDF compilado.
  \item \textbf{Citación}: Todas las referencias bibliográficas seguirán estrictamente la normativa APA 7.ª edición.
\end{itemize}

\subsection*{Criterios de evaluación}
Se valorarán especialmente los siguientes aspectos:

\begin{itemize}
  \item Claridad y rigor en la exposición de los cálculos intermedios.
  \item Justificación matemática y razonamiento lógico de los pasos realizados.
  \item Interpretación crítica de los resultados obtenidos y su contextualización.
  \item Corrección en el uso del lenguaje técnico y la notación matemática y la notación de Dirac.
\end{itemize}

\begin{center}
  \fcolorbox{red}{yellow!20}{%
    \begin{minipage}{0.9\textwidth}
      \textbf{Nota importante:} La rúbrica de evaluación únicamente se aplicará a aquellos trabajos que cumplan todos los requisitos de formato especificados. Los trabajos que no satisfagan estos requisitos podrán ser devueltos sin evaluación.
    \end{minipage}
  }
\end{center}

\section{Rúbrica}

\begin{rubrica}
  \setrubrica{Normativa}{Entregar la actividad en plazo y cumpliendo las indicaciones.}{0,5}{5}
  \setrubrica{Biliografía y citas APA}{Cumplir con las normas de citación y bibliografía según la normativa APA 7.}{0,5}{5}
  \setrubrica{Presentación}{Cálculos ordenados, notación correcta, interpretación física.}{1}{10}
  \setrubrica{Problema 1}{Ortogonalización y proyecciones (0.7 puntos/apartado).}{2.8}{28}
  \setrubrica{Problema 2}{Producto tensorial y sistemas compuestos (0.7 puntos/apartado).}{2.8}{28}
  \setrubrica{Problema 3}{Entrelazamiento cuántico (0.6 puntos/apartado).}{2.4}{24}
\end{rubrica}