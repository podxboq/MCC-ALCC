\maketitulo

\section{Objetivos y pautas de elaboración}

El principal objetivo de esta actividad es consolidar el dominio de los fundamentos matemáticos de la computación cuántica: números complejos, espacios vectoriales y operadores lineales.

En particular, se practicará:
\begin{itemize}
  \item Conversión entre formas de representación de números complejos
  \item Verificación de propiedades algebraicas en espacios vectoriales complejos
  \item Cálculo de valores y vectores propios
  \item Representación matricial de transformaciones lineales
\end{itemize}

\section{Problemas}

\begin{questions}
  \question \textbf{Números complejos y geometría}

  \begin{parts}
    \part Calcular las raíces cúbicas del número complejo $z = -8i$ y representarlas geométricamente en el plano complejo.

    \begin{solution}
      Primero escribimos $z = -8i$ en forma exponencial. Como $|z| = 8$ y $\arg(z) = -\frac{\pi}{2}$ (o $\frac{3\pi}{2}$), tenemos:
      $$z = 8e^{-i\pi/2}$$

      Las raíces cúbicas son:
      $$z_k = \sqrt[3]{8}e^{i\frac{-\pi/2 + 2\pi k}{3}} = 2e^{i\frac{-\pi + 4\pi k}{6}}, \quad k = 0, 1, 2$$

      Calculando cada raíz:
      \begin{align*}
        z_0 & = 2e^{-i\pi/6} = 2\left(\cos\frac{\pi}{6} - i\sin\frac{\pi}{6}\right) = \sqrt{3} - i    \\
        z_1 & = 2e^{i\pi/2} = 2i                                                                      \\
        z_2 & = 2e^{i7\pi/6} = 2\left(\cos\frac{7\pi}{6} + i\sin\frac{7\pi}{6}\right) = -\sqrt{3} - i
      \end{align*}

      Geométricamente, forman un triángulo equilátero inscrito en una circunferencia de radio 2 centrada en el origen.
    \end{solution}

    \part Dadas las amplitudes cuánticas $\alpha_1 = \frac{1+i}{2}$ y $\alpha_2 = \frac{1-i}{2}$, verificar que $|\alpha_1|^2 + |\alpha_2|^2 = 1$ y calcular la interferencia $|\alpha_1 + \alpha_2|^2$.

    \begin{solution}
      Calculamos las magnitudes:
      $$|\alpha_1|^2 = \left|\frac{1+i}{2}\right|^2 = \frac{|1+i|^2}{4} = \frac{1+1}{4} = \frac{1}{2}$$
      $$|\alpha_2|^2 = \left|\frac{1-i}{2}\right|^2 = \frac{|1-i|^2}{4} = \frac{1+1}{4} = \frac{1}{2}$$

      Por tanto: $|\alpha_1|^2 + |\alpha_2|^2 = \frac{1}{2} + \frac{1}{2} = 1$ ✓

      Para la interferencia:
      $$\alpha_1 + \alpha_2 = \frac{1+i}{2} + \frac{1-i}{2} = \frac{2}{2} = 1$$
      $$|\alpha_1 + \alpha_2|^2 = |1|^2 = 1$$

      Esto muestra interferencia constructiva completa (las fases se suman coherentemente).
    \end{solution}

    \part Expresar el número complejo $z = 1 + \sqrt{3}i$ en forma exponencial y calcular $z^{10}$ usando la fórmula de De Moivre.

    \begin{solution}
      Calculamos módulo y argumento:
      $$|z| = \sqrt{1^2 + (\sqrt{3})^2} = \sqrt{4} = 2$$
      $$\arg(z) = \arctan\left(\frac{\sqrt{3}}{1}\right) = \frac{\pi}{3}$$

      Forma exponencial: $z = 2e^{i\pi/3}$

      Por De Moivre:
      $$z^{10} = 2^{10}e^{i10\pi/3} = 1024e^{i10\pi/3}$$

      Simplificando el ángulo: $\frac{10\pi}{3} = 3\pi + \frac{\pi}{3} = \pi + \frac{\pi}{3} = \frac{4\pi}{3}$ (módulo $2\pi$)

      $$z^{10} = 1024\left(\cos\frac{4\pi}{3} + i\sin\frac{4\pi}{3}\right) = 1024\left(-\frac{1}{2} - i\frac{\sqrt{3}}{2}\right) = -512 - 512\sqrt{3}i$$
    \end{solution}

    \part Demostrar que si $\alpha = \frac{3}{5}$ y $\beta = \frac{4i}{5}$, entonces el estado $\ket{\psi} = \alpha\ket{0} + \beta\ket{1}$ está correctamente normalizado y calcular las probabilidades de medir $\ket{0}$ y $\ket{1}$.

    \begin{solution}
      Verificamos normalización:
      $$|\alpha|^2 + |\beta|^2 = \left|\frac{3}{5}\right|^2 + \left|\frac{4i}{5}\right|^2 = \frac{9}{25} + \frac{16}{25} = \frac{25}{25} = 1$$ ✓

      Las probabilidades son:
      $$P(\ket{0}) = |\alpha|^2 = \frac{9}{25} = 0.36 = 36\%$$
      $$P(\ket{1}) = |\beta|^2 = \frac{16}{25} = 0.64 = 64\%$$
    \end{solution}

    \part Calcular $e^{i\pi\sigma_y}$ donde $\sigma_y = \begin{pmatrix} 0 & -i \\ i & 0 \end{pmatrix}$.

    \begin{solution}
      Usamos que $\sigma_y^2 = I$ (propiedad de las matrices de Pauli).

      Por la serie exponencial:
      $$e^{i\pi\sigma_y} = \sum_{n=0}^{\infty} \frac{(i\pi\sigma_y)^n}{n!} = I + i\pi\sigma_y + \frac{(i\pi)^2\sigma_y^2}{2!} + \frac{(i\pi)^3\sigma_y^3}{3!} + \cdots$$

      Como $\sigma_y^2 = I$, $\sigma_y^3 = \sigma_y$, $\sigma_y^4 = I$, etc., podemos separar:
      $$e^{i\pi\sigma_y} = \left(1 - \frac{\pi^2}{2!} + \frac{\pi^4}{4!} - \cdots\right)I + i\left(\pi - \frac{\pi^3}{3!} + \frac{\pi^5}{5!} - \cdots\right)\sigma_y$$
      $$= \cos(\pi)I + i\sin(\pi)\sigma_y = -I + 0 = -I$$
      $$= \begin{pmatrix} -1 & 0 \\ 0 & -1 \end{pmatrix}$$

      Este es el operador de fase global $-1$, que es físicamente equivalente a la identidad.
    \end{solution}
  \end{parts}

  \question \textbf{Operadores lineales y valores propios}

  \begin{parts}
    \part Verificar que el conjunto $\left\{\begin{pmatrix} 1 \\ i \\ 0 \end{pmatrix}, \begin{pmatrix} i \\ 1 \\ 1 \end{pmatrix}, \begin{pmatrix} 0 \\ 1 \\ i \end{pmatrix}\right\}$ es linealmente independiente en $\C^3$.

    \begin{solution}
      Planteamos la ecuación:
      $$\alpha_1\begin{pmatrix} 1 \\ i \\ 0 \end{pmatrix} + \alpha_2\begin{pmatrix} i \\ 1 \\ 1 \end{pmatrix} + \alpha_3\begin{pmatrix} 0 \\ 1 \\ i \end{pmatrix} = \begin{pmatrix} 0 \\ 0 \\ 0 \end{pmatrix}$$

      Esto da el sistema:
      \begin{align*}
        \alpha_1 + i\alpha_2            & = 0 \\
        i\alpha_1 + \alpha_2 + \alpha_3 & = 0 \\
        \alpha_2 + i\alpha_3            & = 0
      \end{align*}

      De la primera ecuación: $\alpha_1 = -i\alpha_2$

      De la tercera ecuación: $\alpha_2 = -i\alpha_3$

      Sustituyendo en la segunda: $i(-i\alpha_2) + \alpha_2 + \alpha_3 = 0 \Rightarrow \alpha_2 + \alpha_2 + \alpha_3 = 0 \Rightarrow 2\alpha_2 + \alpha_3 = 0$

      Usando $\alpha_2 = -i\alpha_3$: $2(-i\alpha_3) + \alpha_3 = 0 \Rightarrow (-2i + 1)\alpha_3 = 0$

      Como $-2i + 1 \neq 0$, entonces $\alpha_3 = 0$, lo que implica $\alpha_2 = 0$ y $\alpha_1 = 0$.

      Por tanto, el conjunto es linealmente independiente.
    \end{solution}

    \part Encontrar los valores propios y vectores propios de la matriz de Hadamard $H = \frac{1}{\sqrt{2}}\begin{pmatrix} 1 & 1 \\ 1 & -1 \end{pmatrix}$.

    \begin{solution}
      El polinomio característico es:
      $$\det(H - \lambda I) = \det\begin{pmatrix} \frac{1}{\sqrt{2}}-\lambda & \frac{1}{\sqrt{2}} \\ \frac{1}{\sqrt{2}} & -\frac{1}{\sqrt{2}}-\lambda \end{pmatrix}$$
      $$= \left(\frac{1}{\sqrt{2}}-\lambda\right)\left(-\frac{1}{\sqrt{2}}-\lambda\right) - \frac{1}{2} = -\frac{1}{2} + \lambda^2 - \frac{1}{2} = \lambda^2 - 1$$

      Valores propios: $\lambda_1 = 1$, $\lambda_2 = -1$

      \textbf{Para $\lambda_1 = 1$:}
      $$(H - I)\vec{v} = 0 \Rightarrow \begin{pmatrix} \frac{1}{\sqrt{2}}-1 & \frac{1}{\sqrt{2}} \\ \frac{1}{\sqrt{2}} & -\frac{1}{\sqrt{2}}-1 \end{pmatrix}\begin{pmatrix} v_1 \\ v_2 \end{pmatrix} = 0$$

      Primera fila: $\left(\frac{1-\sqrt{2}}{\sqrt{2}}\right)v_1 + \frac{v_2}{\sqrt{2}} = 0 \Rightarrow v_2 = (\sqrt{2}-1)v_1$

      Vector propio (normalizado): $\vec{v}_1 = \frac{1}{\sqrt{2+\sqrt{2}}}\begin{pmatrix} 1 \\ \sqrt{2}-1 \end{pmatrix}$ o más simple: $\vec{v}_1 = \frac{1}{\sqrt{2}}\begin{pmatrix} 1 \\ 1 \end{pmatrix}$

      \textbf{Para $\lambda_2 = -1$:}
      De manera similar: $\vec{v}_2 = \frac{1}{\sqrt{2}}\begin{pmatrix} 1 \\ -1 \end{pmatrix}$
    \end{solution}

    \part Dada la transformación lineal $T: \C^2 \to \C^2$ definida por $T\begin{pmatrix} x \\ y \end{pmatrix} = \begin{pmatrix} (1+i)x + y \\ ix - 2y \end{pmatrix}$, encontrar su matriz respecto a la base canónica y calcular su determinante.

    \begin{solution}
      Para la base canónica $\{\vec{e}_1, \vec{e}_2\}$:
      $$T(\vec{e}_1) = T\begin{pmatrix} 1 \\ 0 \end{pmatrix} = \begin{pmatrix} 1+i \\ i \end{pmatrix}$$
      $$T(\vec{e}_2) = T\begin{pmatrix} 0 \\ 1 \end{pmatrix} = \begin{pmatrix} 1 \\ -2 \end{pmatrix}$$

      La matriz es:
      $$[T] = \begin{pmatrix} 1+i & 1 \\ i & -2 \end{pmatrix}$$

      El determinante:
      $$\det([T]) = (1+i)(-2) - (1)(i) = -2 - 2i - i = -2 - 3i$$

      Como $\det([T]) \neq 0$, la transformación es invertible.
    \end{solution}

    \part Verificar que la matriz $U = \frac{1}{\sqrt{2}}\begin{pmatrix} 1 & i \\ i & 1 \end{pmatrix}$ es unitaria calculando $U^\dagger U$.

    \begin{solution}
      Primero calculamos la adjunta:
      $$U^\dagger = \left(\frac{1}{\sqrt{2}}\begin{pmatrix} 1 & i \\ i & 1 \end{pmatrix}\right)^\dagger = \frac{1}{\sqrt{2}}\begin{pmatrix} 1 & -i \\ -i & 1 \end{pmatrix}$$

      Ahora calculamos el producto:
      $$U^\dagger U = \frac{1}{2}\begin{pmatrix} 1 & -i \\ -i & 1 \end{pmatrix}\begin{pmatrix} 1 & i \\ i & 1 \end{pmatrix}$$
      $$= \frac{1}{2}\begin{pmatrix} 1 + (-i)(i) & i + (-i)(1) \\ (-i)(1) + 1(i) & (-i)(i) + 1 \end{pmatrix}$$
      $$= \frac{1}{2}\begin{pmatrix} 1 + 1 & i - i \\ -i + i & 1 + 1 \end{pmatrix} = \frac{1}{2}\begin{pmatrix} 2 & 0 \\ 0 & 2 \end{pmatrix} = \begin{pmatrix} 1 & 0 \\ 0 & 1 \end{pmatrix} = I$$

      Por tanto, $U$ es unitaria. ✓
    \end{solution}

    \part Demostrar que si $A$ y $B$ son matrices hermitianas, entonces $AB$ es hermitiana si y solo si $AB = BA$.

    \begin{solution}
      $(\Rightarrow)$ Supongamos que $AB$ es hermitiana, entonces $(AB)^\dagger = AB$.

      Pero $(AB)^\dagger = B^\dagger A^\dagger = BA$ (usando que $A$ y $B$ son hermitianas).

      Por tanto: $AB = BA$. ✓

      $(\Leftarrow)$ Supongamos que $AB = BA$.

      Entonces: $(AB)^\dagger = B^\dagger A^\dagger = BA = AB$

      Por tanto, $AB$ es hermitiana. ✓

      \textbf{Interpretación física:} En mecánica cuántica, esto significa que dos observables conmutan si y solo si su producto es también un observable (operador hermitiano). Observables que conmutan pueden medirse simultáneamente sin incertidumbre.
    \end{solution}
  \end{parts}
\end{questions}

\section{Requisitos de la actividad}

\subsection*{Modalidad de trabajo}
Esta actividad se realizará de manera \textbf{individual}.

\subsection*{Formato y presentación}
El trabajo deberá cumplir los siguientes requisitos formales:

\begin{itemize}
  \item \textbf{Estructura}: El documento seguirá la estructura de un artículo académico, incluyendo introducción, desarrollo, conclusiones y referencias bibliográficas.
  \item \textbf{Herramientas}: La redacción se realizará íntegramente en \LaTeX.
  \item \textbf{Entrega}: Se entregará un único archivo en formato PDF compilado.
  \item \textbf{Citación}: Todas las referencias bibliográficas seguirán estrictamente la normativa APA 7.ª edición.
\end{itemize}

\subsection*{Criterios de evaluación}
Se valorarán especialmente los siguientes aspectos:

\begin{itemize}
  \item Claridad y rigor en la exposición de los cálculos intermedios.
  \item Justificación matemática y razonamiento lógico de los pasos realizados.
  \item Interpretación crítica de los resultados obtenidos y su contextualización.
  \item Corrección en el uso del lenguaje técnico y la notación matemática.
\end{itemize}

\begin{center}
  \fcolorbox{red}{yellow!20}{%
    \begin{minipage}{0.9\textwidth}
      \textbf{Nota importante:} La rúbrica de evaluación únicamente se aplicará a aquellos trabajos que cumplan todos los requisitos de formato especificados. Los trabajos que no satisfagan estos requisitos podrán ser devueltos sin evaluación.
    \end{minipage}
  }
\end{center}

\section{Rúbrica}

\begin{rubrica}
  \setrubrica{Normativa}{Entregar la actividad en plazo y cumpliendo las indicaciones.}{0,5}{5}
  \setrubrica{Biliografía y citas APA}{Cumplir con las normas de citación y bibliografía según la normativa APA 7.}{0,5}{5}
  \setrubrica{Presentación}{Cálculos ordenados y claramente justificados.}{1}{10}
  \setrubrica{Problema 1}{Números complejos y geometría (0.8 puntos por apartado).}{4}{40}
  \setrubrica{Problema 2}{Operadores y valores propios (0.8 puntos por apartado).}{4}{40}
\end{rubrica}