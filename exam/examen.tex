\caratula

\section{Solicitud}
En la convocatoria del 11 al 16 de julio de 2025 se celebrarán los exámenes del 1 cuatrimestre del Máster en Computación Cuántica per 13440 de la convocatoria ordinaria.

En este sentido, le solicitamos la creación de 14 modelos de examen de la asignatura que está impartiendo (7 para ordinaria y 7 para extraordinaria).

\section{Preámbulo}

\subsection{Instrucciones}

\begin{itemize}
	\item Ten disponible tu documentación oficial para identificarte, en el caso de que se te solicite.
	\item Rellena tus datos personales en todos los espacios fijados para ello y lee atentamente todas las preguntas antes de empezar.
	\item Las preguntas se contestarán en la lengua vehicular de esta asignatura.
	\item Debes contestar en el documento adjunto, respetando en todo momento el espaciado indicado para cada pregunta. Si este es en formato digital, los márgenes, el interlineado, fuente y tamaño de letra vienen dados por defecto y no deben modificarse. En cualquier caso, asegúrate de que la presentación es suficientemente clara y legible.
	\item La entrega del examen en blanco o de un documento distinto del que se le ha facilitado por UNIR para realizar el examen, tendrá una calificación de “0” (suspenso).
	\item Durante el examen y en la corrección por parte del docente, se aplicará el Reglamento de Evaluación Académica de UNIR que regula las consecuencias derivadas de las posibles irregularidades y prácticas académicas incorrectas con relación al plagio y uso inadecuado de materiales y recursos.
	\item No está permitido el uso de material bibliográfico o calculadora de ningún tipo.
	\item Para el examen online. Para introducir una imagen digital, se captura con un dispositivo sin capacidad de conexión inalámbrica y se enviará al ordenador con el que se realiza el examen a través de un cable. No se puede utilizar ninguna aplicación excepto el navegador para cargar/descargar el examen y Word para completarlo. Debe entregarse un único documento en formato PDF. Queda totalmente prohibido entregar documentos tipo imagen (JPG, PNG, etc), word o similar.
\end{itemize}

\subsection{Grupo de preguntas}

\subsubsection{Título}
Calificación del examen.

\subsubsection{Descripción}
Buenos días, a continuación realizarás el examen de álgebra linea y computación cuántica.

El examen consta de 3 preguntas que encontrarás al final del examen. Contesta cada pregunta en el espacio que se te indica en las páginas siguientes.

Las respuestas deben presentarse de manera razonada, detallando claramente los pasos seguidos en el desarrollo.
En caso de que el resultado final sea incorrecto debido a errores de cálculo, pero el procedimiento sea correcto, la puntuación podrá reducirse hasta un 50 \%.

Buena suerte.

\subsection{Códigos}
PER 13224

Asignados a la convocatoria ordinaria (UNIR 2025.06 CONVOCATORIA 27 JUNIO).

\begin{tabular}{llll}
	\hline
	\textbf{Modelos} & \textbf{Convocatoria} & \textbf{Profesor}    & \textbf{Código} \\
	\hline
	A                & Ordinaria             & Francisco Costa Cano & 10264463        \\
	B                & Ordinaria             & Francisco Costa Cano & 10264481        \\
	C                & Ordinaria             & Francisco Costa Cano & 10264482        \\
	D                & Ordinaria             & Francisco Costa Cano & 10264483        \\
	E                & Ordinaria             & Francisco Costa Cano & 10264484        \\
	F                & Ordinaria             & Francisco Costa Cano & 10264485        \\
	1                & Ordinaria             & Francisco Costa Cano & 10264487        \\
	A                & Extraordinaria        & Francisco Costa Cano & 10264489        \\
	B                & Extraordinaria        & Francisco Costa Cano & 10264493        \\
	C                & Extraordinaria        & Francisco Costa Cano & 10264495        \\
	D                & Extraordinaria        & Francisco Costa Cano & 10264498        \\
	E                & Extraordinaria        & Francisco Costa Cano & 10264499        \\
	F                & Extraordinaria        & Francisco Costa Cano & 10264500        \\
	1                & Extraordinaria        & Francisco Costa Cano & 10264501        \\
\end{tabular}