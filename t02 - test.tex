\begin{questions}

  \question El conjunto $W = \left\{\begin{pmatrix} x \\ y \\ z \end{pmatrix} : x + y = 1\right\}$ es:

  \begin{choices}
    \choice Un subespacio vectorial de $\C^3$
    \CorrectChoice No es un subespacio vectorial de $\C^3$
    \choice Es un espacio vectorial completo
    \choice Es la base canónica de $\C^3$
  \end{choices}
  \begin{solution}
    No es subespacio porque no contiene el vector cero. Para $x = y = 0$ tendríamos $0 + 0 \neq 1$.
  \end{solution}

  \question La dimensión de $\C^{2 \times 2}$ (matrices complejas $2 \times 2$) es:

  \begin{choices}
    \choice $2$
    \CorrectChoice $4$
    \choice $8$
    \choice $16$
  \end{choices}
  \begin{solution}
    Una matriz $2 \times 2$ tiene 4 entradas complejas, y el conjunto de todas ellas forma un espacio vectorial de dimensión 4 sobre $\C$.
  \end{solution}

  \question Los vectores $\begin{pmatrix} 1 \\ i \end{pmatrix}$, $\begin{pmatrix} i \\ 1 \end{pmatrix}$ en $\C^2$ son:

  \begin{choices}
    \choice Linealmente dependientes
    \CorrectChoice Linealmente independientes
    \choice Ortogonales
    \choice Ninguna de las anteriores
  \end{choices}
  \begin{solution}
    Para que sean dependientes debe existir $\alpha \in \C$ tal que $\begin{pmatrix} 1 \\ i \end{pmatrix} = \alpha\begin{pmatrix} i \\ 1 \end{pmatrix}$, lo que implica $1 = \alpha i$ y $i = \alpha$. De la segunda ecuación $\alpha = i$, pero sustituyendo en la primera: $1 = i \cdot i = -1$, contradicción.
  \end{solution}

  \question El estado cuántico $\ket{\psi} = \frac{1}{2}\ket{0} + \frac{\sqrt{3}i}{2}\ket{1}$ es:

  \begin{choices}
    \choice No está normalizado
    \CorrectChoice Está normalizado y es válido
    \choice No es un estado cuántico válido
    \choice Representa dos qubits
  \end{choices}
  \begin{solution}
    Verificamos: $\left|\frac{1}{2}\right|^2 + \left|\frac{\sqrt{3}i}{2}\right|^2 = \frac{1}{4} + \frac{3}{4} = 1$. El estado está normalizado.
  \end{solution}

  \question En la parametrización $\ket{\psi} = \cos\frac{\theta}{2}\ket{0} + e^{i\varphi}\sin\frac{\theta}{2}\ket{1}$, el parámetro $\varphi$ representa:

  \begin{choices}
    \choice La probabilidad de medir $\ket{1}$
    \CorrectChoice La fase relativa entre componentes
    \choice El ángulo polar en la esfera de Bloch
    \choice La norma del estado
  \end{choices}
  \begin{solution}
    El parámetro $\varphi$ es la fase relativa que distingue estados con las mismas probabilidades pero diferentes propiedades de interferencia.
  \end{solution}

  \question Dos vectores estado que difieren por una fase global $e^{i\phi}$ son:

  \begin{choices}
    \choice Ortogonales
    \choice Linealmente independientes
    \CorrectChoice Físicamente equivalentes
    \choice Representan estados diferentes medibles
  \end{choices}
  \begin{solution}
    Los estados $\ket{\psi}$ y $e^{i\phi}\ket{\psi}$ son físicamente indistinguibles ya que todas las mediciones dan los mismos resultados: $|e^{i\phi}|^2 = 1$.
  \end{solution}

  \question El espacio de Hilbert de un sistema de 3 qubits tiene dimensión:

  \begin{choices}
    \choice $3$
    \choice $6$
    \CorrectChoice $8$
    \choice $9$
  \end{choices}
  \begin{solution}
    Un sistema de $n$ qubits tiene espacio de estados $(\C^2)^{\otimes n} \cong \C^{2^n}$. Para $n = 3$: $\dim = 2^3 = 8$.
  \end{solution}

  \question Una base de $\C^2$ debe tener:

  \begin{choices}
    \choice Un vector
    \CorrectChoice Dos vectores linealmente independientes
    \choice Tres vectores
    \choice Cuatro vectores
  \end{choices}
  \begin{solution}
    Una base de un espacio de dimensión $n$ debe tener exactamente $n$ vectores linealmente independientes. Para $\C^2$, $n = 2$.
  \end{solution}

  \question El conjunto $\{\ket{+}, \ket{-}\}$ donde $\ket{+} = \frac{\ket{0} + \ket{1}}{\sqrt{2}}$ y $\ket{-} = \frac{\ket{0} - \ket{1}}{\sqrt{2}}$:

  \begin{choices}
    \choice No es una base de $\C^2$
    \choice Es linealmente dependiente
    \CorrectChoice Es una base ortonormal de $\C^2$
    \choice Solo genera un subespacio unidimensional
  \end{choices}
  \begin{solution}
    Son ortogonales: $\braket{+}{-} = \frac{1}{2}(1 - 1) = 0$, normalizados: $\braket{+}{+} = \braket{-}{-} = 1$, y linealmente independientes, por tanto forman una base ortonormal.
  \end{solution}

  \question Si $\{\mathbf{v}_1, \mathbf{v}_2, \mathbf{v}_3\}$ es una base de $\C^3$, entonces:

  \begin{choices}
    \choice Todo vector se puede escribir como combinación lineal de dos de ellos
    \CorrectChoice Todo vector se puede escribir de forma única como combinación lineal de los tres
    \choice Los vectores deben ser ortogonales
    \choice Los vectores deben tener norma 1
  \end{choices}
  \begin{solution}
    Por definición de base, todo vector del espacio se expresa de manera única como combinación lineal de los vectores de la base.
  \end{solution}

\end{questions}