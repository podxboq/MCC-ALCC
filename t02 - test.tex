\begin{questions}

  \question El conjunto $W = \left\{(x, y, z) \mid x + y = 1\right\}$:

  \begin{choices}
    \choice Es un subespacio vectorial de $\C^3$.
    \CorrectChoice No es un subespacio vectorial de $\C^3$.
    \choice Es un espacio vectorial completo.
    \choice Es la base canónica de $\C^3$.
  \end{choices}
  \begin{solution}
    No es subespacio porque no contiene el vector cero. Para $x = y = 0$ tendríamos $0 + 0 \neq 1$.
  \end{solution}

  \question La dimensión de $\C^{2 \times 2}$ (matrices complejas $2 \times 2$) es:

  \begin{choices}
    \CorrectChoice $4$.
    \choice $2$.
    \choice $16$.
    \choice $8$.
  \end{choices}
  \begin{solution}
    Una matriz $2 \times 2$ tiene 4 entradas complejas, y el conjunto de todas ellas forma un espacio vectorial de dimensión 4 sobre $\C$.
  \end{solution}

  \question Los vectores $(1, i)$, $(i, 1)$ en $\C^2$ son:

  \begin{choices}
    \choice Linealmente dependientes.
    \choice Ninguna de las anteriores.
    \choice Linealmente independientes y dependientes.
    \CorrectChoice Linealmente independientes.
  \end{choices}
  \begin{solution}
    Para que sean dependientes debe existir $\alpha \in \C$ tal que $(1, i) = \alpha (i, 1)$, lo que implica $1 = \alpha i$ y $i = \alpha$. De la segunda ecuación $\alpha = i$, pero sustituyendo en la primera: $1 = i \cdot i = -1$, contradicción.
  \end{solution}

  \question El vector $(\frac{1}{2}, \frac{\sqrt{3}i}{2})$ es:

  \begin{choices}
    \CorrectChoice Un vector de dimensión 2 sobre $\C$ como $\R$ espacio vectorial.
    \choice Un vector de dimensión 2 sobre $\C$ como $\C$ espacio vectorial.
    \choice Es un número complejo en forma cartesiana.
    \choice Es un punto en el plano real.
  \end{choices}
  \begin{solution}
    Podemos expresar este elemento como $(\frac{1}{2}, \frac{\sqrt{3}i}{2}) = \frac{1}{2} (1, 0)+ \frac{\sqrt{3}i}{2} (0, i)$, lo que pone el vector con coordenadas reales sobre un espacio vectorial complejo sobre $\R$ y dimensión $2$.
  \end{solution}

  \question El vector cero con respecto a una base, tiene coordenadas:

  \begin{choices}
    \choice El cero es el único vector que no puede ser representado de forma única con respecto a una base.
    \choice El cero no es un vector.
    \choice Depende de la base.
    \CorrectChoice $(0, \dots, 0)$.
  \end{choices}
  \begin{solution}
    El vector cero con respecto a una base tiene coordenadas $(0, \dots, 0)$.
  \end{solution}

  \question Dos subespacios vectoriales de un espacio vectorial $V$, con dimensiones $2$ y $3$ respectivamente. Entonces la dimensión del espacio vectorial $V$ es:

  \begin{choices}
    \choice $5$.
    \choice Como mínimo $2$.
    \CorrectChoice Como mínimo $3$.
    \choice Como máximo $3$.
  \end{choices}
  \begin{solution}
    La dimensión del espacio vectorial $V$ es mayor o igual a la suma de las dimensiones de $V_1$ y $V_2$ menos la dimensión de la intersección.
    Como la intersección es como mucho de dimensión $2$, la dimensión de $V$ es como mínimo $2 + 3 - 2 = 3$.
  \end{solution}

  \question La intersección de dos subespacios vectoriales ¿puede ser el conjunto vacio?:

  \begin{choices}
    \CorrectChoice No.
    \choice Sí.
    \choice Depende de la base.
    \choice Depende de la dimensión.
  \end{choices}
  \begin{solution}
    Dos subespacios vectoriales no pueden intersectarse en el conjunto vacio, porque el cero está en todos los subespacios vectoriales.
  \end{solution}

  \question Una base de $\C^2$ debe tener:

  \begin{choices}
    \choice Un vector.
    \CorrectChoice Dos vectores.
    \choice Tres vectores.
    \choice Cuatro vectores.
  \end{choices}
  \begin{solution}
    Una base de un espacio de dimensión $n$ debe tener exactamente $n$ vectores linealmente independientes. Para $\C^2$, $n = 2$.
  \end{solution}

  \question Dada la base $\{v_1, v_2\}$ de $\C^2$, el conjunto $\{v_1 + v_2, v_1 - v_2\}$, ¿es una base de $\C^2$?:

  \begin{choices}
    \choice Depende de $v_1$ y $v_2$.
    \choice Sí.
    \CorrectChoice No.
    \choice Solo si $v_1$ o $v_2$ es un número real.
  \end{choices}
  \begin{solution}
    Siempre es base de $\C^2$, porque son linealmente independiente y tienen el mismo múmero de vectores que la dimensión.
  \end{solution}

  \question Si $\{v_1, v_2, v_3\}$ es una base de $\C^3$, entonces:

  \begin{choices}
    \choice Todo vector se puede escribir como combinación lineal de dos de ellos.
    \CorrectChoice Todo vector se puede escribir como combinación lineal de los tres.
    \choice Todo vector se puede poner como suma de tres vectores, donde al menos uno de ellos debe ser la base.
    \choice Todo vector es múltiplo de algún vector de la base.
  \end{choices}
  \begin{solution}
    Por definición de base, todo vector del espacio se expresa de manera única como combinación lineal de los vectores de la base.
  \end{solution}

\end{questions}