\documentclass[]{unirTema}

\newcommand{\tq}{\mid}
\newcommand{\K}{\mathrm{K}}
\newcommand{\V}{\mathrm{V}}
\newcommand{\N}{\mathbb{N}}
\newcommand{\Z}{\mathbb{Z}}
\newcommand{\F}{\mathbb{F}}
\newcommand{\Q}{\mathbb{Q}}
\newcommand{\R}{\mathbb{R}}
\newcommand{\C}{\mathbb{C}}
\renewcommand{\H}{\mathcal{H}}
\newcommand{\Cinf}{\mathcal{C}^\infty}
\newcommand{\Cu}{\mathcal{C}^1}
\newcommand{\Rp}{\mathfrak{Re}}
\newcommand{\Ip}{\mathfrak{Im}}
\renewcommand{\d}{\mathrm{d}}
\newcommand{\dm}{\mathrm{d}\mu}
\newcommand{\conj}[1]{\overline{#1}}
\newcommand{\Tras}[1]{{#1}^{\text{T}}}

\DeclareMathOperator{\Log}{Log}
\DeclareMathOperator{\Arg}{Arg}
\DeclareMathOperator{\Dom}{Dom}
\DeclareMathOperator{\Ima}{Im}
\DeclareMathOperator{\sgn}{sgn}
\DeclareMathOperator{\mcd}{MCD}
\DeclareMathOperator{\mcm}{mcm}
\DeclareMathOperator{\Resi}{Res}
\DeclareMathOperator{\Ker}{Ker}
\DeclareMathOperator{\End}{End}
\DeclareMathOperator{\Mat}{Mat}

\newcommand{\parder}[2]{\frac{\partial #1}{\partial #2}}
\newcommand{\dparder}[2]{\dfrac{\partial #1}{\partial #2}}
\newcommand{\tparder}[2]{\partial #1/\partial #2}
\newcommand{\parderr}[3]{\frac{\partial^2 #1}{\partial #2\partial #3}}
\newcommand{\dparderr}[3]{\dfrac{\partial^2 #1}{\partial #2\partial #3}}
\newcommand{\tparderr}[3]{\partial^2 #1/\partial #2\partial #3}
\newcommand{\intx}[1]{\int #1\,dx}
\newcommand{\intt}[1]{\int #1\,dt}
\newcommand{\intdx}[3]{\int_{#1}^{#2} #3\,dx}
\newcommand{\intdt}[3]{\int_{#1}^{#2} #3\,dt}
\newcommand{\intdz}[2]{\int_{#1} #2\,dz}
\newcommand{\set}[1]{\left\{#1\right\}}
\newcommand{\so}{\Rightarrow}
\newcommand{\sii}{\Leftrightarrow}
\newcommand{\by}[1]{\overset{\fbox{\tiny #1}}{=}}
\newcommand{\byref}[1]{\overset{\fbox{\tiny\ref{#1}}}{=}}
\newcommand{\cardinal}[1]{\left|#1\right|}
\newcommand{\maps}[3]{#1 \colon #2\longrightarrow #3}
\newcommand{\equationmaps}[5]{\begin{aligned}[t]#1 \colon #2 &\longrightarrow #3 \\	#4 &\longmapsto #5\end{aligned}}
\newcommand{\coma}{,\thinspace}
\newcommand{\pari}[2]{(#1,\thinspace #2)}
\newcommand{\where}{\mathrel{}\middle|\mathrel{}}
\newcommand{\no}[1]{{\neg}{#1}}
\newcommand{\dcomilla}[1]{{\guillemotleft}#1{\guillemotright}}
\newcommand{\separa}{\vspace*{.75\baselineskip}}
\newcommand{\semisepara}{\vspace*{.25\baselineskip}}
\newcommand{\restrict}[1]{\raisebox{-.5ex}{$|$}_{#1}}

% ========================================
% COMANDOS PERSONALIZADOS PARA COMPUTACIÓN CUÁNTICA
% ========================================

% Estados comunes
\newcommand{\zero}{\ket{0}}
\newcommand{\one}{\ket{1}}
\newcommand{\plus}{\ket{+}}
\newcommand{\minus}{\ket{-}}

% Operadores especiales
\newcommand{\tensor}{\otimes}         % Producto tensorial
\newcommand{\comp}{\circ}             % Composición

% Estados de Bell
\newcommand{\bellphi}{\ket{\Phi^+}}
\newcommand{\bellpsi}{\ket{\Psi^+}}
\newcommand{\bellphiminus}{\ket{\Phi^-}}
\newcommand{\bellpsiminus}{\ket{\Psi^-}}

\newcommand{\floor}[1]{\left\lfloor #1 \right\rfloor}
\newcommand{\ceil}[1]{\left\lceil #1 \right\rceil}


\printanswers

\author{Francisco Costa Cano}
\titulacion{Máster en computación cuántica}
\asignatura{Álgebra lineal en computación cuántica}
\bloque{1}{Fundamentos matemáticos}
\tema{2}{Matriz cambio de base vs matriz cambio de coordenadas}

\begin{document}

\caratula

\section{Espacio vectorial de polinomios}

Vamos a considerar el espacio vectorial $\mathcal{P}_1[x]=\{ax+b\mid a,b\in\mathbb{C}\}$ y las siguientes bases:
\begin{align*}
  \mathcal{B}  & = \{1+x, 1-x\}\,, \\
  \mathcal{B}' & = \{i+x, i-x\}
\end{align*}

\subsection{Matriz cambio de base}
La matriz cambio de base de $\mathcal{B}$ a $\mathcal{B}'$ que llamaremos $M_{\mathcal{B}\rightarrow\mathcal{B}'}$ es la matriz asociada al homomorfismo de espacios vectoriales $f: \mathcal{P}_1[x] \rightarrow \mathcal{P}_1[x]$ que lleva los vectores de $\mathcal{B}$ a los vectores de $\mathcal{B}'$. En este caso, tenemos que
\begin{align*}
  f(1+x) & = i+x \\
  f(1-x) & = i-x
\end{align*}
Para calcular esta matriz tenemos que saber como actua el homomorfismo sobre los vectores de la base canónica de $\mathcal{P}_1[x]$.
\begin{align*}
  f(1) & = \frac{1}{2}(i+x) + \frac{1}{2}(i-x) = i\,, \\
  f(x) & = \frac{1}{2}(i+x) - \frac{1}{2}(i-x) = x\,.
\end{align*}

Por lo tanto, la matriz cambio de base de $\mathcal{B}$ a $\mathcal{B}'$ es
\begin{align*}
  M_{\mathcal{B}\rightarrow\mathcal{B}'} & = \begin{pmatrix}
                                               i & 0 \\
                                               0 & 1
                                             \end{pmatrix}\,.
\end{align*}

Para comprobarlo calculamos la imagen de $1+x$ usando la matriz cambio de base,
\begin{align*}
  f(1+x) & = M_{\mathcal{B}\rightarrow\mathcal{B}'} \begin{pmatrix}
                                                      1 \\
                                                      1
                                                    \end{pmatrix} \\
         & = \begin{pmatrix}
               i & 0 \\
               0 & 1
             \end{pmatrix} \begin{pmatrix}
                             1 \\
                             1
                           \end{pmatrix}                          \\
         & = \begin{pmatrix}
               i \\
               1
             \end{pmatrix} = i+x
\end{align*}

\subsection{No es lo mismo la matriz cambio de base que la matriz cambio de coordenadas}

Tomemos el vector $2\in\mathcal{P}_1[x]$, y veamos sus coordenadas en la base $\mathcal{B}$ y en la base $\mathcal{B}'$.
\begin{align*}
  2 & = 1\cdot(1+x) + 1\cdot(1-x) =(1,1)_\mathcal{B}\,,            \\
  2 & = (-i)\cdot(i+x) + (-i)\cdot(i-x) =(-i,-i)_{\mathcal{B}'}\,.
\end{align*}

Sin embargo, si multiplicamos la matriz cambio de base por las coordenadas de $2$ en la base $\mathcal{B}$ no obtenemos las coordenadas de 2 en la base $\mathcal{B}'$
\begin{align*}
  \begin{pmatrix}
    i & 0 \\
    0 & 1
  \end{pmatrix} \begin{pmatrix}
                  1 \\
                  1
                \end{pmatrix} = \begin{pmatrix}
                                  i \\
                                  1
                                \end{pmatrix} \neq \begin{pmatrix}
                                                     -i \\
                                                     -i
                                                   \end{pmatrix}\,.
\end{align*}

Concluimos que
\[
  M_{\mathcal{B}\rightarrow\mathcal{B}'} 2_{\mathcal{B}} \neq 2_{\mathcal{B}'}\,.
\]

\subsection{Matriz cambio de coordenadas}
Para obtener la matriz cambio de coordenadas de $\mathcal{B}$ a $\mathcal{B}'$ tenemos que calcular como se expresa los vectores de la base $\mathcal{B}$ en la base $\mathcal{B}'$.
\begin{align*}
  1+x & = \frac{1+i}{2i}(i+x) + \frac{1-i}{2i}(i-x)\,, \\
  1-x & = \frac{1-i}{2i}(i+x) + \frac{1+i}{2i}(i-x)\,.
\end{align*}

La matriz cambio de coordenadas es
\[
  M_\mathcal{B}^\mathcal{B'} = \frac{1}{2i} \begin{pmatrix}
    1+i & 1-i \\
    1-i & 1+i
  \end{pmatrix}\,.
\]

Ahora calculamos las coordenadas de $2$ en la base $\mathcal{B}'$ usando la matriz cambio de coordenadas,
\begin{align*}
  2_{\mathcal{B}'} & = M_\mathcal{B}^\mathcal{B'} 2_\mathcal{B}      \\
                   & = \frac{1}{2i} \begin{pmatrix}
                                      1+i & 1-i \\
                                      1-i & 1+i
                                    \end{pmatrix} \begin{pmatrix}
                                                    1 \\
                                                    1
                                                  \end{pmatrix}     \\
                   & = \frac{1}{2i} \begin{pmatrix}
                                      2 \\
                                      2
                                    \end{pmatrix} = \begin{pmatrix}
                                                      -i \\
                                                      -i
                                                    \end{pmatrix}\,.
\end{align*}

Y para comprobarlo realizamos las operaciones
\begin{align*}
  2_{\mathcal{B}'} & = -i(i+x) -i(i-x)   \\
                   & = -i^2 -ix -i^2 +ix \\
                   & = 1 + 1 = 2\,.
\end{align*}

\subsection{Conclusión}

En resumen podemos decir que

\begin{align*}
  M_{\mathcal{B}\rightarrow \mathcal{B'}} 2 & = (-1+i) + (i-1)x  & M_{\mathcal{B}\rightarrow \mathcal{B'}} 2_{\mathcal{B}} & \neq 2_{\mathcal{B}'}   \\
  M_\mathcal{B}^\mathcal{B'} 2_\mathcal{B}  & = 2_{\mathcal{B}'} & M_\mathcal{B}^\mathcal{B'} 2                            & \neq (-1+i) + (i-1)x\,.
\end{align*}

\end{document}