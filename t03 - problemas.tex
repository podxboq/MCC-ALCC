\unirsection{Problemas}

\begin{questions}

  \question Verificar que las siguientes funciones son transformaciones lineales y calcular su núcleo e imagen:
  \begin{parts}
    \part $T: \C^3 \to \C^2$ definida por
    \[
      T(x, y, z) = (x + iy - z, 2x + y)\,.
    \]
    \part $S: \C^2 \to \C^3$ definida por
    \[
      S(u, v) = (u + iv, 2u, u - v)\,.
    \]
  \end{parts}

  \question
  \label{ex:propiedadesKerIma}
  Demostrar las propiedades del resultado~\ref{prop:propiedadesKerIma}.

  \question
  \label{ex:propiedadesOperadores}
  Demostrar las propiedades del resultado~\ref{prop:propiedadesOperadores}.

  \question Encontrar la matriz de la transformación lineal $T: \C^2 \to \C^2$ definida por $T(x, y) = ((1+i)x + y, ix - 2y)$ respecto a:
  \begin{parts}
    \part La base canónica.
    \part La base $\mathcal{B} = \left\{(1, i), (i, 1)\right\}$.
  \end{parts}

  \question Sean $A = \begin{pmatrix} 1 & i \\ 0 & 1 \end{pmatrix}$ y $B = \begin{pmatrix} 1 & 0 \\ i & 1 \end{pmatrix}$. Calcular:
  \begin{parts}
    \part $AB$ y $BA$.
    \part $A^\dagger A$ y $A A^\dagger$.
    \part $\det(A)$, $\det(B)$ y $\det(AB)$.
    \part $\text{tr}(A)$, $\text{tr}(B)$ y $\text{tr}(AB)$.
  \end{parts}

  \question Demostrar que si $A$ es una matriz hermitiana, entonces los valores propios de $A$ son reales.

  \question Demostrar que si $A$ es una matriz unitaria, entonces los valores propios de $A$ tienen módulo 1.

  \question Demostrar que si $A\in\mathcal{M}_2(\C)$, entonces su polinomio característico es
  \[
    p_A(\lambda) = \lambda^2 - \text{tr}(A)\lambda + \det(A)\,.
  \]

  \question Encontrar los valores propios y vectores propios de las siguientes matrices:
  \begin{parts}
    \part $A = \begin{pmatrix} 2 & i \\ -i & 2 \end{pmatrix}$.
    \part $B = \begin{pmatrix} 1 & 1+i \\ 1-i & 2 \end{pmatrix}$.
    \part La matriz de Hadamard $H = \frac{1}{\sqrt{2}}\begin{pmatrix} 1 & 1 \\ 1 & -1 \end{pmatrix}$.
  \end{parts}

  \question Verificar que las siguientes matrices son unitarias y calcular sus inversas:
  \begin{parts}
    \part $U_1 = \begin{pmatrix} \frac{1}{\sqrt{2}} & \frac{i}{\sqrt{2}} \\ \frac{i}{\sqrt{2}} & \frac{1}{\sqrt{2}} \end{pmatrix}$.
    \part $U_2 = \frac{1}{\sqrt{3}}\begin{pmatrix} 1 & 1+i \\ 1-i & -1 \end{pmatrix}$.
  \end{parts}

  \question Demostrar que si $A$ y $B$ son matrices hermitianas, entonces:
  \begin{parts}
    \part $A + B$ es hermitiana.
    \part $AB$ es hermitiana si y solo si $AB = BA$.
    \part $iA$ es antihermitiana (i.e., $(iA)^* = -iA$).
  \end{parts}

  \question
  Considerar la matriz
  \[
    \text{CZ} = \begin{pmatrix} 1 & 0 & 0 & 0 \\ 0 & 1 & 0 & 0 \\ 0 & 0 & 1 & 0 \\ 0 & 0 & 0 & -1 \end{pmatrix}\,.
  \]
  \begin{parts}
    \part Verificar que CZ es unitaria.
    \part Determinar cómo actúa sobre los vectores de la base computacional.
    \part Calcular los valores propios de CZ.
  \end{parts}

  \question Expresar la matriz
  \[
    M = \begin{pmatrix} 3 & 1+2i \\ 1-2i & -1 \end{pmatrix}\,,
  \] como combinación lineal de las matrices de Pauli y la identidad.

  \question Sea $T: V \to V$ un operador lineal.
  \begin{parts}
    \part Demostrar que $T$ es invertible si y solo si $\det([T]) \neq 0$.
    \part Si $T$ es invertible, demostrar que $[T^{-1}] = ([T])^{-1}$.
    \part Demostrar que $\text{tr}(T)$ es independiente de la base elegida.
  \end{parts}

  \question Demostrar que para toda matriz $A\in \mathcal{M}_n(\C)$, se cumple que $A^\dagger A$ y $A A^\dagger$ son hermitianas.

  \question Calcula la descomposición en valores singulares de la matriz
  \[
    A = \begin{pmatrix} 1 & i \\ -i & 1 \end{pmatrix}\,.
  \]
\end{questions}