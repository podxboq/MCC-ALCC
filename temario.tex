\documentclass[]{unirTema}

\newcommand{\tq}{\mid}
\newcommand{\K}{\mathrm{K}}
\newcommand{\V}{\mathrm{V}}
\newcommand{\N}{\mathbb{N}}
\newcommand{\Z}{\mathbb{Z}}
\newcommand{\F}{\mathbb{F}}
\newcommand{\Q}{\mathbb{Q}}
\newcommand{\R}{\mathbb{R}}
\newcommand{\C}{\mathbb{C}}
\renewcommand{\H}{\mathcal{H}}
\newcommand{\Cinf}{\mathcal{C}^\infty}
\newcommand{\Cu}{\mathcal{C}^1}
\newcommand{\Rp}{\mathfrak{Re}}
\newcommand{\Ip}{\mathfrak{Im}}
\renewcommand{\d}{\mathrm{d}}
\newcommand{\dm}{\mathrm{d}\mu}
\newcommand{\conj}[1]{\overline{#1}}
\newcommand{\Tras}[1]{{#1}^{\text{T}}}

\DeclareMathOperator{\Log}{Log}
\DeclareMathOperator{\Arg}{Arg}
\DeclareMathOperator{\Dom}{Dom}
\DeclareMathOperator{\Ima}{Im}
\DeclareMathOperator{\sgn}{sgn}
\DeclareMathOperator{\mcd}{MCD}
\DeclareMathOperator{\mcm}{mcm}
\DeclareMathOperator{\Resi}{Res}
\DeclareMathOperator{\Ker}{Ker}
\DeclareMathOperator{\End}{End}
\DeclareMathOperator{\Mat}{Mat}

\newcommand{\parder}[2]{\frac{\partial #1}{\partial #2}}
\newcommand{\dparder}[2]{\dfrac{\partial #1}{\partial #2}}
\newcommand{\tparder}[2]{\partial #1/\partial #2}
\newcommand{\parderr}[3]{\frac{\partial^2 #1}{\partial #2\partial #3}}
\newcommand{\dparderr}[3]{\dfrac{\partial^2 #1}{\partial #2\partial #3}}
\newcommand{\tparderr}[3]{\partial^2 #1/\partial #2\partial #3}
\newcommand{\intx}[1]{\int #1\,dx}
\newcommand{\intt}[1]{\int #1\,dt}
\newcommand{\intdx}[3]{\int_{#1}^{#2} #3\,dx}
\newcommand{\intdt}[3]{\int_{#1}^{#2} #3\,dt}
\newcommand{\intdz}[2]{\int_{#1} #2\,dz}
\newcommand{\set}[1]{\left\{#1\right\}}
\newcommand{\so}{\Rightarrow}
\newcommand{\sii}{\Leftrightarrow}
\newcommand{\by}[1]{\overset{\fbox{\tiny #1}}{=}}
\newcommand{\byref}[1]{\overset{\fbox{\tiny\ref{#1}}}{=}}
\newcommand{\cardinal}[1]{\left|#1\right|}
\newcommand{\maps}[3]{#1 \colon #2\longrightarrow #3}
\newcommand{\equationmaps}[5]{\begin{aligned}[t]#1 \colon #2 &\longrightarrow #3 \\	#4 &\longmapsto #5\end{aligned}}
\newcommand{\coma}{,\thinspace}
\newcommand{\pari}[2]{(#1,\thinspace #2)}
\newcommand{\where}{\mathrel{}\middle|\mathrel{}}
\newcommand{\no}[1]{{\neg}{#1}}
\newcommand{\dcomilla}[1]{{\guillemotleft}#1{\guillemotright}}
\newcommand{\separa}{\vspace*{.75\baselineskip}}
\newcommand{\semisepara}{\vspace*{.25\baselineskip}}
\newcommand{\restrict}[1]{\raisebox{-.5ex}{$|$}_{#1}}

% ========================================
% COMANDOS PERSONALIZADOS PARA COMPUTACIÓN CUÁNTICA
% ========================================

% Estados comunes
\newcommand{\zero}{\ket{0}}
\newcommand{\one}{\ket{1}}
\newcommand{\plus}{\ket{+}}
\newcommand{\minus}{\ket{-}}

% Operadores especiales
\newcommand{\tensor}{\otimes}         % Producto tensorial
\newcommand{\comp}{\circ}             % Composición

% Estados de Bell
\newcommand{\bellphi}{\ket{\Phi^+}}
\newcommand{\bellpsi}{\ket{\Psi^+}}
\newcommand{\bellphiminus}{\ket{\Phi^-}}
\newcommand{\bellpsiminus}{\ket{\Psi^-}}

\newcommand{\floor}[1]{\left\lfloor #1 \right\rfloor}
\newcommand{\ceil}[1]{\left\lceil #1 \right\rceil}


\printanswers

\begin{document}
\author{Francisco Costa Cano}
\color{negrounir}
\titulacion{Máster en computación cuántica}
\asignatura{Álgebra lineal en computación cuántica}
\bloque{1}{Fundamentos matemáticos}
%\tema{1}{Números complejos y su geometría}
%\portada

\begin{esquemaExplorador}
  \temaEsquema{Fundamentos}{
    \conceptoEsquema{Definición y operaciones}{}
    \temaEsquema{Formas}{
      \conceptoEsquema{Cartesiana}{}
      \conceptoEsquema{Binomial}{}
      \conceptoEsquema{Polar}{}
      \conceptoEsquema{Exponencial}{}
    }
  }
  \temaEsquema{Relaciones}{
    \conceptoEsquema{Fórmula de Euler}{$e^{i\theta} = \cos\theta + i\sin\theta$}
    \conceptoEsquema{Fórmula de De Moivre}{$z^n = r^n(\cos(n\theta) + i\sin(n\theta))$}
  }
  \temaEsquema{Geometria}{
    \conceptoEsquema{Plano complejo}{}
    \conceptoEsquema{Módulo y argumento}{}
    \conceptoEsquema{Distancias y ángulos}{}
    \conceptoEsquema{Raíces $n$-ésimas}{}
  }
  \temaEsquema{Funciones}{
    \conceptoEsquema{Exponencial}{$e^z = \sum_{n=0}^{\infty} \frac{z^n}{n!}$}
    \conceptoEsquema{Seno}{$\sin(z) = \frac{e^{iz} - e^{-iz}}{2i}$}
    \conceptoEsquema{Coseno}{$\cos(z) = \frac{e^{iz} + e^{-iz}}{2}$}
    \conceptoEsquema{Seno hiperbólico}{$\sinh(z) = \frac{e^z - e^{-z}}{2}$}
    \conceptoEsquema{Coseno hiperbólico}{$\cosh(z) = \frac{e^z + e^{-z}}{2}$}
  }
\end{esquemaExplorador}

\unirsection{Ideas clave}

\subsection{Introducción y objetivos}

Los números complejos constituyen el fundamento matemático esencial de la mecánica cuántica y, por tanto, de la computación cuántica. Mientras que en la física clásica las magnitudes se describen mediante números reales, en el mundo cuántico necesitamos la riqueza matemática de los números complejos para describir fenómenos como la superposición, la interferencia y el entrelazamiento.

La estrecha relación entre los números complejos y la mecánica cuántica se refleja en:

\begin{itemize}
  \item Los \textbf{estados cuánticos} son vectores en un espacio vectorial complejo.
  \item Las \textbf{amplitudes cuánticas} son números complejos que determinan las probabilidades de los estados cuánticos.
  \item La \textbf{evolución temporal} de los sistemas cuánticos se describe mediante operadores unitarios con entradas complejas.
  \item La \textbf{interferencia cuántica}, base de muchos algoritmos cuánticos, requiere la aritmética compleja para su comprensión.
\end{itemize}

En este primer capítulo estudiaremos el plano complejo desde el punto de vista algebraico, geométrico y topológico. Empezaremos motivando la existencia de estos números y definiendo el cuerpo $\C$ de los números complejos. También veremos que podemos considerar el cuerpo de los números reales $\R$ como un subcuerpo de $\C$.

Repasaremos las distintas formas de escribir los números complejos: binomial, polar y exponencial. Veremos las ventajas de trabajar con los números complejos en forma exponencial, en particular en el cálculo de las raíces $n$-ésimas de un número complejo. Finalmente repasaremos algunas propiedades de las raíces de polinomios y presentaremos el teorema fundamental del álgebra.

\subsection{El cuerpo de los números complejos}

En esta primera sección recordaremos brevemente algunas de las propiedades algebraicas y geométricas de los números complejos. Empezaremos con el origen de los números complejos.

El propósito de los números complejos es proporcionar soluciones a ecuaciones polinómicas que no tienen solución real. Por ejemplo, las ecuaciones
\[
  x^2 + 1 = 0 \qquad \text{o} \qquad x^2 + 2x + 5 = 0\,,
\]
no tienen solución en el cuerpo de los números reales. Para resolver estas ecuaciones, introducimos el concepto de \textbf{unidad imaginaria}, que denotaremos por $i$, que cumple la relación
\[ i^2 = -1\,. \]
Extendiendo los números reales con esta unidad imaginaria $i$, veremos que podemos resolver todas las ecuaciones polinómicas con coeficientes reales (y, de hecho, complejos).

\begin{defi}[El plano complejo]
  Llamaremos $\C$, \textbf{plano complejo} o \textbf{plano de Argand} a la terna $(\R^2,+,\cdot)$, junto con las operaciones de suma y producto definidas por:
  \begin{itemize}
    \item \textbf{suma}: $(a,b) + (c,d) = (a+c,b+d)$.
    \item \textbf{producto}: $(a,b)\cdot(c,d) = (ac-bd,ad+bc)$.
  \end{itemize}
\end{defi}

Esta manera de representar los números complejos es conocida como la \textbf{forma cartesiana}.

\begin{prop}
  \label{prop:propiedades_cuerpo}
  El plano complejo $\C$ con las operaciones de suma y producto es un cuerpo conmutativo. Es decir, satisface las siguientes propiedades:
  \begin{enumerate}
    \item \textbf{Clausura:} dados $z,w\in\C$ y $z+w,zw\in\C$.
    \item \textbf{Asociativa:} dados $z,w,x\in\C$, $z+(w+x) = (z+w)+x$, y $z(wx) = (zw)x$.
    \item \textbf{Conmutativa:} dados $z,w\in\C$, $z+w = w+z$, y $zw = wz$.
    \item \textbf{Distributiva:} dados $z,w,z\in\C$, $z(w+x) = zw + zx$.
    \item \textbf{Neutro para la suma:} $\exists!\ 0\in\C$ tal que $\forall z\in\C$, $z+0 = z$.
    \item \textbf{Neutro para el producto:} $\exists!\ 1\in\C$ tal que $\forall z\in\C$, $1z = z$.
    \item \textbf{Opuesto para la suma:} $\forall z\in\C$, existe $-z\in\C$ tal que $z+(-z) = 0$.
    \item \textbf{Inverso para el producto:} $\forall z\in\C\setminus\{0\}$, existe $z^{-1}\in\C$ tal que $zz^{-1} = 1$.
  \end{enumerate}
\end{prop}
\begin{proof}
  Se deja al lector que verifique que el plano complejo $\C$ con las operaciones de suma y producto es un cuerpo conmutativo (ejercicio~\ref{ejer:propiedades_cuerpo}).
\end{proof}

\begin{defi}[Números complejos]
  Llamaremos \textbf{números complejos} a los elementos del plano complejo.
\end{defi}

Tal como hemos definido las operaciones entre números complejos podemos identificar el número real $a\in\R$ con el complejo $(a,0)\in\C$. Mediante esta identificación $\R\hookrightarrow\C$ dada por $a\mapsto(a,0)$, podemos entender los números reales como un subcuerpo de los números complejos.

\begin{eje}
  Los elementos que identificamos en el enunciado que demuestran que $\C$ es un cuerpo conmutativo son:
  \begin{itemize}
    \item El elemento neutro para la suma es $0 = (0, 0)$.
    \item El elemento neutro para el producto es $1 = (1, 0)$.
    \item El elemento opuesto para la suma es $-z = (-a,-b)$.
    \item El elemento opuesto para el producto es $z^{-1} = (\frac{a}{a^2+b^2}, \frac{-b}{a^2+b^2})$.
  \end{itemize}
\end{eje}

Teniendo en cuenta esta identificación y el hecho de que
\[(0,1)\cdot(0,1) = (-1,0)\,, \]
obtenemos un elemento especial con una propiedad especial.

\begin{defi}[Unidad imaginaria]
  Llamaremos \textbf{unidad imaginaria} al complejo $i = (0,1)\in\C$.
\end{defi}

Vamos a describir los elementos de $\C$ de forma más cómoda, dándole entidad propia y sin dependencia de la identificación con el plano complejo.
Podemos escribir el complejo $z = (a,b)$ como
\[
  z = (a, b) = (a, 0) + (0, b) = a(1,0) + b(0,1) = a\cdot 1 + b\cdot i = a + bi\,,
\]
esta manera de escribir los números complejos recibe el nombre de \textbf{forma binomial}.

\begin{defi}[Parte real e imaginaria de un complejo]
  Sea $z = a+bi\in\C$ un número complejo. Los números reales
  \[ a = \Rp(z)\ ,\qquad b = \Ip(z) \]
  que lo componen reciben el nombre de \textbf{parte real} y \textbf{parte imaginaria} de $z$ respectivamente. Los números complejos con parte real nula se llaman \textbf{imaginarios}.
\end{defi}

Ya hemos visto que podemos identificar a los números reales con los complejos con parte imaginaria nula.

A diferencia del cuerpo de los números reales, el cuerpo de los números complejos no es un cuerpo ordenado. Dicho de forma más precisa, no se puede definir un orden en $\C$ que sea compatible con las operaciones del cuerpo.

\begin{defi}[Conjugación]
  El \textbf{conjugado} del número complejo $z = a+bi$ es el complejo $\bar z = a-bi$.
\end{defi}
\separa
\begin{nota}
  En algunos libros de texto al número conjugado de $z$ se le denota por $z^*$.
\end{nota}

Notamos que un número complejo $z$ es real si y solo si $z = \bar z$. Por otro lado, $z$ es imaginario si y solo si $\bar z = -z$.

La conjugación satisface las siguientes propiedades.

\begin{prop}
  \label{prop:propiedades_conjugado}
  Si $z$ y $w$ son números complejos, se cumplen las siguientes igualdades:
  \begin{itemize}
    \item Parte real: $\Rp(z) = \frac{z + \conj{z}}{2}$.
    \item Parte imaginaria: $\Ip(z) = \frac{z - \conj{z}}{2i}$.
    \item Conjugación de la suma: $\conj{z + w} = \conj{z} + \conj{w}$.
    \item Conjugación del producto: $\conj{zw} = \conj{z} \cdot \conj{w}$.
    \item Conjugación de la inversa: $\conj{z^{-1}} = (\conj{z})^{-1}$.
    \item Doble conjugación: $\conj{\conj{z}} = z$.
  \end{itemize}
\end{prop}
\begin{proof}
  Se deja al lector que verifique las propiedades del conjugado (ejercicio~\ref{ejer:propiedades_conjugado}).
\end{proof}

\begin{defi}[Módulo]
  El \textbf{módulo} de un número complejo $z$ es el real no negativo definido por
  \[ \abs{z} = \sqrt{z\bar z}\,, \]
  o expresado en forma binomial
  \[ \abs{a + bi} = \sqrt{a^2 + b^2}\,. \]
\end{defi}

Las siguientes propiedades son directamente consecuencia de la definición del módulo.

\begin{prop}
  \label{prop:propiedades_módulo}
  Si $z$ y $w$ son números complejos, se cumplen las siguientes propiedades:
  \begin{itemize}
    \item $\abs{z}\geq 0$.
    \item $\abs{z} = 0$ si y solo si $z = 0$.
    \item $\abs{zw} = \abs{z}\abs{w}$.
    \item $\abs{z} = \abs{\bar{z}}$.
  \end{itemize}
\end{prop}
\begin{proof}
  Se deja al lector que verifique las propiedades del módulo (ejercicio~\ref{ejer:propiedades_módulo}).
\end{proof}

\begin{theo}[Desigualdad triangular]
  \label{theo:desigualdad_triangular}
  Si $z$ y $w$ son números complejos, se cumple que
  \[
    \abs{z+w}\leq\abs{z}+\abs{w}\,.
  \]
\end{theo}
\begin{proof}
  Aplicando la definición del módulo
  \begin{align*}
    \abs{z+w}^2 & = (z+w)\overline{(z+w)}                           \\
                & = z\bar{z} + z\bar{w} + \bar{z}w + \bar{w}\bar{w} \\
                & = \abs{z}^2 + \abs{w}^2 + z\bar{w} + \bar{z}w\,.
  \end{align*}
  Ahora observamos que los dos últimos sumandos por el resultado~\ref{ejer:propiedades_conjugado} se tiene que
  \begin{align*}
    z\bar{w} + \bar{z}w & = z\bar{w} + \overline{z\bar{w}} = 2\Re(z\bar{w})\,.
  \end{align*}
  Además, la parte real de un número complejo siempre es menor que su módulo, por lo que
  \begin{align*}
    \abs{z+w}^2 & = \abs{z}^2 + \abs{w}^2 + 2\Re(z\bar{w})     \\
                & \leq \abs{z}^2 + \abs{w}^2 + 2\abs{z}\abs{w} \\
                & = (\abs{z} + \abs{w})^2\,.
  \end{align*}
  Tomando raices cuadradas obtenemos la desigualdad triangular.
\end{proof}

Para una cantidad finita de sumandos, podemos extender la desigualdad triangular al siguiente resultado.

\begin{theo}[Desigualdad triangular finita]
  \label{theo:desigualdad_triangular_finita}
  Si $z_1, z_2, \ldots, z_n$ son números complejos, se cumple que
  \[
    \abs{z_1 + z_2 + \cdots + z_n}\leq\abs{z_1} + \abs{z_2} + \cdots + \abs{z_n}\,.
  \]
\end{theo}
\begin{proof}
  Por la propiedad asociativa de la suma, y aplicando la desigualdad triangular repetidamente obtenemos
  \begin{align*}
    \abs{z_1 + z_2 + \cdots + z_n} & \leq \abs{z_1} + \abs{z_2 + \cdots + z_n}             \\
                                   & \leq \abs{z_1} + \abs{z_2} + \abs{z_3 + \cdots + z_n} \\
                                   & \leq \cdots                                           \\
                                   & \leq \abs{z_1} + \abs{z_2} + \cdots + \abs{z_n}\,.
  \end{align*}
\end{proof}

La conjugación y el módulo permiten calcular fácilmente el inverso multiplicativo de cualquier complejo $z\in\C^\ast = \C\setminus\{0\}$
\begin{equation}
  \label{eq:inverso_complejo}
  z^{-1} = \frac{\bar z}{\abs{z}^2}\,.
\end{equation}

\begin{eje}
  Para encontrar el inverso de $z = 1 + i$ debemos calcular $\bar z$ y $\abs{z}$.
  \begin{align*}
    z^{-1} & = \frac{\bar z}{\abs{z}^2}      \\
           & = \frac{1 - i}{(1 + i)(1 - i)}  \\
           & = \frac{1 - i}{1 + 1}           \\
           & = \frac{1 - i}{2}           \,.
  \end{align*}
\end{eje}

\subsection{Forma polar}

Recordamos que un punto $(x,y)\in\R^2$, diferente del origen de coordenadas, puede expresarse en \textbf{coordenadas polares} $(r,\theta)$, de forma que
\[ x = r\cos\theta\ ,\quad y = r\sin\theta\,. \]

\begin{defi}
  \label{def:argumento}
  Sea $z=a+bi$ un número complejo no nulo. Diremos que el ángulo $\theta$ es un \textbf{argumento} de $z$ si:
  \begin{align*}
    \cos(\theta) & = \frac{a}{|z|}\,. \\
    \sin(\theta) & = \frac{b}{|z|}\,.
  \end{align*}
  Si $z=0$ diremos que su argumento es nulo.
\end{defi}

Observaremos que el argumento de un número complejo no está completamente definido, pues si $\theta$ cumple las propiedades del argumento, entonces $\theta + 2\pi$ también las cumple. Por lo tanto, podemos considerar el conjunto de todos los argumentos de un número complejo no nulo $z$ como
\[
  \arg z = \left\{\theta\in\R\mid \cos(\theta) = \frac{a}{|z|}\,, \sin(\theta) = \frac{b}{|z|}\right\}\,.
\]

Si restringimos el valor del argumento al intervalo $[0, 2\pi)$, tendremos unicidad en la definición y por este motivo le damos a este valor el nombre de \textbf{argumento principal} y lo denotamos por $\Arg z$.

Observamos por último que de las coordenadas polares, podemos deducir $r = \sqrt{x^2+y^2}$, y que este valor es exactamente el módulo del número complejo asociado.

\begin{defi}[Forma polar]
  Sea $z$ un número  complejo. Llamaremos \textbf{forma polar} a la expresión
  \[ z = |z|(\cos(\Arg z) + i\sin(\Arg z))\,. \]
\end{defi}


\subsection{Forma exponencial}

El interés de la forma polar radica en las siguientes fórmulas, que relacionan los números complejos con las funciones trigonométricas. Estas fórmulas son fundamentales en muchas aplicaciones de los números complejos, incluyendo la computación cuántica.

\begin{theo}[Fórmula de Euler]
  Sea $\theta\in\R$, se cumple
  \begin{equation}
    \label{eq:formula-euler}
    e^{i\theta} = \cos\theta + i\sin\theta\,.
  \end{equation}
\end{theo}
\begin{proof}
  Esta demostración se deja para el final del tema, una vez que se hayan definido las función exponencial y las funciones trigonométricas.
\end{proof}

\begin{theo}[Fórmula de De Moivre]
  Para $z$ un número complejo y $n \in \Z$
  \begin{equation}
    \label{eq:formula-moivre}
    z^n = |z|^n(\cos(n\Arg z) + i\sin(n\Arg z))\,.
  \end{equation}
\end{theo}
\begin{proof}
  Esta demostración se deja para el final del tema, una vez que se hayan probado las propiedades de la función exponencial.
\end{proof}

La fórmula de Euler es muy importante porque motiva la siguiente notación, que será de extrema utilidad a partir de ahora.

\begin{defi}[Forma exponencial]
  \label{def:forma-exponencial}
  Un número complejo $z$ puede escribirse en su \textbf{forma exponencial} como
  \[ z = |z|e^{i\Arg z}\,. \]
\end{defi}

\begin{eje}
  Consideremos el número complejo $z=-1-\sqrt{3}i$, vamos a calcular el conjugado, así como las distintas formas de representación.

  Por definición, el conjugado es el número $\bar{z} = -1 + \sqrt{3}i$ y si lo multiplicamos por $z$ tendremos $z\bar{z}=(-1)^2+(\sqrt{3})^2= 4$.

  El módulo de $z$ es $|z|=\sqrt{z\bar{z}}=\sqrt{4}=2$.

  Para calcular el argumento, como $\Rp(z)\neq 0$ tenemos que
  \[
    \theta = \arctan(\frac{-\sqrt{3}}{-1}) = \arctan(\sqrt{3}) = \frac{\pi}{3}\,.
  \]
  Pero este cálculo no tiene en cuenta el verdadero cuadrante del ángulo. En este ejemplo, el argumento principal de $z$ está en tercer cuadrante, por lo tanto
  \[
    \Arg z = \frac{\pi}{3}+\pi = \frac{4\pi}{3}\,.
  \]

  Expresado $z$ de forma polar sería
  \[
    z=2\cos(\frac{4\pi}{3})+2\sin(\frac{4\pi}{3})i\,,
  \]
  mientras que expresado de forma exponencial sería
  \[
    z=2e^{\frac{4\pi i}{3}}\,.
  \]
\end{eje}

\subsection{Geometría compleja}

Notamos que un complejo de módulo 1 es de la forma $e^{i\theta}$. Geometricamente, la multiplicación $z\mapsto e^{i\theta}z$ es una rotación en el plano complejo de ángulo $\theta$ alrededor del origen.

La forma exponencial permite hacer algunos cálculos fácilmente, especialmente los que involucran productos y cocientes.

\begin{prop}
  Dados $re^{i\theta}$ y $se^{i\varphi}$ dos números complejos expresados en forma exponencial, se tiene:
  \begin{itemize}
    \item $\conj{re^{i\theta}} = re^{-i\theta}$.
    \item $re^{i\theta}\cdot se^{i\varphi} = (rs)e^{i(\theta+\varphi)}$.
    \item Para $r$ no nulo, $\left(re^{i\theta}\right)^{-1} = r^{-1}e^{-i\theta}$.
  \end{itemize}
\end{prop}
\begin{proof}
  Para la primera igualdad debemos observar que
  \[
    \conj{e^{i\theta}} = \cos(\theta)-i\sin(\theta) = \cos(-\theta)+i\sin(-\theta) = e^{-i\theta}\,.
  \]

  Para la segunda igualdad, observamos que
  \begin{align*}
    e^{i\theta} e^{i\varphi} & = (\cos(\theta)+i\sin(\theta))(\cos(\varphi)+i\sin(\varphi))                                                 \\
                             & = \cos(\theta)\cos(\varphi)-\sin(\theta)\sin(\varphi)+i(\sin(\theta)\cos(\varphi)+\cos(\theta)\sin(\varphi)) \\
                             & = \cos(\theta+\varphi)+i\sin(\theta+\varphi)                                                                 \\
                             & = e^{i(\theta+\varphi)}\,.
  \end{align*}

  Por último, para la tercera igualdad, por la definición del inverso de un complejo~\eqref{eq:inverso_complejo}
  \[
    \left(e^{i\theta}\right)^{-1} = \frac{\overline{e^{i\theta}}}{|e^{i\theta}|^2} = \frac{e^{-i\theta}}{1} = e^{-i\theta}\,.
  \]
\end{proof}

Otra ventaja de la notación exponencial es que facilita enormemente el cálculo de las raíces $n$-ésimas.

\begin{defi}[Raíces $n$-ésimas de un complejo]
  Sea $z = re^{i\theta}$ un número complejo no nulo y $n\in\N$. Las \textbf{raíces $n$-ésimas} de $z$ son los complejos $w = \rho e^{i\psi}$ tales que $w^n = z$. Por lo tanto
  \[
    \rho = \sqrt[n]{r}\ ,\qquad \psi = \frac{\theta + 2\pi k}{n}\,,\quad k = 0,1,\dotsc,n-1\,.
  \]
  En particular, las \textbf{raíces $n$-ésimas de la unidad} son
  \[
    \sqrt[n]{1}=\set{e^{i\theta}\ ,\quad \theta = \frac{2\pi k}{n}\,,\quad k = 0,1,\dotsc,n-1}\,.
  \]
\end{defi}

Geometricamente, las raíces $n$-ésimas de la unidad de un complejo son los vértices de un polígono regular de $n$ lados centrado en el origen y circunscrito en una esfera de radio 1.

\begin{eje}
  Vamos a calcular las raíces cúbicas del número complejo $z = -1+i$. Si escribimos el complejo $z$ en forma exponencial, tenemos
  \[ z = \sqrt{2}e^{\frac{3\pi}{4}i}\,. \]

  Si $w = \rho e^{i\psi}$ es una raíz cúbica de $z$, entonces
  \[
    w^3 = \rho^3 e^{3i\psi} = \sqrt{2}e^{\frac{3\pi}{4}i}\Rightarrow \begin{dcases}
      \rho^3 = \sqrt{2} \Rightarrow \rho = \sqrt[6]{2} \\
      3\psi = \frac{3\pi}{4} + 2\pi k\Rightarrow \psi = \frac{(3+8k)\pi}{12}\ ,\quad k\in\Z\,.
    \end{dcases}\,.
  \]

  Por lo tanto, las raíces cúbicas son:
  \begin{align*}
     & k=0:\quad w_0 = \sqrt[6]{2}e^{\frac{\pi}{4}i}\,,    \\
     & k=1:\quad w_1 = \sqrt[6]{2}e^{\frac{11\pi}{12}i}\,, \\
     & k=2:\quad w_2 = \sqrt[6]{2}e^{\frac{19\pi}{12}i}\,.
  \end{align*}
  Como podemos ver en la figura~\ref{fig:raices_cubicas}, estas raíces son los vértices de un triángulo equilátero inscrito en una circunferencia de radio $\sqrt[6]{2}$.
\end{eje}

\begin{figure}[h]
  \centering
  \includegraphics[width=\textwidth]{imgs/raices_cubicas.png}
  \caption{Raíces cúbicas del complejo $z = -1+i$. Fuente: https://complex-analysis.com}.
  \label{fig:raices_cubicas}
\end{figure}

\begin{eje}
  Si tenemos que calcular la raíz cúbica de $27$, la respuesta rápida sería dar solo la raíz real $3$, y el resultado sería incompleto, porque también debemos calcular las otras dos raíces complejas.

  Si escribimos $27$ en forma polar, tenemos $27 = 27e^{i\cdot 0}$.

  Si $w = \rho e^{i\psi}$ es una raíz cúbica de $27$, entonces
  \[
    w^3 = \rho^3 e^{3i\psi} = 27e^{i\cdot 0}\Rightarrow \begin{dcases}
      \rho^3 = 27 \Rightarrow \rho = 3 \\
      3\psi = 2\pi k\Rightarrow \psi = \frac{2\pi k}{3}\ ,\quad k\in\Z\,.
    \end{dcases}\,.
  \]

  Por lo tanto, las raíces cúbicas son:
  \begin{align*}
     & k=0:\quad w_0 = 3e^{i\cdot 0}\,,        \\
     & k=1:\quad w_1 = 3e^{i\frac{2\pi}{3}}\,, \\
     & k=2:\quad w_2 = 3e^{i\frac{4\pi}{3}}\,.
  \end{align*}
\end{eje}

\subsubsection{Geometría euclídea}

Como elementos geométricos heredados de la geometría euclídea de $\R^2$ tenemos:
\begin{itemize}
  \item \textbf{Rectas}: En particular, las rectas paralelas a los ejes $\Rp(z) = a$, o $\Ip(z) = b$.
  \item \textbf{Semiplanos} (abiertos o cerrados). En particular, $\Rp(z) > a$, $\Ip(z) > b$, etc.
  \item \textbf{Bandas}: regiones comprendidas entre dos rectas paralelas.
  \item \textbf{Discos}: (abiertos o cerrados):
        \[ D(z_0,r) = \{ z\in\C\ \vert\ d(z_0,z)<r \}\ ,\qquad \bar D(z_0,r) = \{ z\in\C\ \vert\ d(z_0,z)\leq r \}\,. \]
  \item \textbf{Circunferencias}:
        \[ S(z_0,r) = \{ z\in\C\ \vert\ d(z_0,z) = r \} = \partial D(z_0,r)\,. \]
        En particular, la circunferencia unidad $S_1 = \partial D(0, 1)= \{z\in\C\ \vert\ \abs{z}=1\}$.
  \item \textbf{Coronas circulares}:
        \[ C(z_0,r_1,r_2) = \{ z\in\C\ \vert\ r_1 < d(z_0,z) < r_2 \}\,. \]
  \item El \textbf{módulo} es la distancia del origen al punto, $|z|=d(0, z)$.
  \item El \textbf{conjugado} es la reflexión de respecto al eje real.
  \item La \textbf{suma} corresponde a la suma vectorial en el plano.
\end{itemize}

Es conveniente observar, como se aprecia en la figura~\ref{fig:producto_complejo} que la multiplicación de números complejos tiene dos efectos geométricos:
\begin{itemize}
  \item Una rotación de ángulo la suma de los argumentos.
  \item Una multiplicación de los módulos.
\end{itemize}

\begin{figure}[h]
  \centering
  \includegraphics[width=0.5\textwidth]{imgs/producto_complejo.png}
  \caption{Producto de dos números complejos $2e^{i\frac{\pi}{6}}$ y $3e^{i\frac{\pi}{4}}$. Fuente: Elaboración propia con Geogebra.}
  \label{fig:producto_complejo}
\end{figure}

\subsection{Funciones complejas elementales}

Aunque el estudio de las funciones complejas es ligeramente diferente al de las funciones reales, el estudio de las funciones complejas elementales es similar al de las funciones reales elementales y su definición es una extensión analítica de las funciones reales elementales cumpliendo las mismas propiedades algebraicas.

Por ello solo vamos a dar la definición de las funciones complejas elementales y no su estudio.

\begin{defi}[Función compleja]
  Sea $A$ un subconjunto de $\C$. Una \textbf{función compleja} es una función $f\colon A\to\C$ que toma valores complejos.
\end{defi}

Toda función compleja puede escribirse de forma única como $f = u+iv$, donde $u,v\colon A\to\R$ son funciones reales, llamadas respectivamente \textbf{parte real}, $u = \Rp(f)$, y \textbf{parte imaginaria}, $v = \Ip(f)$.

Ya hemos usado la función exponencial para expresar su valor con exponente un número complejo, pero podemos obtener la formulación analítica de la función exponencial, extendiendo la correspondiente función analítica de la exponencial real.

\begin{defi}[Función exponencial compleja]
  Para $z \in \C$, la función exponencial se define como
  \[
    e^z = \sum_{n=0}^{\infty} \frac{z^n}{n!}\,.
  \]
\end{defi}

La función exponencial compleja, cumple las mismas propiedades que la exponencial real en cuanto a las manipulaciones algebraicas. El siguiente resultado, cuya demostración se basa en la fórmula de Euler~\eqref{eq:formula-euler}.

\begin{prop}
  Sean $z=a+bi$ y $w$ dos números complejos con $a,b\in\R$, se cumple:
  \begin{itemize}
    \item $e^{a+bi} = e^a (\cos(b) + i\sin(b))$.
    \item $e^{z+w} = e^z e^w$.
    \item $(e^z)^n = e^{nz}\ \forall n\in\N$.
  \end{itemize}
\end{prop}

La función exponencial compleja es analítica en todo el plano complejo, contínua e infinitamente derivable, además su derivada es ella misma.

\begin{prop}
  La función exponencial compleja cumple
  \[
    \frac{d}{dz} e^z = e^z\,.
  \]
\end{prop}

Si usamos la fórmula de Euler~\eqref{eq:formula-euler} para el ángulo $\theta = \pi$, obtenemos una de las fórmulas más importantes de la matemática
\[
  e^{i\cdot \pi} + 1 = 0\,.
\]

A partir de la definición de la función exponencial y usando la fórmula de Euler~\eqref{eq:formula-euler}, podemos definir analíticamente también las funciones seno y coseno para números complejos.

\begin{defi}[Funciones trigonométricas complejas]
  Para $z \in \C$, las funciones seno y coseno se definen como
  \begin{align*}
    \cos (z) & = \frac{e^{iz} + e^{-iz}}{2}\,.  \\
    \sin (z) & = \frac{e^{iz} - e^{-iz}}{2i}\,.
  \end{align*}
\end{defi}

Las funciones seno y coseno cumplen algunas de las propiedades que las funciones seno y coseno reales, como por ejemplo que son $2\pi$-periódicas, aunque la principal difrencia es que las funciones seno y coseno complejas no están acotadas.

\begin{prop}
  Sean $z$ y $w$ dos números complejos, se cumple:
  \begin{itemize}
    \item $\cos(z+w) = \cos(z)\cos(w) - \sin(z)\sin(w)$.
    \item $\sin(z+w) = \sin(z)\cos(w) + \cos(z)\sin(w)$.
  \end{itemize}
\end{prop}

De forma análoga definimos analíticamente las funciones seno hiperbólico y coseno hiperbólico para números complejos.

\begin{defi}[Funciones hiperbólicas complejas]
  Para $z \in \C$, las funciones seno hiperbólico y coseno hiperbólico se definen como
  \begin{align*}
    \cosh (z) & = \frac{e^{z} + e^{-z}}{2}\,. \\
    \sinh (z) & = \frac{e^{z} - e^{-z}}{2}\,.
  \end{align*}
\end{defi}

%\unirsection{A fondo}

Los siguientes libros pueden servir de material de apoyo y para profundizar más en los contenidos de este tema.

\textbf{O’Neill, P. V. Matemáticas avanzadas para Ingeniería. Cengage Learning, 2012}

Del primer recurso se recomienda la lectura del capítulo 8.

\textbf{Yanofsy, N. S. y Mannucci, M. A. Quantum Computing for Computer Scientists. Cambridge University Press, 2019}

Se recomienda la lectura del capítulo 1.

\textbf{Asmar, N. K. y Grafakos, L. Complex Analysis with Application. Springer, 2018}

Se recomienda la lectura del capítulo 1.

\textbf{Lang, S. Complex Analysis. Springer, 2003}

Se recomienda la lectura del capítulo 1.

%\unirsection{Problemas}

\begin{questions}
  \question \label{ejer:propiedades_cuerpo} Demostrar las propiedades del cuerpo $\C$ enunciadas en el resultado~\ref{prop:propiedades_cuerpo}.

  \question \label{ejer:propiedades_conjugado} Demostrar las propiedades del conjugado enunciadas en el resultado~\ref{prop:propiedades_conjugado}.

  \question \label{ejer:propiedades_módulo} Demostrar las propiedades del módulo enunciadas en el resultado~\ref{prop:propiedades_módulo}.

  \question Calcular las siguientes operaciones, expresando el resultado en forma binomial:
  \begin{parts}
    \part $(3 + 2i) + (1 - 4i)$
    \part $(2 + i)(3 - 2i)$
    \part $\frac{1 + 2i}{2 - i}$
    \part $|3 + 4i|$
    \part $\conj{(2 - 3i)(1 + i)}$
  \end{parts}

  \question Demostrar que para cualquier par de números complejos $z,w\in\C$ se cumple que
  \[
    \abs{z}-\abs{w}\leq \abs{z-w}\,.
  \]

  \question Convertir a forma polar:
  \begin{parts}
    \part $z_1 = 1 + i$
    \part $z_2 = -2 + 2i\sqrt{3}$
    \part $z_3 = -4i$
  \end{parts}

  \question Demostrar que para cualquier $z \in \C$ se cumple que $|z|^2 = z \cdot \conj{z}$.

  \question Expresar en forma exponencial y realizar las operaciones:
  \begin{parts}
    \part $(1 + i)^8$
    \part $\dfrac{2e^{i\pi/3}}{1 + i\sqrt{3}}$
    \part $\sqrt[3]{-8i}$
  \end{parts}

  \question Encontrar todos los números complejos $z$ que satisfacen:
  \begin{parts}
    \part $|z - i| = |z + 1|$
    \part $|z|^2 + z + \conj{z} = 3$
    \part $z^4 = -16$
  \end{parts}

  \question Un número complejo $z$ satisface $z^3 = 2 + 2i\sqrt{3}$:
  \begin{parts}
    \part Expresar $2 + 2i\sqrt{3}$ en forma exponencial
    \part Encontrar las tres raíces cúbicas
    \part Representar las raíces en el plano complejo
    \part ¿Cuál es la interpretación geométrica de estas raíces?
  \end{parts}

  \question Calcular todas las raíces cuartas de $16i$.

  \question Sean $z, w$ dos números complejos tal que $\bar{z}w\neq 1$. Demostrar que
  \[
    \left|\frac{z-w}{1-\bar{z}w}\right| \leq 1\,.
  \]
  Demostrar que la igualdad se da si y solo si $|z| = |w| = 1$.

  \question Demostrar que para cualquier número complejo $z\neq 1$ se tiene que
  \[
    1+z+\dots+z^n = \frac{z^{n+1}-1}{z-1}\,.
  \]

  \question Calcular el valor en forma binomial de $i^i$.

  \question Sea la función compleja $f(z) = z^2 + 2z + 2$.
  \begin{parts}
    \part Calcular la parte real y la parte imaginaria de $f$.
    \part Si $f(z) = u(z) + i v(z)$ donde $u,v:\C\to\R$ son funciones reales. y $z = x + iy$ donde $x,y\in\R$. Demostrar que
    \[
      \frac{\partial u}{\partial x} = \frac{\partial v}{\partial y}\quad\text{y}\quad \frac{\partial u}{\partial y} = -\frac{\partial v}{\partial x}\,.
    \]
  \end{parts}
\end{questions}

%\tema{2}{Espacios vectoriales complejos}
%\portada

\begin{esquemaExplorador}
  \temaEsquema{Vectores complejos}{
    \conceptoEsquema{Definición y representación}
    \conceptoEsquema{Operaciones vectoriales}
    \conceptoEsquema{Producto escalar complejo}
  }
  \temaEsquema{Estructura algebraica}{
    \conceptoEsquema{Axiomas de espacio vectorial}
    \conceptoEsquema{Subespacios vectoriales}
    \conceptoEsquema{Propiedades fundamentales}
  }
  \temaEsquema{Dependencia e independencia}{
    \conceptoEsquema{Combinaciones lineales}
    \conceptoEsquema{Independencia lineal}
    \conceptoEsquema{Sistemas generadores}
  }
  \temaEsquema{Bases y dimensión}{
    \conceptoEsquema{Concepto de base}
    \conceptoEsquema{Base canónica}
    \conceptoEsquema{Dimensión finita}
  }
  \temaEsquema{Postulados de la mecánica cuántica}{
    \conceptoEsquema{Postulado I: Estados cuánticos}
  }
\end{esquemaExplorador}

\unirsection{Ideas clave}

\subsection{Introducción y objetivos}

En este tema desarrollaremos la teoría general de espacios vectoriales complejos, comenzando con vectores individuales y construyendo progresivamente hasta llegar a conceptos como bases, dimensión y subespacios.

En este tema sirve para repasar los conceptos fundamentales de espacios vectoriales en álgebra lineal, sin usar ninguna propiedad específica de los números complejos, aunque ya trabajamos siempre sobre el cuerpo $\C$.

Esta base teórica será fundamental para comprender en los próximos temas la manipulación de estados cuánticos como vectores.

\subsection{Vectores complejos}

\begin{defi}[Vector complejo]
  Un vector complejo de dimensión $n$ es una $n$-tupla ordenada de números complejos:
  $$\mathbf{v} = \begin{pmatrix} v_1 \\ v_2 \\ \vdots \\ v_n \end{pmatrix}$$
  donde $v_1, v_2, \ldots, v_n \in \C$.

  El conjunto de todos los vectores complejos de dimensión $n$ se denota por $\C^n$.
\end{defi}

\begin{defi}[Operaciones vectoriales]
  Sean $\mathbf{u}, \mathbf{v} \in \C^n$ y $\alpha \in \C$. Definimos:

  \textbf{Suma de vectores:}
  $$\mathbf{u} + \mathbf{v} = \begin{pmatrix} u_1 + v_1 \\ u_2 + v_2 \\ \vdots \\ u_n + v_n \end{pmatrix}$$

  \textbf{Producto por escalar:}
  $$\alpha \mathbf{v} = \begin{pmatrix} \alpha v_1 \\ \alpha v_2 \\ \vdots \\ \alpha v_n \end{pmatrix}$$
\end{defi}

\begin{eje}[Operaciones básicas en $\C^3$]
  Sean $\mathbf{u} = \begin{pmatrix} 1+i \\ 2 \\ 3-i \end{pmatrix}$ y $\mathbf{v} = \begin{pmatrix} 2-i \\ 1+2i \\ -1 \end{pmatrix}$. Entonces:

  $$\mathbf{u} + \mathbf{v} = \begin{pmatrix} (1+i)+(2-i) \\ 2+(1+2i) \\ (3-i)+(-1) \end{pmatrix} = \begin{pmatrix} 3 \\ 3+2i \\ 2-i \end{pmatrix}$$

  $$i \mathbf{u} = \begin{pmatrix} i(1+i) \\ i \cdot 2 \\ i(3-i) \end{pmatrix} = \begin{pmatrix} i-1 \\ 2i \\ 3i+1 \end{pmatrix}$$
\end{eje}

\subsection{Estructura de espacio vectorial complejo}

\begin{defi}[Espacio vectorial complejo]
  Un espacio vectorial complejo es un conjunto $V$ equipado con dos operaciones:
  \begin{itemize}
    \item Una operación de suma: $V \times V \to V$, denotada $(\mathbf{u}, \mathbf{v}) \mapsto \mathbf{u} + \mathbf{v}$
    \item Una operación de producto por escalar: $\C \times V \to V$, denotada $(\alpha, \mathbf{v}) \mapsto \alpha \mathbf{v}$
  \end{itemize}
  que satisfacen los siguientes axiomas para todos $\mathbf{u}, \mathbf{v}, \mathbf{w} \in V$ y $\alpha, \beta \in \C$:
\end{defi}

\begin{enumerate}
  \item \textbf{Axiomas de la suma:}
        \begin{itemize}
          \item (A1) $\mathbf{u} + \mathbf{v} = \mathbf{v} + \mathbf{u}$ (conmutatividad)
          \item (A2) $(\mathbf{u} + \mathbf{v}) + \mathbf{w} = \mathbf{u} + (\mathbf{v} + \mathbf{w})$ (asociatividad)
          \item (A3) Existe $\mathbf{0} \in V$ tal que $\mathbf{v} + \mathbf{0} = \mathbf{v}$ (elemento neutro)
          \item (A4) Para cada $\mathbf{v} \in V$ existe $-\mathbf{v} \in V$ tal que $\mathbf{v} + (-\mathbf{v}) = \mathbf{0}$ (elemento opuesto)
        \end{itemize}

  \item \textbf{Axiomas del producto por escalar:}
        \begin{itemize}
          \item (M1) $\alpha(\beta \mathbf{v}) = (\alpha\beta)\mathbf{v}$ (asociatividad mixta)
          \item (M2) $1 \cdot \mathbf{v} = \mathbf{v}$ (elemento neutro multiplicativo)
        \end{itemize}

  \item \textbf{Axiomas de distributividad:}
        \begin{itemize}
          \item (D1) $\alpha(\mathbf{u} + \mathbf{v}) = \alpha\mathbf{u} + \alpha\mathbf{v}$ (distributividad del escalar respecto a la suma vectorial)
          \item (D2) $(\alpha + \beta)\mathbf{v} = \alpha\mathbf{v} + \beta\mathbf{v}$ (distributividad de la suma escalar)
        \end{itemize}
\end{enumerate}

\begin{eje}[Espacios vectoriales complejos]
  \begin{enumerate}
    \item $\C^n$ con las operaciones estándar es un espacio vectorial complejo.
    \item El conjunto de polinomios complejos de grado menor o igual que $n$:
          $$\mathcal{P}_n(\C) = \{a_0 + a_1 z + a_2 z^2 + \cdots + a_n z^n : a_0, a_1, \ldots, a_n \in \C\}$$
    \item El conjunto de matrices complejas $m \times n$: $\C^{m \times n}$
    \item El conjunto de funciones complejas continuas en un intervalo: $\mathcal{C}([a,b], \C)$
  \end{enumerate}
\end{eje}

\begin{prop}
  En todo espacio vectorial complejo $V$:
  \begin{enumerate}
    \item El elemento neutro $\mathbf{0}$ es único
    \item Para cada $\mathbf{v} \in V$, el opuesto $-\mathbf{v}$ es único
    \item $0 \cdot \mathbf{v} = \mathbf{0}$ para todo $\mathbf{v} \in V$
    \item $\alpha \cdot \mathbf{0} = \mathbf{0}$ para todo $\alpha \in \C$
    \item Si $\alpha \mathbf{v} = \mathbf{0}$, entonces $\alpha = 0$ o $\mathbf{v} = \mathbf{0}$
  \end{enumerate}
\end{prop}

\subsection{Subespacios vectoriales}

\begin{defi}[Subespacio vectorial]
  Sea $V$ un espacio vectorial complejo. Un subconjunto $W \subseteq V$ es un subespacio vectorial de $V$ si:
  \begin{enumerate}
    \item $\mathbf{0} \in W$ (contiene el vector cero)
    \item Si $\mathbf{u}, \mathbf{v} \in W$, entonces $\mathbf{u} + \mathbf{v} \in W$ (cerrado bajo la suma)
    \item Si $\mathbf{v} \in W$ y $\alpha \in \C$, entonces $\alpha \mathbf{v} \in W$ (cerrado bajo el producto por escalar)
  \end{enumerate}
\end{defi}

\begin{eje}[Subespacios de $\C^3$]
  \begin{enumerate}
    \item $W_1 = \left\{\begin{pmatrix} x \\ y \\ 0 \end{pmatrix} : x, y \in \C\right\}$ (el "plano xy")

    \item $W_2 = \left\{\begin{pmatrix} x \\ y \\ z \end{pmatrix} \in \C^3 : x + y - z = 0\right\}$ (un plano que pasa por el origen)
  \end{enumerate}
\end{eje}

\begin{theo}[Caracterización de subespacios]
  Un subconjunto no vacío $W$ de un espacio vectorial $V$ es un subespacio si y solo si es cerrado bajo combinaciones lineales, es decir:
  $$\alpha \mathbf{u} + \beta \mathbf{v} \in W \quad \text{para todos } \mathbf{u}, \mathbf{v} \in W \text{ y } \alpha, \beta \in \C$$
\end{theo}

\subsection{Combinaciones lineales e independencia lineal}

\begin{defi}[Combinación lineal]
  Sea $V$ un espacio vectorial complejo y sean $\mathbf{v}_1, \mathbf{v}_2, \ldots, \mathbf{v}_k \in V$. Una combinación lineal de estos vectores es un vector de la forma:
  $$\alpha_1 \mathbf{v}_1 + \alpha_2 \mathbf{v}_2 + \cdots + \alpha_k \mathbf{v}_k$$
  donde $\alpha_1, \alpha_2, \ldots, \alpha_k \in \C$.
\end{defi}

\begin{defi}[Subespacio generado]
  El subespacio generado por los vectores $\mathbf{v}_1, \mathbf{v}_2, \ldots, \mathbf{v}_k$ es:
  $$\text{gen}\{\mathbf{v}_1, \mathbf{v}_2, \ldots, \mathbf{v}_k\} = \left\{\sum_{i=1}^k \alpha_i \mathbf{v}_i : \alpha_1, \ldots, \alpha_k \in \C\right\}$$
\end{defi}

\begin{eje}[Subespacio generado en $\C^3$]
  Sea $S = \left\{\begin{pmatrix} 1 \\ 0 \\ i \end{pmatrix}, \begin{pmatrix} i \\ 1 \\ 0 \end{pmatrix}\right\}$. Entonces:

  $$\text{gen}(S) = \left\{\alpha \begin{pmatrix} 1 \\ 0 \\ i \end{pmatrix} + \beta \begin{pmatrix} i \\ 1 \\ 0 \end{pmatrix} : \alpha, \beta \in \C\right\} = \left\{\begin{pmatrix} \alpha + \beta i \\ \beta \\ \alpha i \end{pmatrix} : \alpha, \beta \in \C\right\}$$
\end{eje}

\begin{defi}[Independencia lineal]
  Los vectores $\mathbf{v}_1, \mathbf{v}_2, \ldots, \mathbf{v}_k$ en un espacio vectorial $V$ son linealmente independientes si la única solución de la ecuación:
  $$\alpha_1 \mathbf{v}_1 + \alpha_2 \mathbf{v}_2 + \cdots + \alpha_k \mathbf{v}_k = \mathbf{0}$$
  es $\alpha_1 = \alpha_2 = \cdots = \alpha_k = 0$.

  En caso contrario, se dice que son linealmente dependientes.
\end{defi}

\begin{eje}
  Determinar si los vectores $\mathbf{v}_1 = \begin{pmatrix} 1 \\ i \\ 0 \end{pmatrix}$, $\mathbf{v}_2 = \begin{pmatrix} i \\ 0 \\ 1 \end{pmatrix}$, $\mathbf{v}_3 = \begin{pmatrix} 1+i \\ i \\ 1 \end{pmatrix}$ son linealmente independientes.

  \textbf{Solución:} Planteamos la ecuación:
  $$\alpha_1 \begin{pmatrix} 1 \\ i \\ 0 \end{pmatrix} + \alpha_2 \begin{pmatrix} i \\ 0 \\ 1 \end{pmatrix} + \alpha_3 \begin{pmatrix} 1+i \\ i \\ 1 \end{pmatrix} = \begin{pmatrix} 0 \\ 0 \\ 0 \end{pmatrix}$$

  Esto nos da el sistema:
  \begin{align}
    \alpha_1 + i\alpha_2 + (1+i)\alpha_3 & = 0 \\
    i\alpha_1 + i\alpha_3                & = 0 \\
    \alpha_2 + \alpha_3                  & = 0
  \end{align}

  De la segunda ecuación: $\alpha_3 = -\alpha_1$. De la tercera: $\alpha_2 = -\alpha_3 = \alpha_1$. Sustituyendo en la primera:
  $$\alpha_1 + i\alpha_1 + (1+i)(-\alpha_1) = \alpha_1 + i\alpha_1 - \alpha_1 - i\alpha_1 = 0$$

  Esta ecuación se satisface para cualquier $\alpha_1$, por lo que los vectores son linealmente dependientes.
\end{eje}

\begin{prop}
  \begin{enumerate}
    \item Un conjunto que contiene el vector cero es linealmente dependiente
    \item Si un subconjunto de vectores es linealmente dependiente, entonces el conjunto completo es linealmente dependiente
    \item Si un conjunto de vectores es linealmente independiente, entonces cualquier subconjunto también lo es
    \item En $\C^n$, cualquier conjunto de más de $n$ vectores es linealmente dependiente
  \end{enumerate}
\end{prop}

\subsection{Bases y dimensión}

\begin{defi}[Base]
  Un conjunto $\mathcal{B} = \{\mathbf{v}_1, \mathbf{v}_2, \ldots, \mathbf{v}_n\}$ de vectores en un espacio vectorial $V$ es una base de $V$ si:
  \begin{enumerate}
    \item $\mathcal{B}$ es linealmente independiente
    \item $\text{gen}(\mathcal{B}) = V$ (genera todo el espacio)
  \end{enumerate}
\end{defi}

\begin{eje}[Base canónica de $\C^n$]
  La base canónica (o estándar) de $\C^n$ es:
  $$\mathbf{e}_1 = \begin{pmatrix} 1 \\ 0 \\ \vdots \\ 0 \end{pmatrix}, \mathbf{e}_2 = \begin{pmatrix} 0 \\ 1 \\ \vdots \\ 0 \end{pmatrix}, \ldots, \mathbf{e}_n = \begin{pmatrix} 0 \\ 0 \\ \vdots \\ 1 \end{pmatrix}$$

  Cualquier vector $\mathbf{v} = \begin{pmatrix} v_1 \\ v_2 \\ \vdots \\ v_n \end{pmatrix} \in \C^n$ se puede escribir como:
  $$\mathbf{v} = v_1 \mathbf{e}_1 + v_2 \mathbf{e}_2 + \cdots + v_n \mathbf{e}_n$$
\end{eje}

\begin{theo}[Existencia y unicidad de representación]
  Si $\mathcal{B} = \{\mathbf{v}_1, \mathbf{v}_2, \ldots, \mathbf{v}_n\}$ es una base de un espacio vectorial $V$, entonces todo vector $\mathbf{v} \in V$ se puede escribir de manera única como:
  $$\mathbf{v} = \alpha_1 \mathbf{v}_1 + \alpha_2 \mathbf{v}_2 + \cdots + \alpha_n \mathbf{v}_n$$
  Los escalares $\alpha_1, \alpha_2, \ldots, \alpha_n$ se llaman las coordenadas de $\mathbf{v}$ respecto a la base $\mathcal{B}$.
\end{theo}

\begin{defi}[Dimensión]
  La dimensión de un espacio vectorial $V$ es el número de vectores en cualquiera de sus bases. Se denota $\dim(V)$.

  Si $V$ tiene una base finita, se dice que $V$ es finito-dimensional. En caso contrario, es infinito-dimensional.
\end{defi}

\begin{theo}[Propiedades fundamentales de la dimensión]
  Sea $V$ un espacio vectorial complejo de dimensión finita. Entonces:
  \begin{enumerate}
    \item Todas las bases de $V$ tienen el mismo número de elementos
    \item Si $\dim(V) = n$, entonces cualquier conjunto linealmente independiente de $n$ vectores es una base
    \item Si $\dim(V) = n$, entonces cualquier conjunto generador de $n$ vectores es una base
    \item Si $W$ es un subespacio de $V$, entonces $\dim(W) \leq \dim(V)$
  \end{enumerate}
\end{theo}

\begin{eje}[Encontrando una base]
  Encontrar una base para el subespacio $W$ de $\C^4$ definido por:
  $$W = \left\{\begin{pmatrix} x \\ y \\ z \\ w \end{pmatrix} \in \C^4 : x + y - z = 0, \quad 2x - y + w = 0\right\}$$

  \textbf{Solución:} De las ecuaciones del sistema:
  \begin{align}
    x + y - z  & = 0 \Rightarrow z = x + y  \\
    2x - y + w & = 0 \Rightarrow w = y - 2x
  \end{align}

  Por tanto:
  $$W = \left\{\begin{pmatrix} x \\ y \\ x+y \\ y-2x \end{pmatrix} : x, y \in \C\right\} = \left\{x\begin{pmatrix} 1 \\ 0 \\ 1 \\ -2 \end{pmatrix} + y\begin{pmatrix} 0 \\ 1 \\ 1 \\ 1 \end{pmatrix} : x, y \in \C\right\}$$

  Una base para $W$ es $\left\{\begin{pmatrix} 1 \\ 0 \\ 1 \\ -2 \end{pmatrix}, \begin{pmatrix} 0 \\ 1 \\ 1 \\ 1 \end{pmatrix}\right\}$ y $\dim(W) = 2$.
\end{eje}

\subsection{Primer postulado de la mecánica cuántica}

Ahora que hemos establecido la teoría de espacios vectoriales complejos, podemos formular matemáticamente el primer postulado fundamental de la mecánica cuántica.

\begin{resaltado}
  \textbf{Postulado I: Estados cuánticos.}

  El estado de un sistema cuántico se describe completamente mediante un vector unitario en un espacio vectorial complejo.
\end{resaltado}

\begin{eje}[Estado de un cúbit]
  Un cúbit (sistema cuántico de dos niveles) tiene espacio de estados $\mathcal{H} = \mathbb{C}^2$ y se representa como:
  $$\ket{\psi} = \alpha\ket{0} + \beta\ket{1}$$
  donde $\{\ket{0}, \ket{1}\}$ es la base computacional de $\mathbb{C}^2$ y $\alpha, \beta \in \mathbb{C}$ satisfacen:
  $$|\alpha|^2 + |\beta|^2 = 1 \quad \text{(condición de normalización)}$$

  En términos de la base canónica:
  $$\ket{0} = \begin{pmatrix} 1 \\ 0 \end{pmatrix}, \quad \ket{1} = \begin{pmatrix} 0 \\ 1 \end{pmatrix}$$

  Por tanto:
  $$\ket{\psi} = \begin{pmatrix} \alpha \\ \beta \end{pmatrix} \in \mathbb{C}^2$$
\end{eje}

\begin{defi}[Equivalencia por fase global]
  Dos vectores estado que difieren por una fase global son físicamente equivalentes:
  $$\ket{\psi} \sim e^{i\phi}\ket{\psi} \quad \forall \phi \in \mathbb{R}$$

  Esto significa que solo las fases \textbf{relativas} entre componentes tienen significado físico.
\end{defi}

\begin{info}
  En realidad, los estados cuánticos se describen mediante \textbf{vectores de rayos}, que son clases de equivalencia de vectores unitarios bajo la relación de fase global. Esto refleja que las mediciones físicas dependen solo de las diferencias de fase entre componentes del estado. En consecuencia, el espacio de estados de un cúbit es isomorfo al conjunto cociente $\H\cong \mathbb{C}^2 / \sim$, aunque en la práctica se obvia esta distinción y se trabaja directamente con vectores unitarios.
\end{info}

\begin{eje}[Parametrización general de un cúbit]
  Eliminando la fase global, todo cúbit puede escribirse como:
  $$\ket{\psi} = \cos\frac{\theta}{2}\ket{0} + e^{i\varphi}\sin\frac{\theta}{2}\ket{1}$$
  donde $\theta \in [0,\pi]$ y $\varphi \in [0,2\pi]$ son parámetros reales.

  Esta parametrización:
  \begin{itemize}
    \item Elimina la fase global irrelevante.
    \item Usa solo 2 parámetros reales para describir el estado completo.
    \item Corresponde a puntos en la superficie de la esfera de Bloch.
  \end{itemize}
\end{eje}

\begin{eje}[Estados cuánticos importantes]
  \begin{enumerate}
    \item \textbf{Estados de la base computacional:}
          $$\ket{0} = \begin{pmatrix} 1 \\ 0 \end{pmatrix}, \quad \ket{1} = \begin{pmatrix} 0 \\ 1 \end{pmatrix}$$

    \item \textbf{Estados de superposición:}
          $$\ket{+} = \frac{1}{\sqrt{2}}(\ket{0} + \ket{1}) = \frac{1}{\sqrt{2}}\begin{pmatrix} 1 \\ 1 \end{pmatrix}$$
          $$\ket{-} = \frac{1}{\sqrt{2}}(\ket{0} - \ket{1}) = \frac{1}{\sqrt{2}}\begin{pmatrix} 1 \\ -1 \end{pmatrix}$$

    \item \textbf{Estados con fase compleja:}
          $$\ket{i+} = \frac{1}{\sqrt{2}}(\ket{0} + i\ket{1}) = \frac{1}{\sqrt{2}}\begin{pmatrix} 1 \\ i \end{pmatrix}$$
  \end{enumerate}
\end{eje}

% \unirsection{A fondo}

Los siguientes libros pueden servir de material de apoyo y para profundizar más en los contenidos de este tema.

\textbf{O’Neill, P. V. Matemáticas avanzadas para Ingeniería. Cengage Learning, 2012}

Se recomienda la lectura del capítulo 10, explica los conceptos de forma clara y representa diferentes ejemplos del tema.


\textbf{Lang, S. Complex Analysis. Springer, 2003}

Se recomienda la lectura del capítulo 1.

%\unirsection{Problemas}

\begin{questions}


  \question Determinar si los siguientes conjuntos son subespacios de $\C^3$:
  \begin{parts}
    \part $W_1 = \left\{\begin{pmatrix} x \\ y \\ z \end{pmatrix} : x + 2y - z = 1\right\}$
    \part $W_2 = \left\{\begin{pmatrix} x \\ y \\ z \end{pmatrix} : x = \conj{z}\right\}$
    \part $W_3 = \left\{\begin{pmatrix} x \\ y \\ z \end{pmatrix} : |x|^2 + |y|^2 + |z|^2 = 1\right\}$
  \end{parts}


  \question
  Encontrar una base y la dimensión del subespacio de $\C^4$ generado por:
  $$\left\{\begin{pmatrix} 1 \\ i \\ 0 \\ 1 \end{pmatrix}, \begin{pmatrix} i \\ 1 \\ 1 \\ 0 \end{pmatrix}, \begin{pmatrix} 1+i \\ 1+i \\ 1 \\ 1 \end{pmatrix}\right\}$$


  \question
  Verificar que el conjunto $\left\{\begin{pmatrix} 1 \\ 1 \\ 0 \end{pmatrix}, \begin{pmatrix} 1 \\ i \\ 0 \end{pmatrix}, \begin{pmatrix} 0 \\ 0 \\ 1 \end{pmatrix}\right\}$ es linealmente independiente en $\C^3$ y extenderlo a una base de $\C^3$.


  \question
  Calcular el producto escalar hermítico y las normas de:
  $\mathbf{u} = \begin{pmatrix} 1+i \\ 2 \\ 3-i \end{pmatrix}, \quad \mathbf{v} = \begin{pmatrix} 2-i \\ 1+i \\ 1 \end{pmatrix}$


  \question
  Demostrar que si $\mathbf{v}_1, \mathbf{v}_2, \ldots, \mathbf{v}_k$ son linealmente independientes en un espacio vectorial $V$, entonces $\mathbf{v}_1, \mathbf{v}_1 + \mathbf{v}_2, \mathbf{v}_1 + \mathbf{v}_2 + \mathbf{v}_3, \ldots, \mathbf{v}_1 + \mathbf{v}_2 + \cdots + \mathbf{v}_k$ también son linealmente independientes.

  \question
  Sea $V = \mathcal{P}_2(\C)$ el espacio de polinomios complejos de grado menor o igual que 2.
  \begin{parts}
    \part Demostrar que $\{1, z, z^2\}$ es una base de $V$
    \part Encontrar las coordenadas del polinomio $p(z) = (1+i) + 2iz - z^2$ en esta base
    \part Proponer otra base para $V$ y expresar $p(z)$ en ella
  \end{parts}


  \question
  En el espacio $\C^{2 \times 2}$ de matrices complejas $2 \times 2$:
  \begin{parts}
    \part Verificar que las matrices de Pauli junto con la identidad forman una base:
    $\mathcal{B} = \left\{I = \begin{pmatrix} 1 & 0 \\ 0 & 1 \end{pmatrix}, \sigma_x = \begin{pmatrix} 0 & 1 \\ 1 & 0 \end{pmatrix}, \sigma_y = \begin{pmatrix} 0 & -i \\ i & 0 \end{pmatrix}, \sigma_z = \begin{pmatrix} 1 & 0 \\ 0 & -1 \end{pmatrix}\right\}$
    \part Expresar la matriz $A = \begin{pmatrix} 2+i & 1-i \\ 1+i & 2-i \end{pmatrix}$ como combinación lineal de esta base
  \end{parts}


  \question
  Sean $V$ y $W$ espacios vectoriales complejos de dimensiones finitas $m$ y $n$ respectivamente.
  \begin{parts}
    \part Demostrar que $\dim(V \times W) = \dim(V) + \dim(W)$
    \part Si $f: V \to W$ es una transformación lineal, demostrar que $\dim(V) = \dim(\Ker(f)) + \dim(\Ima(f))$
  \end{parts}

  \question Verificar que los siguientes vectores representan estados cuánticos válidos:
  \begin{parts}
    \part $\ket{\psi_1} = \frac{3}{5}\ket{0} + \frac{4i}{5}\ket{1}$
    \part $\ket{\psi_2} = \frac{1}{\sqrt{3}}\ket{0} + \frac{\sqrt{2}}{\sqrt{3}}\ket{1}$
    \part $\ket{\psi_3} = \frac{1+i}{2}\ket{0} + \frac{1-i}{2}\ket{1}$
  \end{parts}

  \question Para el estado cuántico $\ket{\psi} = \cos\frac{\pi}{6}\ket{0} + e^{i\pi/3}\sin\frac{\pi}{6}\ket{1}$:
  \begin{parts}
    \part Escribir el estado en forma matricial
    \part Calcular las probabilidades de medir $\ket{0}$ y $\ket{1}$
    \part Determinar los valores de $\theta$ y $\varphi$ en la parametrización estándar
  \end{parts}
\end{questions} 
%\tema{3}{Operadores lineales y representación matricial}
%\portada

\begin{esquemaExplorador}
  \temaEsquema{Transformaciones lineales}{
    \conceptoEsquema{Definición y propiedades}
    \conceptoEsquema{Núcleo e imagen}
    \conceptoEsquema{Inyectividad y sobreyectividad}
  }
  \temaEsquema{Operadores lineales}{
    \conceptoEsquema{Operadores como endomorfismos}
    \conceptoEsquema{Composición de operadores}
    \conceptoEsquema{Operador inverso}
  }
  \temaEsquema{Representación matricial}{
    \conceptoEsquema{Matriz asociada a una transformación}
    \conceptoEsquema{Cambio de base}
    \conceptoEsquema{Isomorfismo con matrices}
  }
  \temaEsquema{Matrices complejas}{
    \conceptoEsquema{Operaciones matriciales}
    \conceptoEsquema{Matrices especiales}
  }
\end{esquemaExplorador}

\unirsection{Ideas clave}

\subsection{Introducción y objetivos}

Los operadores lineales constituyen el puente conceptual entre la estructura algebraica abstracta de los espacios vectoriales y su representación computacional mediante matrices. En computación cuántica, este concepto es fundamental ya que todas las operaciones que pueden realizarse sobre sistemas cuánticos se describen mediante operadores lineales unitarios.

La importancia de los operadores lineales en computación cuántica se refleja en múltiples aspectos:

\begin{itemize}
  \item Las \textbf{puertas cuánticas} son operadores unitarios que actúan sobre qubits.
  \item La \textbf{evolución temporal} de sistemas cuánticos se describe mediante operadores unitarios.
  \item Los \textbf{observables cuánticos} son operadores hermitianos.
  \item Los \textbf{algoritmos cuánticos} se construyen como secuencias de operadores lineales.
\end{itemize}

En este tema desarrollaremos la teoría de transformaciones lineales entre espacios vectoriales complejos, con especial énfasis en los operadores (transformaciones de un espacio en sí mismo) y su representación matricial. Esta conexión entre conceptos abstractos y representaciones concretas es esencial para la implementación práctica de algoritmos cuánticos.

\subsection{Transformaciones lineales}

\begin{defi}[Transformación lineal]
  Sean $V$ y $W$ espacios vectoriales complejos. Una función $T: V \to W$ es una transformación lineal si para todos $\mathbf{u}, \mathbf{v} \in V$ y $\alpha, \beta \in \C$:
  $$T(\alpha \mathbf{u} + \beta \mathbf{v}) = \alpha T(\mathbf{u}) + \beta T(\mathbf{v})\,.$$

  Equivalentemente, $T$ es lineal si:
  \begin{enumerate}
    \item $T(\mathbf{u} + \mathbf{v}) = T(\mathbf{u}) + T(\mathbf{v})$ (preserva la suma).
    \item $T(\alpha \mathbf{v}) = \alpha T(\mathbf{v})$ (preserva el producto por escalar).
  \end{enumerate}
\end{defi}

\begin{eje}[Transformaciones lineales básicas]
  \begin{enumerate}
    \item \textbf{Transformación identidad:} $I: V \to V$ definida por $I(\mathbf{v}) = \mathbf{v}$.

    \item \textbf{Transformación cero:} $O: V \to W$ definida por $O(\mathbf{v}) = \mathbf{0}$.

    \item \textbf{Escalamiento:} $S_\alpha: \C^n \to \C^n$ definida por $S_\alpha(\mathbf{v}) = \alpha \mathbf{v}$.

    \item \textbf{Proyección:} $P: \C^3 \to \C^2$ definida por $P\begin{pmatrix} x \\ y \\ z \end{pmatrix} = \begin{pmatrix} x \\ y \end{pmatrix}$.

    \item \textbf{Rotación en $\C^2$:} $R_\theta: \C^2 \to \C^2$ definida por
          $$R_\theta\begin{pmatrix} x \\ y \end{pmatrix} = \begin{pmatrix} x\cos\theta - y\sin\theta \\ x\sin\theta + y\cos\theta \end{pmatrix}\,.$$
  \end{enumerate}
\end{eje}

\begin{prop}
  \label{prop:propiedadesTL}
  Si $T: V \to W$ es una transformación lineal, entonces:
  \begin{enumerate}
    \item $T(\mathbf{0}) = \mathbf{0}$.
    \item $T(-\mathbf{v}) = -T(\mathbf{v})$ para todo $\mathbf{v} \in V$.
    \item $T$ está completamente determinada por su acción sobre cualquier base de $V$. Es decir, si existe $\mathcal{B} = \{\mathbf{v}_1, \ldots, \mathbf{v}_n\}$ una base de $V$ y $U:V\to W$ otra transformación lineal tal que $U(\mathbf{v}_i) = T(\mathbf{v}_i)$ para todo $i$, entonces $U = T$.
  \end{enumerate}
\end{prop}

\begin{defi}[Núcleo]
  Sea $T: V \to W$ una transformación lineal. Llamamos \textbf{núcleo (o kernel)} de $T$ a $\Ker(T) = \{\mathbf{v} \in V : T(\mathbf{v}) = \mathbf{0}\}$
\end{defi}

\begin{prop}
  Sea $T: V \to W$ una transformación lineal. Entonces:
  \begin{enumerate}
    \item $\Ker(T)$ es un subespacio de $V$.
    \item $\Ima(T)$ es un subespacio de $W$.
    \item $T$ es inyectiva si y solo si $\Ker(T) = \{\mathbf{0}\}$.
  \end{enumerate}
\end{prop}

\begin{eje}[Cálculo de núcleo e imagen]
  Considerar la transformación $T: \C^3 \to \C^2$ definida por:
  $$T\begin{pmatrix} x \\ y \\ z \end{pmatrix} = \begin{pmatrix} x + iy \\ 2x - z \end{pmatrix}$$

  \textbf{Núcleo:} Resolver $T(\mathbf{v}) = \mathbf{0}$:
  \begin{align*}
    x + iy & = 0 \\
    2x - z & = 0
  \end{align*}

  De la primera ecuación: $x = -iy$. De la segunda: $z = 2x = -2iy$. Por tanto:
  $$\Ker(T) = \left\{\begin{pmatrix} -iy \\ y \\ -2iy \end{pmatrix} : y \in \C\right\} = \text{gen}\left\{\begin{pmatrix} -i \\ 1 \\ -2i \end{pmatrix}\right\}$$

  \textbf{Imagen:} Como $T$ es lineal, $\Ima(T) = \text{gen}\{T(\mathbf{e}_1), T(\mathbf{e}_2), T(\mathbf{e}_3)\}$:
  \begin{align*}
    T(\mathbf{e}_1) = T\begin{pmatrix} 1 \\ 0 \\ 0 \end{pmatrix} = \begin{pmatrix} 1 \\ 2 \end{pmatrix}, \quad
    T(\mathbf{e}_2) = T\begin{pmatrix} 0 \\ 1 \\ 0 \end{pmatrix} = \begin{pmatrix} i \\ 0 \end{pmatrix}, \quad
    T(\mathbf{e}_3) = T\begin{pmatrix} 0 \\ 0 \\ 1 \end{pmatrix} = \begin{pmatrix} 0 \\ -1 \end{pmatrix}
  \end{align*}

  Como los vectores $\mathbf{e}_2$ y $\mathbf{e}_3$ son linealmente independientes en $\C^2$, $\Ima(T) = \C^2$.
\end{eje}

\begin{theo}
  \label{th:descomposicion}
  Sea $T: V \to W$ una transformación lineal entre espacios vectoriales de dimensión finita, entonces:
  \begin{itemize}
    \item Existe $V^\prime \subseteq V$ tal que $V = \Ker(T) \oplus V^\prime$.
    \item Existe $W^\prime \subseteq W$ tal que $W = \Ima(T) \oplus W^\prime$.
    \item $V/\Ker(T) \cong \Ima(T)$.
  \end{itemize}
\end{theo}

Como consecuencia directa del Teorema~\ref{th:descomposicion} obtenemos el siguiente resultado fundamental.

\begin{theo}
  Sea $T: V \to W$ una transformación lineal entre espacios vectoriales de dimensión finita. Entonces:
  $$\dim(V) = \dim(\Ker(T)) + \dim(\Ima(T))$$
\end{theo}

\subsection{Operadores lineales}

\begin{defi}[Operador lineal]
  Un operador lineal en un espacio vectorial $V$ es una transformación lineal $T: V \to V$. El conjunto de todos los operadores lineales en $V$ se denota $\mathcal{L}(V)$.
\end{defi}

Los operadores lineales tienen propiedades especiales debido a que el espacio de salida coincide con el de entrada, lo que permite usar conceptos como la composición o la inversa.

\begin{prop}
  La composición de operadores lineales satisface:
  \begin{enumerate}
    \item \textbf{Asociatividad:} $(R \comp S) \comp T = R \comp (S \comp T)$.
    \item \textbf{Elemento neutro:} $I \comp T = T \comp I = T$.
    \item \textbf{Distributividad:} $R \comp (S + T) = R \comp S + R \comp T$.
  \end{enumerate}
\end{prop}

\begin{theo}[Caracterización de la invertibilidad]
  Un operador $T \in \mathcal{L}(V)$ es invertible si y solo si es biyectivo. Equivalentemente:
  \begin{enumerate}
    \item $T$ es inyectivo ($\Ker(T) = \{\mathbf{0}\}$).
    \item $T$ es sobreyectivo ($\Ima(T) = V$).
  \end{enumerate}
\end{theo}

Aplicando el Teorema~\ref{th:descomposicion} a operadores lineales obtenemos la siguiente consecuencia.
\begin{theo}
  Sea $T \in \mathcal{L}(V)$ un operador lineal en un espacio vectorial de dimensión finita. Entonces:
  $$V = \Ker(T) \oplus \Ima(T)$$
\end{theo}

\subsection{Representación matricial de transformaciones lineales}

La representación matricial establece una correspondencia biyectiva entre transformaciones lineales y matrices, permitiendo cálculos computacionales eficientes.

\begin{defi}[Matriz de una transformación lineal]
  Sean $V$ y $W$ espacios vectoriales de dimensiones finitas con bases $\mathcal{B}_V = \{\mathbf{v}_1, \ldots, \mathbf{v}_n\}$ y $\mathcal{B}_W = \{\mathbf{w}_1, \ldots, \mathbf{w}_m\}$ respectivamente. Sea $T: V \to W$ una transformación lineal.

  Para cada $j = 1, \ldots, n$, expresamos $T(\mathbf{v}_j)$ en la base $\mathcal{B}_W$:
  $$T(\mathbf{v}_j) = \sum_{i=1}^m a_{ij} \mathbf{w}_i$$

  La matriz de $T$ respecto a las bases $\mathcal{B}_V$ y $\mathcal{B}_W$ es:
  $$[T]_{\mathcal{B}_V}^{\mathcal{B}_W} = \begin{pmatrix} a_{11} & a_{12} & \cdots & a_{1n} \\ a_{21} & a_{22} & \cdots & a_{2n} \\ \vdots & \vdots & \ddots & \vdots \\ a_{m1} & a_{m2} & \cdots & a_{mn} \end{pmatrix}$$
\end{defi}

En el caso de considerarse $\mathcal{B}_V$ y $\mathcal{B}_W$ las bases canónicas, la matriz de la transformación se denota simplemente como $[T]$.

\begin{nota}
  La $j$-ésima columna de $[T]_{\mathcal{B}_V}^{\mathcal{B}_W}$ contiene las coordenadas de $T(\mathbf{v}_j)$ respecto a la base $\mathcal{B}_W$.
\end{nota}

\begin{eje}[Matriz de una transformación]
  Sea $T: \C^2 \to \C^3$ definida por $T\begin{pmatrix} x \\ y \end{pmatrix} = \begin{pmatrix} x + iy \\ 2x \\ x - y \end{pmatrix}$.

  Usando las bases canónicas:
  \begin{align*}
    T(\mathbf{e}_1) = T\begin{pmatrix} 1 \\ 0 \end{pmatrix} = \begin{pmatrix} 1 \\ 2 \\ 1 \end{pmatrix}, \quad
    T(\mathbf{e}_2) = T\begin{pmatrix} 0 \\ 1 \end{pmatrix} = \begin{pmatrix} i \\ 0 \\ -1 \end{pmatrix}
  \end{align*}

  Por tanto:
  $$[T] = \begin{pmatrix} 1 & i \\ 2 & 0 \\ 1 & -1 \end{pmatrix}$$
\end{eje}

\begin{theo}[Cambio de base]
  \label{th:cambio_base}
  Sean $T: V \to W$ una transformación lineal entre espacios vectoriales de la misma dimensión finita, $\mathcal{B}_1$ y $\mathcal{B}_2$ dos bases de $V$ y $\mathcal{B}_3$ y $\mathcal{B}_4$ dos bases de $W$. Entonces existe una matriz $P$ invertible tal que las matrices $[T]_{\mathcal{B}_1}^{\mathcal{B}_3}$ y $[T]_{\mathcal{B}_2}^{\mathcal{B}_4}$ están relacionadas por:
  $$[T]_{\mathcal{B}_2}^{\mathcal{B}_4} = P^{-1}[T]_{\mathcal{B}_1}^{\mathcal{B}_3}P$$
  Llamamos a $P$ la matriz de \textbf{cambio de base}.
\end{theo}

El teorema de cambio de base~\ref{th:cambio_base} es muy importante, pues nos permite verificar muchas propiedades sobre las transformaciones lineales, con solo comprobarlo con la matriz asociada a las bases canónicas.

\begin{theo}[Correspondencia entre transformaciones y matrices]
  La representación matricial establece un isomorfismo de espacios vectoriales:
  $$\mathcal{L}(V, W) \cong \C^{m \times n}$$
  donde $\dim(V) = n$ y $\dim(W) = m$.

  Además, si $S: U \to V$ y $T: V \to W$ son transformaciones lineales, entonces:
  $$[T \comp S] = [T][S]$$
\end{theo}

\subsection{Matrices complejas como operadores}

Cada matriz compleja $A \in \C^{m \times n}$ define naturalmente una transformación lineal $T_A: \C^n \to \C^m$ mediante $T_A(\mathbf{v}) = A\mathbf{v}$. Esta correspondencia permite estudiar propiedades algebraicas de las matrices a través de la teoría de operadores.

Junto con las operaciones matriciales estándar, se definen conceptos clave como la transpuesta conjugada, las matrices hermitianas y unitarias, que son fundamentales en computación cuántica.

\begin{defi}
  Sea $A\in \C^{n}$ una matriz cuadrada, llamamos \textbf{matriz adjunta} a la matriz $A^\dagger$ definida por:
  \[
    (A^\dagger)_{ij} = \conj{A_{ji}}\,.
  \]
\end{defi}

\begin{defi}
  Sea $A \in \C^{n}$ una matriz cuadrada, decimos que es:
  \begin{itemize}
    \item
          \textbf{Hermítica} si $A^\dagger = A$.
    \item
          \textbf{Unitaria} si $A^\dagger A = AA^\dagger = I$.
    \item
          \textbf{Normal} si $A^\dagger A = AA^\dagger$.
  \end{itemize}

\end{defi}

Estudiaremos con más detalle estas matrices en temas posteriores, ya que son fundamentales en computación cuántica.

\begin{eje}[Matrices de Pauli]
  Las matrices de Pauli son matrices hermitianas fundamentales en computación cuántica:
  \begin{align}
    \sigma_x = \begin{pmatrix} 0 & 1 \\ 1 & 0 \end{pmatrix}, \quad
    \sigma_y = \begin{pmatrix} 0 & -i \\ i & 0 \end{pmatrix}, \quad
    \sigma_z = \begin{pmatrix} 1 & 0 \\ 0 & -1 \end{pmatrix}
  \end{align}

  Estas matrices satisfacen:
  \begin{itemize}
    \item $\sigma_i^2 = I$ para $i = x, y, z$.
    \item $\sigma_x\sigma_y = i\sigma_z$, $\sigma_y\sigma_z = i\sigma_x$, $\sigma_z\sigma_x = i\sigma_y$.
    \item $\{\sigma_x, \sigma_y\} = \sigma_x\sigma_y + \sigma_y\sigma_x = 0$ (anticonmutan).
    \item Junto con la identidad $I$, forman una base de $\C^{2 \times 2}$.
  \end{itemize}
\end{eje}

\begin{eje}[Matriz de Hadamard]
  La matriz de Hadamard es una matriz unitaria fundamental:
  $H = \frac{1}{\sqrt{2}}\begin{pmatrix} 1 & 1 \\ 1 & -1 \end{pmatrix}$.

  Se verifica que $H^2 = I$ y transforma la base computacional en:
  \begin{align}
    H\ket{0} = H\begin{pmatrix} 1 \\ 0 \end{pmatrix} = \frac{1}{\sqrt{2}}\begin{pmatrix} 1 \\ 1 \end{pmatrix} = \frac{\ket{0} + \ket{1}}{\sqrt{2}} = \ket{+} \\
    H\ket{1} = H\begin{pmatrix} 0 \\ 1 \end{pmatrix} = \frac{1}{\sqrt{2}}\begin{pmatrix} 1 \\ -1 \end{pmatrix} = \frac{\ket{0} - \ket{1}}{\sqrt{2}} = \ket{-}
  \end{align}
\end{eje}

\subsection{Valores y vectores propios}

\begin{defi}[Valor y vector propio]
  Sea $A \in \C^{n \times n}$. Un escalar $\lambda \in \C$ es un \textbf{valor propio} de $A$ si existe un vector no nulo $\mathbf{v} \in \C^n$ tal que:
  $A\mathbf{v} = \lambda \mathbf{v}$.

  El vector $\mathbf{v}$ se llama \textbf{vector propio} asociado al valor propio $\lambda$.
\end{defi}

Para obtener los valores propios de una matriz $A$, se resuelve el sistema $(A - \lambda I)\mathbf{v} = \mathbf{0}$. Este sistema tiene soluciones no triviales si y solo si $\det(A - \lambda I) = 0$.

\begin{defi}[Polinomio característico]
  El \textbf{polinomio característico} de $A \in \C^{n \times n}$ es:
  $p_A(\lambda) = \det(A - \lambda I)$.
\end{defi}

\begin{prop}
  Los valores propios de $A$ son las raíces de $p_A(\lambda)$.
\end{prop}

\begin{eje}
  Consideremos la matriz de Pauli $\sigma_z = \begin{pmatrix} 1 & 0 \\ 0 & -1 \end{pmatrix}$, cuyo polinomio característico es:

  \[
    p_{\sigma_z}(\lambda) = \det\begin{pmatrix} 1-\lambda & 0 \\ 0 & -1-\lambda \end{pmatrix} = (1-\lambda)(-1-\lambda) = \lambda^2 - 1\,.
  \]

  Los valores propios son $\lambda_1 = 1$ y $\lambda_2 = -1$. Ahora calculemos los vectores propios.

  Para $\lambda_1 = 1$, $(\sigma_z - I)\mathbf{v} = \begin{pmatrix} 0 & 0 \\ 0 & -2 \end{pmatrix}\begin{pmatrix} v_1 \\ v_2 \end{pmatrix} = \mathbf{0} \Rightarrow v_2 = 0$. Por tanto, $\mathbf{v}_1 = \begin{pmatrix} v_1 \\ 0 \end{pmatrix}$ con $v_1 \neq 0$.

  Para $\lambda_2 = -1$, $(\sigma_z + I)\mathbf{v} = \begin{pmatrix} 2 & 0 \\ 0 & 0 \end{pmatrix}\begin{pmatrix} v_1 \\ v_2 \end{pmatrix} = \mathbf{0} \Rightarrow v_1 = 0$. Por tanto, $\mathbf{v}_2 = \begin{pmatrix} 0 \\ v_2 \end{pmatrix}$ con $v_2 \neq 0$.

  Podemos obtener una base ortonormal de $\C^2$ con los vectores propios de $\sigma_z$ obteniendo:
  \begin{itemize}
    \item Para $\lambda_1 = 1$: $\mathbf{v}_1 = \begin{pmatrix} 1 \\ 0 \end{pmatrix} = \ket{0}$
    \item Para $\lambda_2 = -1$: $\mathbf{v}_2 = \begin{pmatrix} 0 \\ 1 \end{pmatrix} = \ket{1}$
  \end{itemize}
\end{eje}

\begin{theo}[Diagonalización]
  Una matriz $A \in \C^{n \times n}$ es diagonalizable si y solo si tiene $n$ vectores propios linealmente independientes. En tal caso, existe una matriz invertible $P$ tal que:
  $P^{-1}AP = D$
  donde $D$ es diagonal con los valores propios de $A$ en la diagonal.
\end{theo}

\subsection{Espacios de operadores}

\begin{defi}[Espacio de operadores lineales]
  El conjunto $\mathcal{L}(V)$ de todos los operadores lineales en un espacio vectorial $V$ forma un espacio vectorial con las operaciones:
  \begin{itemize}
    \item $(S + T)(\mathbf{v}) = S(\mathbf{v}) + T(\mathbf{v})$
    \item $(\alpha T)(\mathbf{v}) = \alpha T(\mathbf{v})$
  \end{itemize}
\end{defi}

\begin{prop}
  Si $\dim(V) = n$, entonces $\dim(\mathcal{L}(V)) = n^2$.
\end{prop}

\begin{eje}[Base del espacio de operadores en $\C^2$]
  Una base para $\mathcal{L}(\C^2)$ está dada por las matrices:
  \[
    E_{11} = \begin{pmatrix} 1 & 0 \\ 0 & 0 \end{pmatrix}, E_{12} = \begin{pmatrix} 0 & 1 \\ 0 & 0 \end{pmatrix}, E_{21} = \begin{pmatrix} 0 & 0 \\ 1 & 0 \end{pmatrix}, E_{22} = \begin{pmatrix} 0 & 0 \\ 0 & 1 \end{pmatrix}.
  \]

  Cualquier operador $T \in \mathcal{L}(\C^2)$ se puede escribir como:
  $T = \sum_{i,j=1}^2 t_{ij} E_{ij}$.
\end{eje}

Alternativamente, se puede usar la llamada base de Pauli:
\[
  \{I, \sigma_x, \sigma_y, \sigma_z\}
\]
%\unirsection{Problemas}

\begin{questions}

  \question Verificar que las siguientes funciones son transformaciones lineales y calcular su núcleo e imagen:
  \begin{parts}
    \part $T: \C^3 \to \C^2$ definida por $T\begin{pmatrix} x \\ y \\ z \end{pmatrix} = \begin{pmatrix} x + iy - z \\ 2x + y \end{pmatrix}$
    \part $S: \C^2 \to \C^3$ definida por $S\begin{pmatrix} u \\ v \end{pmatrix} = \begin{pmatrix} u + iv \\ 2u \\ u - v \end{pmatrix}$
  \end{parts}

  \question Encontrar la matriz de la transformación lineal $T: \C^2 \to \C^2$ definida por $T\begin{pmatrix} x \\ y \end{pmatrix} = \begin{pmatrix} (1+i)x + y \\ ix - 2y \end{pmatrix}$ respecto a:
  \begin{parts}
    \part La base canónica
    \part La base $\mathcal{B} = \left\{\begin{pmatrix} 1 \\ i \end{pmatrix}, \begin{pmatrix} i \\ 1 \end{pmatrix}\right\}$
  \end{parts}

  \question Sean $A = \begin{pmatrix} 1 & i \\ 0 & 1 \end{pmatrix}$ y $B = \begin{pmatrix} 1 & 0 \\ i & 1 \end{pmatrix}$. Calcular:
  \begin{parts}
    \part $AB$ y $BA$
    \part $A^*$ y $B^*$
    \part $\det(A)$, $\det(B)$ y $\det(AB)$
    \part $\text{tr}(A)$, $\text{tr}(B)$ y $\text{tr}(AB)$
  \end{parts}

  \question Encontrar los valores propios y vectores propios de las siguientes matrices:
  \begin{parts}
    \part $A = \begin{pmatrix} 2 & i \\ -i & 2 \end{pmatrix}$
    \part $B = \begin{pmatrix} 1 & 1+i \\ 1-i & 2 \end{pmatrix}$
    \part La matriz de Hadamard $H = \frac{1}{\sqrt{2}}\begin{pmatrix} 1 & 1 \\ 1 & -1 \end{pmatrix}$
  \end{parts}

  \question Verificar que las siguientes matrices son unitarias y calcular sus inversas:
  \begin{parts}
    \part $U_1 = \begin{pmatrix} \frac{1}{\sqrt{2}} & \frac{i}{\sqrt{2}} \\ \frac{i}{\sqrt{2}} & \frac{1}{\sqrt{2}} \end{pmatrix}$
    \part $U_2 = \frac{1}{\sqrt{3}}\begin{pmatrix} 1 & 1+i \\ 1-i & -1 \end{pmatrix}$
  \end{parts}

  \question Demostrar que si $A$ y $B$ son matrices hermitianas, entonces:
  \begin{parts}
    \part $A + B$ es hermitiana
    \part $AB$ es hermitiana si y solo si $AB = BA$
    \part $iA$ es antihermitiana (i.e., $(iA)^* = -iA$)
  \end{parts}

  \question
  Considerar la puerta cuántica de fase controlada:
  $\text{CZ} = \begin{pmatrix} 1 & 0 & 0 & 0 \\ 0 & 1 & 0 & 0 \\ 0 & 0 & 1 & 0 \\ 0 & 0 & 0 & -1 \end{pmatrix}$
  \begin{parts}
    \part Verificar que CZ es unitaria
    \part Determinar cómo actúa sobre los estados de la base computacional
    \part Calcular los valores propios de CZ
  \end{parts}

  \question Expresar la matriz $M = \begin{pmatrix} 3 & 1+2i \\ 1-2i & -1 \end{pmatrix}$ como combinación lineal de las matrices de Pauli y la identidad.

  \question Sea $T: V \to V$ un operador lineal en un espacio vectorial de dimensión finita.
  \begin{parts}
    \part Demostrar que $T$ es invertible si y solo si $\det([T]) \neq 0$
    \part Si $T$ es invertible, demostrar que $[T^{-1}] = ([T])^{-1}$
    \part Demostrar que $\text{tr}(T)$ es independiente de la base elegida
  \end{parts}

\end{questions}
%\bloque{2}{Espacios cuánticos}
\tema{4}{Espacios de Hilbert}
\portada

\begin{esquemaExplorador}
  \temaEsquema{Producto interno complejo}{
    \conceptoEsquema{Definición}{}
    \conceptoEsquema{Propiedades fundamentales}{}
    \conceptoEsquema{Norma inducida}{}
    \conceptoEsquema{Distancia asociada a la norma}{}
  }
  \temaEsquema{Espacios de Hilbert}{
    \conceptoEsquema{Desigualdad de Cauchy-Schwarz}{$|\langle u, v \rangle|^2 \leq \langle u, u \rangle \langle v, v \rangle$}
    \conceptoEsquema{Ortogonalidad}{}
    \conceptoEsquema{Proyecciones ortogonales}{}
    \conceptoEsquema{Completitud}{}
    \conceptoEsquema{Bases ortonormales}{}
    \conceptoEsquema{Teorema de Parseval}{$\|v\|^2 = \sum_{k=1}^n |\langle v, e_k \rangle|^2$}
  }
  \temaEsquema{Operadores en espacios de Hilbert}{
    \conceptoEsquema{Operadores hermitianos}{$A^\dagger = A$}
    \conceptoEsquema{Operadores unitarios}{$U^\dagger U = UU^\dagger = I$}
    \conceptoEsquema{Teorema espectral}{$A = \sum_{i} \lambda_i e_i e_i^\dagger$}
  }
\end{esquemaExplorador}

\unirsection{Ideas clave}

\subsection{Introducción y objetivos}

Los espacios de Hilbert proporcionan el marco matemático fundamental para la mecánica cuántica y, por tanto, para la computación cuántica. Estos espacios combinan la estructura algebraica de los espacios vectoriales con una geometría rica derivada del producto interno, permitiendo definir conceptos como longitudes, ángulos y ortogonalidad que son esenciales para la interpretación física de los fenómenos cuánticos.

La importancia de los espacios de Hilbert en computación cuántica se manifiesta en múltiples aspectos fundamentales:

\begin{itemize}
  \item Los \textbf{estados cuánticos} se representan como vectores unitarios en espacios de Hilbert complejos.
  \item Las \textbf{probabilidades cuánticas} se calculan mediante productos internos: $P = |\langle\psi|\phi\rangle|^2$.
  \item Los \textbf{observables físicos} corresponden a operadores hermitianos en estos espacios.
  \item La \textbf{evolución unitaria} preserva el producto interno y, por tanto, las probabilidades.
  \item Los \textbf{algoritmos cuánticos} manipulan información mediante transformaciones que respetan la estructura de espacio de Hilbert.
\end{itemize}

En este tema desarrollaremos la teoría de espacios con producto interno, culminando en los espacios de Hilbert y su aplicación a sistemas cuánticos. Esta base teórica es esencial para comprender tanto los fundamentos conceptuales como las implementaciones prácticas de la computación cuántica.

Mientras no se diga lo contrario, durante todo el tema estaremos trabajando con espacios vectoriales complejos de dimensión finita.

\subsection{Producto interno}

\begin{defi}[Producto interno]
  Sea $V$ un espacio vectorial. Un \textbf{producto interno} en $V$ es una función $\langle \cdot, \cdot \rangle: V \times V \to \C$ que satisface para todo $u, v, w \in V$ y $\alpha, \beta \in \C$:

  \begin{enumerate}
    \item \textbf{Linealidad en el segundo argumento}
          \[
            \pint{w}{\alpha u + \beta v}= \alpha\pint{w}{u} + \beta\pint{w}{v}\,.
          \]

    \item \textbf{Antisimetría hermítica}
          \[
            \langle u, v \rangle = \conj{\langle v, u \rangle}\,.
          \]

    \item \textbf{Positividad}
          \[
            \langle v, v \rangle \geq 0\,,
          \]
          con igualdad si y solo si $v = 0$.
  \end{enumerate}
\end{defi}
\semisepara

\begin{eje}[Productos internos estándar]
  Algunos ejemplos de productos internos en espacios vectoriales complejos son:
  \begin{enumerate}
    \item En $\C^n$ $\langle u, v \rangle = \sum_{k=1}^n  \conj{u_k}v_k$.

    \item En $\C^{m \times n}$ $\langle A, B \rangle = \text{tr}(A B^\dagger) = \sum_{i,j} \conj{A_{ij}}B_{ij}$.

    \item En $L^2([a,b])$ $\langle f, g \rangle = \int_a^b \conj{f(x)}g(x) \, dx$.
  \end{enumerate}
\end{eje}
Veamos algunos resultados básicos sobre el producto interno que son importantes recordar.
\begin{prop}
  \label{prop:propiedadesProductoInterno}
  Sean $u,v,w \in V$ vectores y $\alpha, \beta \in \C$. Entonces:
  \begin{enumerate}
    \item $\pint{\alpha v + \beta w}{u} = \conj{\alpha}\langle v, u\rangle + \conj{\beta}\langle w, u\rangle$.
    \item $\langle u, 0 \rangle = 0$.
    \item $\abs{\langle u, v \rangle} = \abs{\langle v, u \rangle}$.
  \end{enumerate}
\end{prop}
\begin{proof}
  Se deja al lector como ejercicio (~\ref{ex:propiedadesProductoInterno}).
\end{proof}

Nuestro objetivo es dotar a los espacios vectoriales de una estructura geométrica que nos permita definir conceptos como ángulos, longitudes y ortogonalidad. Para ello, necesitamos de una poderosa herramienta, la distancia, que nos permitirá establecer la noción de proximidad e incluso dotar al espacio de una topología adecuada.

Para ello, partimos de la definición de norma inducida por un producto interno.
\begin{defi}[Norma inducida]
  El producto interno define la \textbf{norma inducida} en $V$ definida por
  \[
    \|v\| = \sqrt{\langle v, v \rangle}\,.
  \]
\end{defi}

\begin{prop}
  \label{prop:propiedadesNorma}
  Sean $u,v \in V$ vectores. La norma inducida por un producto interno satisface:
  \begin{enumerate}
    \item $\|v\| \geq 0$, con igualdad si y solo si $v = 0$.
    \item $\|\alpha v\| = |\alpha| \|v\|$ para todo $\alpha \in \C$.
    \item $\|u + v\| \leq \|u\| + \|v\|$ (desigualdad triangular).
  \end{enumerate}
\end{prop}
\begin{proof}
  Se deja al lector como ejercicio (~\ref{ex:propiedadesNorma}).
\end{proof}

\begin{theo}[Desigualdad de Cauchy-Schwarz]
  \label{th:cauchy_schwarz}
  Sea $V$ un espacio vectorial con producto interno, para todo $u, v \in V$ se cumple que
  \[
    |\langle u, v \rangle| \leq \|u\| \|v\|\,.
  \]
  La igualdad se da si y solo si $u$ y $v$ son linealmente dependientes.
\end{theo}

\begin{proof}
  Si $u=0$, la desigualdad se cumple trivialmente pues ambos lados son cero.

  Supongamos que $u \neq 0$. Consideremos el vector
  \[
    w = v - \frac{\langle u, v \rangle}{\|u\|^2} u\,.
  \]
  Por la propiedad de positividad del producto interno, sabemos que $\langle w, w \rangle \geq 0$. Desarrollando el producto interno
  \begin{align*}
    0 \leq \langle w, w \rangle & = \left\langle v - \frac{\langle u, v \rangle}{\|u\|^2} u, v - \frac{\langle u, v \rangle}{\|u\|^2} u \right\rangle                                                                                                              \\
                                & = \langle v, v \rangle - \conj{\frac{\langle u, v \rangle}{\|u\|^2}} \langle v, u \rangle - \frac{\langle u, v \rangle}{\|u\|^2} \langle u, v \rangle + \left|\frac{\langle u, v \rangle}{\|u\|^2}\right|^2 \langle u, u \rangle \\
                                & = \|v\|^2 - \frac{|\langle v, u \rangle|^2}{\|u\|^2} - \frac{|\langle u, v \rangle|^2}{\|u\|^2} + \frac{|\langle u, v \rangle|^2}{\|u\|^2}                                                                                       \\
                                & = \|v\|^2 - \frac{|\langle u, v \rangle|^2}{\|u\|^2}\,.
  \end{align*}
  Por lo tanto,
  \[
    \frac{|\langle u, v \rangle|^2}{\|u\|^2} \leq \|v\|^2 \implies |\langle u, v \rangle|^2 \leq \|u\|^2 \|v\|^2 \implies |\langle u, v \rangle| \leq \|u\| \|v\|\,.
  \]
  La igualdad se cumple si y solo si $\langle w, w \rangle = 0$, lo que implica que $w = 0$, es decir
  \[
    v = \frac{\langle u, v \rangle}{\|u\|^2} u\,,
  \]
  por lo que $u$ y $v$ son linealmente dependientes.
\end{proof}

Por último, la norma nos permite definir el concepto de distancia, que es esencial para realizar procesos comparativos y de recuperación de estados cuánticos. Estos y otros temas que relacionados con espacios métricos se necesitan en el campo de la información cuántica.

\begin{defi}[Distancia]
  Sea $V$ un espacio vectorial, definimos \textbf{distancia} sobre $V$ a una aplicación $d: V \times V \to \R$ que satisface para todos $u, v, w \in V$:
  \begin{enumerate}
    \item $d(u, v) \geq 0$, siendo cero si y solo si $u=v$.
    \item $d(u, v) = d(v, u)$.
    \item $d(u, v) \leq d(u, w) + d(w, v)$.
  \end{enumerate}

\end{defi}

Todos los espacios vectoriales que dispongan de una norma, pueden definir una función distancia.

\begin{prop}
  \label{prop:distanciaAsociadaNorma}
  Sea $V$ un espacio vectorial con una norma, la función definida por
  \[
    d(u, v) = \| u - v \|\,,
  \]
  es una distancia sobre $V$. Llamamos a esta distancia, \textbf{distancia asociada a la norma}.
\end{prop}

\begin{proof}
  Se deja al lector como ejercicio (~\ref{ex:distanciaAsociadaNorma}).
\end{proof}

\begin{eje}
  En $\C^n$ a partir del producto interno podemos construir la siguiente función de distancia
  \[
    d(u, v) = \|u-v\| = \sqrt{\pint{u-v}{u-v}} = \sqrt{\sum_{k=1}^n  |u_k - v_k|^2}\,.
  \]
\end{eje}

\subsection{Ortogonalidad y proyecciones}

\begin{defi}[Ortogonalidad]
  Dos vectores $u, v$ en un espacio con producto interno son \textbf{ortogonales} si $\langle u, v \rangle = 0$. Se denota $u \perp v$.

  Dos subespacios $W_1, W_2$ de un espacio con producto interno $V$ son \textbf{ortogonales} si $\langle u, v \rangle = 0$ para todo $u \in W_1$ y $v \in W_2$. Se denota $W_1 \perp W_2$.
\end{defi}

\begin{defi}[Conjunto ortonormal]
  Un conjunto $\{v_1, v_2, \ldots\}$ de vectores es:
  \begin{itemize}
    \item \textbf{Ortogonal} si $\langle v_i, v_j \rangle = 0$ para $i \neq j$.
    \item \textbf{Ortonormal} si es ortogonal y además $\|v_i\| = 1$ para todo $i$.
  \end{itemize}

  O expresado de forma más compacta, un conjunto es ortonormal si $\langle v_i, v_j \rangle = \delta_{ij}$.
\end{defi}

El producto interno, es una herramienta fundamental para conocer cuanto de un vector está contenido en la dirección de otro vector.
Si $V$ es un espacio vectorial con producto interno, y $\mathcal{B}=\{v_1, \ldots, v_n\}$ es una base ortonormal de $V$, todo vector $v$ expresado como combinación lineal de los elementos de la base
\[
  v = \sum_{i=1}^n a_i v_i\,,
\]
nos permite expresar el producto interno con un vector de la base $v_k$ como
\[
  \pint{v_k}{v} = \sum_{i=1}^n a_i \pint{v_k}{v_i} = a_k\,,
\]
y de esta igualdad deducimos que los coeficientes de la combinación lineal son simplemente los productos internos con los vectores de la base ortonormal
\[
  v = \sum_{k=1}^n \pint{v_k}{v} v_k\,.
\]

La igualdad anterior podemos interpretarla como la descomposición de $v$ en suma de vectores en la misma dirección que los vectores de la base ortonormal, y donde la magnitud de cada vector es el producto interno de $v$ con el vector de la base.

Es decir que la descomposición de $v$ en la base ortonormal es simplemente la proyección ortogonal de $v$ sobre el subespacio generado por la base ortonormal.

\begin{defi}[Proyección ortogonal]
  Sea $W$ un subespacio de un espacio con producto interno $V$, y sea $\{v_1, \ldots, v_k\}$ una base ortonormal de $W$. La \textbf{proyección ortogonal} de $v \in V$ sobre $W$ es
  $$\text{proj}_W(v) = \sum_{i=1}^k \pint{v_i}{v} v_i\,.$$
\end{defi}

\begin{prop}
  La proyección ortogonal sobre un subespacio vectorial $W$ de un espacio vectorial $V$ con producto interno es una aplicación lineal que cumple
  \[
    \text{proj}_W(\text{proj}_W(v)) = \text{proj}_W(v)\,,
  \]
  para todo $v\in V$.
\end{prop}
\begin{proof}
  Sea $v\in V$ y $\{e_1, \ldots, e_k\}$ una base ortonormal de $W$. Entonces
  \[
    \text{proj}_W(v) = \sum_{i=1}^k \langle v, e_i \rangle e_i\,.
  \]
  Por lo que
  \begin{align*}
    \text{proj}_W(\text{proj}_W(v)) & = \sum_{i=1}^k \langle \text{proj}_W(v), e_i \rangle e_i                                                                                        \\
                                    & = \sum_{i=1}^k \langle \sum_{j=1}^k \langle v, e_j \rangle e_j, e_i \rangle e_i                                                                 \\
                                    & = \sum_{i=1}^k \sum_{j=1}^k \langle v, e_j \rangle \langle e_j, e_i \rangle e_i = \sum_{i=1}^k \langle v, e_i \rangle e_i = \text{proj}_W(v)\,.
  \end{align*}
\end{proof}

\begin{theo}[Teorema de proyección]
  Sea $V$ un espacio con producto interno y $\mathcal{B}$ una base ortonormal. Para cualquier $\mathcal{B}^\prime\subset\mathcal{B}$ y $v \in V$
  se cumple que
  \[
    v - \text{proj}_{\mathcal{B}^\prime}(v) \perp \left<\mathcal{B}^\prime\right>\,.
  \]
\end{theo}
\begin{proof}
  Sea $v\in V$ y $\mathcal{B}=\{e_1, \ldots, e_k\}$. Entonces
  \begin{align*}
    \langle v, e_i\rangle - \langle\sum_{j=1}^k \langle v, e_j \rangle e_j, e_i\rangle
     & = \langle v, e_i\rangle - \langle\sum_{j=1}^k \langle v, e_j \rangle e_j, e_i\rangle  \\
     & = \langle v, e_i\rangle - \sum_{j=1}^k \langle v, e_j \rangle \langle e_j, e_i\rangle \\
     & = \langle v, e_i\rangle - \sum_{j=1}^k \langle v, e_j \rangle \delta_{ji}             \\
     & = \langle v, e_i\rangle - \langle v, e_i \rangle                                      \\
     & = 0\,.
  \end{align*}
\end{proof}

El teorema anterior nos permite definir el subespacio ortogonal a $W$ como
\[
  W^\perp = \{v \in V \mid v \perp W\} = \langle v - \text{proj}_W(v)\mid \forall v \in V \rangle\,,
\]
que es fácil comprobar que además está en suma directa con $W$
\[
  V=W\oplus W^\perp\,.
\]

Este proceso de obtener espacios ortogonales, es fundamental para la construcción de una base ortonormal de $V$ a partir de un conjunto linealmente independiente. Pero para poder garantizar su existencia, hemos necesitado partir de una base ortonormal del espacio vectorial.

La pregunta que surge es: ¿Existe un proceso que nos permita obtener una base ortonormal de un espacio vectorial a partir de un conjunto linealmente independiente?

\begin{theo}[Proceso de Gram-Schmidt]
  \label{th:gram_schmidt}
  Todo conjunto linealmente independiente $\{v_1, \ldots, v_n\}$ puede transformarse en un conjunto ortonormal $\{w_1, \ldots, w_n\}$ que genera el mismo subespacio.

  El proceso es:
  \begin{align*}
    u_1    & = v_1                                           \\
    w_1    & = \frac{u_1}{\|u_1\|}                           \\
    u_2    & = v_2 - \text{proj}_{w_1}(v_2)                  \\
    w_2    & = \frac{u_2}{\|u_2\|}                           \\
    \vdots &                                                 \\
    u_k    & = v_k - \sum_{j=1}^{k-1} \text{proj}_{w_j}(v_k) \\
    w_k    & = \frac{u_k}{\|u_k\|}
  \end{align*}
\end{theo}

El proceso de Gram-Schmidt nos permite construir para cualquier espacio vectorial una base ortonormal, que por su importancia en computación cuántica, es completamente imprescindible conocer.

\begin{eje}[Ortogonalización en $\C^3$]
  Ortogonalizar el conjunto $\left\{v_1 = (1, i, 0), v_2 = (0, 1, i), v_3 = (i, 0, 1)\right\}$.

  Primero tenemos que asegurar que los vectores son linealmente independientes.
  Si partimos de una combinación lineal igualada a cero tendremos el sistema de ecuaciones
  \[
    0 = \alpha (1, i, 0) + \beta (0, 1, i) + \gamma (i, 0, 1)\Rightarrow
    \begin{cases}
      \alpha + \gamma i = 0    \\
      \alpha i + \beta = 0 \,. \\
      \beta i + \gamma = 0
    \end{cases}
  \]
  Multiplicando la primera ecuación por $i$ y restando la segunda, obtenemos que $\beta=-\gamma$, que substituyendo en la tercera sale $\gamma=\beta=0$ y usando la primera ecuación $\alpha=0$.

  Ahora pasemos a desarrollar el método de ortogonalización de Gram--Schmidt:
  \begin{align*}
    \textbf{1. } u_1 & = v_1 = (1, i, 0)                                                                                                                                                                               \\
    \textbf{2. } w_1 & =  \frac{u_1}{\|u_1\|} = \frac{1}{\sqrt{2}} (1, i, 0)                                                                                                                                           \\
    \textbf{3. } u_2 & = v_2 - \text{proj}_{w_1}(v_2) = (0, 1, i) - \langle v_2, w_1 \rangle w_1 = (0, 1, i) - \frac{i}{\sqrt{2}}w_1                                                                                   \\
                     & = (0, 1, i) + \frac{i}{2}(1, i, 0) = \frac{1}{2}(i, 1, 2i)                                                                                                                                      \\
    \textbf{4. } w_2 & = \frac{u_2}{\|u_2\|}=\frac{1}{\sqrt{6}} (i, 1, 2i)                                                                                                                                             \\
    \textbf{5. } u_3 & = v_3 - \text{proj}_{w_1}(v_3) - \text{proj}_{w_2}(v_3) = (i, 0, 1) - \langle v_3, w_1 \rangle w_1 - \langle v_3, w_2 \rangle w_2                                                               \\
                     & = (i, 0, 1) - \frac{i}{\sqrt{2}}w_1 - \frac{1-2i}{\sqrt{6}}w_2                                                                  = (i, 0, 1) - \frac{1}{2} (1, i, 0) - \frac{1-2i}{6} (i, 1, 2i) \\
                     & = \frac{1}{3}(-1+i, 1+i, 1-i)                                                                                                                                                                   \\
    \textbf{6. } w_3 & = \frac{u_3}{\|u_3\|}=\frac{1}{\sqrt{6}}(-1+i, 1+i, 1-i)\,.
  \end{align*}
\end{eje}

\subsection{Espacios de Hilbert}

Le hmos dedicado mucho tiempo a estudiar los espacios vectoriales con producto interno, pero todavía no hemos nombrado el concepto que da título a este tema, y como en breve comprobarás, apenas le dedicaremos unas pocas líneas a su definición.

La sutileza del concepto de espacio de Hilbert, nos obliga a conocer con profundidad la estructura topológica de los espacios vectoriales, conocer la métrica inducida por el producto interno y dominar el análisis funcional para el estudio de la convergencia de sucesiones de vectores.

Todos estos detalles, para la computación cuántica son indiferentes, por la finitud de la dimensión de los espacios vectoriales que nos interesan. Aún así veremos las definiciones para poder entender el concepto.

\begin{defi}[Sucesión de Cauchy]
  En un espacio vectorial con producto interno, una sucesión $(v_n)$ es de Cauchy si para todo $\epsilon \in\R^+$ existe $N\in\N$ tal que
  $$\|v_m - v_n\| < \epsilon \quad \forall m, n > N$$
\end{defi}

Un resultado importante es que en espacios vectoriales con producto interno, toda sucesión convergente es de Cauchy. El recíproco no es cierto en general. Los espacios topológicos completos son los que cumplen esta propiedad.

\begin{defi}[Espacio de Hilbert]
  Un espacio de Hilbert es un espacio vectorial con producto interno completo, es decir, donde toda sucesión de Cauchy converge.
\end{defi}

La importancia de este concepto, es garantizar que las sucesiones que tienden a un único vector, éste forma parte del espacio vectorial.

\begin{eje}
  Un ejemplo de espacio topológico no completo es $\Q$, donde la sucesión
  \[
    x_{n+1}=\frac{1}{2}(x_n+\frac{2}{x_n})\,,
  \]
  con $x_0=1$ cuyo límite es $\sqrt{2}$, que no es un número racional.

\end{eje}
\begin{prop}
  Todo espacio vectorial con producto interno de dimensión finita es completo y, por tanto, es un espacio de Hilbert.
\end{prop}

Como ya habíamos comentado, nuestro objeto de estudio será un espacio vectorial complejo de dimensión finita, que cuenta con un producto interno, y por el resultado anterior, un espacio de Hilbert. Esto nos permite hablar de espacios vectoriales y de espacios de Hilbert indistintamente.

Sin embargo, haremos una distinción entre espacios vectoriales y espacios de Hilbert, sobre todo a partir del siguiente tema, usando el concepto de espacio de Hilbert para contextos cuánticos y de espacios vectoriales para contextos matemáticos.

\subsection{Operadores en espacios vectoriales con producto interno}

\begin{defi}[Operador adjunto]
  Sea $T\in\mathcal{L}(V)$ un operador lineal. El operador \textbf{adjunto} $T^\dagger\in\mathcal{L}(V)$ es el único operador que satisface
  \begin{equation*}
    \langle T(u), v \rangle = \langle u, T^\dagger (v) \rangle\,,
  \end{equation*}
  para todo $u,v \in V$.
\end{defi}

Sobre la base canónica de $V$, aplicando la definición del operador adjunto es
\begin{align*}
  \langle T(e_j), e_i \rangle         & = \langle \sum_{k=1}^n [T]_{k,j} e_k, e_i \rangle = [T]_{i,j}\,.                            \\
  \langle e_j, T^\dagger(e_i) \rangle & = \langle e_j, \sum_{k=1}^n [T^\dagger]_{k,i} e_k \rangle = \overline{[T^\dagger]_{j,i}}\,.
\end{align*}

Como ambas expresiones son iguales, se cumple que $[T]_{i,j} = \overline{[T^\dagger]_{j,i}}$, es decir, que la matriz asociada a $T^\dagger$ es la matriz adjunta de la matriz asociada a $T$:
\[
  [T^\dagger] = [T]^\dagger\,.
\]

\begin{prop}
  El operador adjunto satisface:
  \begin{enumerate}
    \item $(T^\dagger)^\dagger = T$.
    \item $(S + T)^\dagger = S^\dagger + T^\dagger$.
    \item $(\alpha T)^\dagger = \conj{\alpha} T^\dagger$.
    \item $(ST)^\dagger = T^\dagger S^\dagger$.
  \end{enumerate}
\end{prop}

\begin{defi}
  Sea $T\in\mathcal{L}(V)$ un operador lineal. Diremos que $T$ es:
  \begin{itemize}
    \item \textbf{Hermitiano (autoadjunto):} $T^\dagger = T$.
    \item \textbf{Unitario:} $T^\dagger T = TT^\dagger = I$.
    \item \textbf{Normal:} $T^\dagger T = TT^\dagger$.
  \end{itemize}
\end{defi}

El motivo de dar a estas definiciones a los operadores, con los mismos nombres que las que dimos para matrices, es que son equivalentes.

\begin{prop}
  \begin{itemize}
    \item Sea $A \in \mathcal{M}_n(\C)$. Entonces $A$ es hermitiana, unitaria o normal si y solo si $T_A$ es hermitiano, unitario o normal, respectivamente.
    \item Sea $T \in \mathcal{L}(V)$. Entonces $T$ es hermitiano, unitario o normal si y solo si $[T]$ es hermitiana, unitaria o normal, respectivamente.
  \end{itemize}
\end{prop}


\begin{prop}
  Si $T$ es un operador lineal hermitiano, entonces:
  \begin{enumerate}
    \item Todos los valores propios de $T$ son reales.
    \item Vectores propios correspondientes a valores propios distintos son ortogonales.
    \item $\langle Tv, v \rangle \in \R$ para todo $v$.
  \end{enumerate}
\end{prop}

\begin{prop}
  Si $U$ es un operador lineal unitario, entonces:
  \begin{enumerate}
    \item $U$ preserva el producto interno: $\langle Uu, Uv \rangle = \langle u, v \rangle$, para todo $u$ y $v$.
    \item $U$ preserva la norma: $\|Uv\| = \|v\|$.
    \item Todos los valores propios tienen módulo 1.
  \end{enumerate}
\end{prop}

\begin{theo}[Teorema espectral para dimensiones finitas]
  Sea $T$ un operador normal en un espacio vectorial complejo de dimensión finita $V$.  Entonces existe una base ortonormal de $V$ formada por los vectores propios de $T$. Ademas la representación matricial de $T$ con respecto a esta base es una matriz diagonal formada por los valores propios de $T$.
\end{theo}

Cuando se habla del los valores espectrales de un operador o matriz normal, se refiere a los valores propios de este, ordenados de menor a mayor.

\begin{eje}[Diagonalización espectral]
  Considerar el operador hermitiano
  \[
    A = \begin{pmatrix} 2 & 1-i \\ 1+i & 3 \end{pmatrix}\,.
  \]
  Para calcular los valores propios, primero obtenemos el polinomio característico
  \[
    \det(A - \lambda I) = (2-\lambda)(3-\lambda) - (1-i)(1+i) = \lambda^2 - 5\lambda + 4\,.
  \]
  Y resolvemos $\lambda^2 - 5\lambda + 4 = 0$, obteniendo los valores propios $\lambda_1 = 1$ y $\lambda_2 = 4$.

  Como $A$ es hermítica, sus valores propios son reales, tal como se esperaba.

  El espacio propio asociado a $\lambda_1$ es $\text{gen}\left\{(-1+i, 1)\right\}$, mientras que el espacio propio asociado a $\lambda_2$ es $\text{gen}\left\{(1, 1+i)\right\}$.

  Observamos que los espacios propios son ortogonales:
  \[
    \left\langle (-1+i, 1), (1, 1+i) \right\rangle = (-1+i)(1) + 1(1-i) = 0\,.
  \]
  Aplicando Gram-Schmidt obtenemos la base ortonormal
  \[
    \left\{\frac{1}{\sqrt{3}}(-1+i, 1), \frac{1}{\sqrt{3}}(1, 1+i)\right\}\,.
  \]
  La matriz unitaria de cambio de coordenadas $U$ se forma con los vectores propios normalizados como columnas
  \[
    U = \frac{1}{\sqrt{3}}\begin{pmatrix} -1+i & 1   \\
                1    & 1+i\end{pmatrix}\,.
  \]

  La matriz diagonal D se forma con los valores propios en la diagonal, en el mismo orden que sus vectores propios asociados en U
  \[
    D = \begin{pmatrix} 1 & 0 \\ 0 & 4 \end{pmatrix}\,.
  \]

  Finalmente, verificamos la diagonalización espectral
  \begin{align*}
    A & = U D U^\dagger = \frac{1}{\sqrt{3}}\begin{pmatrix} -1+i & 1   \\
                1    & 1+i\end{pmatrix}\begin{pmatrix} 1 & 0 \\ 0 & 4 \end{pmatrix}\frac{1}{\sqrt{3}}\begin{pmatrix} -1-i & 1   \\
                1    & 1-i\end{pmatrix} \\
      & = \frac{1}{3}\begin{pmatrix} -1+i & 4      \\
                1    & 4(1+i)\end{pmatrix}\begin{pmatrix} -1-i & 1   \\
                1    & 1-i\end{pmatrix}                                                                                   \\
      & = \frac{1}{3}\begin{pmatrix} 6    & 3-3i \\
                3+3i & 9\end{pmatrix} = \begin{pmatrix} 2 & 1-i \\ 1+i & 3 \end{pmatrix}\,.
  \end{align*}
\end{eje}

Una de las operaciones más habituales en computación cuántica, es el cálculo de la exponencial de la matriz de una operación unitaria. La descomposición espectral de este tipo de matrices será crucial para este cálculo.

\subsection{Exponencial de una matriz}

Sin entrar en cuestiones sobre existencia y convergencia de series, vamos a definir la exponencial de una matriz $A$ de forma similar a la exponencial de un número real.

\begin{defi}
  Sea $T \in \mathcal{L}(V)$.
  Definimos la exponencial de $T$ como la exponencial de la matriz $[T]$.
  \[
    e^T = e^{[T]} = \sum_{k=0}^\infty \frac{[T]^k}{k!}\,.
  \]
\end{defi}

Para nuestro objetivo, podemos usar la descomposición espectral para calcular la exponencial. Sólo debemos observar que si $A$ es una matriz unitaria y su descomposición espectral es $U D U^\dagger$, entonces
\begin{align*}
  A^k & = (U D U^\dagger)^k = (U D U^\dagger)(U D U^\dagger)\dots (U D U^\dagger) \\
      & = U D (U^\dagger U)D (U^\dagger U)\dots D (U^\dagger U)DU^\dagger         \\
      & = U D D\dots DU^\dagger = U D^k U^\dagger\,.
\end{align*}

Por lo tanto, la exponencial de una matriz unitaria se puede calcular como
\begin{align*}
  e^A & = \sum_{k=0}^\infty \frac{A^k}{k!} = \sum_{k=0}^\infty \frac{U D^k U^\dagger}{k!} = U \sum_{k=0}^\infty \frac{D^k}{k!} U^\dagger = U e^D U^\dagger\,.
\end{align*}

Para las matrices diagonales, la exponencial es simplemente la exponencial de cada elemento de la diagonal. Si $D=\text{diag}(\lambda_1, \lambda_2, \dots, \lambda_n)$, entonces
\begin{align*}
  e^D & = \text{diag}(e^{\lambda_1}, e^{\lambda_2}, \dots, e^{\lambda_n})\,.
\end{align*}

\begin{eje}[Exponencial de una matriz diagonal]
  \label{eje:exponencial_z}
  Considerar la matriz diagonal
  \[
    Z = \begin{pmatrix} 1 & 0 \\ 0 & -1 \end{pmatrix}\,.
  \]

  Entonces
  \[
    e^Z = \begin{pmatrix} e & 0 \\ 0 & e^{-1} \end{pmatrix}\,.
  \]
\end{eje}

\begin{eje}[Exponencial de una matriz unitaria]
  \label{eje:exponencial_x}
  Considerar la matriz unitaria
  \[
    X = \begin{pmatrix} 0 & 1 \\ 1 & 0 \end{pmatrix}\,,
  \]
  que tiene la descomposición espectral
  \[
    X = H Z H^\dagger = \frac{1}{\sqrt{2}}\begin{pmatrix} 1 & 1 \\ 1 & -1 \end{pmatrix}\begin{pmatrix} 1 & 0 \\ 0 & -1 \end{pmatrix}\frac{1}{\sqrt{2}}\begin{pmatrix} 1 & 1 \\ 1 & -1 \end{pmatrix}\,.
  \]

  Entonces
  \[
    e^X = H e^Z H^\dagger = \frac{1}{\sqrt{2}}\begin{pmatrix} 1 & 1 \\ 1 & -1 \end{pmatrix}\begin{pmatrix} e & 0 \\ 0 & e^{-1} \end{pmatrix}\frac{1}{\sqrt{2}}\begin{pmatrix} 1 & 1 \\ 1 & -1 \end{pmatrix} = \frac{1}{2}\begin{pmatrix} e+e^{-1} & e-e^{-1} \\ e-e^{-1} & e+e^{-1} \end{pmatrix}\,.
  \]

  La forma más habitual de ver desarrollada la exponencial de la matriz $X$ es mediante su expresión trigonométrica, para ello necesitamos conocer las siguientes identidades:
  \begin{align*}
    \sinh z & = \frac{e^z - e^{-z}}{2}    \\
    \cosh z & = \frac{e^z + e^{-z}}{2}\,.
  \end{align*}

  Entonces
  \[
    e^X = \begin{pmatrix} \cosh 1 & \sinh 1 \\ \sinh 1 & \cosh 1 \end{pmatrix}\,.
  \]
\end{eje}


%\unirsection{Problemas}

\begin{questions}

  \question Verificar que las siguientes funciones definen productos internos y calcular la norma inducida:
  \begin{parts}
    \part En $\C^2$: $\langle \mathbf{u}, \mathbf{v} \rangle = 2u_1\conj{v_1} + u_1\conj{v_2} + u_2\conj{v_1} + 3u_2\conj{v_2}$
    \part En $\C^{2 \times 2}$: $\langle A, B \rangle = \text{tr}(A^* B)$
  \end{parts}

  \question Aplicar el proceso de Gram-Schmidt para ortogonalizar:
  $$\left\{\begin{pmatrix} 1 \\ 1 \\ i \end{pmatrix}, \begin{pmatrix} 1 \\ i \\ 0 \end{pmatrix}, \begin{pmatrix} i \\ 1 \\ 1 \end{pmatrix}\right\}$$

  \question Sea $W = \text{gen}\left\{\begin{pmatrix} 1 \\ i \\ 0 \end{pmatrix}, \begin{pmatrix} 0 \\ 1 \\ i \end{pmatrix}\right\} \subset \C^3$.
  \begin{parts}
    \part Encontrar una base ortonormal para $W$
    \part Calcular $\text{proj}_W\begin{pmatrix} 1 \\ 1 \\ 1 \end{pmatrix}$
    \part Encontrar el complemento ortogonal $W^\perp$
  \end{parts}

  \question Para las matrices $A = \begin{pmatrix} 2 & 1+i \\ 1-i & 3 \end{pmatrix}$ y $B = \begin{pmatrix} 1 & 2i \\ -2i & 1 \end{pmatrix}$:
  \begin{parts}
    \part Verificar cuáles son hermitianas
    \part Calcular los valores propios y vectores propios de las hermitianas
    \part Diagonalizar las matrices hermitianas usando una base ortonormal
  \end{parts}

  \question Demostrar que las siguientes matrices son unitarias y encontrar su descomposición espectral:
  \begin{parts}
    \part $U_1 = \frac{1}{\sqrt{2}}\begin{pmatrix} 1 & i \\ i & 1 \end{pmatrix}$
    \part $U_2 = \frac{1}{\sqrt{2}}\begin{pmatrix} 1 & 1 \\ 1 & -1 \end{pmatrix}$ (matriz de Hadamard)
  \end{parts}

  \question Un qubit se encuentra en el estado $\ket{\psi} = \frac{3}{5}\ket{0} + \frac{4i}{5}\ket{1}$.
  \begin{parts}
    \part Verificar que el estado está normalizado
    \part Calcular las probabilidades de medir $\sigma_x$, $\sigma_y$ y $\sigma_z$
    \part Determinar el estado después de medir $\sigma_z$ y obtener el resultado $+1$
  \end{parts}

  \question Demostrar las siguientes propiedades para operadores en espacios de Hilbert:
  \begin{parts}
    \part Si $T$ es hermitiano, entonces $\langle T\mathbf{v}, \mathbf{v} \rangle \in \R$ para todo $\mathbf{v}$
    \part Si $U$ es unitario, entonces $U$ preserva el producto interno
    \part El producto de dos operadores unitarios es unitario
  \end{parts}

  \question Para el operador $H = \begin{pmatrix} 1 & 0 & 0 \\ 0 & 0 & 1 \\ 0 & 1 & 0 \end{pmatrix}$:
  \begin{parts}
    \part Verificar que $H$ es hermitiano
    \part Encontrar todos los valores propios y vectores propios
    \part Construir la descomposición espectral $H = \sum_i \lambda_i P_i$ donde $P_i$ son proyecciones ortogonales
  \end{parts}

  \question Para el observable $\hat{A} = \begin{pmatrix} 2 & 1-i \\ 1+i & 3 \end{pmatrix}$:
  \begin{parts}
    \part Verificar que $\hat{A}$ es hermitiano
    \part Encontrar los valores propios y vectores propios
    \part Escribir la descomposición espectral de $\hat{A}$
  \end{parts}

  \question Un cúbit se encuentra en el estado $\ket{\psi} = \frac{1}{\sqrt{5}}(2\ket{0} + i\ket{1})$:
  \begin{parts}
    \part Calcular las probabilidades de medir $\sigma_x$, $\sigma_y$ y $\sigma_z$
    \part Determinar los valores esperados $\langle\sigma_x\rangle$, $\langle\sigma_y\rangle$ y $\langle\sigma_z\rangle$
    \part Si se mide $\sigma_x$ y se obtiene el resultado $+1$, ¿cuál es el estado después de la medición?
  \end{parts}

  \question Demostrar que si $\hat{A}$ y $\hat{B}$ son operadores hermitianos que conmutan ($[\hat{A}, \hat{B}] = 0$), entonces tienen una base común de vectores propios.

  \question Para el operador $\hat{H} = \frac{1}{\sqrt{2}}(\sigma_x + \sigma_z)$:
  \begin{parts}
    \part Verificar que $\hat{H}$ es hermitiano
    \part Encontrar la descomposición espectral de $\hat{H}$
    \part Si un cúbit está en estado $\ket{0}$, ¿cuáles son las probabilidades de medir $\pm 1$ para el observable $\hat{H}$?
  \end{parts}

  \question Sea $V = \mathbb{C}^2$ con la base $\mathcal{B} = \left\{\begin{pmatrix} 1 \\ i \end{pmatrix}, \begin{pmatrix} 1 \\ -i \end{pmatrix}\right\}$.
  \begin{parts}
    \part Verificar que $\mathcal{B}$ es efectivamente una base de $V$.
    \part Encontrar la base dual $\mathcal{B}^* = \{e_1^*, e_2^*\}$ correspondiente.
    \part Expresar el funcional $f(x,y) = x + 2y$ como combinación lineal de elementos de $\mathcal{B}^*$.
    \part Calcular $e_1^*\begin{pmatrix} 2+i \\ 1-i \end{pmatrix}$ y $e_2^*\begin{pmatrix} 2+i \\ 1-i \end{pmatrix}$.
  \end{parts}

  \question En un espacio de Hilbert $\mathcal{H} = \mathbb{C}^2$ con el producto interno estándar:
  \begin{parts}
    \part Usar el teorema de Riesz-Fréchet para encontrar el vector $y_f \in \mathcal{H}$ que representa el funcional $f(x) = \begin{pmatrix} 1-i & 2+i \end{pmatrix} x$
    \part Verificar que $f(x) = \langle x, y_f \rangle$ para cualquier $x \in \mathcal{H}$.
  \end{parts}

  \question Considerar el operador $T: \mathbb{C}^2 \to \mathbb{C}^2$ definido por la matriz $T = \begin{pmatrix} 1 & i \\ 0 & 2 \end{pmatrix}$.
  \begin{parts}
    \part Encontrar la representación matricial del operador adjunto $T^*$.
    \part Verificar que $\langle Tx, y \rangle = \langle x, T^*y \rangle$ para $x = \begin{pmatrix} 1 \\ i \end{pmatrix}$ y $y = \begin{pmatrix} 1+i \\ 1-i \end{pmatrix}$.
    \part Determinar si $T$ es hermitiano, unitario o normal.
  \end{parts}

  \question Considere el estado cuántico normalizado en el espacio de Hilbert $\mathcal{H} \cong \mathbb{C}^2$:
  $$ |\psi\rangle = \frac{1}{\sqrt{5}} \begin{pmatrix} 1+i \\ 2 \end{pmatrix} $$

  \question Sea el operador lineal $A$ actuando sobre $\mathbb{C}^2$, representado por la matriz:
  $$A = \begin{pmatrix} 0 & -i \\ i & 0 \end{pmatrix}$$

  \begin{parts}
    \part Calcule el operador adjunto $A^\dagger$ (transpuesta conjugada de $A$).
    \part Determine si $A$ es un operador \textbf{hermítico} (autoadjunto). ¿Qué implicación física tiene esta propiedad en la Mecánica Cuántica?
    \part Calcule los valores propios de $A$ y demuestre que son reales.
  \end{parts}

  \question Considere el operador hermítico (observable) $H$ dado por:
  $$H = \begin{pmatrix} 3 & 1 \\ 1 & 3 \end{pmatrix}$$

  \begin{parts}
    \part Encuentre los valores propios $\lambda_1$ y $\lambda_2$.
    \part Determine los vectores propios normalizados $v_1$ y $v_2$ asociados a estos valores propios.
    \part Escriba la matriz diagonal $D$ y la matriz unitaria de cambio de base $U$ (la matriz de los estados propios). Confirme la descomposición espectral: $H = U D U^\dagger$.
  \end{parts}

  \question Considere el sistema compuesto de dos qubits. El espacio de Hilbert es $\mathcal{H}_{AB} = \mathcal{H}_A \otimes \mathcal{H}_B$, donde $\mathcal{H}_A \cong \mathbb{C}^2$ y $\mathcal{H}_B \cong \mathbb{C}^2$.

  \begin{parts}
    \part ¿Cuál es la dimensión de $\mathcal{H}_{AB}$? Enumere todos los elementos de la base canónica tensorial.
    \part Considere el tensor $t = \frac{1}{\sqrt{2}}(e_0\otimes e_1 - e_1\otimes e_0)$. Intente encontrar dos vectores $a \in \mathcal{H}_A$ y $b \in \mathcal{H}_B$ tal que $t = a \otimes b$. ¿Es $t$ un tensor simple (estado separable)?
  \end{parts}

  \question Sea el operador $X$ (la matriz NOT cuántica o Pauli $X$) y el operador $Z$ (Pauli $Z$):
  $$ X = \begin{pmatrix} 0 & 1 \\ 1 & 0 \end{pmatrix}, \quad Z = \begin{pmatrix} 1 & 0 \\ 0 & -1 \end{pmatrix} $$

  \begin{parts}
    \part Calcule el operador compuesto $C = X \otimes Z$ utilizando el producto de Kronecker. Escriba la matriz resultante en $\mathbb{C}^4$.
    \part El estado $|01\rangle$ se representa como el vector:
    $$ |01\rangle = |0\rangle \otimes |1\rangle = \begin{pmatrix} 1 \\ 0 \end{pmatrix} \otimes \begin{pmatrix} 0 \\ 1 \end{pmatrix} = \begin{pmatrix} 0 \\ 1 \\ 0 \\ 0 \end{pmatrix} $$
    Calcule el resultado de aplicar el operador $C$ a este estado, es decir, $C|01\rangle$.
    \part Verifique el resultado anterior aplicando la propiedad del producto tensorial de operadores directamente sobre el tensor simple:
    $$ C|01\rangle = (X \otimes Z) (|0\rangle \otimes |1\rangle) = (X|0\rangle) \otimes (Z|1\rangle)\,. $$
  \end{parts}

  \question Para el sistema de dos cúbits en el estado $\ket{\psi} = \frac{1}{\sqrt{3}}(\ket{00} + \ket{01} + \ket{11})$:
  \begin{parts}
    \part Verificar que el estado está normalizado
    \part Calcular las probabilidades de medir cada estado de la base computacional
    \part Determinar si el estado es separable o entrelazado
    \part Calcular la probabilidad de medir el primer cúbit en $\ket{0}$
  \end{parts}

  \question Para los cuatro estados de Bell:
  \begin{parts}
    \part Verificar que forman una base ortonormal
    \part Expresar cada estado de Bell como combinación lineal de la base computacional
    \part Demostrar que todos son entrelazados usando la descomposición de Schmidt
    \part Calcular las probabilidades de medir cada cúbit individualmente
  \end{parts}

  \question Determine si el estado $\ket{\psi} = \frac{1}{\sqrt{3}}(\ket{00} + \ket{01} + \ket{10})$ es entrelazado.

  \question Para $A = \sigma_x$ y $B = \sigma_y$, calcule explícitamente $(A \otimes B)\ket{01}$.

\end{questions}
%\tema{5}{Espacio dual y producto tensorial}
%\portada

\begin{esquemaExplorador}
  \temaEsquema{Espacio dual}{
    \conceptoEsquema{Definición}
    \conceptoEsquema{Base dual}
    \conceptoEsquema{Teorema de Riesz-Fréchet}
  }
  \temaEsquema{Dualidad en espacios de Hilbert}{
    \conceptoEsquema{Isomorfismo canónico}
    \conceptoEsquema{Representación de funcionales}
    \conceptoEsquema{Aplicaciones a computación cuántica}
  }
  \temaEsquema{Producto tensorial}{
    \conceptoEsquema{Definición}
    \conceptoEsquema{Propiedades}
    \conceptoEsquema{Base tensorial}
    \conceptoEsquema{Producto tensorial de operadores y matrices}
    \conceptoEsquema{Ejemplos y aplicaciones}
  }
  \temaEsquema{Postulados de la mecánica cuántica}{
    \conceptoEsquema{Postulado IV: Composición de sistemas cuánticos}
  }
  \temaEsquema{Estados entrelazados}{
    \conceptoEsquema{Definición de entrelazamiento}
    \conceptoEsquema{Estados de Bell}
    \conceptoEsquema{Descomposición de Schmidt}
    \conceptoEsquema{Producto exterior}
  }
\end{esquemaExplorador}

\unirsection{Ideas clave}

\subsection{Introducción y objetivos}


Hasta ahora hemos explorado la columna vertebral matemática de la Mecánica Cuántica: los espacios vectoriales complejos ($\C^n$) y los espacios de Hilbert ($\H$), que sirven como el \textbf{espacio de estados} para un \textbf{sistema cuántico simple} (un \textit{cúbit}, por ejemplo).

\begin{itemize}
  \item Un \textbf{cúbit} se describe en $\mathcal{H}_1 \cong \mathbb{C}^2$.
  \item Un operador lineal (una medición, una compuerta) en este cúbit es una matriz en $\mathcal{L}(\mathcal{H}_1)$.
\end{itemize}

El concepto de espacio dual es fundamental en el análisis funcional y proporciona una perspectiva profunda sobre la estructura de los espacios vectoriales. En computación cuántica, el espacio dual adquiere un significado especial a través de la notación de Dirac, donde los \textit{bras} representan elementos del espacio dual de los \textit{kets}.

La importancia del espacio dual en matemáticas y física se manifiesta en varios aspectos:

\begin{itemize}
  \item Proporciona un marco \textbf{unificado} para entender funcionales lineales y formas multilineales.
  \item Permite la \textbf{caracterización completa} de propiedades geométricas mediante funcionales.
  \item Es esencial para la teoría de \textbf{operadores adjuntos} y autoadjuntos.
  \item Conecta la \textbf{notación bra-ket} con la estructura matemática.
\end{itemize}

Sin embargo, la verdadera potencia de la Computación Cuántica radica en la \textbf{superposición} y el \textbf{entrelazamiento} de \textbf{múltiples sistemas} (varios cúbits).

\paragraph{El problema de composición:}
Si tenemos dos sistemas cuánticos independientes, $S_A$ y $S_B$, descritos por sus respectivos espacios de Hilbert $\mathcal{H}_A$ y $\mathcal{H}_B$, ¿cuál es el espacio de estados $\mathcal{H}_{AB}$ que describe el \textbf{sistema compuesto} $S_{AB}$?

\begin{itemize}
  \item Si $S_A$ tiene dimensión $n$ ($\text{dim}(\mathcal{H}_A) = n$) y $S_B$ tiene dimensión $m$ ($\text{dim}(\mathcal{H}_B) = m$), el espacio compuesto $\mathcal{H}_{AB}$ debe tener dimensión \textbf{$nm$}.
  \item \textbf{La suma directa} ($\mathcal{H}_A \oplus \mathcal{H}_B$), que usamos para combinar estados linealmente independientes, tiene dimensión $n+m$. \textbf{No sirve}.
  \item La operación correcta para combinar espacios de estados y construir el espacio de estados compuesto es el \textbf{producto tensorial} ($\otimes$).
\end{itemize}

El \textbf{producto tensorial} es el mecanismo matemático que nos permite describir rigurosamente estados entrelazados como el famoso estado de Bell $\frac{1}{\sqrt{2}}(|00\rangle + |11\rangle)$, que no puede ser separado en un simple producto de estados individuales.

\subsection{Espacio dual}

\begin{defi}[Espacio dual]
  Sea $V$ un espacio vectorial. El \textbf{espacio dual} de $V$, denotado $V^*$, es el espacio vectorial de todos los funcionales lineales sobre $V$:
  $$V^* = \{f: V \to \C : f \text{ es lineal}\}$$

  Las operaciones en $V^*$ se definen puntualmente:
  \begin{align}
    (f + g)(x)    & = f(x) + g(x) & \forall x \in V                \\
    (\alpha f)(x) & = \alpha f(x) & \forall x \in V, \alpha \in \C
  \end{align}
\end{defi}

\begin{theo}[Dimensión del espacio dual]
  Si $V$ es un espacio vectorial de dimensión finita $n$ sobre $\C$, entonces $\dim V^* = n$.
\end{theo}

\begin{proof}
  Sea $\{e_1, e_2, \ldots, e_n\}$ una base de $V$. Definimos los funcionales $\{e_1^*, e_2^*, \ldots, e_n^*\}$ mediante:
  $$e_i^*(e_j) = \delta_{ij} = \begin{cases}
      1 & \text{si } i = j    \\
      0 & \text{si } i \neq j
    \end{cases}$$

  Para cualquier $f \in V^*$ y cualquier $x = \sum_{i=1}^n x_i e_i \in V$:
  $$f(x) = f\left(\sum_{i=1}^n x_i e_i\right) = \sum_{i=1}^n x_i f(e_i) = \sum_{i=1}^n f(e_i) e_i^*(x)$$

  Por tanto, $f = \sum_{i=1}^n f(e_i) e_i^*$, lo que muestra que $\{e_1^*, e_2^*, \ldots, e_n^*\}$ genera $V^*$. La independencia lineal se verifica fácilmente, por lo que constituye una base de $V^*$.
\end{proof}

La demostración anterior proporciona una construcción explícita de una base del espacio dual a partir de una base del espacio original, que será útil en aplicaciones prácticas.

\begin{defi}[Base dual]
  Si $\{e_1, e_2, \ldots, e_n\}$ es una base de $V$, la \textbf{base dual} $\{e_1^*, e_2^*, \ldots, e_n^*\}$ de $V^*$ se define por:
  $$e_i^*(e_j) = \delta_{ij}$$
\end{defi}

\subsection{Teorema de representación de Riesz-Fréchet}
Los resultados expuestos a continuación requieren de conocimientos avanzados en análisis funcional y teoría de espacios de Hilbert. Se recomienda consultar las referencias bibliográficas al final del tema para una comprensión más profunda. Se incluyen para entender la importancia del espacio dual y su conexión con la notación de Dirac.

En espacios de Hilbert, existe una correspondencia especial entre el espacio y su dual.

\begin{theo}[Teorema de Riesz-Fréchet]
  Sea $\mathcal{H}$ un espacio de Hilbert complejo con producto interno $\langle \cdot, \cdot \rangle$. Para cada funcional lineal continuo $f \in \mathcal{H}'$, existe un único elemento $y_f \in \mathcal{H}$ tal que:
  $$f(x) = \langle x, y_f \rangle \quad \forall x \in \mathcal{H}$$

  Además, $\|f\|_{\mathcal{H}'} = \|y_f\|_{\mathcal{H}}$.

  La aplicación $\Phi: \mathcal{H} \to \mathcal{H}'$ definida por $\Phi(y)(x) = \langle x, y \rangle$ es un isomorfismo antilineal isométrico.
\end{theo}

\subsection{Aplicaciones computacionales}

\begin{eje}[Mediciones cuánticas como funcionales]
  En mecánica cuántica, una medición de un observable $A$ en un estado $|\psi\rangle$ se puede interpretar como la evaluación de un funcional:

  El valor esperado $\langle A \rangle$ se puede escribir como:
  $$\langle A \rangle = f_A(|\psi\rangle)$$

  donde $f_A$ es el funcional definido por $f_A(|\phi\rangle) = \ket{\phi}^*(A\ket{\phi})$.
\end{eje}

\begin{eje}[Fidelidad como producto en el dual]
  La fidelidad cuántica entre dos estados $|\psi\rangle$ y $|\phi\rangle$ se define como:
  $$F(|\psi\rangle, |\phi\rangle) = |\langle\psi, \phi\rangle|^2$$

  Esto representa la evaluación del funcional $\langle\psi|$ sobre el vector $|\phi\rangle$, seguida de la toma del módulo al cuadrado.
\end{eje}

\subsection{Producto tensorial en espacios vectoriales}

Sean $V$ y $W$ dos espacios vectoriales complejos. El \textbf{producto tensorial} de $V$ y $W$, denotado por $V \otimes W$, es, informalmente, el espacio vectorial "más pequeño" que contiene todos los productos formales de la forma $v \otimes w$, para $v \in V$ y $w \in W$, y que respeta la \textbf{bilinealidad}.

\begin{defi}
  El \textbf{producto tensorial} de dos espacios vectoriales complejos $V$ y $W$, denotado por $V \otimes W$, es el espacio vectorial generado por los productos tensoriales simples $v \otimes w$, donde $v \in V$ y $w \in W$, sujeto a las relaciones de bilinealidad.
\end{defi}

El producto tensorial $V \otimes W$ se construye a partir del espacio vectorial libre $F(V \times W)$ generado por los pares ordenados $(v, w) \in V \times W$, factorizando por el subespacio $R$ generado por las relaciones de \textbf{bilinealidad}:

\begin{enumerate}
  \item \textbf{Linealidad en el primer argumento:}
        $$(\alpha v_1 + \beta v_2, w) - \alpha (v_1, w) - \beta (v_2, w)$$
  \item \textbf{Linealidad en el segundo argumento:}
        $$(v, \alpha w_1 + \beta w_2) - \alpha (v, w_1) - \beta (v, w_2)$$
\end{enumerate}
donde $v, v_1, v_2 \in V$, $w, w_1, w_2 \in W$, y $\alpha, \beta \in \mathbb{K}$.

El \textbf{producto tensorial} es el espacio cociente:
$$V \otimes W := F(V \times W) / R$$

El \textbf{tensor simple} $v \otimes w$ es la clase de equivalencia del par $(v, w)$ en el espacio cociente.

\begin{theo}[Propiedad universal]
  El producto tensorial $V \otimes W$ es un espacio vectorial, junto con una aplicación \textbf{bilineal} $\tau: V \times W \to V \otimes W$, que satisface la siguiente propiedad universal:

  Para cualquier espacio vectorial $Z$ y cualquier aplicación \textbf{bilineal} $f: V \times W \to Z$, existe una \textbf{única} aplicación \textbf{lineal} $\tilde{f}: V \otimes W \to Z$ tal que el diagrama conmuta: $f = \tilde{f} \circ \tau$.
\end{theo}

Esta propiedad es la que asegura que $V \otimes W$ es el espacio vectorial \textbf{generado} por los productos $v \otimes w$ con las mínimas relaciones necesarias para preservar la estructura bilineal.

\subsection{Base canónica}

Al trabajar con espacios vectoriales de dimensión finita, podemos entender cómo se construyen las bases en el producto tensorial y por tanto su estructura.

\begin{prop}
  Si $\{v_i\}_{i=1}^n$ es una base para $V$ y $\{w_j\}_{j=1}^m$ es una base para $W$, entonces el conjunto de \textbf{tensores simples}
  $$
    \mathcal{B}_{V \otimes W} = \{v_i \otimes w_j\}_{i=1,\ldots,n}^{j=1,\ldots,m}
  $$
  es una base para $V \otimes W$.
\end{prop}

Como consecuencia del resultado anterior, todo elemento $t \in V \otimes W$ puede escribirse de manera única como una combinación lineal de los elementos de la base canónica:
$$t = \sum_{i=1}^n \sum_{j=1}^m \alpha_{ij} (v_i \otimes w_j)$$
donde $\alpha_{ij} \in \mathbb{C}$.

Además, la dimensión del espacio tensorial es el producto de las dimensiones de los espacios originales:
$$\text{dim}(V \otimes W) = \text{dim}(V) \cdot \text{dim}(W) = n \cdot m$$


\begin{defi}
  Sea $V$ y $W$ dos espacios vectoriales complejos.
  Un elemento $t \in V \otimes W$ que puede escribirse como $t = v \otimes w$, con $v\in V$ y $w\in W$, se llama \textbf{tensor simple} o \textbf{producto separable}.
\end{defi}

Sin embargo, la mayoría de los elementos de $V \otimes W$ son \textbf{combinaciones lineales} de tensores simples:
$$t = \sum_{k} \alpha_k (v_k \otimes w_k)$$

\begin{nota}
  Un tensor que \textbf{no} puede escribirse como un tensor simple es lo que se conoce en la cuántica como un \textbf{estado entrelazado}.
\end{nota}

\subsection{Propiedades del producto tensorial}

Recordemos que espacios vectoriales de la misma dimensión son isomorfos, pero no \textit{iguales}. Por tanto, las siguientes propiedades se entienden como isomorfismos naturales entre espacios vectoriales.
\begin{itemize}
  \item \textbf{Asociatividad:}
        $$(U \otimes V) \otimes W \cong U \otimes (V \otimes W) \cong U \otimes V \otimes W$$
  \item \textbf{Conmutatividad:}
        $$V \otimes W \cong W \otimes V$$
\end{itemize}

\subsection{Producto tensorial de operadores y matrices}
Podemos extender la definición del producto tensorial a operadores lineales entre espacios vectoriales de manera natural.
\begin{defi}
  Si $A: V_1 \to V_2$ y $B: W_1 \to W_2$ son operadores lineales, podemos definir el \textbf{operador tensorial} $A \otimes B$ en el espacio tensorial $V_1 \otimes W_1$, como
  \begin{align*}
    A \otimes B : V_1 \times W_1                            & \to V_2 \times W_2                                                    \\
    \sum_{i=1}^n \sum_{j=1}^m \alpha_{ij} (v_i \otimes w_j) & \mapsto \sum_{i=1}^n \sum_{j=1}^m \alpha_{ij} (A(v_i) \otimes B(w_j))
  \end{align*}

\end{defi}

De igual manera, podemos definir el producto tensorial de matrices, siendo esta forma la más cómoda de trabajar.

Sean $A \in M_{n \times m}(\C)$ y $B \in M_{p \times q}(\C)$ dos matrices complejas. El producto tensorial matricial $A \otimes B$ es una matriz en $M_{np \times mq}(\C)$ definida por
$$
  (A \otimes B)_{(i,j),(k,l)} = A_{i,k} B_{j,l}
$$
para $1 \leq i \leq n$, $1 \leq j \leq m$, $1 \leq k \leq p$, $1 \leq l \leq q$. Es decir, si
$$A = \begin{pmatrix}
    a_{11} & a_{12} & \ldots & a_{1m} \\
    a_{21} & a_{22} & \ldots & a_{2m} \\
    \vdots & \vdots & \ddots & \vdots \\
    a_{n1} & a_{n2} & \ldots & a_{nm}
  \end{pmatrix} \text{ y } \quad B = \begin{pmatrix}
    b_{11} & b_{12} & \ldots & b_{1q} \\
    b_{21} & b_{22} & \ldots & b_{2q} \\
    \vdots & \vdots & \ddots & \vdots \\
    b_{p1} & b_{p2} & \ldots & b_{pq}
  \end{pmatrix},$$
entonces
$$A \otimes B = \begin{pmatrix}
    a_{11}B & a_{12}B & \ldots & a_{1m}B \\
    a_{21}B & a_{22}B & \ldots & a_{2m}B \\
    \vdots  & \vdots  & \ddots & \vdots  \\
    a_{n1}B & a_{n2}B & \ldots & a_{nm}B
  \end{pmatrix}.$$

\subsection{Sistemas cuánticos compuestos}

\begin{resaltado}
  \textbf{Postulado IV: Composición de sistemas}

  El espacio de estados de un sistema cuántico compuesto, es el formado por el producto tensorial de los subsistemas que lo componen.
\end{resaltado}

\begin{eje}[Sistema de dos cúbits]
  Para dos cúbits independientes, el espacio de estados es $\mathbb{C}^2 \otimes \mathbb{C}^2 \cong \mathbb{C}^4$, con base computacional:
  \begin{align}
    \ket{0} \otimes \ket{0} = \begin{pmatrix} 1 \\ 0 \\ 0 \\ 0 \end{pmatrix} \\
    \ket{0} \otimes \ket{1} = \begin{pmatrix} 0 \\ 1 \\ 0 \\ 0 \end{pmatrix} \\
    \ket{1} \otimes \ket{0} = \begin{pmatrix} 0 \\ 0 \\ 1 \\ 0 \end{pmatrix} \\
    \ket{1} \otimes \ket{1} = \begin{pmatrix} 0 \\ 0 \\ 0 \\ 1 \end{pmatrix}
  \end{align}
\end{eje}

\subsection{Separabilidad y entrelazamiento cuántico}

\begin{defi}[Estado separable]
  Un estado $\ket{\psi} \in \mathcal{H}_A \otimes \mathcal{H}_B$ es \textbf{separable} si existen $\ket{\alpha} \in \mathcal{H}_A$ y $\ket{\beta} \in \mathcal{H}_B$ tales que:
  $$\ket{\psi} = \ket{\alpha} \otimes \ket{\beta}$$

  Un estado es \textbf{entrelazado} si no es separable.
\end{defi}

\begin{eje}[Estados separables]
  Los siguientes estados de dos cúbits son separables:
  \begin{itemize}
    \item $\ket{0} \otimes \ket{0}$
    \item $\frac{1}{\sqrt{2}}(\ket{0}\otimes\ket{1} + \ket{1}\otimes\ket{1}) = \frac{1}{\sqrt{2}}(\ket{0} + \ket{1}) \otimes \ket{1}$
    \item $\frac{1}{2}(\ket{0}\otimes\ket{0} + \ket{0}\otimes\ket{1} + \ket{1}\otimes\ket{0} + \ket{1}\otimes\ket{1}) = \ket{+} \otimes \ket{+}$
  \end{itemize}
\end{eje}

\begin{defi}[Estados de Bell]
  Los cuatro estados de Bell forman una base ortonormal de estados entrelazados para dos cúbits:
  \begin{align}
    \ket{\Phi^+} & = \frac{1}{\sqrt{2}}(\ket{0} \otimes \ket{0} + \ket{1} \otimes \ket{1}) \quad \text{(estado EPR)}      \\
    \ket{\Phi^-} & = \frac{1}{\sqrt{2}}(\ket{0} \otimes \ket{0} - \ket{1} \otimes \ket{1})                                \\
    \ket{\Psi^+} & = \frac{1}{\sqrt{2}}(\ket{0} \otimes \ket{1} + \ket{1} \otimes \ket{0})                                \\
    \ket{\Psi^-} & = \frac{1}{\sqrt{2}}(\ket{0} \otimes \ket{1} - \ket{1} \otimes \ket{0}) \quad \text{(estado singlete)}
  \end{align}
\end{defi}

\begin{eje}[Verificación de entrelazamiento para $\ket{\Phi^+}$]
  Si $\ket{\Phi^+}$ fuera separable, existirían $\ket{\alpha} = a\ket{0} + b\ket{1}$ y $\ket{\beta} = c\ket{0} + d\ket{1}$ tales que:
  $$\ket{\Phi^+} = (a\ket{0} + b\ket{1}) \otimes (c\ket{0} + d\ket{1}) = ac\ket{0}\otimes\ket{0} + ad\ket{0}\otimes\ket{1} + bc\ket{1}\otimes\ket{0} + bd\ket{1}\otimes\ket{1}$$

  Comparando con $\ket{\Phi^+} = \frac{1}{\sqrt{2}}(\ket{0} \otimes \ket{0} + \ket{1} \otimes \ket{1})$:
  $$ac = \frac{1}{\sqrt{2}}, \quad ad = 0, \quad bc = 0, \quad bd = \frac{1}{\sqrt{2}}$$

  De $ad = 0$ y $bc = 0$, o bien $a = c = 0$ o bien $b = d = 0$, pero esto contradice $ac = bd = \frac{1}{\sqrt{2}} \neq 0$.

  Por tanto, $\ket{\Phi^+}$ es entrelazado.
\end{eje}

\begin{info}
  Los estados de Bell exhiben correlaciones cuánticas no locales: medir un cúbit instantáneamente determina el resultado de medir el otro cúbit, independientemente de la distancia que los separe. Esta propiedad es fundamental para protocolos como la teleportación cuántica y la criptografía cuántica.
\end{info}

\begin{theo}[Descomposición de Schmidt]
  Todo estado puro $\ket{\psi} \in \mathcal{H}_A \otimes \mathcal{H}_B$ puede escribirse como:
  $$\ket{\psi} = \sum_{i} \sqrt{\lambda_i} \ket{u_i} \otimes \ket{v_i}$$
  donde $\{\ket{u_i}\}$ y $\{\ket{v_i}\}$ son bases ortonormales, $\lambda_i \geq 0$, y los $\lambda_i$ son únicos.

  El \textbf{número de Schmidt} es la cantidad de $\lambda_i > 0$ y caracteriza el entrelazamiento:
  \begin{itemize}
    \item Número de Schmidt = 1 $\Leftrightarrow$ Estado separable.
    \item Número de Schmidt > 1 $\Leftrightarrow$ Estado entrelazado.
  \end{itemize}
\end{theo}

Una vez que hemos definido el producto tensorial, podemos definir el producto exterior.

\begin{defi}
  Dado un espacio de Hilbert $\mathcal{H}$, se define el \textbf{producto exterior} de dos vectores $u, v \in \mathcal{H}$ por:
  $$u \wedge v = u \otimes v^\dagger\,.$$
\end{defi}

Podemos interpretar el producto exterior como un operador lineal considerando su acción sobre un vector $w \in \mathcal{H}$:
$$ (u \wedge v)(w) = (u \otimes v^\dagger)(w) = \langle v , w \rangle u\,.$$
%\unirsection{Problemas}

\begin{questions}

  \question Sea $V = \mathbb{C}^2$ con la base $\mathcal{B} = \left\{\mqty(1\\ i), \mqty(1\\ -i)\right\}$.
  \begin{parts}
    \part Encontrar la base dual $\mathcal{B}^* = \{e_1^*, e_2^*\}$.
    \part Expresar el funcional $f(x,y) = x + 2y$ como combinación lineal de $\mathcal{B}^*$.
    \part Calcular $e_1^*\mqty(2+i\\ 1-i)$ y $e_2^*\mqty(2+i\\ 1-i)$.
  \end{parts}

  \question Sea el operador lineal $A$ actuando sobre $\mathbb{C}^2$, representado por la matriz:
  $$A = \begin{pmatrix} 0 & -i \\ i & 0 \end{pmatrix}\,.$$

  \begin{parts}
    \part Calcule el operador adjunto $A^\dagger$.
    \part Determine si $A$ es un operador \textbf{hermítico}.
    \part Calcule los valores propios de $A$ y demuestre que son reales.
  \end{parts}

  \question Considere el operador hermítico $H$ dado por:
  $$H = \begin{pmatrix} 3 & 1 \\ 1 & 3 \end{pmatrix}\,.$$

  \begin{parts}
    \part Encuentre los valores propios $\lambda_1$ y $\lambda_2$.
    \part Determine los vectores propios normalizados $v_1$ y $v_2$ asociados a estos valores propios.
    \part Escriba la matriz diagonal $D$ y la matriz unitaria de cambio de coordenadas $U$ (la matriz de los estados propios). Confirme la descomposición espectral: $H = U D U^\dagger$.
  \end{parts}

  \question Considere el sistema compuesto de dos espacios de Hilbert $\mathcal{H}_{AB} = \mathcal{H}_A \otimes \mathcal{H}_B$, donde $\mathcal{H}_A \cong \mathbb{C}^2$ y $\mathcal{H}_B \cong \mathbb{C}^2$.

  \begin{parts}
    \part ¿Cuál es la dimensión de $\mathcal{H}_{AB}$? Enumere todos los elementos de la base canónica tensorial.
    \part Considere el tensor $t = \frac{1}{\sqrt{2}}(e_0\otimes e_1 - e_1\otimes e_0)$. Intente encontrar dos vectores $a \in \mathcal{H}_A$ y $b \in \mathcal{H}_B$ tal que $t = a \otimes b$. ¿Es $t$ un tensor simple (estado separable)?
  \end{parts}

  \question Sea el operador $X$ (la matriz NOT cuántica o Pauli $X$) y el operador $Z$ (Pauli $Z$):
  $$ X = \begin{pmatrix} 0 & 1 \\ 1 & 0 \end{pmatrix}, \quad Z = \begin{pmatrix} 1 & 0 \\ 0 & -1 \end{pmatrix}\,. $$

  \begin{parts}
    \part Calcule el operador compuesto $C = X \otimes Z$ utilizando el producto de Kronecker. Escriba la matriz resultante en $\mathbb{C}^4$.
    \part Calcule el resultado de aplicar el operador $C$ al vector $e_1\otimes e_2$.
  \end{parts}

  \question Para el vector $v = \frac{1}{\sqrt{3}}(e_0\otimes e_0 + e_0\otimes e_1 + e_1\otimes e_1)$:
  \begin{parts}
    \part Verificar que el vector está normalizado.
    \part Determinar si el estado es separable o entrelazado.
  \end{parts}

  \question Calcule explícitamente $(e_0\otimes e_1)(X \otimes Y)$. Donde $X$ es la matriz Pauli $X$ y $Y$ es la matriz Pauli $Y$.

\end{questions}
%\tema{6}{Notación de Dirac}
%\portada

\begin{esquemaExplorador}
  \temaEsquema{Notación bra-ket}{
    \conceptoEsquema{Kets y bras}
    \conceptoEsquema{Productos internos}
    \conceptoEsquema{Operadores externos}
  }
  \temaEsquema{Estados cuánticos}{
    \conceptoEsquema{cúbits y normalización}
    \conceptoEsquema{Superposición cuántica}
    \conceptoEsquema{Interpretación probabilística}
  }
  \temaEsquema{Sistemas de múltiples cúbits}{
    \conceptoEsquema{Producto tensorial}
    \conceptoEsquema{Base computacional}
    \conceptoEsquema{Estados entrelazados}
  }
  \temaEsquema{Mediciones y observables}{
    \conceptoEsquema{Regla de Born}
    \conceptoEsquema{Colapso del estado}
    \conceptoEsquema{Medidas en diferentes bases}
  }
  \temaEsquema{Dinámicas cuánticas}{
    \conceptoEsquema{Acción de operadores}
    \conceptoEsquema{Evolución libre}
  }
\end{esquemaExplorador}

\unirsection{Ideas clave}

\subsection{Introducción y objetivos}

La notación de Dirac, también conocida como notación bra-ket, proporciona un formalismo elegante y poderoso para trabajar con estados cuánticos y operadores. Desarrollada por Paul Dirac, esta notación no solo simplifica los cálculos algebraicos, sino que también captura de manera intuitiva los conceptos físicos fundamentales de la mecánica cuántica.

En computación cuántica, la notación de Dirac es indispensable porque:

\begin{itemize}
  \item Proporciona una representación \textbf{concisa y clara} de estados cuánticos complejos
  \item Facilita el cálculo de \textbf{probabilidades cuánticas} mediante productos internos
  \item Permite expresar \textbf{operadores cuánticos} de manera natural y eficiente
  \item Conecta directamente la \textbf{estructura matemática} con la \textbf{interpretación física}
  \item Es el lenguaje estándar para describir \textbf{algoritmos cuánticos} y \textbf{protocolos cuánticos}
\end{itemize}

Este tema establece el puente definitivo entre la matemática abstracta de los espacios de Hilbert y la implementación práctica de sistemas cuánticos. Desarrollaremos desde los conceptos más básicos de la notación hasta aplicaciones avanzadas en sistemas de múltiples cúbits, proporcionando la base para la notación para todos los temas posteriores del curso.

\subsection{Fundamentos de la notación bra-ket}

\begin{defi}[Ket]
  Un \textbf{ket}, denotado $\ket{\psi}$, representa un vector unitario en un espacio de Hilbert complejo. Formalmente, $\ket{\psi} \in \H$ donde $\H$ es el espacio de estados del sistema cuántico.
\end{defi}

La base canónica $\set{e_1, e_2, \ldots, e_n}$ de $\C^n$, ahora se denota usando la notación de Dirac como los kets $\set{\ket{1}, \ket{2}, \ldots, \ket{n}}$ de $\H$. Aunque algunas veces es más conveniente usar la notación $\set{\ket{0}, \ket{1}, \ldots, \ket{n-1}}$.

En términos de coordenadas, si $\{\ket{i}\}$ es una base ortonormal:
$$\ket{\psi} = \sum_i c_i \ket{i}\so c_i = \braket{\psi}{i}$$
donde $c_i \in \C$ son las llamadas \textbf{amplitudes de probabilidad}.

\begin{defi}[Bracket]
  El \textbf{bracket} es el producto interno entre dos kets $\ket{\phi}$ y $\ket{\psi}$, se denota $\braket{\phi}{\psi}$:
  $$\braket{\phi}{\psi} = \langle \phi , \psi \rangle \in \C\,.$$
\end{defi}

Recordando las definiciones del dual de un espacio vectorial, dado un ket $\ket{\phi}$, el funcional asociado es un elemento del espacio dual $\H^*$, así, el producto interno se puede interpretar como la acción del funcional asociado.

\begin{defi}[Bra]
  Dado un ket $\ket{\phi}$, el funcional asociado del espacio dual lo llamamos \textbf{bra}, denotado $\bra{\phi}$ y que actúa sobre un ket $\ket{\psi}$ mediante la notación:
  $$\braket{\phi}{\psi} = \bra{\phi}(\ket{\psi})\,.$$
\end{defi}

Para cada ket $\ket{\psi}$, el bra correspondiente a nivel de coeficiente respecto de una base ortogonal y su base dual se cumple:
$$\bra{\psi} = \ket{\psi}^\dagger\,.$$

\begin{eje}[Notación básica en $\C^2$]
  Para un cúbit, los estados base se escriben:
  $$\ket{0} = \begin{pmatrix} 1 \\ 0 \end{pmatrix}, \quad \ket{1} = \begin{pmatrix} 0 \\ 1 \end{pmatrix}$$

  Los bras correspondientes son:
  $$\bra{0} = \begin{pmatrix} 1 & 0 \end{pmatrix}, \quad \bra{1} = \begin{pmatrix} 0 & 1 \end{pmatrix}$$

  Los productos internos fundamentales:
  \begin{align*}
    \braket{0}{0} & = 1, \quad \braket{1}{1} = 1 \\
    \braket{0}{1} & = 0, \quad \braket{1}{0} = 0
  \end{align*}
\end{eje}

\begin{defi}[Operador externo]
  El producto externo de un ket y un bra se denota $\ketbra{\psi}{\phi}$ y representa el operador lineal definido por:
  \[
    \ketbra{\psi}{\phi} (\ket{\chi}) = \braket{\phi}{\chi}\ket{\psi}
  \]
\end{defi}

Normalmente, no se escriben los paréntesis y se entiende que el operador actúa sobre un ket a su derecha.
\[
  \ketbra{\psi}{\phi} \ket{\chi} = \braket{\phi}{\chi}\ket{\psi}
\]

\begin{eje}[Operadores de proyección]
  Los operadores de proyección sobre los estados base son:
  \begin{align*}
    P_0 = \ketbra{0}{0} & = \begin{pmatrix} 1 \\ 0 \end{pmatrix} \begin{pmatrix} 1 & 0 \end{pmatrix} = \begin{pmatrix} 1 & 0 \\ 0 & 0 \end{pmatrix} \\
    P_1 = \ketbra{1}{1} & = \begin{pmatrix} 0 \\ 1 \end{pmatrix} \begin{pmatrix} 0 & 1 \end{pmatrix} = \begin{pmatrix} 0 & 0 \\ 0 & 1 \end{pmatrix}
  \end{align*}

  Se verifica que $P_0 + P_1 = I$ (relación de completitud).
\end{eje}

\begin{theo}[Relación de completitud]
  Para cualquier base ortonormal $\{\ket{i}\}$ de un espacio de Hilbert de dimensión finita:
  $$\sum_i \ketbra{i}{i} = I$$

  Esta identidad permite expresar cualquier operador o estado en términos de la base elegida.
\end{theo}

\subsection{Estados cuánticos y cúbits}

\begin{defi}[Estado cuántico puro]
  Un estado cuántico puro se representa mediante un ket normalizado:
  $$\ket{\psi} \in \H, \quad \braket{\psi}{\psi} = 1$$

  Para un cúbit, el estado general es:
  $$\ket{\psi} = \alpha \ket{0} + \beta \ket{1}$$
  donde $\alpha, \beta \in \C$ y $|\alpha|^2 + |\beta|^2 = 1$.
\end{defi}

\begin{eje}[Estados cuánticos importantes]
  \begin{enumerate}
    \item \textbf{Estados computacionales:}
          $$\ket{0} = \begin{pmatrix} 1 \\ 0 \end{pmatrix}, \quad \ket{1} = \begin{pmatrix} 0 \\ 1 \end{pmatrix}$$

    \item \textbf{Estados de superposición balanceada:}
          $$\ket{+} = \frac{\ket{0} + \ket{1}}{\sqrt{2}}, \quad \ket{-} = \frac{\ket{0} - \ket{1}}{\sqrt{2}}$$

    \item \textbf{Estados con fase compleja:}
          $$\ket{i+} = \frac{\ket{0} + i\ket{1}}{\sqrt{2}}, \quad \ket{i-} = \frac{\ket{0} - i\ket{1}}{\sqrt{2}}$$
  \end{enumerate}
\end{eje}

\begin{defi}[Interpretación probabilística - Regla de Born]
  Si un sistema se encuentra en el estado $\ket{\psi}$ y se mide un observable con vectores propios ortonormales $\{\ket{\phi_i}\}$, la probabilidad de obtener el resultado correspondiente al estado $\ket{\phi_i}$ es:
  $$P(\phi_i) = |\braket{\psi}{\phi_i}|^2$$
\end{defi}

\begin{eje}[Cálculo de probabilidades]
  Para el estado $\ket{\psi} = \frac{3}{5}\ket{0} + \frac{4i}{5}\ket{1}$:

  \textbf{Verificación de normalización:}
  $$\braket{\psi}{\psi} = \left|\frac{3}{5}\right|^2 + \left|\frac{4i}{5}\right|^2 = \frac{9}{25} + \frac{16}{25} = 1$$

  \textbf{Probabilidades en la base computacional:}
  \begin{align*}
    P(0) & = |\braket{\psi}{0}|^2 = \left|\frac{3}{5}\right|^2 = \frac{9}{25}   \\
    P(1) & = |\braket{\psi}{1}|^2 = \left|\frac{4i}{5}\right|^2 = \frac{16}{25}
  \end{align*}

  \textbf{Probabilidades en la base $\{\ket{+}, \ket{-}\}$:}
  $$\braket{\psi}{+} = \frac{1}{\sqrt{2}}\left(\frac{3}{5} + \frac{4i}{5}\right) = \frac{3 + 4i}{5\sqrt{2}}$$
  $$P(+) = \left|\frac{3 + 4i}{5\sqrt{2}}\right|^2 = \frac{9 + 16}{50} = \frac{1}{2}$$
\end{eje}

\begin{defi}[Colapso cuántico]
  Después de una medición que da como resultado el estado $\ket{\phi_i}$, el sistema colapsa al estado:
  $$\ket{\psi'} = \frac{P_i \ket{\psi}}{\|P_i \ket{\psi}\|} = \frac{\braket{\psi}{\phi_i}}{|\braket{\psi}{\phi_i}|} \ket{\phi_i}$$
  donde $P_i = \ketbra{\phi_i}{\phi_i}$ es el proyector sobre $\ket{\phi_i}$.
\end{defi}

\subsection{Operadores en notación de Dirac}

\begin{defi}[Valor esperado]
  El valor esperado de un operador $A$ en el estado $\ket{\psi}$ es:
  $$\langle A \rangle = \bra{\psi}A\ket{\psi}$$
\end{defi}

\begin{eje}[Matrices de Pauli en notación de Dirac]
  Las matrices de Pauli se pueden expresar como:
  \begin{align*}
    \sigma_x & = \ketbra{0}{1} + \ketbra{1}{0}    \\
    \sigma_y & = -i\ketbra{0}{1} + i\ketbra{1}{0} \\
    \sigma_z & = \ketbra{0}{0} - \ketbra{1}{1}
  \end{align*}

  \textbf{Verificación para $\sigma_x$:}
  \begin{align*}
    \sigma_x\ket{0} & = (\ketbra{0}{1} + \ketbra{1}{0})\ket{0} = \ket{0}\braket{1}{0} + \ket{1}\braket{0}{0} = \ket{1} \\
    \sigma_x\ket{1} & = (\ketbra{0}{1} + \ketbra{1}{0})\ket{1} = \ket{0}\braket{1}{1} + \ket{1}\braket{0}{1} = \ket{0}
  \end{align*}
\end{eje}

\begin{eje}[Valor esperado de $\sigma_z$]
  Para el estado $\ket{\psi} = \alpha\ket{0} + \beta\ket{1}$:
  \begin{align*}
    \langle \sigma_z \rangle & = \bra{\psi}\sigma_z\ket{\psi}                                                                              \\
                             & = (\conj{\alpha}\bra{0} + \conj{\beta}\bra{1})(\ketbra{0}{0} - \ketbra{1}{1})(\alpha\ket{0} + \beta\ket{1}) \\
                             & = |\alpha|^2 - |\beta|^2
  \end{align*}

  Este resultado tiene una interpretación física clara: es la diferencia entre las probabilidades de medir $+1$ y $-1$.
\end{eje}

\subsection{Sistemas de múltiples cúbits}

\begin{defi}[Producto tensorial de estados]
  Para sistemas compuestos, los estados se forman mediante el producto tensorial:
  $$\ket{\psi} \tensor \ket{\phi} = \ket{\psi\phi}$$

  Para $n$ cúbits, el espacio de estados es $(\C^2)^{\tensor n} \cong \C^{2^n}$.
\end{defi}

\begin{defi}[Base computacional para múltiples cúbits]
  Para $n$ cúbits, la base computacional está formada por:
  $$\{\ket{b_1 b_2 \cdots b_n} : b_i \in \{0,1\}\}\,.$$
\end{defi}

\begin{eje}[Sistema de dos cúbits]
  La base computacional para dos cúbits es:
  \begin{align*}
    \ket{00} = \ket{0} \tensor \ket{0} & = \begin{pmatrix} 1 \\ 0 \\ 0 \\ 0 \end{pmatrix} \\
    \ket{01} = \ket{0} \tensor \ket{1} & = \begin{pmatrix} 0 \\ 1 \\ 0 \\ 0 \end{pmatrix} \\
    \ket{10} = \ket{1} \tensor \ket{0} & = \begin{pmatrix} 0 \\ 0 \\ 1 \\ 0 \end{pmatrix} \\
    \ket{11} = \ket{1} \tensor \ket{1} & = \begin{pmatrix} 0 \\ 0 \\ 0 \\ 1 \end{pmatrix}
  \end{align*}

  Un estado general de dos cúbits se escribe:
  $$\ket{\psi} = \alpha_{00}\ket{00} + \alpha_{01}\ket{01} + \alpha_{10}\ket{10} + \alpha_{11}\ket{11}$$
  con $\sum_{ij} |\alpha_{ij}|^2 = 1$.
\end{eje}

\begin{eje}[Estados de Bell]
  Los cuatro estados de Bell forman una base ortonormal de estados entrelazados para dos cúbits:
  \begin{align*}
    \ket{\Phi^+} & = \frac{\ket{00} + \ket{11}}{\sqrt{2}} \\
    \ket{\Phi^-} & = \frac{\ket{00} - \ket{11}}{\sqrt{2}} \\
    \ket{\Psi^+} & = \frac{\ket{01} + \ket{10}}{\sqrt{2}} \\
    \ket{\Psi^-} & = \frac{\ket{01} - \ket{10}}{\sqrt{2}}
  \end{align*}
\end{eje}

\subsection{Mediciones y observables}

\begin{defi}[Observable en notación de Dirac]
  Un observable es un operador hermitiano $\hat{O}$ que puede escribirse en su descomposición espectral:
  $$\hat{O} = \sum_i \lambda_i \ketbra{\phi_i}{\phi_i}$$
  donde $\lambda_i \in \R$ son los valores propios y $\{\ket{\phi_i}\}$ es una base ortonormal de vectores propios.
\end{defi}

\begin{eje}[Medición de $\sigma_z$ en diferentes bases]
  \textbf{Base computacional:}
  $$\sigma_z = (+1)\ketbra{0}{0} + (-1)\ketbra{1}{1}$$

  Para el estado $\ket{\psi} = \alpha\ket{0} + \beta\ket{1}$:
  - $P(+1) = |\braket{0}{\psi}|^2 = |\alpha|^2$
  - $P(-1) = |\braket{1}{\psi}|^2 = |\beta|^2$

  \textbf{Base de $\sigma_x$ (base $\{\ket{+}, \ket{-}\}$):}
  $$\sigma_x = (+1)\ketbra{+}{+} + (-1)\ketbra{-}{-}$$

  Para el mismo estado:
  \begin{align*}
    \braket{+}{\psi} & = \frac{\alpha + \beta}{\sqrt{2}} \Rightarrow P(+1) = \frac{|\alpha + \beta|^2}{2} \\
    \braket{-}{\psi} & = \frac{\alpha - \beta}{\sqrt{2}} \Rightarrow P(-1) = \frac{|\alpha - \beta|^2}{2}
  \end{align*}
\end{eje}

\subsection{Dinámicas cuánticas en notación de Dirac}

\begin{defi}[Evolución unitaria]
  La evolución temporal de un sistema cuántico cerrado se describe mediante un operador unitario $U(t)$:
  $$\ket{\psi(t)} = U(t)\ket{\psi(0)}$$

  Para hamiltonianos independientes del tiempo:
  $$U(t) = e^{-i\hat{H}t/\hbar}$$
  donde $\hat{H}$ es el hamiltoniano del sistema.
\end{defi}

\begin{eje}[Evolución libre de un cúbit]
  Para un cúbit con hamiltoniano $\hat{H} = \frac{\omega}{2}\sigma_z$:
  $$U(t) = e^{-i\omega t\sigma_z/2} = \cos\frac{\omega t}{2}I - i\sin\frac{\omega t}{2}\sigma_z$$

  $$U(t) = \begin{pmatrix} e^{-i\omega t/2} & 0 \\ 0 & e^{i\omega t/2} \end{pmatrix}$$

  Si el estado inicial es $\ket{\psi(0)} = \alpha\ket{0} + \beta\ket{1}$:
  $$\ket{\psi(t)} = \alpha e^{-i\omega t/2}\ket{0} + \beta e^{i\omega t/2}\ket{1}$$

  Las probabilidades $|\alpha|^2$ y $|\beta|^2$ se mantienen constantes, pero las fases evolucionan.
\end{eje}

\begin{info}
  El Postulado V garantiza que la información cuántica se conserva durante la evolución temporal de sistemas cerrados. Esto es fundamental para el diseño de algoritmos cuánticos, donde las operaciones se implementan mediante secuencias de operadores unitarios.
\end{info}
%\unirsection{Problemas}

\begin{questions}

  \question Expresar los siguientes estados en notación de Dirac y verificar su normalización:
  \begin{parts}
    \part $(\frac{1}{\sqrt{3}}, \frac{\sqrt{2}}{\sqrt{3}})^t$.
    \part $(\frac{1+i}{2}, \frac{1-i}{2})^t$.
    \part $\frac{1}{2}(1, 1, 1, 1)^t$.
  \end{parts}

  \question Para el estado $\ket{\psi} = \frac{1}{\sqrt{5}}(2\ket{0} + i\ket{1})$:
  \begin{parts}
    \part Calcular las probabilidades de medir $\ket{0}$ y $\ket{1}$.
    \part Calcular las probabilidades de medir $\ket{+}$ y $\ket{-}$.
    \part Determinar el estado después de medir $\ket{+}$.
  \end{parts}

  \question Verificar las siguientes identidades usando notación de Dirac:
  \begin{parts}
    \part $X = \ketbra{0}{1} + \ketbra{1}{0}$.
    \part $Y = -i\ketbra{0}{1} + i\ketbra{1}{0}$.
    \part $Z = \ketbra{0}{0} - \ketbra{1}{1}$.
    \part $I = \ketbra{0}{0} + \ketbra{1}{1}$.
  \end{parts}

  \question Calcular los valores esperados $\expval{X}{\psi}$, $\expval{Y}{\psi}$ y $\expval{Z}{\psi}$ para:
  \begin{parts}
    \part $\ket{\psi} = \ket{0}$.
    \part $\ket{\psi} = \ket{+}$.
    \part $\ket{\psi} = \frac{\ket{0} + i\ket{1}}{\sqrt{2}}$.
  \end{parts}

  \question Para el sistema de dos cúbits en el estado $\ket{\psi} = \frac{1}{\sqrt{3}}(\ket{00} + \ket{01} + \ket{11})$:
  \begin{parts}
    \part Verificar que el estado está normalizado.
    \part Calcular las probabilidades de medir cada estado de la base computacional.
    \part Determinar si el estado es separable o entrelazado.
    \part Calcular la probabilidad de medir el primer cúbit en $\ket{0}$.
  \end{parts}

  \question Para los cuatro estados de Bell:
  \begin{parts}
    \part Verificar que forman una base ortonormal.
    \part Expresar cada estado de Bell como combinación lineal de la base computacional.
    \part Demostrar que todos son maximalmente entrelazados.
    \part Calcular las probabilidades de medir cada cúbit individualmente.
  \end{parts}

  \question Verificar que $\ket{\Phi^+}$ está entrelazado.
  \begin{solution}

    Si $\ket{\Phi^+}$ fuera separable, existirían $\ket{\alpha} = a\ket{0} + b\ket{1}$ y $\ket{\beta} = c\ket{0} + d\ket{1}$ tales que:
    $$\ket{\Phi^+} = (a\ket{0} + b\ket{1}) \otimes (c\ket{0} + d\ket{1}) = ac\ket{0}\otimes\ket{0} + ad\ket{0}\otimes\ket{1} + bc\ket{1}\otimes\ket{0} + bd\ket{1}\otimes\ket{1}$$

    Comparando con $\ket{\Phi^+} = \frac{1}{\sqrt{2}}(\ket{0} \otimes \ket{0} + \ket{1} \otimes \ket{1})$:
    $$ac = \frac{1}{\sqrt{2}}, \quad ad = 0, \quad bc = 0, \quad bd = \frac{1}{\sqrt{2}}$$

    De $ad = 0$ y $bc = 0$, o bien $a = c = 0$ o bien $b = d = 0$, pero esto contradice $ac = bd = \frac{1}{\sqrt{2}} \neq 0$.

    Por tanto, $\ket{\Phi^+}$ es entrelazado.
  \end{solution}

  \question Determine si el estado $\ket{\psi} = \frac{1}{\sqrt{3}}(\ket{00} + \ket{01} + \ket{10})$ es entrelazado.


  \question Para las matrices de Pauli $X$ y $Y$, calcule explícitamente $(X \otimes Y)\ket{01}$.

  \question Demuestre que la puerta CNOT puede crear entrelazamiento aplicándola a estados separables apropiados. Proporcione al menos dos ejemplos específicos.

  \question Para el estado $\ket{\psi} = \cos\frac{\pi}{8}\ket{0} + e^{i\pi/4}\sin\frac{\pi}{8}\ket{1}$:
  \begin{parts}
    \part Calcular las probabilidades de medir $Z$ en los valores $\pm 1$.
    \part Calcular las probabilidades de medir $X$ en los valores $\pm 1$.
    \part Si se mide primero $Z$ y se obtiene $+1$, ¿cuáles son las probabilidades para una medición posterior de $X$?
  \end{parts}

  \question Un cúbit evoluciona bajo el hamiltoniano $\hat{H} = \frac{\pi}{4}Y$:
  \begin{parts}
    \part Calcular el operador de evolución $U(t) = e^{-i\hat{H}t}$.
    \part Si el estado inicial es $\ket{0}$, determinar $\ket{\psi(t)}$.
    \part ¿En qué instante $t$ el estado se convierte en $\ket{1}$?
  \end{parts}

  \question Para la rotación $R_z(\theta) = e^{-i\theta Z/2}$:
  \begin{parts}
    \part Escribir la forma matricial explícita de $R_z(\theta)$.
    \part Demostrar que $R_z(\theta) = \cos\frac{\theta}{2}I - i\sin\frac{\theta}{2}Z$.
    \part Aplicar $R_z(\pi/2)$ al estado $\ket{+}$ y expresar el resultado.
  \end{parts}

  \question Demostrar que la composición de rotaciones alrededor del mismo eje se suma: $$R_z(\alpha)R_z(\beta) = R_z(\alpha + \beta)\,.$$

  \question Dadas las amplitudes cuánticas $\alpha = \frac{2}{3}$ y $\beta = \frac{\sqrt{5}i}{3}$:
  \begin{parts}
    \part Verificar que $|\alpha|^2 + |\beta|^2 = 1$.
    \part Expresar cada amplitud en forma $re^{i\theta}$.
    \part Calcular las probabilidades asociadas a cada amplitud.
  \end{parts}

  \question Un sistema cuántico tiene amplitudes $\alpha_1 = \frac{1}{\sqrt{2}}e^{i\pi/4}$ y $\alpha_2 = \frac{1}{\sqrt{2}}e^{i3\pi/4}$:
  \begin{parts}
    \part Calcular la amplitud total $\alpha_1 + \alpha_2$.
    \part Determinar si hay interferencia constructiva, destructiva o parcial.
    \part Calcular la probabilidad total $|\alpha_1 + \alpha_2|^2$.
  \end{parts}

  \question Tres amplitudes cuánticas tienen la misma magnitud $\frac{1}{\sqrt{3}}$ pero fases diferentes: $0$, $\frac{3\pi}{4}$ y $\frac{3\pi}{2}$:
  \begin{parts}
    \part Escribir las tres amplitudes en forma binomial.
    \part Calcular la suma total de las tres amplitudes expresada en forma binomial.
    \part Explicar por qué el resultado tiene sentido físicamente.
  \end{parts}

  \begin{solution}
    \begin{parts}
      \part
      \begin{itemize}
        \item $\frac{1}{\sqrt{3}} \cdot e^{i \cdot 0} = \frac{1}{\sqrt{3}}$.
        \item $\frac{1}{\sqrt{3}} \cdot e^{i \cdot \frac{3\pi}{4}} = \frac{1}{\sqrt{3}} \left(\cos\left(\frac{3\pi}{4}\right) + i\sin\left(\frac{3\pi}{4}\right)\right) = \frac{1}{\sqrt{3}} \left(-\frac{\sqrt{2}}{2} + i\frac{\sqrt{2}}{2}\right)$.
        \item $\frac{1}{\sqrt{3}} \cdot e^{i \cdot \frac{3\pi}{2}} = \frac{1}{\sqrt{3}} \left(\cos\left(\frac{3\pi}{2}\right) + i\sin\left(\frac{3\pi}{2}\right)\right) = \frac{-i}{\sqrt{3}}$.
      \end{itemize}

      \part
      \begin{align*}
        \frac{1}{\sqrt{3}} \left(1 + \left(-\frac{\sqrt{2}}{2} - i\frac{\sqrt{2}}{2}\right) - i\right) = \frac{1}{\sqrt{3}} \left(\left(1 - \frac{\sqrt{2}}{2}\right) - \left(1 + \frac{\sqrt{2}}{2}\right) i\right)
      \end{align*}

      \part La suma de las tres amplitudes resulta en una amplitud con magnitud menor que la de cada una de las individuales, lo que indica interferencia destructiva parcial. Esto tiene sentido físicamente porque las fases diferentes causan que las ondas asociadas a las amplitudes se cancelen parcialmente entre sí.
    \end{parts}

  \end{solution}

  \question Para las amplitudes $\alpha = ae^{i\phi}$ y $\beta = be^{i\psi}$ con $a, b \in \mathbb{R}^+$:
  \begin{parts}
    \part Demostrar que $|\alpha + \beta|^2 = a^2 + b^2 + 2ab\cos(\psi - \phi)$.
    \part ¿Para qué diferencia de fases se obtiene interferencia máxima?
    \part ¿Para qué diferencia de fases se obtiene interferencia mínima?
  \end{parts}

  \question Dos amplitudes cuánticas $\alpha_1$ y $\alpha_2$ tienen magnitudes $\frac{1}{\sqrt{5}}$ y $\frac{2}{\sqrt{5}}$ respectivamente:
  \begin{parts}
    \part Si sus fases son $\theta_1 = 0$ y $\theta_2 = \pi/3$, calcular $|\alpha_1 + \alpha_2|^2$.
    \part Encontrar el valor de $\theta_2$ que maximiza $|\alpha_1 + \alpha_2|^2$.
    \part Encontrar el valor de $\theta_2$ que minimiza $|\alpha_1 + \alpha_2|^2$.
  \end{parts}

  \question Calcular la magnitud y fase de los siguientes números complejos:
  \begin{parts}
    \part $z_1 = 3 + 4i$.
    \part $z_2 = -2 + 2i\sqrt{3}$.
    \part $z_3 = -5i$.
    \part $z_4 = 7$.
  \end{parts}

\end{questions}

%\bloque{3}{Operadores y dinámicas}
%\tema{7}{Operadores lineales en computación cuántica}
%\portada

\begin{esquemaExplorador}
  \temaEsquema{Operadores en $\mathcal{C}$}{
    \conceptoEsquema{Matriz adjunta y operadores especiales}{}
    \conceptoEsquema{Operadores hermitianos y unitarios}{}
    \conceptoEsquema{Descomposición espectral}{}
  }
  \temaEsquema{Puertas Cuánticas de Un cúbit}{
    \conceptoEsquema{Matrices de Pauli: X, Y, Z}{}
    \conceptoEsquema{Puerta de Hadamard}{}
    \conceptoEsquema{Puertas de fase: S, T, rotaciones}{}
  }
  \temaEsquema{Geometría en la Esfera de Bloch}{
    \conceptoEsquema{Rotaciones como operadores unitarios}{}
    \conceptoEsquema{Visualización geométrica}{}
    \conceptoEsquema{Generación de puertas arbitrarias}{}
  }
  \temaEsquema{Operadores en $\mathcal{C}^n$}{
    \conceptoEsquema{Sistemas de múltiples cúbits}{}
    \conceptoEsquema{Operadores producto tensorial}{}
    \conceptoEsquema{Operadores no separables}{}
  }
  \temaEsquema{Puertas Cuánticas Controladas}{
    \conceptoEsquema{CNOT y puertas controladas generales}{}
    \conceptoEsquema{Puertas de múltiples controles}{}
    \conceptoEsquema{Universalidad cuántica}{}
  }
  \temaEsquema{Evolución Temporal y Hamiltonianos}{
    \conceptoEsquema{Operadores hermitianos como generadores}{}
    \conceptoEsquema{Matrices exponenciales}{}
    \conceptoEsquema{Simulación de evolución cuántica}{}
  }
\end{esquemaExplorador}

\unirsection{Ideas clave}

\subsection{Introducción y objetivos}

Habiendo establecido la teoría general de operadores lineales en el Tema 3, ahora nos enfocamos en los operadores específicos en $\mathcal{C}$ y $\mathcal{C}^n$ que forman el corazón de la computación cuántica. Estos operadores, cuando satisfacen la condición de unitariedad, se conocen como \textbf{puertas cuánticas} y constituyen los bloques de construcción fundamentales de los algoritmos cuánticos.

Los objetivos específicos de este tema son:

\begin{itemize}
  \item Caracterizar completamente los operadores unitarios y hermitianos en $\mathcal{C}$.
  \item Estudiar las puertas cuánticas fundamentales y su interpretación geométrica.
  \item Analizar cómo los operadores producto tensorial describen sistemas multicúbit.
  \item Desarrollar puertas controladas y entender su implementación.
  \item Conectar operadores hermitianos con la evolución temporal cuántica.
\end{itemize}

\subsection{Operadores lineales en $\mathcal{C}$}

Empecemos con el espacio de un solo cúbit, $\mathcal{C}$. Aquí, los operadores se representan como matrices $2 \times 2$, y su estudio es fundamental para entender la computación cuántica. La dimensión del espacio de matrices $2 \times 2$ es 4, lo que implica que éste es el número de matrices linealmente independientes que cualquier base va a tener.

\begin{eje}
  Las matrices de Pauli $X, Y, Z$ junto con la identidad $I$ es una base.

  Al coincidir el número de matrices del conjunto con la dimensión del espacio, podemos estudiar solo una condición de base, ser linealmente independiente o generar el espacio.

  Veamos que son linealmente independientes. Si existen $a, b, c, d \in \mathcal{C}$ tales que $aI + bX + cY + dZ = 0$, entonces
  \begin{align*}
    0 & = aI + bX + cY + dZ = a\begin{pmatrix} 1 & 0 \\ 0 & 1 \end{pmatrix}+b \begin{pmatrix} 0 & 1 \\ 1 & 0 \end{pmatrix}    +c \begin{pmatrix} 0 & -i \\ i & 0 \end{pmatrix}+d \begin{pmatrix} 1 & 0 \\ 0 & -1 \end{pmatrix} \\
      & = \begin{pmatrix} a+d & b - ic \\ b + ic & a - d \end{pmatrix}\Rightarrow
    \begin{cases}
      a + d = 0  \\
      b - ic = 0 \\
      b + ic = 0 \\
      a - d = 0
    \end{cases} \Rightarrow a = b = c = d = 0\,.
  \end{align*}
\end{eje}

Cualquier matriz hermítica $A = \begin{pmatrix} a & b \\ \overline{b} & c \end{pmatrix}$ con $a, c \in \mathbb{R}$ y $b \in \mathcal{C}$ puede expresarse en esta base de la siguiente manera
$$A = \frac{a+c}{2}I + \Rp(b)X - \Ip(b)Y+ \frac{a-c}{2}Z\,. $$

El postulado 3 de la teoría cuántica establece que la ecuación de Schrödinger para la evolución temporal de un cúbit debe ser un operador unitario.

En cuanto a los operadores unitarios, la condición $U^\dagger U = I$ impone restricciones no lineales sobre los coeficientes de la combinación lineal. Por lo tanto, el conjunto de operadores unitarios no forma un espacio vectorial, sino un grupo bajo la multiplicación de matrices.

\begin{defi}
  Llamaremos \textbf{puertas cuánticas} a los operadores unitarios en $\mathcal{C}^n$.
\end{defi}

\begin{theo}[Parametrización de puertas cuánticas de un cúbit]
  Toda puerta cuántica en $\mathcal{C}$ puede escribirse como
  $$U = \begin{pmatrix} e^{i\alpha} & 0 \\ 0 & e^{i\beta} \end{pmatrix} \begin{pmatrix} \cos\frac{\theta}{2} & \sin\frac{\theta}{2} \\ -\sin\frac{\theta}{2} & \cos\frac{\theta}{2} \end{pmatrix} \begin{pmatrix} e^{-i\alpha} & 0 \\ 0 & e^{-i\beta} \end{pmatrix}\,,$$
  donde $\alpha, \beta, \theta \in \mathbb{R}$.
\end{theo}

\begin{info}
  Debería notarse que esta parametrización no es compatible con el grupo de operadores unitarios $2 \times 2$ isomorfo a $U(2)$ y con dimensión real 4, como cabría esperar. En computación cuántica, típicamente ignoramos fases globales, y es por ello que en realidad estamos trabajando con el grupo $SU(2)/\mathbb{Z}_2 \cong SO(3)$.
\end{info}

\subsection{La esfera de Bloch}
La esfera de Bloch es una representación geométrica de los estados de un cúbit. Cada punto en la superficie de la esfera corresponde a un estado del cúbit.

Todo ket $\ket{a}$ de un cúbit puede ser expresado a partir de dos valores reales, $\theta$ y $\phi$, como:
\[
  \ket{a}=\cos\left(\frac{\theta}{2}\right)\ket{0}+e^{i\phi}\sin\left(\frac{\theta}{2}\right)\ket{1}\ (\theta,\phi)\in [0, \pi]\times[0, 2\pi]
\]

\begin{figure}[H]
  \centering
  \begin{tikzpicture}
    [line cap=round, line join=round]
    \clip (-2.19,-2.49) rectangle (2.66,2.58);
    \draw [shift={(0,0)}, fill, fill opacity=0.1] (0,0) -- (56.7:0.4) arc (56.7:90.:0.4) -- cycle;
    \draw [shift={(0,0)}, fill, fill opacity=0.1] (0,0) -- (-135.7:0.4) arc (-135.7:-33.2:0.4) -- cycle;
    \draw (0,0) circle (2cm);
    \draw [rotate around={0.:(0.,0.)},dash pattern=on 3pt off 3pt] (0,0) ellipse (2cm and 0.9cm);
    \draw (0,0)-- (0.70,1.07);
    \draw (0,0) -- (0,2);
    \draw (0,0) -- (-0.81,-0.79);
    \draw (0,0) -- (2,0);
    \draw [dotted] (0.7,1) -- (0.7,-0.46);
    \draw [dotted] (0,0) -- (0.7,-0.46);
    \draw (-0.08,-0.3) node[anchor=north west] {$\phi$};
    \draw (0.01,0.9) node[anchor=north west] {$\theta$};
    \draw (0.4,1.65) node[anchor=north west] {$\ket{a}$};
    \scriptsize
    \draw [fill] (0,0) circle (1.5pt);
    \draw [fill] (0.7,1.1) circle (0.5pt);
  \end{tikzpicture}
  \caption{Representación de un ket $\ket{a}\in\H$ en la esfera de Bloch}
\end{figure}

Por convenio, se dibuja la esfera sobre los tres ejes espaciales X, Y, Z de tal modo que el ecuador quede en el plano XY, igualmente se considera el eje Z perpendicular a dicho plano.
Los kets de la base canónica computacional se sitúan sobre el eje Z, estando el ket $\ket{0}$ en el eje positivo y el ket $\ket{1}$ sobre el eje negativo.

\begin{figure}[H]
  \centering
  \begin{tikzpicture}
    [line cap=round, line join=round]
    \draw (0,0) circle (2cm);
    \draw [rotate around={0.:(0.,0.)},dash pattern=on 3pt off 3pt] (0,0) ellipse (2cm and 0.9cm);
    \draw (0,0) -- (0,2);
    \draw (0,0) -- (0,-2);
    \draw (0,0) -- (2,0);
    \draw (0,0) -- (-2,0);
    \draw (0,0) -- (-0.81,-0.79);
    \draw (0,0) -- (0.81,0.79);
    \draw (-0.3,3) node[anchor=north west] {Z};
    \draw (-0.3,2.6) node[anchor=north west] {$\ket{0}$};
    \draw (-0.3,-2) node[anchor=north west] {$\ket{1}$};
    \draw (2,0.3) node[anchor=north west] {Y};
    \draw (-1.2,-0.8) node[anchor=north west] {X};
  \end{tikzpicture}
  \caption{Ejes espaciales y representación de la base canónica computacional en la esfera de Bloch}
\end{figure}

\begin{eje}
  Así por ejemplo, el estado $\ket{+} = \frac{1}{\sqrt{2}}(\ket{0} + \ket{1})$ corresponde al punto en el ecuador con $\phi=0$.

  Mientras que $\ket{-} = \frac{1}{\sqrt{2}}(\ket{0} - \ket{1})$ corresponde al punto en el ecuador con $\phi=\pi$.
\end{eje}

\begin{info}
  La esfera de Bloch proporciona una visualización intuitiva de las operaciones cuánticas como rotaciones en el espacio tridimensional. Cada operador unitario en $\mathcal{C}$ corresponde a una rotación de la esfera.
\end{info}

\subsection{Puertas cuánticas fundamentales de un cúbit}

Las matrices de Pauli son las puertas cuánticas más fundamentales y corresponden con rotaciones de $\pi/2$ alrededor de los ejes de la esfera de Bloch. Como puertas dentro del contexto de la computación cuántica, se suelen denotar por las letras $X, Y, Z$, aunque también reciben otros nombres.

\begin{defi}[Puertas de Pauli]
  \begin{align*}
    X & = \begin{pmatrix} 0 & 1 \\ 1 & 0 \end{pmatrix}  & \text{(puerta NOT)}     \\
    Y & = \begin{pmatrix} 0 & -i \\ i & 0 \end{pmatrix}                           \\
    Z & = \begin{pmatrix} 1 & 0 \\ 0 & -1 \end{pmatrix} & \text{(puerta de fase)}
  \end{align*}
\end{defi}

La puerta $X$ a veces se representa como $\oplus$ porque esencialmente es un sumador binario. Para ello solo tenemos que ver como actúa sobre los estados de la base computacional
\begin{align*}
  \ket{0}X & = \ket{1} \quad \text{que es } \ket{0\oplus 1}    \\
  \ket{1}X & = \ket{0} \quad \text{que es } \ket{1\oplus 1}\,,
\end{align*}
donde $\oplus$ es la operación de \textbf{suma binaria}.
Por este motivo, podemos expresar la acción de $X$ sobre la base computacional como
\[
  \ket{a}X = \ket{a\oplus 1}\,,
\]
para $a\in\{0,1\}$.

Para la puerta $Z$ podemos representar su acción sobre la base computacional como
\[
  \ket{a}Z = (-1)^a \ket{a}\,,
\]
para $a\in\{0,1\}$.

\begin{prop}
  La representación de las puertas de Pauli en términos del producto exterior es:
  \begin{enumerate}
    \item $X = \ketbra{0}{1} + \ketbra{1}{0}$.
    \item $Y = i\ketbra{0}{1} - i\ketbra{1}{0}$.
    \item $Z = \ketbra{0}{0} - \ketbra{1}{1}$.
  \end{enumerate}
\end{prop}

Otra puerta comúmente usada es la puerta de Hadamard, que es una puerta cuántica que crea superposiciones equiprobables.

\begin{defi}[Puerta de Hadamard]
  $$H = \frac{1}{\sqrt{2}}\begin{pmatrix} 1 & 1 \\ 1 & -1 \end{pmatrix}$$
\end{defi}

La puerta de Hadamard es fundamental para crear superposiciones, y su acción sobre los estados de la base computacional es:
\begin{itemize}
  \item $\ket{0}H = \frac{1}{\sqrt{2}}(\ket{0} + \ket{1}) = \ket{+}$.
  \item $\ket{1}H = \frac{1}{\sqrt{2}}(\ket{0} - \ket{1}) = \ket{-}$.
\end{itemize}

\begin{defi}[Base de Hadamard]
  Llamaremos base de Hadamard a la base ortonormal $\{\ket{+}, \ket{-}\}$.
\end{defi}
Las siguientes identidades conectan las puertas de Pauli y la puerta de Hadamard:
\begin{align*}
  HXH & = Z  \\
  HZH & = X  \\
  HYH & = -Y
\end{align*}

Para la puerta $H$ podemos representar su acción sobre la base computacional como
\[
  \ket{a}H = \frac{1}{\sqrt{2}}(\ket{0} + (-1)^a \ket{1})\,,
\]
para $a\in\{0,1\}$.

\begin{prop}
  La representación de la puerta de Hadamard en términos del producto exterior es:
  \begin{align*}
    H & = \frac{1}{\sqrt{2}}(\ketbra{0}{0} + \ketbra{1}{1})
  \end{align*}
\end{prop}

Otras puertas más genéricas son las puertas de fase.

\begin{defi}[Puertas de fase]
  \begin{align*}
    R_\phi & = \begin{pmatrix} 1 & 0 \\ 0 & e^{i\phi} \end{pmatrix}  & \text{(puerta de fase general)} \\
    S      & = \begin{pmatrix} 1 & 0 \\ 0 & i \end{pmatrix}          & R_{\frac{\pi}{2}}               \\
    T      & = \begin{pmatrix} 1 & 0 \\ 0 & e^{i\pi/4} \end{pmatrix} & R_{\frac{\pi}{4}}
  \end{align*}
\end{defi}

Las puertas de fase introducen una fase relativa entre los estados $\ket{0}$ y $\ket{1}$, y son cruciales para muchos algoritmos cuánticos. Estas puertas se relacionan entre sí mediante las siguientes identidades:
\begin{itemize}
  \item $S^2 = Z$
  \item $T^2 = S$
  \item $T^4 = Z$
  \item $T^8 = I$
\end{itemize}

\begin{prop}
  La representación de las puertas de fase en términos del producto exterior es:
  \begin{enumerate}
    \item $R_\phi = \ketbra{0}{0} + e^{i\phi}\ketbra{1}{1}$.
    \item $S = \ketbra{0}{0} + i\ketbra{1}{1}$.
    \item $T = \ketbra{0}{0} + e^{i\pi/4}\ketbra{1}{1}$.
  \end{enumerate}
\end{prop}


\subsection{Rotaciones en la Esfera de Bloch}

Toda rotación en la esfera de Bloch corresponde a un operador unitario en $\mathcal{C}$:

\begin{defi}[Rotaciones de Pauli]
  Las rotaciones alrededor de los ejes de Pauli son:
  \begin{align*}
    R_x(\theta) & = e^{-i\theta X/2} = \cos\frac{\theta}{2}I - i\sin\frac{\theta}{2}X = \begin{pmatrix} \cos\frac{\theta}{2} & -i\sin\frac{\theta}{2} \\ -i\sin\frac{\theta}{2} & \cos\frac{\theta}{2} \end{pmatrix} \\
    R_y(\theta) & = e^{-i\theta Y/2} = \cos\frac{\theta}{2}I - i\sin\frac{\theta}{2}Y = \begin{pmatrix} \cos\frac{\theta}{2} & -\sin\frac{\theta}{2} \\ \sin\frac{\theta}{2} & \cos\frac{\theta}{2} \end{pmatrix}    \\
    R_z(\theta) & = e^{-i\theta Z/2} = \cos\frac{\theta}{2}I - i\sin\frac{\theta}{2}Z = \begin{pmatrix} e^{-i\theta/2} & 0 \\ 0 & e^{i\theta/2} \end{pmatrix}
  \end{align*}
\end{defi}

\begin{theo}[Descomposición universal para $SU(2)$]
  Toda matriz unitaria $U \in SU(2)$ puede descomponerse como:
  $$U = e^{i\alpha}R_z(\beta)R_y(\gamma)R_z(\delta)$$
  para algunos ángulos reales $\alpha, \beta, \gamma, \delta$.
\end{theo}

\begin{eje}[Implementación de rotación arbitraria]
  Para implementar una rotación alrededor del vector $\hat{n} = (n_x, n_y, n_z)$ por ángulo $\theta$:
  $$R_{\hat{n}}(\theta) = \cos\frac{\theta}{2}I - i\sin\frac{\theta}{2}(n_xX + n_yY + n_zZ)$$
\end{eje}

\subsection{Operadores en $\mathcal{C}^n$: Sistemas multicúbit}

Recordemos que el espacio de estados de un sistema de $n$ cúbits es el producto tensorial de $n$ copias de $\mathcal{C}$ y tiene dimensión $2^n$, y por construcción de la base computacional, la base canónica de $\mathcal{C}^n$ es
$$\{\ket{b_1 b_2 \ldots b_n} : b_i \in \{0,1\}\}$$

\begin{eje}[Sistema de 2 cúbits]
  Para $n = 2$, tenemos $\mathcal{C}^2$ con base $\{\ket{00}, \ket{01}, \ket{10}, \ket{11}\}$.

  Algunos ejemplo de como actuan las puertas construidas como tensoriales:
  \begin{align*}
    \ket{01}(X \otimes I) & = \ket{0}X \otimes \ket{1}I = \ket{1} \otimes \ket{1} = \ket{11}    \\
    \ket{10}(I \otimes Z) & = \ket{1}I \otimes \ket{0}Z = \ket{1} \otimes \ket{0} = \ket{10}\,.
  \end{align*}
\end{eje}

Sin embargo, no todas las puertas sobre dos cúbits se puede poner como un producto tensorial de dos puertas de un cúbit. Por ejemplo, la puerta CNOT.

\begin{defi}[Puerta CNOT]
  La puerta Controlled-NOT actúa sobre dos cúbits, y se define por
  $$\text{CNOT} = \begin{pmatrix} 1 & 0 & 0 & 0 \\ 0 & 1 & 0 & 0 \\ 0 & 0 & 0 & 1 \\ 0 & 0 & 1 & 0 \end{pmatrix}\,.$$
\end{defi}
La puerta CNOT puede expresarse como suma de productos tensoriales de puertas de un cúbit
$$\text{CNOT} = \ket{0}\bra{0} \otimes I + \ket{1}\bra{1} \otimes X\,.$$
Y como productos externos también se puede expresar como
$$\text{CNOT} = \ketbra{00}{00} + \ketbra{11}{11} + \ketbra{01}{10} + \ketbra{10}{01}\,.$$

\begin{eje}
  \begin{align*}
    \ket{00}\text{CNOT} & = \ket{00} \\
    \ket{01}\text{CNOT} & = \ket{01} \\
    \ket{10}\text{CNOT} & = \ket{11} \\
    \ket{11}\text{CNOT} & = \ket{10}
  \end{align*}
  El nombre de CNOT es Control NOT, pues el primer cúbit actúa como control, esta puerta solo actúa si el primer cúbit está en $\ket{1}$, en tal caso aplica $X$ al segundo cúbit.
\end{eje}

Algunas veces debemos aplicar puertas de un cúbit a sistemas de varios cúbits, y simplemente por el nombre de la puerta no es suficiente para saber dónde actúa la puerta.
Lo habitual es usar el producto tensorial con la unidad para construir nuevas puertas.

\begin{eje}
  La puerta de Hadamard sobre tres cúbits pero actuando solo sobre el segundo se puede construir como $I \otimes H \otimes I$.
\end{eje}

Una notación más compacta y clara es indicar con un subíndice el cúbit sobre el que actúa la puerta.

\begin{eje}
  $H_2$ denota la puerta de Hadamard aplicada al cúbit 2.
  \[
    \ket{001}H_2 = \frac{\ket{001}+\ket{011}}{\sqrt{2}}
  \]
\end{eje}

\subsubsection{Puertas controladas generales}

\begin{defi}[Puerta controlada general]
  Para cualquier operador unitario $U$ de un cúbit:
  $$C_U = \ket{0}\bra{0} \otimes I + \ket{1}\bra{1} \otimes U = \begin{pmatrix} I & 0 \\ 0 & U \end{pmatrix}$$
\end{defi}

\begin{eje}[Otras puertas controladas importantes]
  \begin{itemize}
    \item \textbf{Controlled-Z}: $C_Z = \text{diag}(1, 1, 1, -1)$
    \item \textbf{Controlled-Hadamard}: $C_H = \begin{pmatrix} I & 0 \\ 0 & H \end{pmatrix}$
    \item \textbf{Controlled-Phase}: $C_{R_\phi} = \text{diag}(1, 1, 1, e^{i\phi})$
  \end{itemize}
\end{eje}

Para puertas controladas de dos cúbits, a veces se ejecuta el control sobre el segundo cúbit, pero en estas situaciones, no podemos identificar claramente el control y el objetivo solo por el nombre de la puerta.

En estas situaciones es necesario indicar donde está el cúbit de control.
\begin{defi}
  Sea $U$ una puerta cuántica de un cúbit. Denotamos por $U^j_k$ a la puerta controlada de $U$ con el cúbit $j$ como control y el cúbit $k$ como objetivo.
\end{defi}

\begin{eje}
  \begin{itemize}
    \item La puerta controlada CNOT se puede escribir como $X^1_2$.
    \item La puerta controlada Hadamard se puede escribir como $H^1_2$.
    \item La puerta controlada $X^2_1$ esta representada por la matriz
          $$\begin{pmatrix} 1 & 0 & 0 & 0 \\ 0 & 0 & 0 & 1 \\ 0 & 0 & 1 & 0 \\ 0 & 1 & 0 & 0 \end{pmatrix}\,.$$
  \end{itemize}
\end{eje}

\begin{prop}
  Se dan las siguientes igualdades:
  \begin{align*}
    X^2_1 & = (H\otimes H)X^1_2 (H\otimes H)\,. \\
    Z^2_1 & = Z^1_2\,.                          \\
    X^1_2 & = H_1Z^1_2 H_2\,.
  \end{align*}
\end{prop}

\subsubsection{Puerta SWAP}
\begin{defi}
  La puerta SWAP intercambia dos cúbits
  $$\text{SWAP} = \begin{pmatrix} 1 & 0 & 0 & 0 \\ 0 & 0 & 1 & 0 \\ 0 & 1 & 0 & 0 \\ 0 & 0 & 0 & 1 \end{pmatrix}\,.$$
\end{defi}

\begin{prop}
  La puerta SWAP cumple
  \[
    \text{SWAP} = X^2_1 X^1_2 X^2_1\,.
  \]
\end{prop}

\subsubsection{Puertas de múltiples controles}

\begin{defi}
  Sea $U$ una puerta cuántica de un cúbit. Denotamos por $U^{j_1 j_2 \ldots j_m}_k$ a la puerta controlada de $U$ con el cúbit $j_1, j_2, \ldots, j_m$ como control y el cúbit $k$ como objetivo.
\end{defi}

\begin{defi}[Puerta Toffoli (CCNOT)]
  La puerta Toffoli es un CNOT con dos controles
  $$\text{CCNOT} = X^{12}_3 = \begin{pmatrix} 1 & 0 & 0 & 0 & 0 & 0 & 0 & 0 \\ 0 & 1 & 0 & 0 & 0 & 0 & 0 & 0 \\ 0 & 0 & 1 & 0 & 0 & 0 & 0 & 0 \\ 0 & 0 & 0 & 1 & 0 & 0 & 0 & 0 \\ 0 & 0 & 0 & 0 & 1 & 0 & 0 & 0 \\ 0 & 0 & 0 & 0 & 0 & 1 & 0 & 0 \\ 0 & 0 & 0 & 0 & 0 & 0 & 0 & 1 \\ 0 & 0 & 0 & 0 & 0 & 0 & 1 & 0 \end{pmatrix}\,.$$

  Aplica $X$ al tercer cúbit solo si los primeros dos están en $\ket{11}$.
\end{defi}

\subsection{Evolución Temporal y Hamiltonianos}

Recordemos que la evolución temporal de un sistema cuántico aislado está gobernada por la ecuación de Schrödinger estacionaria
$$\frac{d}{dt}\ket{\psi(t)} = -\frac{i}{\hbar}H\ket{\psi(t)}\,,$$
donde $H$ es el Hamiltoniano (operador hermitiano).

La solución formal es:
$$\ket{\psi(t)} = e^{-iHt/\hbar}\ket{\psi(0)}$$

\begin{eje}[Evolución bajo Hamiltoniano de Pauli]
  Para $H = \omega Z/2$ tenemos que
  $$e^{-iHt} = e^{-i\omega t Z/2} = \cos\frac{\omega t}{2}I - i\sin\frac{\omega t}{2}Z = \begin{pmatrix} e^{-i\omega t/2} & 0 \\ 0 & e^{i\omega t/2} \end{pmatrix}\,.$$

  Un cúbit inicialmente en $\ket{+}$ evoluciona como
  $$\ket{\psi(t)} = \frac{1}{\sqrt{2}}(e^{-i\omega t/2}\ket{0} + e^{i\omega t/2}\ket{1})\,.$$
\end{eje}

Como sabemos calcular la exponencial de una matriz a partir de sus valores propios y vectores propios, podemos calcular la exponencial de un Hamiltoniano.

Para una matriz hermitiana $H$ con descomposición espectral $H = \sum_i \lambda_i P_i$ donde $P_i$ son las proyecciones sobre los subespacios propios, tenemos que
$$e^{-iHt} = \sum_i e^{-i\lambda_i t} P_i\,.$$

\begin{eje}[Exponencial de combinaciones de Pauli]
  Para $H = \alpha X + \beta Y + \gamma Z$ con $|\vec{n}| = \sqrt{\alpha^2 + \beta^2 + \gamma^2}$,
  $$e^{-iHt} = \cos(|\vec{n}|t)I - i\sin(|\vec{n}|t)\frac{\vec{n} \cdot \vec{\sigma}}{|\vec{n}|}\,.$$
\end{eje}

\begin{theo}
  Cualquier operador unitario en $\mathcal{C}^n$ puede aproximarse arbitrariamente bien usando solo las puertas $H$, $T$, y CNOT.
\end{theo}

\begin{info}
  Este resultado es fundamental para la computación cuántica práctica, ya que muestra que solo necesitamos implementar físicamente un pequeño conjunto de puertas para realizar cualquier cálculo cuántico.
\end{info}

\begin{eje}[Estados de Bell]
  A partir del estádo básico $\ket{00}$ podemos construir cualquier estado de Bell aplicando las puertas $H$ y $X^1_2$. Por ejemplo
  \[
    \ket{00}H_1 X^1_2 = \frac{1}{\sqrt{2}}(\ket{00} + \ket{10})X^1_2 = \frac{1}{\sqrt{2}}(\ket{00} + \ket{11}) = \ket{\Phi^+}\,.
  \]
\end{eje}

\begin{eje}[Estado GHZ]
  El estado Greenberger-Horne-Zeilinger (GHZ) es un estado de tres cúbits que es un estado de Bell generalizado. Su construcción es similar a la de los estados de Bell, pero en lugar de aplicar $X^1_2$ aplicamos $X^1_2 X^2_3$.
  \[
    \ket{000}H_1 X^1_2 X^2_3 = \frac{1}{\sqrt{2}}(\ket{000} + \ket{111}) = \ket{\text{GHZ}}\,.
  \]
\end{eje}

Por último veamos el teorema de no clonación. Este teorema nos impide clonar estados cuánticos arbitrariamente y es el motivo por el que no podemos implementar clonadores cuánticos o dicho de otra manera, no podemos hacer copias de estados cuánticos.

\begin{theo}[Teorema de no clonación]
  No existe un operador unitario que pueda clonar estados cuánticos arbitrariamente.
\end{theo}

%
\unirsection{Problemas}

\begin{questions}

  \question Demuestre que las siguientes matrices son unitarias:
  \begin{parts}
    \part $U_1 = \frac{1}{\sqrt{2}}\begin{pmatrix} 1 & i \\ i & 1 \end{pmatrix}$
    \part $U_2 = \frac{1}{\sqrt{3}}\begin{pmatrix} 1 & \sqrt{2} \\ \sqrt{2} & -1 \end{pmatrix}$
  \end{parts}


  \question Exprese la puerta $Y$ como producto de rotaciones $R_x$ y $R_z$.

  \question Construya explícitamente la matriz $4 \times 4$ para la puerta Controlled-$R_y(\pi/3)$.

  \question Un qubit inicialmente en estado $\ket{+}$ evoluciona bajo el Hamiltoniano $H = \sigma_x + \sigma_z$.
  \begin{parts}
    \part Calcule $e^{-iHt}$
    \part Determine $\ket{\psi(t)}$
    \part ¿Cuál es el período de oscilación?
  \end{parts}

  \question Utilizando la notación de Dirac, demuestre que la puerta CSWAP puede escribirse como
  \[
    \text{CSWAP} = \ket{0}\bra{0} \otimes I \otimes I + \ket{1}\bra{1} \otimes \text{SWAP}\,.
  \]

  \question Demuestre que la puerta CNOT puede construirse a partir de puertas Hadamard y CZ.

  \question Muestre que la puerta SWAP puede construirse a partir de tres puertas CNOT.

  \question Comprueba que la matriz de Hadamard es equivalente a $\dfrac{X+Z}{\sqrt{2}}$.

  \question Demuestra que todo operador unitario y hermítico tiene autovalores $\pm 1$.

  \question Demuestra las siguientes igualdades:
  \begin{parts}
    \part $H X H = Z$.
    \part $H Z H = X$.
    \part $H Y H = -Y$.
    \part $X^2_1 = H_1 H_2 X^1_2 H_1 H_2$.
    \part $Z^2_1 = Z^1_2$.
  \end{parts}

  \question Considere el siguiente operador unitario sobre 3 cúbits
  \[
    U = H_1(\text{SWAP})^1_{2,3}H_1\,.
  \]

  Dado dos cúbits $\ket{a}$ y $\ket{b}$,
  \begin{parts}
    \part ¿cuál es el resultado de aplicar $U$ al estado $\ket{0ab}$?
    \part ¿Cual es la probabilidad de medir el primer cúbit en estado $\ket{0}$ tras aplicar $U$?
    \part Indica como podemos usar este operador para comparar si los cúbits $\ket{a}$ y $\ket{b}$ son iguales.
  \end{parts}

\end{questions}


%\tema{8}{Estados probabilísticos}
%\portada

\begin{esquemaExplorador}
  \temaEsquema{Estados}{
    \conceptoEsquema{Estados puros}{$\ket{a}$}
    \conceptoEsquema{Estados mixtos}{$\{(p_1, |\psi_1\rangle), \ldots, (p_k, |\psi_k\rangle)\}$}
  }
  \temaEsquema{Matriz de densidad}{
    \conceptoEsquema{Definición y propiedades}{$\sum_{i=1} p_i |\psi_i\rangle\langle\psi_i|$}
    \conceptoEsquema{Operador densidad para estados puros}{$\rho = |\psi\rangle\langle\psi|$}
    \conceptoEsquema{Estados mixtos y mezclas estadísticas}{}
    \conceptoEsquema{Criterio de pureza}{}
  }
  \temaEsquema{Sistemas compuestos}{
    \conceptoEsquema{Producto tensorial de matrices de densidad}{$\rho_{AB} = \rho_A \otimes \rho_B$}
    \conceptoEsquema{Evolución matriz de densidad}{}
  }
  \temaEsquema{Traza parcial}{
    \conceptoEsquema{Traza parcial y sistemas abiertos}{$\text{Tr}_B(\rho_{AB})$}
    \conceptoEsquema{Decoherencia cuántica}{}
  }
  \temaEsquema{Canales cuánticos}{
    \conceptoEsquema{Representación de Kraus}{$\rho' = \sum_k E_k \rho E_k^\dagger$}
    \conceptoEsquema{Operaciones completamente positivas}{}
  }
  \temaEsquema{Información cuántica}{
    \conceptoEsquema{Entropía de von Neumann}{$S(\rho) = -\text{Tr}(\rho \log \rho)$}
    \conceptoEsquema{Información mutua cuántica}{$I(A:B) = S(\rho_A) + S(\rho_B) - S(\rho_{AB})$}
    \conceptoEsquema{Fidelidad cuántica}{$F(\rho, \sigma) = \left(\text{Tr}\sqrt{\sqrt{\rho}\sigma\sqrt{\rho}}\right)^2$}
    \conceptoEsquema{Distancia de traza}{$D(\rho, \sigma) = \frac{1}{2}\text{Tr}|\rho - \sigma|$}
  }
\end{esquemaExplorador}

\unirsection{Ideas clave}

\subsection{Introducción y objetivos}

En los temas anteriores hemos trabajado con el formalismo de vectores estado para describir sistemas cuánticos, donde cada estado se representa mediante un vector unitario en un espacio de Hilbert. Sin embargo, este enfoque tiene limitaciones importantes cuando consideramos sistemas cuánticos realistas que pueden encontrarse en estados de los cuales no tenemos información completa, o cuando estudiamos subsistemas de sistemas cuánticos compuestos.

El formalismo del operador de densidad constituye una parte fundamental de la mecánica cuántica moderna y de la computación cuántica, introducido inicialmente por von Neumann en el marco de la teorı́a cuántica estadı́stica, permitió describir de manera rigurosa otros tipos de sistemas cuánticos que no se pueden representar puramente mediante vectores estado.
El operador densidad es esencial para describir sistemas abiertos, decoherencia y mezclas estadı́sticas, aspectos imprescindibles en la computación
cuántica realista.

Junto con el operador de densidad, la matriz de densidad permite describir procesos de decoherencia, canales cuánticos, entrelazamiento y operaciones sobre sistemas compuestos. En particular, los estados de Bell, constituyen un ejemplo paradigmático de estados puros máximamente entrelazados cuya estructura se comprende de forma natural mediante la matriz de densidad y las trazas parciales.

En este tema analizaremos el formalismo de la matriz de densidad y sus aplicaciones en la teorı́a cuántica estadı́stica y la computación cuántica realista.
Veremos algunos conceptos básicos que:

\begin{itemize}
  \item Permite describir \textbf{estados mixtos} que surgen por decoherencia o incertidumbre clásica.
  \item Facilita el análisis de \textbf{subsistemas cuánticos} mediante la traza parcial.
  \item Proporciona herramientas para cuantificar la \textbf{calidad de estados cuánticos} mediante medidas como la fidelidad.
  \item Es esencial para entender el \textbf{entrelazamiento cuántico} y la separabilidad de estados.
  \item Conecta directamente con la \textbf{teoría de información cuántica} y protocolos de comunicación cuántica.
\end{itemize}

Vamos a establecer las bases matemáticas necesarias para comprender sistemas cuánticos realistas, incluyendo efectos de ruido, decoherencia y la estructura de correlaciones cuánticas en sistemas compuestos.

\subsection{Estados puros y estados mixtos}

\begin{defi}
  Un estado cuántico \textbf{mixto} representa una situación donde el sistema puede estar en uno de varios estados puros $|\psi_i\rangle$ con probabilidades clásicas $p_i$.
  $$\text{Estado mixto} = \{(p_1, |\psi_1\rangle), (p_2, |\psi_2\rangle), \ldots, (p_k, |\psi_k\rangle)\}\,,$$

  donde
  \[
    p_i \geq 0\quad \text{y}\quad \sum_{i=1}^k p_i = 1\,.
  \]
\end{defi}

Esta descripción surge cuando hay incertidumbre clásica sobre cuál es el estado real del sistema.

\begin{eje}
  Considere las siguientes situaciones que requieren estados mixtos:

  \begin{enumerate}
    \item \textbf{Preparación incierta:} Alice prepara un cúbit en el estado $|0\rangle$ con probabilidad $1/2$ y en el estado $|1\rangle$ con probabilidad $1/2$. Bob recibe el cúbit pero no sabe cuál estado fue preparado.

    \item \textbf{Decoherencia:} Un cúbit inicialmente en superposición $\frac{1}{\sqrt{2}}(|0\rangle + |1\rangle)$ interactúa con el ambiente y pierde coherencia, resultando en una mezcla estadística.

    \item \textbf{Subsistema:} Dado un sistema bipartito en estado entrelazado $|\Phi^+\rangle = \frac{1}{\sqrt{2}}(|00\rangle + |11\rangle)$, cada subsistema individual está en un estado mixto.
  \end{enumerate}
\end{eje}

\subsection{Matriz de densidad}

\begin{defi}[Operador densidad]
  El operador densidad de un sistema cuántico mixto $\{(p_i, |\psi_i\rangle)\}$, es el operador $\rho$ definido por
  $$\rho = \sum_{i=1} p_i |\psi_i\rangle\langle\psi_i|\,.$$
\end{defi}

Para un estado puro $|\psi\rangle$ el operador densidad es el proyector
$$\rho = |\psi\rangle\langle\psi|\,.$$

La matriz asociada, que también se denota como $\rho$ en una base ortonormal $\{|e_j\rangle\}$, tiene las entradas $\rho = (\rho)_{mn}$ calculadas como
$$\rho_{mn} = \langle e_m|\rho|e_n\rangle\,.$$

\begin{eje}
  \begin{enumerate}
    \item \textbf{Estado puro} $|0\rangle$:
          $$\rho_0 = |0\rangle\langle 0| = \begin{pmatrix} 1 & 0 \\ 0 & 0 \end{pmatrix}\,.$$

    \item \textbf{Estado de superposición} $|\psi\rangle = \frac{1}{\sqrt{2}}(|0\rangle + |1\rangle)$:
          $$\rho_\psi = |\psi\rangle\langle\psi| = \frac{1}{2}\begin{pmatrix} 1 & 1 \\ 1 & 1 \end{pmatrix}\,.$$

    \item \textbf{Estado mixto equiprobable}:
          $$\rho_{\text{mix}} = \frac{1}{2}|0\rangle\langle 0| + \frac{1}{2}|1\rangle\langle 1| = \frac{1}{2}\begin{pmatrix} 1 & 0 \\ 0 & 1 \end{pmatrix} = \frac{I}{2}\,.$$
  \end{enumerate}
\end{eje}

\begin{prop}
  Sea $\rho = \sum_{i=1}^k p_i \ketbra{\psi_i}$ un operador densidad. Entonces:
  \begin{enumerate}
    \item $\rho$ es hermitiana: $\rho = \rho^\dagger$.
    \item $\rho$ es positiva semidefinida: $\langle\phi|\rho|\phi\rangle \geq 0$ para todo $|\phi\rangle$.
    \item $\text{Tr}(\rho) = 1$ (normalización).
  \end{enumerate}
\end{prop}
\begin{proof}
  \begin{enumerate}
    \item Para demostrar que $\rho$ es hermitiana, tenemos que tener en cuenta las propiedades de la traspuesta conjutada y que $p_i\in \R$, que $\ket{\psi_i}^\dagger = \bra{\psi_i}$ y $\bra{\psi_i}^\dagger = \ket{\psi_i}$ para todo $i$. Entonces
          \[
            \rho^\dagger = \sum_{i=1}^k p_i (\ketbra{\psi_i})^\dagger = \sum_{i=1}^k p_i \bra{\psi_i}^\dagger\ket{\psi_i}^\dagger  = \sum_{i=1}^k p_i \ketbra{\psi_i} = \rho\,.
          \]
    \item Para cualquier ket $\ket{\phi}$ tenemos
          \begin{align*}
            \expval{\rho}{\phi} & = \sum_{i=1}^{k}p_i\braket{\phi}{\psi_i}\braket{\psi_i}{\phi}   = \sum_{i=1}^{k}p_i\braket{\phi}{\psi_i}\braket{\phi_i}{\psi}^* \\
                                & = \sum_{i=1}^{k}p_i|\braket{\phi}{\psi_i}|^2                    \geq 0\,.
          \end{align*}
    \item $\text{Tr}(\rho) = \sum_{i=1}^k p_i = 1$.
  \end{enumerate}
\end{proof}

\begin{theo}[Criterio de pureza]
  Un estado descrito por el operador densidad $\rho$ es puro si y solo si $\text{Tr}(\rho^2) = 1$.

  Equivalentemente, el estado es mixto si y solo si $\text{Tr}(\rho^2) < 1$.
\end{theo}
\begin{proof}
  Como $\rho$ es un operador hermitiano, existe una base ortonormal $\{\ket{i}\}$ en la que $\rho$ es diagonal:
  \[
    \rho = \sum_i \lambda_i \ketbra{i}\,,
  \]
  donde $\lambda_i$ son los valores propios de $\rho$. Por las propiedades de la matriz de densidad, sabemos que $\lambda_i \ge 0$ y $\sum_i \lambda_i = 1$. Esto implica que $0 \le \lambda_i \le 1$ para todo $i$.

  La traza de $\rho^2$ se calcula como:
  \[
    \text{Tr}(\rho^2) = \sum_i \lambda_i^2\,.
  \]

  Dado que $0 \le \lambda_i \le 1$, se cumple que $\lambda_i^2 \le \lambda_i$, dándose la igualdad únicamente si $\lambda_i = 0$ o $\lambda_i = 1$. Sumando sobre todos los índices $i$:
  \[
    \text{Tr}(\rho^2) = \sum_i \lambda_i^2 \le \sum_i \lambda_i = 1\,.
  \]

  La igualdad $\text{Tr}(\rho^2) = 1$ ocurre si y solo si $\lambda_i^2 = \lambda_i$ para todo $i$. Teniendo en cuenta la restricción de normalización $\sum_i \lambda_i = 1$, esto fuerza a que exactamente un valor propio sea 1 (digamos $\lambda_k=1$) y todos los demás sean 0. En tal caso:
  \[
    \rho = \ketbra{k}\,,
  \]
  lo cual corresponde, por definición, a un estado puro. Si $\text{Tr}(\rho^2) < 1$, entonces el estado no es un proyector de rango 1, correspondiendo a una mezcla estadística.
\end{proof}

\begin{eje}
  Sea $\rho = \frac{3}{4}\ketbra{0}{0} + \frac{1}{4}\ketbra{1}{1}$ la matriz de densidad de un estado. Necesitamos saber si el estado es mixto o puro.
  la matriz de densidad es
  \[
    \rho = \begin{pmatrix} 3/4 & 0 \\ 0 & 1/4 \end{pmatrix}\,.
  \]

  Calculamos la traza de $\rho^2$
  \[
    \text{Tr}(\rho^2) = \text{Tr}\begin{pmatrix} (3/4)^2 & 0 \\ 0 & (1/4)^2 \end{pmatrix} = \frac{9}{16} + \frac{1}{16} = \frac{10}{16} = \frac{5}{8} < 1\,.
  \]
  Concluimos que el estado es mixto.
\end{eje}
\subsection{Evolución de la matriz de densidad}

\begin{theo}[Evolución unitaria]
  Si un sistema con matriz de densidad $\rho$ evoluciona bajo un operador unitario $U$, la nueva matriz de densidad es:
  $$\rho' = U\rho U^\dagger$$

  Esta transformación preserva todas las propiedades de un operador densidad válido.
\end{theo}
\begin{proof}
  Si $\rho^\prime$ es la matriz de densidad del sistema después de la evolución unitaria $U$, entonces
  \begin{align*}
    \rho' & = \sum_i p_i \ketbra{U\psi_i}                         \\
          & = \sum_i p_i U \ketbra{\psi_i} U^\dagger              \\
          & = U \left(\sum_i p_i \ketbra{\psi_i}\right) U^\dagger \\
          & = U \rho U^\dagger\,.
  \end{align*}
\end{proof}

\begin{eje}
  Consideremos la matriz de densidad que representa un estado mixto en un cúbit equiprobable $\rho = \frac{1}{2}I$ y apliquemos cualquier puerta cuántica $U$.
  $$\rho' = U\rho U^\dagger = U \cdot \frac{1}{2}I \cdot U^\dagger = \frac{1}{2}UU^\dagger = \frac{1}{2}I = \rho\,.$$

  El estado mixto completamente aleatorio es invariante bajo cualquier operación unitaria.
\end{eje}

\begin{eje}[Aplicación de puertas cuánticas]
  Sea en un cúbit en el estado mixto con matriz de densidad
  \[
    \rho = \begin{pmatrix} 3/4 & 0 \\ 0 & 1/4 \end{pmatrix}\,,
  \]
  y apliquemos la puerta Hadamard $H$ al estado, la evolución dará lugar a la nueva matriz de densidad
  \begin{align*}
    \rho' & = H\rho H^\dagger = \frac{1}{\sqrt{2}}\begin{pmatrix} 1 & 1 \\ 1 & -1 \end{pmatrix}\begin{pmatrix} 3/4 & 0 \\ 0 & 1/4 \end{pmatrix}\frac{1}{\sqrt{2}}\begin{pmatrix} 1 & 1 \\ 1 & -1 \end{pmatrix}^\dagger \\
          & = \frac{1}{2}\begin{pmatrix} 3/4 & 1/4 \\ 3/4 & -1/4 \end{pmatrix}\begin{pmatrix} 1 & 1 \\ 1 & -1 \end{pmatrix}                                                                                            \\
          & = \frac{1}{2}\begin{pmatrix} 1 & 1/2 \\ 1/2 & 1 \end{pmatrix}\,.
  \end{align*}
\end{eje}

\subsection{Representación en la esfera de Bloch}

Para un cúbit $\ket{a}=\alpha\ket{0} + \beta\ket{1}$, el operador densidad es
\[
  \rho = |\alpha|^2 \ketbra{0} + |\beta|^2 \ketbra{1} + \alpha\beta^* \ketbra{0}{1} + \alpha^*\beta \ketbra{1}{0}\,.
\]

En forma matricial y expresado en términos de la base de Pauli, tenemos
\begin{align*}
  \rho & = \begin{pmatrix} |\alpha|^2 & \alpha\beta^* \\ \alpha^*\beta & |\beta|^2 \end{pmatrix}                                                   \\
       & = \frac{1}{2}\begin{pmatrix} 1 + (|\alpha|^2 - |\beta|^2) & 2\alpha\beta^* \\ 2\alpha^*\beta & 1 - (|\alpha|^2 - |\beta|^2) \end{pmatrix} \\
       & = \frac{1}{2}\left[I + (|\alpha|^2 - |\beta|^2)Z + 2\text{Re}(\alpha\beta^*)X + 2\text{Im}(\alpha\beta^*)Y\right]\,.
\end{align*}

Podemos escribir la matriz de densidad de forma más compacta como
$$\rho = \frac{1}{2}(I + \vec{r} \cdot \vec{\sigma})\,,$$
donde $\vec{r} = (r_x, r_y, r_z)$ lo llamamos el \textbf{vector de Bloch} y $\vec{\sigma} = (X, Y, Z)$.

El vector de Bloch admite la siguiente interpretación geométrica:
\begin{itemize}
  \item \textbf{Estados puros:} $|\vec{r}| = 1$ (superficie de la esfera).
  \item \textbf{Estados mixtos:} $|\vec{r}| < 1$ (interior de la esfera).
  \item \textbf{Estado completamente mixto:} $\vec{r} = 0$ (centro de la esfera).
\end{itemize}

El vector de Bloch se puede calcular a partir de la matriz de densidad como
$$\vec{r} = (\text{Tr}(\rho X), \text{Tr}(\rho Y), \text{Tr}(\rho Z))\,.$$

\begin{eje}[Cálculo del vector de Bloch]
  Para el estado $\rho = \frac{3}{4}|0\rangle\langle 0| + \frac{1}{4}|1\rangle\langle 1| = \begin{pmatrix} 3/4 & 0 \\ 0 & 1/4 \end{pmatrix}$, calculamos el vector de Bloch:
  \begin{align*}
    \rho X & = \begin{pmatrix} 3/4 & 0 \\ 0 & 1/4 \end{pmatrix}\begin{pmatrix} 0 & 1 \\ 1 & 0 \end{pmatrix} = \begin{pmatrix} 0 & 3/4 \\ 1/4 & 0 \end{pmatrix}\,.    \\
    \rho Y & = \begin{pmatrix} 3/4 & 0 \\ 0 & 1/4 \end{pmatrix}\begin{pmatrix} 0 & -i \\ i & 0 \end{pmatrix} = \begin{pmatrix} 0 & -3i/4 \\ i/4 & 0 \end{pmatrix}\,. \\
    \rho Z & = \begin{pmatrix} 3/4 & 0 \\ 0 & 1/4 \end{pmatrix}\begin{pmatrix} 1 & 0 \\ 0 & -1 \end{pmatrix} = \begin{pmatrix} 3/4 & 0 \\ 0 & -1/4 \end{pmatrix}\,.
  \end{align*}
  Entonces
  $$\vec{r} = \left(\text{Tr}(\rho X), \text{Tr}(\rho Y), \text{Tr}(\rho Z)\right) = (0, 0, 1/2)\,.$$

  Para calcular si el estado es puro o mixto, observamos que la norma del vector de Bloch es $|\vec{r}| = 1/2 < 1$, confirmando que es un estado mixto.
\end{eje}

\subsection{Sistemas compuestos y traza parcial}

\begin{defi}[Producto tensorial de operadores densidad]
  Para sistemas independientes descritos por matrices de densidad $\rho_A$ y $\rho_B$, el sistema compuesto se describe por
  $$\rho_{AB} = \rho_A \otimes \rho_B\,.$$

  Sin embargo, no todos los estados de sistemas compuestos pueden escribirse en esta forma separable.
\end{defi}

\begin{defi}[Traza parcial]
  Sea $T$ un operador sobre un sistema bipartito $A\otimes B$, la traza parcial de $T$ sobre el subsistema $B$ se define como el operador $\text{Tr}_B(T)$ sobre el subsistema $A$ dado por la propiedad
  $$\matrixelement{\phi}{\text{Tr}_B(T)}{\psi} = \sum_i \matrixelement{\phi \otimes \ket{i}}{T}{\psi \otimes \ket{i}}\,,$$
  donde $\ket{\phi},\ket{\psi}\in A$ y $\{\ket{i}\}$ es cualquier base ortonormal del subsistema $B$.
\end{defi}

\begin{eje}
  Consideremos el estado producto
  $$\ket{\psi} = \ket{0+} = \frac{\ket{00} + \ket{01}}{\sqrt{2}}\,.$$

  La matriz de densidad del sistema compuesto es
  $$\rho_{AB} = \ket{\psi}\bra{\psi} = \ket{0}\bra{0} \otimes \ket{+}\bra{+}\,.$$

  En forma matricial usando la base computacional, es
  $$\rho_{AB} = \frac{1}{2}\begin{pmatrix}
      1 & 1 & 0 & 0 \\
      1 & 1 & 0 & 0 \\
      0 & 0 & 0 & 0 \\
      0 & 0 & 0 & 0
    \end{pmatrix}\,.$$

  Para obtener la matriz de densidad del subsistema $A$, realizamos la traza parcial sobre $B$. Si $\ket{\phi}=a\ket{0}+b\ket{1}$ y $\ket{\psi}=c\ket{0}+d\ket{1}$ son dos kets arbitrarios en el subsistema $A$, calculamos los siguientes términos:
  \begin{equation}
    \label{eq:traza_parcial_eje_1}
    \matrixelement{\phi\otimes \ket{0}}{\rho_{AB}}{\psi\otimes \ket{0}} = \frac{1}{2}ac^*\,,
  \end{equation}
  \begin{equation}
    \label{eq:traza_parcial_eje_2}
    \matrixelement{\phi\otimes \ket{1}}{\rho_{AB}}{\psi\otimes \ket{1}} = \frac{1}{2}ac^*\,.
  \end{equation}

  Por definición de traza parcial, sumamos ambos resultados obtenemos
  \begin{align*}
    \eqref{eq:traza_parcial_eje_1} + \eqref{eq:traza_parcial_eje_2} & = \frac{1}{2}(ac^* + ac^*) = ac^*\,.
  \end{align*}

  Por lo tanto la matriz de densidad del subsistema $A$ es
  \[
    \text{Tr}_B(\rho_{AB}) = \ket{0}\bra{0}\,.
  \]

\end{eje}

\begin{prop}
  Sea $T$ un operador sobre un sistema bipartito $A\otimes B$, definido como el producto tensorial de dos operadores $T_A \otimes T_B$, la traza parcial de $T$ sobre el subsistema $B$ se obtiene como
  \[
    \text{Tr}_B(T_A \otimes T_B) = \text{Tr}(T_B) T_A\,.
  \]
\end{prop}

\begin{eje}
  Volviendo al ejemplo anterior, la matriz de densidad del sistema compuesto es
  $$\rho_{AB} = \frac{1}{2}\begin{pmatrix}
      1 & 1 & 0 & 0 \\
      1 & 1 & 0 & 0 \\
      0 & 0 & 0 & 0 \\
      0 & 0 & 0 & 0
    \end{pmatrix} = \frac{1}{2}\begin{pmatrix}
      1 & 0 \\
      0 & 0
    \end{pmatrix}\otimes \begin{pmatrix}
      1 & 1 \\
      1 & 1
    \end{pmatrix}\,.$$

  Usando el resultado anterior, para obtener la matriz de densidad del subsistema $A$, realizamos la traza parcial sobre $B$.
  \[
    \text{Tr}_B(\rho_{AB}) = \text{Tr}(\rho_B)\rho_A=\begin{pmatrix} 1 & 0 \\ 0 & 0 \end{pmatrix}=\outerproduct{0}{0}\,.
  \]

\end{eje}


\begin{eje}
  Considere el estado entrelazado $|\Phi^+\rangle = \frac{1}{\sqrt{2}}(|00\rangle + |11\rangle)$ con matriz de densidad:
  $$\rho_{AB} = |\Phi^+\rangle\langle\Phi^+| = \frac{1}{2}\begin{pmatrix} 1 & 0 & 0 & 1 \\ 0 & 0 & 0 & 0 \\ 0 & 0 & 0 & 0 \\ 1 & 0 & 0 & 1 \end{pmatrix}$$

  La traza parcial sobre el subsistema $B$ en la base $\{|0\rangle_B, |1\rangle_B\}$:
  \begin{align*}
    \rho_A & = \langle 0|_B \rho_{AB} |0\rangle_B + \langle 1|_B \rho_{AB} |1\rangle_B \\
           & = \frac{1}{2}|0\rangle_A\langle 0|_A + \frac{1}{2}|1\rangle_A\langle 1|_A \\
           & = \frac{1}{2}\begin{pmatrix} 1 & 0 \\ 0 & 1 \end{pmatrix} = \frac{I_A}{2}
  \end{align*}

  El subsistema $A$ está en un estado mixto completamente aleatorio, a pesar de que el sistema total está en un estado puro.
\end{eje}

\begin{prop}
  La operación de traza parcial satisface:
  \begin{enumerate}
    \item Linealidad: $\text{Tr}_B(\alpha\rho + \beta\sigma) = \alpha\text{Tr}_B(\rho) + \beta\text{Tr}_B(\sigma)$.
    \item Preservación de la traza: $\text{Tr}(\text{Tr}_B(\rho_{AB})) = \text{Tr}(\rho_{AB})$.
    \item $\text{Tr}_B(\rho_{AB}) = \rho_A \text{Tr}(\rho_B)$.
  \end{enumerate}
\end{prop}

\subsection{Decoherencia cuántica}

La \textbf{decoherencia cuántica} es el proceso por el cual un sistema cuántico pierde su coherencia debido a interacciones incontroladas con el entorno, transformando estados puros en estados mixtos.

\begin{eje}[Decoherencia de un cúbit]
  Considere un cúbit inicialmente en el estado de superposición $|\psi\rangle = \frac{1}{\sqrt{2}}(|0\rangle + |1\rangle)$ que sufre decoherencia de desfase. Después de un tiempo $t$, el estado viene representado por la matriz de densidad
  $$\rho(t) = \frac{1}{2}\begin{pmatrix} 1 & e^{-\gamma t} \\ e^{-\gamma t} & 1 \end{pmatrix}\,.$$
  donde $\gamma > 0$ es la tasa de decoherencia.
  \begin{itemize}
    \item En $t = 0$: $\rho(0) = |\psi\rangle\langle\psi|$, estado puro.
    \item Cuando $t \to \infty$: $\rho(\infty) = \frac{I}{2}$, estado maximalmente mixto.
    \item La pureza decae como: $\text{Tr}(\rho(t)^2) = \frac{1}{2}(1 + e^{-2\gamma t})$.
  \end{itemize}
\end{eje}

\subsection{Canales cuánticos}

\begin{defi}[Canal cuántico]
  Un canal cuántico es una aplicación completamente positiva que preserva la traza
  $$\mathcal{E}: \mathcal{L}(\mathcal{H}_A) \to \mathcal{L}(\mathcal{H}_B)$$

  Esta aplicación describe la evolución más general posible de un sistema cuántico abierto.
\end{defi}

\begin{theo}[Representación de Kraus]
  Todo canal cuántico $\mathcal{E}$ puede representarse mediante operadores de Kraus $\{K_i\}$
  $$\mathcal{E}(\rho) = \sum_i K_i \rho K_i^\dagger\,,$$

  donde los operadores satisfacen la condición de completitud
  $$\sum_i K_i^\dagger K_i = I\,.$$
\end{theo}

\begin{eje}[Canal de despolarización]
  El canal de despolarización para un cúbit con probabilidad $p$ tiene la forma:
  $$\mathcal{E}(\rho) = (1-p)\rho + \frac{p}{3}(X \rho X + Y \rho Y + Z \rho Z)$$

  Los operadores de Kraus son:
  $$K_0 = \sqrt{1-p}I, \quad K_1 = \sqrt{\frac{p}{3}}X, \quad K_2 = \sqrt{\frac{p}{3}}Y, \quad K_3 = \sqrt{\frac{p}{3}}Z\,.$$
\end{eje}

\subsection{Medidas de información cuántica}

\begin{defi}[Entropía de von Neumann]
  La entropía de von Neumann de un estado cuántico descrito por la matriz de densidad $\rho$ se define como
  $$S(\rho) = -\text{Tr}(\rho \log_2 \rho)\,.$$
\end{defi}

Si $\rho$ tiene descomposición espectral $\rho = \sum_i \lambda_i |e_i\rangle\langle e_i|$, entonces
$$S(\rho) = -\sum_i \lambda_i \log_2 \lambda_i$$

La entropía cuantifica el grado de "mezcla" o "desorden" del estado cuántico.

\begin{prop}
  \begin{enumerate}
    \item $S(\rho) \geq 0$ para toda matriz de densidad $\rho$.
    \item $S(\rho) = 0$ si y solo si $\rho$ representa un estado puro.
    \item Para un sistema de dimensión $d$: $S(\rho) \leq \log_2 d$, con igualdad si y solo si $\rho = \frac{I}{d}$.
    \item La entropía es invariante bajo transformaciones unitarias: $S(U\rho U^\dagger) = S(\rho)$.
    \item Subaditividad: $S(\rho_{AB}) \leq S(\rho_A) + S(\rho_B)$.
  \end{enumerate}
\end{prop}

\begin{eje}[Cálculo de entropía]
  Calculemos la entropía de von Neumann para los siguientes estados:
  \begin{enumerate}
    \item \textbf{Estado puro:} Para cualquier $|\psi\rangle$, $S(|\psi\rangle\langle\psi|) = 0$.

    \item \textbf{Estado mixto equiprobable:} Para $\rho = \frac{I}{2}$ en un cúbit
          $$S(\rho) = -\frac{1}{2}\log_2\frac{1}{2} - \frac{1}{2}\log_2\frac{1}{2} = 1\,.$$

    \item \textbf{Estado mixto general:} Para $\rho = p|0\rangle\langle 0| + (1-p)|1\rangle\langle 1|$
          $$S(\rho) = -p\log_2 p - (1-p)\log_2(1-p) = H(p)\,,$$
          donde $H(p)$ es la entropía de Shannon clásica.
  \end{enumerate}
\end{eje}

\begin{defi}[Información mutua cuántica]
  Para un estado bipartito $\rho_{AB}$, la información mutua cuántica se define como:
  $$I(A:B) = S(\rho_A) + S(\rho_B) - S(\rho_{AB})$$

  Esta cantidad mide las correlaciones totales (clásicas y cuánticas) entre los subsistemas $A$ y $B$.
\end{defi}

\begin{defi}[Fidelidad cuántica]
  La fidelidad entre dos estados cuánticos descritos por matrices de densidad $\rho$ y $\sigma$ se define como:
  $$F(\rho, \sigma) = \text{Tr}\left(\sqrt{\sqrt{\rho}\sigma\sqrt{\rho}}\right)$$

  Para estados puros $|\psi\rangle$ y $|\phi\rangle$:
  $$F(|\psi\rangle, |\phi\rangle) = |\langle\psi|\phi\rangle|$$

  La fidelidad mide qué tan cercanos están dos estados cuánticos.
\end{defi}

\begin{prop}
  \begin{enumerate}
    \item $0 \leq F(\rho, \sigma) \leq 1$.
    \item $F(\rho, \sigma) = F(\sigma, \rho)$ (simetría).
    \item $F(\rho, \rho) = 1$ (normalización).
    \item $F(\rho, \sigma) = 1$ si y solo si $\rho = \sigma$.
    \item Para operadores unitarios: $F(U\rho U^\dagger, U\sigma U^\dagger) = F(\rho, \sigma)$.
  \end{enumerate}
\end{prop}

\begin{defi}[Distancia de traza]
  La distancia de traza entre dos matrices de densidad $\rho$ y $\sigma$ se define como:
  $$D(\rho, \sigma) = \frac{1}{2}\text{Tr}(|\rho - \sigma|)$$

  donde $|A| = \sqrt{A^\dagger A}$ es el valor absoluto del operador $A$.
\end{defi}

\begin{prop}
  \begin{enumerate}
    \item $0 \leq D(\rho, \sigma) \leq 1$.
    \item $D(\rho, \sigma) = 0$ si y solo si $\rho = \sigma$.
    \item $D(\rho, \sigma) = D(\sigma, \rho)$ (simetría).
    \item $D(\rho, \tau) \leq D(\rho, \sigma) + D(\sigma, \tau)$ (desigualdad triangular).
    \item Para operadores unitarios: $D(U\rho U^\dagger, U\sigma U^\dagger) = D(\rho, \sigma)$.
  \end{enumerate}
\end{prop}

\begin{theo}[Relación entre fidelidad y distancia de traza]
  Para cualesquiera dos matrices de densidad $\rho$ y $\sigma$:
  $$1 - F(\rho, \sigma) \leq D(\rho, \sigma) \leq \sqrt{2(1 - F(\rho, \sigma))}$$

  Ambas medidas proporcionan información complementaria sobre la proximidad entre estados cuánticos.
\end{theo}

\subsection{Aplicaciones en computación cuántica}

\begin{eje}[Fidelidad en protocolos cuánticos]
  En el protocolo de teleportación cuántica, la fidelidad entre el estado original $|\psi\rangle$ y el estado reconstruido $\rho_{\text{out}}$ mide la calidad del protocolo:
  $$F = \langle\psi|\rho_{\text{out}}|\psi\rangle$$

  Para teleportación perfecta, $F = 1$. En presencia de ruido o errores de medición, $F < 1$, y el protocolo es útil solo si $F > 2/3$ (límite clásico).
\end{eje}

\begin{eje}[Caracterización del entrelazamiento]
  Para un estado bipartito $\rho_{AB}$, el grado de entrelazamiento puede cuantificarse mediante:

  \begin{itemize}
    \item \textbf{Entropía de entrelazamiento:} $E(\rho_{AB}) = S(\rho_A) = S(\rho_B)$ para estados puros.
    \item \textbf{Negatividad:} Basada en la traza de la transposición parcial.
    \item \textbf{Concurrencia:} Medida específica para sistemas de dos cúbits.
  \end{itemize}

  Estas medidas son fundamentales para protocolos de información cuántica como criptografía cuántica y computación distribuida.
\end{eje}
%\unirsection{Problemas}

\begin{questions}

  \question Calcule la matriz de densidad para un ensemble que contiene:
  \begin{parts}
    \part 40\% de cúbits en estado $\ket{0}$
    \part 35\% de cúbits en estado $\ket{1}$
    \part 25\% de cúbits en estado $\ket{+}$
  \end{parts}

  \question Para la matriz de densidad del ejercicio anterior:
  \begin{parts}
    \part Calcule $\text{Tr}(\rho^2)$
    \part Determine si es un estado puro o mixto
    \part Calcule la entropía de von Neumann $S(\rho)$
    \part Encuentre el vector de Bloch $\vec{r}$ y verifique que $|\vec{r}| < 1$
  \end{parts}

  \question Un cúbit en estado $\ket{\psi} = \cos(\theta/2)\ket{0} + e^{i\phi}\sin(\theta/2)\ket{1}$ experimenta un canal de desfase puro que transforma:
  $$\rho = \ket{\psi}\bra{\psi} \to \rho' = \begin{pmatrix} \cos^2(\theta/2) & e^{-\gamma t}\cos(\theta/2)\sin(\theta/2)e^{i\phi} \\ e^{-\gamma t}\cos(\theta/2)\sin(\theta/2)e^{-i\phi} & \sin^2(\theta/2) \end{pmatrix}$$
  \begin{parts}
    \part Verifique que $\rho'$ es una matriz de densidad válida
    \part Calcule $\text{Tr}(\rho'^2)$ en función de $\gamma$ y $t$
    \part Determine en qué instante el estado se convierte en completamente mixto
  \end{parts}

  \question Considere el estado de Bell $\ket{\Phi^+} = \frac{1}{\sqrt{2}}(\ket{00} + \ket{11})$:
  \begin{parts}
    \part Escriba la matriz de densidad $\rho_{AB} = \ket{\Phi^+}\bra{\Phi^+}$ en forma matricial
    \part Calcule la traza parcial $\rho_A = \text{Tr}_B(\rho_{AB})$
    \part Calcule la entropía de von Neumann $S(\rho_A)$
    \part Compare $S(\rho_{AB})$ con $S(\rho_A)$. ¿Qué indica esto sobre el entrelazamiento?
  \end{parts}

  \question Para el canal de despolarización con operadores de Kraus:
  $$K_0 = \sqrt{1-p}I, \quad K_1 = \sqrt{\frac{p}{3}}\sigma_x, \quad K_2 = \sqrt{\frac{p}{3}}\sigma_y, \quad K_3 = \sqrt{\frac{p}{3}}\sigma_z$$
  \begin{parts}
    \part Verificar que satisfacen la condición de completitud $\sum_i K_i^\dagger K_i = I$
    \part Aplicar el canal al estado puro $\ket{0}$ y obtener $\rho'$
    \part Calcular $\text{Tr}(\rho'^2)$ en función de $p$
    \part Determine el valor de $p$ para el cual el estado de salida es completamente mixto
  \end{parts}

  \question Calcule la fidelidad y la distancia de traza entre los siguientes pares de estados:
  \begin{parts}
    \part $\rho_1 = \ket{0}\bra{0}$ y $\rho_2 = \ket{1}\bra{1}$
    \part $\rho_1 = \ket{+}\bra{+}$ y $\rho_2 = \frac{I}{2}$
    \part $\rho_1 = \frac{3}{4}\ket{0}\bra{0} + \frac{1}{4}\ket{1}\bra{1}$ y $\rho_2 = \frac{1}{4}\ket{0}\bra{0} + \frac{3}{4}\ket{1}\bra{1}$
  \end{parts}

  \question Para el estado bipartito $\rho_{AB} = \frac{1}{2}\ket{\Phi^+}\bra{\Phi^+} + \frac{1}{2}\ket{00}\bra{00}$ donde $\ket{\Phi^+} = \frac{1}{\sqrt{2}}(\ket{00} + \ket{11})$:
  \begin{parts}
    \part Calcule las matrices de densidad reducidas $\rho_A$ y $\rho_B$
    \part Calcule las entropías $S(\rho_A)$, $S(\rho_B)$ y $S(\rho_{AB})$
    \part Calcule la información mutua cuántica $I(A:B) = S(\rho_A) + S(\rho_B) - S(\rho_{AB})$
    \part Interprete el resultado: ¿qué nos dice sobre las correlaciones entre los subsistemas?
  \end{parts}

\end{questions}
\end{document}