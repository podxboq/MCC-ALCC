\portada

\begin{esquemaExplorador}
  \temaEsquema{Espacio dual}{
    \conceptoEsquema{Definición}
    \conceptoEsquema{Base dual}
    \conceptoEsquema{Teorema de Riesz-Fréchet}
  }
  \temaEsquema{Dualidad en espacios de Hilbert}{
    \conceptoEsquema{Isomorfismo canónico}
    \conceptoEsquema{Representación de funcionales}
    \conceptoEsquema{Aplicaciones a computación cuántica}
  }
  \temaEsquema{Producto tensorial}{
    \conceptoEsquema{Definición}
    \conceptoEsquema{Propiedades}
    \conceptoEsquema{Base tensorial}
    \conceptoEsquema{Producto tensorial de operadores y matrices}
    \conceptoEsquema{Ejemplos y aplicaciones}
  }
  \temaEsquema{Postulados de la mecánica cuántica}{
    \conceptoEsquema{Postulado IV: Composición de sistemas cuánticos}
  }
  \temaEsquema{Estados entrelazados}{
    \conceptoEsquema{Definición de entrelazamiento}
    \conceptoEsquema{Estados de Bell}
    \conceptoEsquema{Descomposición de Schmidt}
    \conceptoEsquema{Producto exterior}
  }
\end{esquemaExplorador}

\unirsection{Ideas clave}

\subsection{Introducción y objetivos}


Hasta ahora hemos explorado la columna vertebral matemática de la Mecánica Cuántica: los espacios vectoriales complejos ($\C^n$) y los espacios de Hilbert ($\H$), que sirven como el \textbf{espacio de estados} para un \textbf{sistema cuántico simple} (un \textit{cúbit}, por ejemplo).

\begin{itemize}
  \item Un \textbf{cúbit} se describe en $\mathcal{H}_1 \cong \mathbb{C}^2$.
  \item Un operador lineal (una medición, una compuerta) en este cúbit es una matriz en $\mathcal{L}(\mathcal{H}_1)$.
\end{itemize}

El concepto de espacio dual es fundamental en el análisis funcional y proporciona una perspectiva profunda sobre la estructura de los espacios vectoriales. En computación cuántica, el espacio dual adquiere un significado especial a través de la notación de Dirac, donde los \textit{bras} representan elementos del espacio dual de los \textit{kets}.

La importancia del espacio dual en matemáticas y física se manifiesta en varios aspectos:

\begin{itemize}
  \item Proporciona un marco \textbf{unificado} para entender funcionales lineales y formas multilineales.
  \item Permite la \textbf{caracterización completa} de propiedades geométricas mediante funcionales.
  \item Es esencial para la teoría de \textbf{operadores adjuntos} y autoadjuntos.
  \item Conecta la \textbf{notación bra-ket} con la estructura matemática.
\end{itemize}

Sin embargo, la verdadera potencia de la Computación Cuántica radica en la \textbf{superposición} y el \textbf{entrelazamiento} de \textbf{múltiples sistemas} (varios cúbits).

\paragraph{El problema de composición:}
Si tenemos dos sistemas cuánticos independientes, $S_A$ y $S_B$, descritos por sus respectivos espacios de Hilbert $\mathcal{H}_A$ y $\mathcal{H}_B$, ¿cuál es el espacio de estados $\mathcal{H}_{AB}$ que describe el \textbf{sistema compuesto} $S_{AB}$?

\begin{itemize}
  \item Si $S_A$ tiene dimensión $n$ ($\text{dim}(\mathcal{H}_A) = n$) y $S_B$ tiene dimensión $m$ ($\text{dim}(\mathcal{H}_B) = m$), el espacio compuesto $\mathcal{H}_{AB}$ debe tener dimensión \textbf{$nm$}.
  \item \textbf{La suma directa} ($\mathcal{H}_A \oplus \mathcal{H}_B$), que usamos para combinar estados linealmente independientes, tiene dimensión $n+m$. \textbf{No sirve}.
  \item La operación correcta para combinar espacios de estados y construir el espacio de estados compuesto es el \textbf{producto tensorial} ($\otimes$).
\end{itemize}

El \textbf{producto tensorial} es el mecanismo matemático que nos permite describir rigurosamente estados entrelazados como el famoso estado de Bell $\frac{1}{\sqrt{2}}(|00\rangle + |11\rangle)$, que no puede ser separado en un simple producto de estados individuales.

\subsection{Espacio dual}

\begin{defi}[Espacio dual]
  Sea $V$ un espacio vectorial. El \textbf{espacio dual} de $V$, denotado $V^*$, es el espacio vectorial de todos los funcionales lineales sobre $V$:
  $$V^* = \{f: V \to \C : f \text{ es lineal}\}$$

  Las operaciones en $V^*$ se definen puntualmente:
  \begin{align}
    (f + g)(x)    & = f(x) + g(x) & \forall x \in V                \\
    (\alpha f)(x) & = \alpha f(x) & \forall x \in V, \alpha \in \C
  \end{align}
\end{defi}

\begin{theo}[Dimensión del espacio dual]
  Si $V$ es un espacio vectorial de dimensión finita $n$ sobre $\C$, entonces $\dim V^* = n$.
\end{theo}

\begin{proof}
  Sea $\{e_1, e_2, \ldots, e_n\}$ una base de $V$. Definimos los funcionales $\{e_1^*, e_2^*, \ldots, e_n^*\}$ mediante:
  $$e_i^*(e_j) = \delta_{ij} = \begin{cases}
      1 & \text{si } i = j    \\
      0 & \text{si } i \neq j
    \end{cases}$$

  Para cualquier $f \in V^*$ y cualquier $x = \sum_{i=1}^n x_i e_i \in V$:
  $$f(x) = f\left(\sum_{i=1}^n x_i e_i\right) = \sum_{i=1}^n x_i f(e_i) = \sum_{i=1}^n f(e_i) e_i^*(x)$$

  Por tanto, $f = \sum_{i=1}^n f(e_i) e_i^*$, lo que muestra que $\{e_1^*, e_2^*, \ldots, e_n^*\}$ genera $V^*$. La independencia lineal se verifica fácilmente, por lo que constituye una base de $V^*$.
\end{proof}

La demostración anterior proporciona una construcción explícita de una base del espacio dual a partir de una base del espacio original, que será útil en aplicaciones prácticas.

\begin{defi}[Base dual]
  Si $\{e_1, e_2, \ldots, e_n\}$ es una base de $V$, la \textbf{base dual} $\{e_1^*, e_2^*, \ldots, e_n^*\}$ de $V^*$ se define por:
  $$e_i^*(e_j) = \delta_{ij}$$
\end{defi}

\subsection{Teorema de representación de Riesz-Fréchet}
Los resultados expuestos a continuación requieren de conocimientos avanzados en análisis funcional y teoría de espacios de Hilbert. Se recomienda consultar las referencias bibliográficas al final del tema para una comprensión más profunda. Se incluyen para entender la importancia del espacio dual y su conexión con la notación de Dirac.

En espacios de Hilbert, existe una correspondencia especial entre el espacio y su dual.

\begin{theo}[Teorema de Riesz-Fréchet]
  Sea $\mathcal{H}$ un espacio de Hilbert complejo con producto interno $\langle \cdot, \cdot \rangle$. Para cada funcional lineal continuo $f \in \mathcal{H}'$, existe un único elemento $y_f \in \mathcal{H}$ tal que:
  $$f(x) = \langle x, y_f \rangle \quad \forall x \in \mathcal{H}$$

  Además, $\|f\|_{\mathcal{H}'} = \|y_f\|_{\mathcal{H}}$.

  La aplicación $\Phi: \mathcal{H} \to \mathcal{H}'$ definida por $\Phi(y)(x) = \langle x, y \rangle$ es un isomorfismo antilineal isométrico.
\end{theo}

\subsection{Aplicaciones computacionales}

\begin{eje}[Mediciones cuánticas como funcionales]
  En mecánica cuántica, una medición de un observable $A$ en un estado $|\psi\rangle$ se puede interpretar como la evaluación de un funcional:

  El valor esperado $\langle A \rangle$ se puede escribir como:
  $$\langle A \rangle = f_A(|\psi\rangle)$$

  donde $f_A$ es el funcional definido por $f_A(|\phi\rangle) = \ket{\phi}^*(A\ket{\phi})$.
\end{eje}

\begin{eje}[Fidelidad como producto en el dual]
  La fidelidad cuántica entre dos estados $|\psi\rangle$ y $|\phi\rangle$ se define como:
  $$F(|\psi\rangle, |\phi\rangle) = |\langle\psi, \phi\rangle|^2$$

  Esto representa la evaluación del funcional $\langle\psi|$ sobre el vector $|\phi\rangle$, seguida de la toma del módulo al cuadrado.
\end{eje}

\subsection{Producto tensorial en espacios vectoriales}

Sean $V$ y $W$ dos espacios vectoriales complejos. El \textbf{producto tensorial} de $V$ y $W$, denotado por $V \otimes W$, es, informalmente, el espacio vectorial "más pequeño" que contiene todos los productos formales de la forma $v \otimes w$, para $v \in V$ y $w \in W$, y que respeta la \textbf{bilinealidad}.

\begin{defi}
  El \textbf{producto tensorial} de dos espacios vectoriales complejos $V$ y $W$, denotado por $V \otimes W$, es el espacio vectorial generado por los productos tensoriales simples $v \otimes w$, donde $v \in V$ y $w \in W$, sujeto a las relaciones de bilinealidad.
\end{defi}

El producto tensorial $V \otimes W$ se construye a partir del espacio vectorial libre $F(V \times W)$ generado por los pares ordenados $(v, w) \in V \times W$, factorizando por el subespacio $R$ generado por las relaciones de \textbf{bilinealidad}:

\begin{enumerate}
  \item \textbf{Linealidad en el primer argumento:}
        $$(\alpha v_1 + \beta v_2, w) - \alpha (v_1, w) - \beta (v_2, w)$$
  \item \textbf{Linealidad en el segundo argumento:}
        $$(v, \alpha w_1 + \beta w_2) - \alpha (v, w_1) - \beta (v, w_2)$$
\end{enumerate}
donde $v, v_1, v_2 \in V$, $w, w_1, w_2 \in W$, y $\alpha, \beta \in \mathbb{K}$.

El \textbf{producto tensorial} es el espacio cociente:
$$V \otimes W := F(V \times W) / R$$

El \textbf{tensor simple} $v \otimes w$ es la clase de equivalencia del par $(v, w)$ en el espacio cociente.

\begin{theo}[Propiedad universal]
  El producto tensorial $V \otimes W$ es un espacio vectorial, junto con una aplicación \textbf{bilineal} $\tau: V \times W \to V \otimes W$, que satisface la siguiente propiedad universal:

  Para cualquier espacio vectorial $Z$ y cualquier aplicación \textbf{bilineal} $f: V \times W \to Z$, existe una \textbf{única} aplicación \textbf{lineal} $\tilde{f}: V \otimes W \to Z$ tal que el diagrama conmuta: $f = \tilde{f} \circ \tau$.
\end{theo}

Esta propiedad es la que asegura que $V \otimes W$ es el espacio vectorial \textbf{generado} por los productos $v \otimes w$ con las mínimas relaciones necesarias para preservar la estructura bilineal.

\subsection{Base canónica}

Al trabajar con espacios vectoriales de dimensión finita, podemos entender cómo se construyen las bases en el producto tensorial y por tanto su estructura.

\begin{prop}
  Si $\{v_i\}_{i=1}^n$ es una base para $V$ y $\{w_j\}_{j=1}^m$ es una base para $W$, entonces el conjunto de \textbf{tensores simples}
  $$
    \mathcal{B}_{V \otimes W} = \{v_i \otimes w_j\}_{i=1,\ldots,n}^{j=1,\ldots,m}
  $$
  es una base para $V \otimes W$.
\end{prop}

Como consecuencia del resultado anterior, todo elemento $t \in V \otimes W$ puede escribirse de manera única como una combinación lineal de los elementos de la base canónica:
$$t = \sum_{i=1}^n \sum_{j=1}^m \alpha_{ij} (v_i \otimes w_j)$$
donde $\alpha_{ij} \in \mathbb{C}$.

Además, la dimensión del espacio tensorial es el producto de las dimensiones de los espacios originales:
$$\text{dim}(V \otimes W) = \text{dim}(V) \cdot \text{dim}(W) = n \cdot m$$


\begin{defi}
  Sea $V$ y $W$ dos espacios vectoriales complejos.
  Un elemento $t \in V \otimes W$ que puede escribirse como $t = v \otimes w$, con $v\in V$ y $w\in W$, se llama \textbf{tensor simple} o \textbf{producto separable}.
\end{defi}

Sin embargo, la mayoría de los elementos de $V \otimes W$ son \textbf{combinaciones lineales} de tensores simples:
$$t = \sum_{k} \alpha_k (v_k \otimes w_k)$$

\begin{nota}
  Un tensor que \textbf{no} puede escribirse como un tensor simple es lo que se conoce en la cuántica como un \textbf{estado entrelazado}.
\end{nota}

\subsection{Propiedades del producto tensorial}

Recordemos que espacios vectoriales de la misma dimensión son isomorfos, pero no \textit{iguales}. Por tanto, las siguientes propiedades se entienden como isomorfismos naturales entre espacios vectoriales.
\begin{itemize}
  \item \textbf{Asociatividad:}
        $$(U \otimes V) \otimes W \cong U \otimes (V \otimes W) \cong U \otimes V \otimes W$$
  \item \textbf{Conmutatividad:}
        $$V \otimes W \cong W \otimes V$$
\end{itemize}

\subsection{Producto tensorial de operadores y matrices}
Podemos extender la definición del producto tensorial a operadores lineales entre espacios vectoriales de manera natural.
\begin{defi}
  Si $A: V_1 \to V_2$ y $B: W_1 \to W_2$ son operadores lineales, podemos definir el \textbf{operador tensorial} $A \otimes B$ en el espacio tensorial $V_1 \otimes W_1$, como
  \begin{align*}
    A \otimes B : V_1 \times W_1                            & \to V_2 \times W_2                                                    \\
    \sum_{i=1}^n \sum_{j=1}^m \alpha_{ij} (v_i \otimes w_j) & \mapsto \sum_{i=1}^n \sum_{j=1}^m \alpha_{ij} (A(v_i) \otimes B(w_j))
  \end{align*}

\end{defi}

De igual manera, podemos definir el producto tensorial de matrices, siendo esta forma la más cómoda de trabajar.

Sean $A \in M_{n \times m}(\C)$ y $B \in M_{p \times q}(\C)$ dos matrices complejas. El producto tensorial matricial $A \otimes B$ es una matriz en $M_{np \times mq}(\C)$ definida por
$$
  (A \otimes B)_{(i,j),(k,l)} = A_{i,k} B_{j,l}
$$
para $1 \leq i \leq n$, $1 \leq j \leq m$, $1 \leq k \leq p$, $1 \leq l \leq q$. Es decir, si
$$A = \begin{pmatrix}
    a_{11} & a_{12} & \ldots & a_{1m} \\
    a_{21} & a_{22} & \ldots & a_{2m} \\
    \vdots & \vdots & \ddots & \vdots \\
    a_{n1} & a_{n2} & \ldots & a_{nm}
  \end{pmatrix} \text{ y } \quad B = \begin{pmatrix}
    b_{11} & b_{12} & \ldots & b_{1q} \\
    b_{21} & b_{22} & \ldots & b_{2q} \\
    \vdots & \vdots & \ddots & \vdots \\
    b_{p1} & b_{p2} & \ldots & b_{pq}
  \end{pmatrix},$$
entonces
$$A \otimes B = \begin{pmatrix}
    a_{11}B & a_{12}B & \ldots & a_{1m}B \\
    a_{21}B & a_{22}B & \ldots & a_{2m}B \\
    \vdots  & \vdots  & \ddots & \vdots  \\
    a_{n1}B & a_{n2}B & \ldots & a_{nm}B
  \end{pmatrix}.$$

\subsection{Sistemas cuánticos compuestos}

\begin{resaltado}
  \textbf{Postulado IV: Composición de sistemas}

  El espacio de estados de un sistema cuántico compuesto, es el formado por el producto tensorial de los subsistemas que lo componen.
\end{resaltado}

\begin{eje}[Sistema de dos cúbits]
  Para dos cúbits independientes, el espacio de estados es $\mathbb{C}^2 \otimes \mathbb{C}^2 \cong \mathbb{C}^4$, con base computacional:
  \begin{align}
    \ket{0} \otimes \ket{0} = \begin{pmatrix} 1 \\ 0 \\ 0 \\ 0 \end{pmatrix} \\
    \ket{0} \otimes \ket{1} = \begin{pmatrix} 0 \\ 1 \\ 0 \\ 0 \end{pmatrix} \\
    \ket{1} \otimes \ket{0} = \begin{pmatrix} 0 \\ 0 \\ 1 \\ 0 \end{pmatrix} \\
    \ket{1} \otimes \ket{1} = \begin{pmatrix} 0 \\ 0 \\ 0 \\ 1 \end{pmatrix}
  \end{align}
\end{eje}

\subsection{Separabilidad y entrelazamiento cuántico}

\begin{defi}[Estado separable]
  Un estado $\ket{\psi} \in \mathcal{H}_A \otimes \mathcal{H}_B$ es \textbf{separable} si existen $\ket{\alpha} \in \mathcal{H}_A$ y $\ket{\beta} \in \mathcal{H}_B$ tales que:
  $$\ket{\psi} = \ket{\alpha} \otimes \ket{\beta}$$

  Un estado es \textbf{entrelazado} si no es separable.
\end{defi}

\begin{eje}[Estados separables]
  Los siguientes estados de dos cúbits son separables:
  \begin{itemize}
    \item $\ket{0} \otimes \ket{0}$
    \item $\frac{1}{\sqrt{2}}(\ket{0}\otimes\ket{1} + \ket{1}\otimes\ket{1}) = \frac{1}{\sqrt{2}}(\ket{0} + \ket{1}) \otimes \ket{1}$
    \item $\frac{1}{2}(\ket{0}\otimes\ket{0} + \ket{0}\otimes\ket{1} + \ket{1}\otimes\ket{0} + \ket{1}\otimes\ket{1}) = \ket{+} \otimes \ket{+}$
  \end{itemize}
\end{eje}

\begin{defi}[Estados de Bell]
  Los cuatro estados de Bell forman una base ortonormal de estados entrelazados para dos cúbits:
  \begin{align}
    \ket{\Phi^+} & = \frac{1}{\sqrt{2}}(\ket{0} \otimes \ket{0} + \ket{1} \otimes \ket{1}) \quad \text{(estado EPR)}      \\
    \ket{\Phi^-} & = \frac{1}{\sqrt{2}}(\ket{0} \otimes \ket{0} - \ket{1} \otimes \ket{1})                                \\
    \ket{\Psi^+} & = \frac{1}{\sqrt{2}}(\ket{0} \otimes \ket{1} + \ket{1} \otimes \ket{0})                                \\
    \ket{\Psi^-} & = \frac{1}{\sqrt{2}}(\ket{0} \otimes \ket{1} - \ket{1} \otimes \ket{0}) \quad \text{(estado singlete)}
  \end{align}
\end{defi}

\begin{eje}[Verificación de entrelazamiento para $\ket{\Phi^+}$]
  Si $\ket{\Phi^+}$ fuera separable, existirían $\ket{\alpha} = a\ket{0} + b\ket{1}$ y $\ket{\beta} = c\ket{0} + d\ket{1}$ tales que:
  $$\ket{\Phi^+} = (a\ket{0} + b\ket{1}) \otimes (c\ket{0} + d\ket{1}) = ac\ket{0}\otimes\ket{0} + ad\ket{0}\otimes\ket{1} + bc\ket{1}\otimes\ket{0} + bd\ket{1}\otimes\ket{1}$$

  Comparando con $\ket{\Phi^+} = \frac{1}{\sqrt{2}}(\ket{0} \otimes \ket{0} + \ket{1} \otimes \ket{1})$:
  $$ac = \frac{1}{\sqrt{2}}, \quad ad = 0, \quad bc = 0, \quad bd = \frac{1}{\sqrt{2}}$$

  De $ad = 0$ y $bc = 0$, o bien $a = c = 0$ o bien $b = d = 0$, pero esto contradice $ac = bd = \frac{1}{\sqrt{2}} \neq 0$.

  Por tanto, $\ket{\Phi^+}$ es entrelazado.
\end{eje}

\begin{info}
  Los estados de Bell exhiben correlaciones cuánticas no locales: medir un cúbit instantáneamente determina el resultado de medir el otro cúbit, independientemente de la distancia que los separe. Esta propiedad es fundamental para protocolos como la teleportación cuántica y la criptografía cuántica.
\end{info}

\begin{theo}[Descomposición de Schmidt]
  Todo estado puro $\ket{\psi} \in \mathcal{H}_A \otimes \mathcal{H}_B$ puede escribirse como:
  $$\ket{\psi} = \sum_{i} \sqrt{\lambda_i} \ket{u_i} \otimes \ket{v_i}$$
  donde $\{\ket{u_i}\}$ y $\{\ket{v_i}\}$ son bases ortonormales, $\lambda_i \geq 0$, y los $\lambda_i$ son únicos.

  El \textbf{número de Schmidt} es la cantidad de $\lambda_i > 0$ y caracteriza el entrelazamiento:
  \begin{itemize}
    \item Número de Schmidt = 1 $\Leftrightarrow$ Estado separable.
    \item Número de Schmidt > 1 $\Leftrightarrow$ Estado entrelazado.
  \end{itemize}
\end{theo}

Una vez que hemos definido el producto tensorial, podemos definir el producto exterior.

\begin{defi}
  Dado un espacio de Hilbert $\mathcal{H}$, se define el \textbf{producto exterior} de dos vectores $u, v \in \mathcal{H}$ por:
  $$u \wedge v = u \otimes v^\dagger\,.$$
\end{defi}

Podemos interpretar el producto exterior como un operador lineal considerando su acción sobre un vector $w \in \mathcal{H}$:
$$ (u \wedge v)(w) = (u \otimes v^\dagger)(w) = \langle v , w \rangle u\,.$$