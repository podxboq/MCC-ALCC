\unirsection{Problemas}

\begin{questions}

  \question Expresar los siguientes estados en notación de Dirac y verificar su normalización:
  \begin{parts}
    \part $(\frac{1}{\sqrt{3}}, \frac{\sqrt{2}}{\sqrt{3}})^t$.
    \part $(\frac{1+i}{2}, \frac{1-i}{2})^t$.
    \part $\frac{1}{2}(1, 1, 1, 1)^t$.
  \end{parts}

  \question Para el estado $\ket{\psi} = \frac{1}{\sqrt{5}}(2\ket{0} + i\ket{1})$:
  \begin{parts}
    \part Calcular las probabilidades de medir $\ket{0}$ y $\ket{1}$.
    \part Calcular las probabilidades de medir $\ket{+}$ y $\ket{-}$.
    \part Determinar el estado después de medir $\ket{+}$.
  \end{parts}

  \question Verificar las siguientes identidades usando notación de Dirac:
  \begin{parts}
    \part $X = \ketbra{0}{1} + \ketbra{1}{0}$.
    \part $Y = -i\ketbra{0}{1} + i\ketbra{1}{0}$.
    \part $Z = \ketbra{0}{0} - \ketbra{1}{1}$.
    \part $I = \ketbra{0}{0} + \ketbra{1}{1}$.
  \end{parts}

  \question Calcular los valores esperados $\expval{X}{\psi}$, $\expval{Y}{\psi}$ y $\expval{Z}{\psi}$ para:
  \begin{parts}
    \part $\ket{\psi} = \ket{0}$.
    \part $\ket{\psi} = \ket{+}$.
    \part $\ket{\psi} = \frac{\ket{0} + i\ket{1}}{\sqrt{2}}$.
  \end{parts}

  \question Para el sistema de dos cúbits en el estado $\ket{\psi} = \frac{1}{\sqrt{3}}(\ket{00} + \ket{01} + \ket{11})$:
  \begin{parts}
    \part Verificar que el estado está normalizado.
    \part Calcular las probabilidades de medir cada estado de la base computacional.
    \part Determinar si el estado es separable o entrelazado.
    \part Calcular la probabilidad de medir el primer cúbit en $\ket{0}$.
  \end{parts}

  \question Para los cuatro estados de Bell:
  \begin{parts}
    \part Verificar que forman una base ortonormal.
    \part Expresar cada estado de Bell como combinación lineal de la base computacional.
    \part Demostrar que todos son maximalmente entrelazados.
    \part Calcular las probabilidades de medir cada cúbit individualmente.
  \end{parts}

  \question Verificar que $\ket{\Phi^+}$ está entrelazado.
  \begin{solution}

    Si $\ket{\Phi^+}$ fuera separable, existirían $\ket{\alpha} = a\ket{0} + b\ket{1}$ y $\ket{\beta} = c\ket{0} + d\ket{1}$ tales que:
    $$\ket{\Phi^+} = (a\ket{0} + b\ket{1}) \otimes (c\ket{0} + d\ket{1}) = ac\ket{0}\otimes\ket{0} + ad\ket{0}\otimes\ket{1} + bc\ket{1}\otimes\ket{0} + bd\ket{1}\otimes\ket{1}$$

    Comparando con $\ket{\Phi^+} = \frac{1}{\sqrt{2}}(\ket{0} \otimes \ket{0} + \ket{1} \otimes \ket{1})$:
    $$ac = \frac{1}{\sqrt{2}}, \quad ad = 0, \quad bc = 0, \quad bd = \frac{1}{\sqrt{2}}$$

    De $ad = 0$ y $bc = 0$, o bien $a = c = 0$ o bien $b = d = 0$, pero esto contradice $ac = bd = \frac{1}{\sqrt{2}} \neq 0$.

    Por tanto, $\ket{\Phi^+}$ es entrelazado.
  \end{solution}

  \question Determine si el estado $\ket{\psi} = \frac{1}{\sqrt{3}}(\ket{00} + \ket{01} + \ket{10})$ es entrelazado.


  \question Para las matrices de Pauli $X$ y $Y$, calcule explícitamente $(X \otimes Y)\ket{01}$.

  \question Demuestre que la puerta CNOT puede crear entrelazamiento aplicándola a estados separables apropiados. Proporcione al menos dos ejemplos específicos.

  \question Para el estado $\ket{\psi} = \cos\frac{\pi}{8}\ket{0} + e^{i\pi/4}\sin\frac{\pi}{8}\ket{1}$:
  \begin{parts}
    \part Calcular las probabilidades de medir $Z$ en los valores $\pm 1$.
    \part Calcular las probabilidades de medir $X$ en los valores $\pm 1$.
    \part Si se mide primero $Z$ y se obtiene $+1$, ¿cuáles son las probabilidades para una medición posterior de $X$?
  \end{parts}

  \question Un cúbit evoluciona bajo el hamiltoniano $\hat{H} = \frac{\pi}{4}Y$:
  \begin{parts}
    \part Calcular el operador de evolución $U(t) = e^{-i\hat{H}t}$.
    \part Si el estado inicial es $\ket{0}$, determinar $\ket{\psi(t)}$.
    \part ¿En qué instante $t$ el estado se convierte en $\ket{1}$?
  \end{parts}

  \question Para la rotación $R_z(\theta) = e^{-i\theta Z/2}$:
  \begin{parts}
    \part Escribir la forma matricial explícita de $R_z(\theta)$.
    \part Demostrar que $R_z(\theta) = \cos\frac{\theta}{2}I - i\sin\frac{\theta}{2}Z$.
    \part Aplicar $R_z(\pi/2)$ al estado $\ket{+}$ y expresar el resultado.
  \end{parts}

  \question Demostrar que la composición de rotaciones alrededor del mismo eje se suma: $$R_z(\alpha)R_z(\beta) = R_z(\alpha + \beta)\,.$$

  \question Dadas las amplitudes cuánticas $\alpha = \frac{2}{3}$ y $\beta = \frac{\sqrt{5}i}{3}$:
  \begin{parts}
    \part Verificar que $|\alpha|^2 + |\beta|^2 = 1$.
    \part Expresar cada amplitud en forma $re^{i\theta}$.
    \part Calcular las probabilidades asociadas a cada amplitud.
  \end{parts}

  \question Un sistema cuántico tiene amplitudes $\alpha_1 = \frac{1}{\sqrt{2}}e^{i\pi/4}$ y $\alpha_2 = \frac{1}{\sqrt{2}}e^{i3\pi/4}$:
  \begin{parts}
    \part Calcular la amplitud total $\alpha_1 + \alpha_2$.
    \part Determinar si hay interferencia constructiva, destructiva o parcial.
    \part Calcular la probabilidad total $|\alpha_1 + \alpha_2|^2$.
  \end{parts}

  \question Tres amplitudes cuánticas tienen la misma magnitud $\frac{1}{\sqrt{3}}$ pero fases diferentes: $0$, $\frac{3\pi}{4}$ y $\frac{3\pi}{2}$:
  \begin{parts}
    \part Escribir las tres amplitudes en forma binomial.
    \part Calcular la suma total de las tres amplitudes expresada en forma binomial.
    \part Explicar por qué el resultado tiene sentido físicamente.
  \end{parts}

  \begin{solution}
    \begin{parts}
      \part
      \begin{itemize}
        \item $\frac{1}{\sqrt{3}} \cdot e^{i \cdot 0} = \frac{1}{\sqrt{3}}$.
        \item $\frac{1}{\sqrt{3}} \cdot e^{i \cdot \frac{3\pi}{4}} = \frac{1}{\sqrt{3}} \left(\cos\left(\frac{3\pi}{4}\right) + i\sin\left(\frac{3\pi}{4}\right)\right) = \frac{1}{\sqrt{3}} \left(-\frac{\sqrt{2}}{2} + i\frac{\sqrt{2}}{2}\right)$.
        \item $\frac{1}{\sqrt{3}} \cdot e^{i \cdot \frac{3\pi}{2}} = \frac{1}{\sqrt{3}} \left(\cos\left(\frac{3\pi}{2}\right) + i\sin\left(\frac{3\pi}{2}\right)\right) = \frac{-i}{\sqrt{3}}$.
      \end{itemize}

      \part
      \begin{align*}
        \frac{1}{\sqrt{3}} \left(1 + \left(-\frac{\sqrt{2}}{2} - i\frac{\sqrt{2}}{2}\right) - i\right) = \frac{1}{\sqrt{3}} \left(\left(1 - \frac{\sqrt{2}}{2}\right) - \left(1 + \frac{\sqrt{2}}{2}\right) i\right)
      \end{align*}

      \part La suma de las tres amplitudes resulta en una amplitud con magnitud menor que la de cada una de las individuales, lo que indica interferencia destructiva parcial. Esto tiene sentido físicamente porque las fases diferentes causan que las ondas asociadas a las amplitudes se cancelen parcialmente entre sí.
    \end{parts}

  \end{solution}

  \question Para las amplitudes $\alpha = ae^{i\phi}$ y $\beta = be^{i\psi}$ con $a, b \in \mathbb{R}^+$:
  \begin{parts}
    \part Demostrar que $|\alpha + \beta|^2 = a^2 + b^2 + 2ab\cos(\psi - \phi)$.
    \part ¿Para qué diferencia de fases se obtiene interferencia máxima?
    \part ¿Para qué diferencia de fases se obtiene interferencia mínima?
  \end{parts}

  \question Dos amplitudes cuánticas $\alpha_1$ y $\alpha_2$ tienen magnitudes $\frac{1}{\sqrt{5}}$ y $\frac{2}{\sqrt{5}}$ respectivamente:
  \begin{parts}
    \part Si sus fases son $\theta_1 = 0$ y $\theta_2 = \pi/3$, calcular $|\alpha_1 + \alpha_2|^2$.
    \part Encontrar el valor de $\theta_2$ que maximiza $|\alpha_1 + \alpha_2|^2$.
    \part Encontrar el valor de $\theta_2$ que minimiza $|\alpha_1 + \alpha_2|^2$.
  \end{parts}

  \question Calcular la magnitud y fase de los siguientes números complejos:
  \begin{parts}
    \part $z_1 = 3 + 4i$.
    \part $z_2 = -2 + 2i\sqrt{3}$.
    \part $z_3 = -5i$.
    \part $z_4 = 7$.
  \end{parts}

\end{questions}
