\portada

\begin{esquemaExplorador}
  \temaEsquema{Estados}{
    \conceptoEsquema{Estados puros}{$\ket{a}$}
    \conceptoEsquema{Estados mixtos}{$\{(p_1, |\psi_1\rangle), \ldots, (p_k, |\psi_k\rangle)\}$}
  }
  \temaEsquema{Matriz de densidad}{
    \conceptoEsquema{Definición y propiedades}{$\sum_{i=1} p_i |\psi_i\rangle\langle\psi_i|$}
    \conceptoEsquema{Operador densidad para estados puros}{$\rho = |\psi\rangle\langle\psi|$}
    \conceptoEsquema{Estados mixtos y mezclas estadísticas}{}
    \conceptoEsquema{Criterio de pureza}{}
  }
  \temaEsquema{Sistemas compuestos}{
    \conceptoEsquema{Producto tensorial de matrices de densidad}{$\rho_{AB} = \rho_A \otimes \rho_B$}
    \conceptoEsquema{Evolución matriz de densidad}{}
  }
  \temaEsquema{Traza parcial}{
    \conceptoEsquema{Traza parcial y sistemas abiertos}{$\text{Tr}_B(\rho_{AB})$}
    \conceptoEsquema{Decoherencia cuántica}{}
  }
  \temaEsquema{Canales cuánticos}{
    \conceptoEsquema{Representación de Kraus}{$\rho' = \sum_k E_k \rho E_k^\dagger$}
    \conceptoEsquema{Operaciones completamente positivas}{}
  }
  \temaEsquema{Información cuántica}{
    \conceptoEsquema{Entropía de von Neumann}{$S(\rho) = -\text{Tr}(\rho \log \rho)$}
    \conceptoEsquema{Información mutua cuántica}{$I(A:B) = S(\rho_A) + S(\rho_B) - S(\rho_{AB})$}
    \conceptoEsquema{Fidelidad cuántica}{$F(\rho, \sigma) = \left(\text{Tr}\sqrt{\sqrt{\rho}\sigma\sqrt{\rho}}\right)^2$}
    \conceptoEsquema{Distancia de traza}{$D(\rho, \sigma) = \frac{1}{2}\text{Tr}|\rho - \sigma|$}
  }
\end{esquemaExplorador}

\unirsection{Ideas clave}

\subsection{Introducción y objetivos}

En los temas anteriores hemos trabajado con el formalismo de vectores estado para describir sistemas cuánticos, donde cada estado se representa mediante un vector unitario en un espacio de Hilbert. Sin embargo, este enfoque tiene limitaciones importantes cuando consideramos sistemas cuánticos realistas que pueden encontrarse en estados de los cuales no tenemos información completa, o cuando estudiamos subsistemas de sistemas cuánticos compuestos.

El formalismo del operador de densidad constituye una parte fundamental de la mecánica cuántica moderna y de la computación cuántica, introducido inicialmente por von Neumann en el marco de la teorı́a cuántica estadı́stica, permitió describir de manera rigurosa otros tipos de sistemas cuánticos que no se pueden representar puramente mediante vectores estado.
El operador densidad es esencial para describir sistemas abiertos, decoherencia y mezclas estadı́sticas, aspectos imprescindibles en la computación
cuántica realista.

Junto con el operador de densidad, la matriz de densidad permite describir procesos de decoherencia, canales cuánticos, entrelazamiento y operaciones sobre sistemas compuestos. En particular, los estados de Bell, constituyen un ejemplo paradigmático de estados puros máximamente entrelazados cuya estructura se comprende de forma natural mediante la matriz de densidad y las trazas parciales.

En este tema analizaremos el formalismo de la matriz de densidad y sus aplicaciones en la teorı́a cuántica estadı́stica y la computación cuántica realista.
Veremos algunos conceptos básicos que:

\begin{itemize}
  \item Permite describir \textbf{estados mixtos} que surgen por decoherencia o incertidumbre clásica.
  \item Facilita el análisis de \textbf{subsistemas cuánticos} mediante la traza parcial.
  \item Proporciona herramientas para cuantificar la \textbf{calidad de estados cuánticos} mediante medidas como la fidelidad.
  \item Es esencial para entender el \textbf{entrelazamiento cuántico} y la separabilidad de estados.
  \item Conecta directamente con la \textbf{teoría de información cuántica} y protocolos de comunicación cuántica.
\end{itemize}

Vamos a establecer las bases matemáticas necesarias para comprender sistemas cuánticos realistas, incluyendo efectos de ruido, decoherencia y la estructura de correlaciones cuánticas en sistemas compuestos.

\subsection{Estados puros y estados mixtos}

\begin{defi}
  Un estado cuántico \textbf{mixto} representa una situación donde el sistema puede estar en uno de varios estados puros $|\psi_i\rangle$ con probabilidades clásicas $p_i$.
  $$\text{Estado mixto} = \{(p_1, |\psi_1\rangle), (p_2, |\psi_2\rangle), \ldots, (p_k, |\psi_k\rangle)\}\,,$$

  donde
  \[
    p_i \geq 0\quad \text{y}\quad \sum_{i=1}^k p_i = 1\,.
  \]
\end{defi}

Esta descripción surge cuando hay incertidumbre clásica sobre cuál es el estado real del sistema.

\begin{eje}
  Considere las siguientes situaciones que requieren estados mixtos:

  \begin{enumerate}
    \item \textbf{Preparación incierta:} Alice prepara un cúbit en el estado $|0\rangle$ con probabilidad $1/2$ y en el estado $|1\rangle$ con probabilidad $1/2$. Bob recibe el cúbit pero no sabe cuál estado fue preparado.

    \item \textbf{Decoherencia:} Un cúbit inicialmente en superposición $\frac{1}{\sqrt{2}}(|0\rangle + |1\rangle)$ interactúa con el ambiente y pierde coherencia, resultando en una mezcla estadística.

    \item \textbf{Subsistema:} Dado un sistema bipartito en estado entrelazado $|\Phi^+\rangle = \frac{1}{\sqrt{2}}(|00\rangle + |11\rangle)$, cada subsistema individual está en un estado mixto.
  \end{enumerate}
\end{eje}

\subsection{Matriz de densidad}

\begin{defi}[Operador densidad]
  El operador densidad de un sistema cuántico mixto $\{(p_i, |\psi_i\rangle)\}$, es el operador $\rho$ definido por
  $$\rho = \sum_{i=1} p_i |\psi_i\rangle\langle\psi_i|\,.$$
\end{defi}

Para un estado puro $|\psi\rangle$ el operador densidad es el proyector
$$\rho = |\psi\rangle\langle\psi|\,.$$

La matriz asociada, que también se denota como $\rho$ en una base ortonormal $\{|e_j\rangle\}$, tiene las entradas $\rho = (\rho)_{mn}$ calculadas como
$$\rho_{mn} = \langle e_m|\rho|e_n\rangle\,.$$

\begin{eje}
  \begin{enumerate}
    \item \textbf{Estado puro} $|0\rangle$:
          $$\rho_0 = |0\rangle\langle 0| = \begin{pmatrix} 1 & 0 \\ 0 & 0 \end{pmatrix}\,.$$

    \item \textbf{Estado de superposición} $|\psi\rangle = \frac{1}{\sqrt{2}}(|0\rangle + |1\rangle)$:
          $$\rho_\psi = |\psi\rangle\langle\psi| = \frac{1}{2}\begin{pmatrix} 1 & 1 \\ 1 & 1 \end{pmatrix}\,.$$

    \item \textbf{Estado mixto equiprobable}:
          $$\rho_{\text{mix}} = \frac{1}{2}|0\rangle\langle 0| + \frac{1}{2}|1\rangle\langle 1| = \frac{1}{2}\begin{pmatrix} 1 & 0 \\ 0 & 1 \end{pmatrix} = \frac{I}{2}\,.$$
  \end{enumerate}
\end{eje}

\begin{prop}
  Sea $\rho = \sum_{i=1}^k p_i \ketbra{\psi_i}$ un operador densidad. Entonces:
  \begin{enumerate}
    \item $\rho$ es hermitiana: $\rho = \rho^\dagger$.
    \item $\rho$ es positiva semidefinida: $\langle\phi|\rho|\phi\rangle \geq 0$ para todo $|\phi\rangle$.
    \item $\text{Tr}(\rho) = 1$ (normalización).
  \end{enumerate}
\end{prop}
\begin{proof}
  \begin{enumerate}
    \item Para demostrar que $\rho$ es hermitiana, tenemos que tener en cuenta las propiedades de la traspuesta conjutada y que $p_i\in \R$, que $\ket{\psi_i}^\dagger = \bra{\psi_i}$ y $\bra{\psi_i}^\dagger = \ket{\psi_i}$ para todo $i$. Entonces
          \[
            \rho^\dagger = \sum_{i=1}^k p_i (\ketbra{\psi_i})^\dagger = \sum_{i=1}^k p_i \bra{\psi_i}^\dagger\ket{\psi_i}^\dagger  = \sum_{i=1}^k p_i \ketbra{\psi_i} = \rho\,.
          \]
    \item Para cualquier ket $\ket{\phi}$ tenemos
          \begin{align*}
            \expval{\rho}{\phi} & = \sum_{i=1}^{k}p_i\braket{\phi}{\psi_i}\braket{\psi_i}{\phi}   = \sum_{i=1}^{k}p_i\braket{\phi}{\psi_i}\braket{\phi_i}{\psi}^* \\
                                & = \sum_{i=1}^{k}p_i|\braket{\phi}{\psi_i}|^2                    \geq 0\,.
          \end{align*}
    \item $\text{Tr}(\rho) = \sum_{i=1}^k p_i = 1$.
  \end{enumerate}
\end{proof}

\begin{theo}[Criterio de pureza]
  Un estado descrito por el operador densidad $\rho$ es puro si y solo si $\text{Tr}(\rho^2) = 1$.

  Equivalentemente, el estado es mixto si y solo si $\text{Tr}(\rho^2) < 1$.
\end{theo}
\begin{proof}
  Como $\rho$ es un operador hermitiano, existe una base ortonormal $\{\ket{i}\}$ en la que $\rho$ es diagonal:
  \[
    \rho = \sum_i \lambda_i \ketbra{i}\,,
  \]
  donde $\lambda_i$ son los valores propios de $\rho$. Por las propiedades de la matriz de densidad, sabemos que $\lambda_i \ge 0$ y $\sum_i \lambda_i = 1$. Esto implica que $0 \le \lambda_i \le 1$ para todo $i$.

  La traza de $\rho^2$ se calcula como:
  \[
    \text{Tr}(\rho^2) = \sum_i \lambda_i^2\,.
  \]

  Dado que $0 \le \lambda_i \le 1$, se cumple que $\lambda_i^2 \le \lambda_i$, dándose la igualdad únicamente si $\lambda_i = 0$ o $\lambda_i = 1$. Sumando sobre todos los índices $i$:
  \[
    \text{Tr}(\rho^2) = \sum_i \lambda_i^2 \le \sum_i \lambda_i = 1\,.
  \]

  La igualdad $\text{Tr}(\rho^2) = 1$ ocurre si y solo si $\lambda_i^2 = \lambda_i$ para todo $i$. Teniendo en cuenta la restricción de normalización $\sum_i \lambda_i = 1$, esto fuerza a que exactamente un valor propio sea 1 (digamos $\lambda_k=1$) y todos los demás sean 0. En tal caso:
  \[
    \rho = \ketbra{k}\,,
  \]
  lo cual corresponde, por definición, a un estado puro. Si $\text{Tr}(\rho^2) < 1$, entonces el estado no es un proyector de rango 1, correspondiendo a una mezcla estadística.
\end{proof}

\begin{eje}
  Sea $\rho = \frac{3}{4}\ketbra{0}{0} + \frac{1}{4}\ketbra{1}{1}$ la matriz de densidad de un estado. Necesitamos saber si el estado es mixto o puro.
  la matriz de densidad es
  \[
    \rho = \begin{pmatrix} 3/4 & 0 \\ 0 & 1/4 \end{pmatrix}\,.
  \]

  Calculamos la traza de $\rho^2$
  \[
    \text{Tr}(\rho^2) = \text{Tr}\begin{pmatrix} (3/4)^2 & 0 \\ 0 & (1/4)^2 \end{pmatrix} = \frac{9}{16} + \frac{1}{16} = \frac{10}{16} = \frac{5}{8} < 1\,.
  \]
  Concluimos que el estado es mixto.
\end{eje}
\subsection{Evolución de la matriz de densidad}

\begin{theo}[Evolución unitaria]
  Si un sistema con matriz de densidad $\rho$ evoluciona bajo un operador unitario $U$, la nueva matriz de densidad es:
  $$\rho' = U\rho U^\dagger$$

  Esta transformación preserva todas las propiedades de un operador densidad válido.
\end{theo}
\begin{proof}
  Si $\rho^\prime$ es la matriz de densidad del sistema después de la evolución unitaria $U$, entonces
  \begin{align*}
    \rho' & = \sum_i p_i \ketbra{U\psi_i}                         \\
          & = \sum_i p_i U \ketbra{\psi_i} U^\dagger              \\
          & = U \left(\sum_i p_i \ketbra{\psi_i}\right) U^\dagger \\
          & = U \rho U^\dagger\,.
  \end{align*}
\end{proof}

\begin{eje}
  Consideremos la matriz de densidad que representa un estado mixto en un cúbit equiprobable $\rho = \frac{1}{2}I$ y apliquemos cualquier puerta cuántica $U$.
  $$\rho' = U\rho U^\dagger = U \cdot \frac{1}{2}I \cdot U^\dagger = \frac{1}{2}UU^\dagger = \frac{1}{2}I = \rho\,.$$

  El estado mixto completamente aleatorio es invariante bajo cualquier operación unitaria.
\end{eje}

\begin{eje}[Aplicación de puertas cuánticas]
  Sea en un cúbit en el estado mixto con matriz de densidad
  \[
    \rho = \begin{pmatrix} 3/4 & 0 \\ 0 & 1/4 \end{pmatrix}\,,
  \]
  y apliquemos la puerta Hadamard $H$ al estado, la evolución dará lugar a la nueva matriz de densidad
  \begin{align*}
    \rho' & = H\rho H^\dagger = \frac{1}{\sqrt{2}}\begin{pmatrix} 1 & 1 \\ 1 & -1 \end{pmatrix}\begin{pmatrix} 3/4 & 0 \\ 0 & 1/4 \end{pmatrix}\frac{1}{\sqrt{2}}\begin{pmatrix} 1 & 1 \\ 1 & -1 \end{pmatrix}^\dagger \\
          & = \frac{1}{2}\begin{pmatrix} 3/4 & 1/4 \\ 3/4 & -1/4 \end{pmatrix}\begin{pmatrix} 1 & 1 \\ 1 & -1 \end{pmatrix}                                                                                            \\
          & = \frac{1}{2}\begin{pmatrix} 1 & 1/2 \\ 1/2 & 1 \end{pmatrix}\,.
  \end{align*}
\end{eje}

\subsection{Representación en la esfera de Bloch}

Para un cúbit $\ket{a}=\alpha\ket{0} + \beta\ket{1}$, el operador densidad es
\[
  \rho = |\alpha|^2 \ketbra{0} + |\beta|^2 \ketbra{1} + \alpha\beta^* \ketbra{0}{1} + \alpha^*\beta \ketbra{1}{0}\,.
\]

En forma matricial y expresado en términos de la base de Pauli, tenemos
\begin{align*}
  \rho & = \begin{pmatrix} |\alpha|^2 & \alpha\beta^* \\ \alpha^*\beta & |\beta|^2 \end{pmatrix}                                                   \\
       & = \frac{1}{2}\begin{pmatrix} 1 + (|\alpha|^2 - |\beta|^2) & 2\alpha\beta^* \\ 2\alpha^*\beta & 1 - (|\alpha|^2 - |\beta|^2) \end{pmatrix} \\
       & = \frac{1}{2}\left[I + (|\alpha|^2 - |\beta|^2)Z + 2\text{Re}(\alpha\beta^*)X + 2\text{Im}(\alpha\beta^*)Y\right]\,.
\end{align*}

Podemos escribir la matriz de densidad de forma más compacta como
$$\rho = \frac{1}{2}(I + \vec{r} \cdot \vec{\sigma})\,,$$
donde $\vec{r} = (r_x, r_y, r_z)$ lo llamamos el \textbf{vector de Bloch} y $\vec{\sigma} = (X, Y, Z)$.

El vector de Bloch admite la siguiente interpretación geométrica:
\begin{itemize}
  \item \textbf{Estados puros:} $|\vec{r}| = 1$ (superficie de la esfera).
  \item \textbf{Estados mixtos:} $|\vec{r}| < 1$ (interior de la esfera).
  \item \textbf{Estado completamente mixto:} $\vec{r} = 0$ (centro de la esfera).
\end{itemize}

El vector de Bloch se puede calcular a partir de la matriz de densidad como
$$\vec{r} = (\text{Tr}(\rho X), \text{Tr}(\rho Y), \text{Tr}(\rho Z))\,.$$

\begin{eje}[Cálculo del vector de Bloch]
  Para el estado $\rho = \frac{3}{4}|0\rangle\langle 0| + \frac{1}{4}|1\rangle\langle 1| = \begin{pmatrix} 3/4 & 0 \\ 0 & 1/4 \end{pmatrix}$, calculamos el vector de Bloch:
  \begin{align*}
    \rho X & = \begin{pmatrix} 3/4 & 0 \\ 0 & 1/4 \end{pmatrix}\begin{pmatrix} 0 & 1 \\ 1 & 0 \end{pmatrix} = \begin{pmatrix} 0 & 3/4 \\ 1/4 & 0 \end{pmatrix}\,.    \\
    \rho Y & = \begin{pmatrix} 3/4 & 0 \\ 0 & 1/4 \end{pmatrix}\begin{pmatrix} 0 & -i \\ i & 0 \end{pmatrix} = \begin{pmatrix} 0 & -3i/4 \\ i/4 & 0 \end{pmatrix}\,. \\
    \rho Z & = \begin{pmatrix} 3/4 & 0 \\ 0 & 1/4 \end{pmatrix}\begin{pmatrix} 1 & 0 \\ 0 & -1 \end{pmatrix} = \begin{pmatrix} 3/4 & 0 \\ 0 & -1/4 \end{pmatrix}\,.
  \end{align*}
  Entonces
  $$\vec{r} = \left(\text{Tr}(\rho X), \text{Tr}(\rho Y), \text{Tr}(\rho Z)\right) = (0, 0, 1/2)\,.$$

  Para calcular si el estado es puro o mixto, observamos que la norma del vector de Bloch es $|\vec{r}| = 1/2 < 1$, confirmando que es un estado mixto.
\end{eje}

\subsection{Sistemas compuestos y traza parcial}

\begin{defi}[Producto tensorial de operadores densidad]
  Para sistemas independientes descritos por matrices de densidad $\rho_A$ y $\rho_B$, el sistema compuesto se describe por
  $$\rho_{AB} = \rho_A \otimes \rho_B\,.$$

  Sin embargo, no todos los estados de sistemas compuestos pueden escribirse en esta forma separable.
\end{defi}

\begin{defi}[Traza parcial]
  Sea $T$ un operador sobre un sistema bipartito $A\otimes B$, la traza parcial de $T$ sobre el subsistema $B$ se define como el operador $\text{Tr}_B(T)$ sobre el subsistema $A$ dado por la propiedad
  $$\matrixelement{\phi}{\text{Tr}_B(T)}{\psi} = \sum_i \matrixelement{\phi \otimes \ket{i}}{T}{\psi \otimes \ket{i}}\,,$$
  donde $\ket{\phi},\ket{\psi}\in A$ y $\{\ket{i}\}$ es cualquier base ortonormal del subsistema $B$.
\end{defi}

\begin{eje}
  Consideremos el estado producto
  $$\ket{\psi} = \ket{0+} = \frac{\ket{00} + \ket{01}}{\sqrt{2}}\,.$$

  La matriz de densidad del sistema compuesto es
  $$\rho_{AB} = \ket{\psi}\bra{\psi} = \ket{0}\bra{0} \otimes \ket{+}\bra{+}\,.$$

  En forma matricial usando la base computacional, es
  $$\rho_{AB} = \frac{1}{2}\begin{pmatrix}
      1 & 1 & 0 & 0 \\
      1 & 1 & 0 & 0 \\
      0 & 0 & 0 & 0 \\
      0 & 0 & 0 & 0
    \end{pmatrix}\,.$$

  Para obtener la matriz de densidad del subsistema $A$, realizamos la traza parcial sobre $B$. Si $\ket{\phi}=a\ket{0}+b\ket{1}$ y $\ket{\psi}=c\ket{0}+d\ket{1}$ son dos kets arbitrarios en el subsistema $A$, calculamos los siguientes términos:
  \begin{equation}
    \label{eq:traza_parcial_eje_1}
    \matrixelement{\phi\otimes \ket{0}}{\rho_{AB}}{\psi\otimes \ket{0}} = \frac{1}{2}ac^*\,,
  \end{equation}
  \begin{equation}
    \label{eq:traza_parcial_eje_2}
    \matrixelement{\phi\otimes \ket{1}}{\rho_{AB}}{\psi\otimes \ket{1}} = \frac{1}{2}ac^*\,.
  \end{equation}

  Por definición de traza parcial, sumamos ambos resultados obtenemos
  \begin{align*}
    \eqref{eq:traza_parcial_eje_1} + \eqref{eq:traza_parcial_eje_2} & = \frac{1}{2}(ac^* + ac^*) = ac^*\,.
  \end{align*}

  Por lo tanto la matriz de densidad del subsistema $A$ es
  \[
    \text{Tr}_B(\rho_{AB}) = \ket{0}\bra{0}\,.
  \]

\end{eje}

\begin{prop}
  Sea $T$ un operador sobre un sistema bipartito $A\otimes B$, definido como el producto tensorial de dos operadores $T_A \otimes T_B$, la traza parcial de $T$ sobre el subsistema $B$ se obtiene como
  \[
    \text{Tr}_B(T_A \otimes T_B) = \text{Tr}(T_B) T_A\,.
  \]
\end{prop}

\begin{eje}
  Volviendo al ejemplo anterior, la matriz de densidad del sistema compuesto es
  $$\rho_{AB} = \frac{1}{2}\begin{pmatrix}
      1 & 1 & 0 & 0 \\
      1 & 1 & 0 & 0 \\
      0 & 0 & 0 & 0 \\
      0 & 0 & 0 & 0
    \end{pmatrix} = \frac{1}{2}\begin{pmatrix}
      1 & 0 \\
      0 & 0
    \end{pmatrix}\otimes \begin{pmatrix}
      1 & 1 \\
      1 & 1
    \end{pmatrix}\,.$$

  Usando el resultado anterior, para obtener la matriz de densidad del subsistema $A$, realizamos la traza parcial sobre $B$.
  \[
    \text{Tr}_B(\rho_{AB}) = \text{Tr}(\rho_B)\rho_A=\begin{pmatrix} 1 & 0 \\ 0 & 0 \end{pmatrix}=\outerproduct{0}{0}\,.
  \]

\end{eje}


\begin{eje}
  Considere el estado entrelazado $|\Phi^+\rangle = \frac{1}{\sqrt{2}}(|00\rangle + |11\rangle)$ con matriz de densidad:
  $$\rho_{AB} = |\Phi^+\rangle\langle\Phi^+| = \frac{1}{2}\begin{pmatrix} 1 & 0 & 0 & 1 \\ 0 & 0 & 0 & 0 \\ 0 & 0 & 0 & 0 \\ 1 & 0 & 0 & 1 \end{pmatrix}$$

  La traza parcial sobre el subsistema $B$ en la base $\{|0\rangle_B, |1\rangle_B\}$:
  \begin{align*}
    \rho_A & = \langle 0|_B \rho_{AB} |0\rangle_B + \langle 1|_B \rho_{AB} |1\rangle_B \\
           & = \frac{1}{2}|0\rangle_A\langle 0|_A + \frac{1}{2}|1\rangle_A\langle 1|_A \\
           & = \frac{1}{2}\begin{pmatrix} 1 & 0 \\ 0 & 1 \end{pmatrix} = \frac{I_A}{2}
  \end{align*}

  El subsistema $A$ está en un estado mixto completamente aleatorio, a pesar de que el sistema total está en un estado puro.
\end{eje}

\begin{prop}
  La operación de traza parcial satisface:
  \begin{enumerate}
    \item Linealidad: $\text{Tr}_B(\alpha\rho + \beta\sigma) = \alpha\text{Tr}_B(\rho) + \beta\text{Tr}_B(\sigma)$.
    \item Preservación de la traza: $\text{Tr}(\text{Tr}_B(\rho_{AB})) = \text{Tr}(\rho_{AB})$.
    \item $\text{Tr}_B(\rho_{AB}) = \rho_A \text{Tr}(\rho_B)$.
  \end{enumerate}
\end{prop}

\subsection{Decoherencia cuántica}

La \textbf{decoherencia cuántica} es el proceso por el cual un sistema cuántico pierde su coherencia debido a interacciones incontroladas con el entorno, transformando estados puros en estados mixtos.

\begin{eje}[Decoherencia de un cúbit]
  Considere un cúbit inicialmente en el estado de superposición $|\psi\rangle = \frac{1}{\sqrt{2}}(|0\rangle + |1\rangle)$ que sufre decoherencia de desfase. Después de un tiempo $t$, el estado viene representado por la matriz de densidad
  $$\rho(t) = \frac{1}{2}\begin{pmatrix} 1 & e^{-\gamma t} \\ e^{-\gamma t} & 1 \end{pmatrix}\,.$$
  donde $\gamma > 0$ es la tasa de decoherencia.
  \begin{itemize}
    \item En $t = 0$: $\rho(0) = |\psi\rangle\langle\psi|$, estado puro.
    \item Cuando $t \to \infty$: $\rho(\infty) = \frac{I}{2}$, estado maximalmente mixto.
    \item La pureza decae como: $\text{Tr}(\rho(t)^2) = \frac{1}{2}(1 + e^{-2\gamma t})$.
  \end{itemize}
\end{eje}

\subsection{Canales cuánticos}

\begin{defi}[Canal cuántico]
  Un canal cuántico es una aplicación completamente positiva que preserva la traza
  $$\mathcal{E}: \mathcal{L}(\mathcal{H}_A) \to \mathcal{L}(\mathcal{H}_B)$$

  Esta aplicación describe la evolución más general posible de un sistema cuántico abierto.
\end{defi}

\begin{theo}[Representación de Kraus]
  Todo canal cuántico $\mathcal{E}$ puede representarse mediante operadores de Kraus $\{K_i\}$
  $$\mathcal{E}(\rho) = \sum_i K_i \rho K_i^\dagger\,,$$

  donde los operadores satisfacen la condición de completitud
  $$\sum_i K_i^\dagger K_i = I\,.$$
\end{theo}

\begin{eje}[Canal de despolarización]
  El canal de despolarización para un cúbit con probabilidad $p$ tiene la forma:
  $$\mathcal{E}(\rho) = (1-p)\rho + \frac{p}{3}(X \rho X + Y \rho Y + Z \rho Z)$$

  Los operadores de Kraus son:
  $$K_0 = \sqrt{1-p}I, \quad K_1 = \sqrt{\frac{p}{3}}X, \quad K_2 = \sqrt{\frac{p}{3}}Y, \quad K_3 = \sqrt{\frac{p}{3}}Z\,.$$
\end{eje}

\subsection{Medidas de información cuántica}

\begin{defi}[Entropía de von Neumann]
  La entropía de von Neumann de un estado cuántico descrito por la matriz de densidad $\rho$ se define como
  $$S(\rho) = -\text{Tr}(\rho \log_2 \rho)\,.$$
\end{defi}

Si $\rho$ tiene descomposición espectral $\rho = \sum_i \lambda_i |e_i\rangle\langle e_i|$, entonces
$$S(\rho) = -\sum_i \lambda_i \log_2 \lambda_i$$

La entropía cuantifica el grado de "mezcla" o "desorden" del estado cuántico.

\begin{prop}
  \begin{enumerate}
    \item $S(\rho) \geq 0$ para toda matriz de densidad $\rho$.
    \item $S(\rho) = 0$ si y solo si $\rho$ representa un estado puro.
    \item Para un sistema de dimensión $d$: $S(\rho) \leq \log_2 d$, con igualdad si y solo si $\rho = \frac{I}{d}$.
    \item La entropía es invariante bajo transformaciones unitarias: $S(U\rho U^\dagger) = S(\rho)$.
    \item Subaditividad: $S(\rho_{AB}) \leq S(\rho_A) + S(\rho_B)$.
  \end{enumerate}
\end{prop}

\begin{eje}[Cálculo de entropía]
  Calculemos la entropía de von Neumann para los siguientes estados:
  \begin{enumerate}
    \item \textbf{Estado puro:} Para cualquier $|\psi\rangle$, $S(|\psi\rangle\langle\psi|) = 0$.

    \item \textbf{Estado mixto equiprobable:} Para $\rho = \frac{I}{2}$ en un cúbit
          $$S(\rho) = -\frac{1}{2}\log_2\frac{1}{2} - \frac{1}{2}\log_2\frac{1}{2} = 1\,.$$

    \item \textbf{Estado mixto general:} Para $\rho = p|0\rangle\langle 0| + (1-p)|1\rangle\langle 1|$
          $$S(\rho) = -p\log_2 p - (1-p)\log_2(1-p) = H(p)\,,$$
          donde $H(p)$ es la entropía de Shannon clásica.
  \end{enumerate}
\end{eje}

\begin{defi}[Información mutua cuántica]
  Para un estado bipartito $\rho_{AB}$, la información mutua cuántica se define como:
  $$I(A:B) = S(\rho_A) + S(\rho_B) - S(\rho_{AB})$$

  Esta cantidad mide las correlaciones totales (clásicas y cuánticas) entre los subsistemas $A$ y $B$.
\end{defi}

\begin{defi}[Fidelidad cuántica]
  La fidelidad entre dos estados cuánticos descritos por matrices de densidad $\rho$ y $\sigma$ se define como:
  $$F(\rho, \sigma) = \text{Tr}\left(\sqrt{\sqrt{\rho}\sigma\sqrt{\rho}}\right)$$

  Para estados puros $|\psi\rangle$ y $|\phi\rangle$:
  $$F(|\psi\rangle, |\phi\rangle) = |\langle\psi|\phi\rangle|$$

  La fidelidad mide qué tan cercanos están dos estados cuánticos.
\end{defi}

\begin{prop}
  \begin{enumerate}
    \item $0 \leq F(\rho, \sigma) \leq 1$.
    \item $F(\rho, \sigma) = F(\sigma, \rho)$ (simetría).
    \item $F(\rho, \rho) = 1$ (normalización).
    \item $F(\rho, \sigma) = 1$ si y solo si $\rho = \sigma$.
    \item Para operadores unitarios: $F(U\rho U^\dagger, U\sigma U^\dagger) = F(\rho, \sigma)$.
  \end{enumerate}
\end{prop}

\begin{defi}[Distancia de traza]
  La distancia de traza entre dos matrices de densidad $\rho$ y $\sigma$ se define como:
  $$D(\rho, \sigma) = \frac{1}{2}\text{Tr}(|\rho - \sigma|)$$

  donde $|A| = \sqrt{A^\dagger A}$ es el valor absoluto del operador $A$.
\end{defi}

\begin{prop}
  \begin{enumerate}
    \item $0 \leq D(\rho, \sigma) \leq 1$.
    \item $D(\rho, \sigma) = 0$ si y solo si $\rho = \sigma$.
    \item $D(\rho, \sigma) = D(\sigma, \rho)$ (simetría).
    \item $D(\rho, \tau) \leq D(\rho, \sigma) + D(\sigma, \tau)$ (desigualdad triangular).
    \item Para operadores unitarios: $D(U\rho U^\dagger, U\sigma U^\dagger) = D(\rho, \sigma)$.
  \end{enumerate}
\end{prop}

\begin{theo}[Relación entre fidelidad y distancia de traza]
  Para cualesquiera dos matrices de densidad $\rho$ y $\sigma$:
  $$1 - F(\rho, \sigma) \leq D(\rho, \sigma) \leq \sqrt{2(1 - F(\rho, \sigma))}$$

  Ambas medidas proporcionan información complementaria sobre la proximidad entre estados cuánticos.
\end{theo}

\subsection{Aplicaciones en computación cuántica}

\begin{eje}[Fidelidad en protocolos cuánticos]
  En el protocolo de teleportación cuántica, la fidelidad entre el estado original $|\psi\rangle$ y el estado reconstruido $\rho_{\text{out}}$ mide la calidad del protocolo:
  $$F = \langle\psi|\rho_{\text{out}}|\psi\rangle$$

  Para teleportación perfecta, $F = 1$. En presencia de ruido o errores de medición, $F < 1$, y el protocolo es útil solo si $F > 2/3$ (límite clásico).
\end{eje}

\begin{eje}[Caracterización del entrelazamiento]
  Para un estado bipartito $\rho_{AB}$, el grado de entrelazamiento puede cuantificarse mediante:

  \begin{itemize}
    \item \textbf{Entropía de entrelazamiento:} $E(\rho_{AB}) = S(\rho_A) = S(\rho_B)$ para estados puros.
    \item \textbf{Negatividad:} Basada en la traza de la transposición parcial.
    \item \textbf{Concurrencia:} Medida específica para sistemas de dos cúbits.
  \end{itemize}

  Estas medidas son fundamentales para protocolos de información cuántica como criptografía cuántica y computación distribuida.
\end{eje}