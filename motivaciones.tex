\documentclass[]{unirTema}

\newcommand{\tq}{\mid}
\newcommand{\K}{\mathrm{K}}
\newcommand{\V}{\mathrm{V}}
\newcommand{\N}{\mathbb{N}}
\newcommand{\Z}{\mathbb{Z}}
\newcommand{\F}{\mathbb{F}}
\newcommand{\Q}{\mathbb{Q}}
\newcommand{\R}{\mathbb{R}}
\newcommand{\C}{\mathbb{C}}
\renewcommand{\H}{\mathcal{H}}
\newcommand{\Cinf}{\mathcal{C}^\infty}
\newcommand{\Cu}{\mathcal{C}^1}
\newcommand{\Rp}{\mathfrak{Re}}
\newcommand{\Ip}{\mathfrak{Im}}
\renewcommand{\d}{\mathrm{d}}
\newcommand{\dm}{\mathrm{d}\mu}
\newcommand{\conj}[1]{\overline{#1}}
\newcommand{\Tras}[1]{{#1}^{\text{T}}}

\DeclareMathOperator{\Log}{Log}
\DeclareMathOperator{\Arg}{Arg}
\DeclareMathOperator{\Dom}{Dom}
\DeclareMathOperator{\Ima}{Im}
\DeclareMathOperator{\sgn}{sgn}
\DeclareMathOperator{\mcd}{MCD}
\DeclareMathOperator{\mcm}{mcm}
\DeclareMathOperator{\Resi}{Res}
\DeclareMathOperator{\Ker}{Ker}
\DeclareMathOperator{\End}{End}
\DeclareMathOperator{\Mat}{Mat}

\newcommand{\parder}[2]{\frac{\partial #1}{\partial #2}}
\newcommand{\dparder}[2]{\dfrac{\partial #1}{\partial #2}}
\newcommand{\tparder}[2]{\partial #1/\partial #2}
\newcommand{\parderr}[3]{\frac{\partial^2 #1}{\partial #2\partial #3}}
\newcommand{\dparderr}[3]{\dfrac{\partial^2 #1}{\partial #2\partial #3}}
\newcommand{\tparderr}[3]{\partial^2 #1/\partial #2\partial #3}
\newcommand{\intx}[1]{\int #1\,dx}
\newcommand{\intt}[1]{\int #1\,dt}
\newcommand{\intdx}[3]{\int_{#1}^{#2} #3\,dx}
\newcommand{\intdt}[3]{\int_{#1}^{#2} #3\,dt}
\newcommand{\intdz}[2]{\int_{#1} #2\,dz}
\newcommand{\set}[1]{\left\{#1\right\}}
\newcommand{\so}{\Rightarrow}
\newcommand{\sii}{\Leftrightarrow}
\newcommand{\by}[1]{\overset{\fbox{\tiny #1}}{=}}
\newcommand{\byref}[1]{\overset{\fbox{\tiny\ref{#1}}}{=}}
\newcommand{\cardinal}[1]{\left|#1\right|}
\newcommand{\maps}[3]{#1 \colon #2\longrightarrow #3}
\newcommand{\equationmaps}[5]{\begin{aligned}[t]#1 \colon #2 &\longrightarrow #3 \\	#4 &\longmapsto #5\end{aligned}}
\newcommand{\coma}{,\thinspace}
\newcommand{\pari}[2]{(#1,\thinspace #2)}
\newcommand{\where}{\mathrel{}\middle|\mathrel{}}
\newcommand{\no}[1]{{\neg}{#1}}
\newcommand{\dcomilla}[1]{{\guillemotleft}#1{\guillemotright}}
\newcommand{\separa}{\vspace*{.75\baselineskip}}
\newcommand{\semisepara}{\vspace*{.25\baselineskip}}
\newcommand{\restrict}[1]{\raisebox{-.5ex}{$|$}_{#1}}

% ========================================
% COMANDOS PERSONALIZADOS PARA COMPUTACIÓN CUÁNTICA
% ========================================

% Estados comunes
\newcommand{\zero}{\ket{0}}
\newcommand{\one}{\ket{1}}
\newcommand{\plus}{\ket{+}}
\newcommand{\minus}{\ket{-}}

% Operadores especiales
\newcommand{\tensor}{\otimes}         % Producto tensorial
\newcommand{\comp}{\circ}             % Composición

% Estados de Bell
\newcommand{\bellphi}{\ket{\Phi^+}}
\newcommand{\bellpsi}{\ket{\Psi^+}}
\newcommand{\bellphiminus}{\ket{\Phi^-}}
\newcommand{\bellpsiminus}{\ket{\Psi^-}}

\newcommand{\floor}[1]{\left\lfloor #1 \right\rfloor}
\newcommand{\ceil}[1]{\left\lceil #1 \right\rceil}

\printanswers

\begin{document}
\author{Francisco Costa Cano}
\titulacion{Máster en computación cuántica}
\asignatura{Álgebra lineal en computación cuántica}
\bloque{1}{Fundamentos matemáticos}
\tema{1}{Números complejos y su geometría}

\begin{questions}

  \question \textbf{Números complejos y su geometría.}

  ¿Es realmente necesario usar números complejos para describir la mecánica cuántica, o son simplemente un artificio matemático conveniente?

  Los números complejos representan mucho más que una extensión abstracta de los números reales. En el contexto de la mecánica cuántica, se convierten en una herramienta matemática indispensable para capturar la naturaleza profunda de los sistemas físicos a escala atómica.

  \begin{solution}
    A lo largo de este tema, exploraremos por qué los números complejos no son un mero artificio matemático, sino una herramienta matemática necesaria para comprender la realidad física a nivel cuántico.
  \end{solution}

  \question \textbf{Espacios vectoriales complejos.}

  ¿Puede un vector representar algo más que una dirección y magnitud en el espacio físico?

  Los espacios vectoriales complejos nos invitan a expandir nuestra comprensión de lo que significa un vector. Más allá de ser simples flechas en un plano, los vectores se convierten en objetos matemáticos sofisticados capaces de codificar información cuántica fundamental.

  \begin{solution}

    \begin{itemize}
      \item \textbf{Vectores como estados cuánticos:} En mecánica cuántica, un vector no solo representa una dirección, sino un estado completo de un sistema físico.

      \item \textbf{Superposición y complejidad:} Los vectores en espacios complejos pueden representar estados superpuestos, donde un sistema existe simultáneamente en múltiples configuraciones.

      \item \textbf{Información probabilística:} Cada componente vectorial lleva consigo información sobre probabilidades de medición y comportamiento cuántico.
    \end{itemize}

  \end{solution}
  \question \textbf{Operadores lineales y representación matricial.}

  ¿Por qué no todas las matrices pueden transformar estados cuánticos?

  A primera vista, las matrices parecen herramientas matemáticas neutrales, capaces de transformar cualquier dato numérico. Sin embargo, en el mundo cuántico, las transformaciones no son tan simples ni arbitrarias.

  \begin{solution}
    Los operadores lineales en computación cuántica son mucho más que simples transformaciones matriciales. Representan las únicas formas físicamente permitidas de manipular información cuántica, con restricciones fundamentales que reflejan la naturaleza profunda de los sistemas cuánticos.

    \begin{itemize}
      \item \textbf{Conservación de probabilidad:} Los operadores cuánticos deben preservar la suma total de probabilidades en un sistema.

      \item \textbf{Reversibilidad:} Las transformaciones cuánticas deben ser reversibles, imponiendo condiciones estrictas a los operadores.

      \item \textbf{Unitariedad:} Igualmente estas operaciones deben ser representadas por matrices unitarias, que conservan la norma del vector de estado.
    \end{itemize}
  \end{solution}

  \question ¿Cómo capturan los espacios de Hilbert la incertidumbre y probabilidad inherentes a la mecánica cuántica?

  La mecánica cuántica nos presenta un mundo donde la certeza absoluta es imposible. Los sistemas cuánticos no se describen mediante valores fijos, sino mediante distribuciones de probabilidad que evolucionan de formas sorprendentes y contraintuitivas.

  \begin{solution}
    En los espacios de Hilbert, la incertidumbre no es un defecto, sino una característica fundamental:

    \begin{enumerate}
      \item Los vectores representan distribuciones de probabilidad
      \item Los operadores transforman estas distribuciones
      \item Las mediciones colapsan estados de probabilidad
      \item La ortogonalidad captura relaciones probabilísticas entre estados
    \end{enumerate}

    La mecánica cuántica nos enseña que la incertidumbre no es un límite de nuestro conocimiento, sino una propiedad intrínseca de la naturaleza, y los espacios de Hilbert son el lenguaje matemático que nos permite describirla.
  \end{solution}


  %\bloque{2}{Espacios cuánticos}
  %\question \textbf{Espacios de Hilbert}
  %\question \textbf{Espacio dual y producto tensorial}
  %\question \textbf{Notación de Dirac}
  %\bloque{3}{Operadores y dinámicas}
  %\question \textbf{Operadores lineales en computación cuántica}
  %\question \textbf{Estados probabilísticos}
\end{questions}
\end{document}