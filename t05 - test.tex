\begin{questions}

  \question La dimensión del espacio dual $V^*$ de un espacio vectorial $V$ de dimensión finita $n$ es:

  \begin{choices}
    \choice $n^2$
    \choice $2n$
    \CorrectChoice $n$
    \choice $n-1$
  \end{choices}
  \begin{solution}
    El espacio dual de un espacio de dimensión finita tiene la misma dimensión que el espacio original: $\dim(V^*) = \dim(V) = n$.
  \end{solution}

  \question Si $\{e_1, e_2\}$ es una base de $V$, la base dual $\{e_1^*, e_2^*\}$ satisface:

  \begin{choices}
    \choice $e_1^*(e_1) = 0$
    \CorrectChoice $e_i^*(e_j) = \delta_{ij}$
    \choice $e_1^*(e_2) = 1$
    \choice $e_1^* = e_2^*$
  \end{choices}
  \begin{solution}
    Por definición de base dual: $e_i^*(e_j) = \delta_{ij}$, donde $\delta_{ij} = 1$ si $i = j$ y $0$ si $i \neq j$.
  \end{solution}

  \question La dimensión del producto tensorial $\C^2 \otimes \C^3$ es:

  \begin{choices}
    \choice $5$
    \CorrectChoice $6$
    \choice $8$
    \choice $9$
  \end{choices}
  \begin{solution}
    $\dim(V \otimes W) = \dim(V) \cdot \dim(W) = 2 \cdot 3 = 6$.
  \end{solution}

  \question El estado $\ket{\psi} = \frac{1}{\sqrt{2}}(\ket{00} + \ket{11})$ es:

  \begin{choices}
    \choice Separable
    \CorrectChoice Entrelazado
    \choice Un estado de un solo qubit
    \choice No está normalizado
  \end{choices}
  \begin{solution}
    No puede escribirse como $\ket{\psi}_A \otimes \ket{\phi}_B$, por tanto es entrelazado. Si fuera separable, tendríamos $(a\ket{0} + b\ket{1}) \otimes (c\ket{0} + d\ket{1}) = ac\ket{00} + ad\ket{01} + bc\ket{10} + bd\ket{11}$, lo que requeriría $ad = bc = 0$ pero $ac = bd = \frac{1}{\sqrt{2}}$, lo cual es imposible.
  \end{solution}

  \question El producto $(e_1\otimes e_2)(X \otimes Y)$ donde $X$ e $Y$ son matrices de Pauli es igual a:

  \begin{choices}
    \choice $e_2 \otimes e_1$
    \choice $e_1 \otimes e_2$
    \CorrectChoice $-ie_2 \otimes e_1$
    \choice $ie_1 \otimes e_2$
  \end{choices}
  \begin{solution}
    $(e_1\otimes e_2)(X \otimes Y)(e_1 X) \otimes (e_2 Y) = e_2 \otimes (-ie_1) = -ie_2 \otimes e_1$.
  \end{solution}

  \question El producto externo $u\wedge v$ es:

  \begin{choices}
    \choice Un número complejo
    \choice Un vector
    \CorrectChoice Un operador lineal
    \choice Una matriz diagonal
  \end{choices}
  \begin{solution}
    El producto externo es un operador lineal que actúa como $(u\wedge v)(w) = \left<v, w\right>u$.
  \end{solution}

  \question El teorema de Riesz-Fréchet establece que en un espacio de Hilbert:

  \begin{choices}
    \choice El espacio dual es de mayor dimensión
    \CorrectChoice Cada funcional lineal continuo puede representarse como producto interno
    \choice No existe isomorfismo con el dual
    \choice Los funcionales son todos discontinuos
  \end{choices}
  \begin{solution}
    El teorema de Riesz-Fréchet establece que para cada funcional lineal continuo $f$ existe único $y_f$ tal que $f(x) = \langle x, y_f \rangle$.
  \end{solution}

  \question Para un espacio vectorial de dimensión $2$, el producto tensorial $n$ veces tiene dimensión:

  \begin{choices}
    \choice $n$
    \choice $n^2$
    \CorrectChoice $2^n$
    \choice $2n$
  \end{choices}
  \begin{solution}
    El espacio tensorial $n$ veces tiene tiene dimensión $2^n$.
  \end{solution}

  \question El producto externo $u \wedge v$ puede interpretarse como:
  \begin{choices}
    \choice Una proyección sobre $u$
    \choice Una proyección sobre $v$
    \CorrectChoice Una transformación que proyecta sobre $u$ según la componente en $v$
    \choice Una transformación que proyecta sobre $v$ según la componente en $u$
  \end{choices}
  \begin{solution}
    El producto externo actúa como $(u \wedge v)(w) = \left<v, w\right> u$, proyectando $w$ sobre $v$ y escalando $u$ por ese valor.
  \end{solution}

  \question Si $\{e_1, e_2\}$ es una base de $V$, la base dual $\{e_1^*, e_2^*\}$ está definida por:
  \begin{choices}
    \choice $e_i^*(e_j) = 1$ para todo $i,j$
    \choice $e_i^*(e_j) = 0$ para todo $i,j$
    \CorrectChoice $e_i^*(e_j) = \delta_{ij}$
    \choice $e_1^* = e_2^*$
  \end{choices}
  \begin{solution}
    La base dual está definida por la relación $e_i^*(e_j) = \delta_{ij}$, donde $\delta_{ij}$ es el delta de Kronecker.
  \end{solution}

\end{questions}