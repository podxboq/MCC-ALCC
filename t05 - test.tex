\begin{questions}

  \question La dimensión del espacio dual $V^*$ de un espacio vectorial $V$ de dimensión finita $n$ es:

  \begin{choices}
    \choice $n^2$
    \choice $2n$
    \CorrectChoice $n$
    \choice $n-1$
  \end{choices}
  \begin{solution}
    El espacio dual de un espacio de dimensión finita tiene la misma dimensión que el espacio original: $\dim(V^*) = \dim(V) = n$.
  \end{solution}

  \question Si $\{e_1, e_2\}$ es una base de $V$, la base dual $\{e_1^*, e_2^*\}$ satisface:

  \begin{choices}
    \choice $e_1^*(e_1) = 0$
    \CorrectChoice $e_i^*(e_j) = \delta_{ij}$
    \choice $e_1^*(e_2) = 1$
    \choice $e_1^* = e_2^*$
  \end{choices}
  \begin{solution}
    Por definición de base dual: $e_i^*(e_j) = \delta_{ij}$, donde $\delta_{ij} = 1$ si $i = j$ y $0$ si $i \neq j$.
  \end{solution}

  \question La dimensión del producto tensorial $\C^2 \otimes \C^3$ es:

  \begin{choices}
    \choice $5$
    \CorrectChoice $6$
    \choice $8$
    \choice $9$
  \end{choices}
  \begin{solution}
    $\dim(V \otimes W) = \dim(V) \cdot \dim(W) = 2 \cdot 3 = 6$.
  \end{solution}

  \question El estado $\ket{\psi} = \frac{1}{\sqrt{2}}(\ket{00} + \ket{11})$ es:

  \begin{choices}
    \choice Separable
    \CorrectChoice Entrelazado
    \choice Un estado de un solo qubit
    \choice No está normalizado
  \end{choices}
  \begin{solution}
    No puede escribirse como $\ket{\psi}_A \otimes \ket{\phi}_B$, por tanto es entrelazado. Si fuera separable, tendríamos $(a\ket{0} + b\ket{1}) \otimes (c\ket{0} + d\ket{1}) = ac\ket{00} + ad\ket{01} + bc\ket{10} + bd\ket{11}$, lo que requeriría $ad = bc = 0$ pero $ac = bd = \frac{1}{\sqrt{2}}$, lo cual es imposible.
  \end{solution}

  \question Los cuatro estados de Bell forman:

  \begin{choices}
    \choice Una base de $\C^2$
    \CorrectChoice Una base de $\C^4$
    \choice Un conjunto linealmente dependiente
    \choice Estados separables
  \end{choices}
  \begin{solution}
    Los cuatro estados de Bell $\{\ket{\Phi^+}, \ket{\Phi^-}, \ket{\Psi^+}, \ket{\Psi^-}\}$ forman una base ortonormal de $\C^4 = \C^2 \otimes \C^2$.
  \end{solution}

  \question El producto $(X \otimes Y)\ket{01}$ donde $X$ e $Y$ son matrices de Pauli es igual a:

  \begin{choices}
    \choice $\ket{01}$
    \choice $\ket{10}$
    \CorrectChoice $-i\ket{10}$
    \choice $i\ket{01}$
  \end{choices}
  \begin{solution}
    $(X \otimes Y)\ket{01} = (X\ket{0}) \otimes (Y\ket{1}) = \ket{1} \otimes (-i\ket{0}) = -i\ket{10}$.
  \end{solution}

  \question El número de Schmidt del estado $\ket{\Phi^+} = \frac{1}{\sqrt{2}}(\ket{00} + \ket{11})$ es:

  \begin{choices}
    \choice $1$
    \CorrectChoice $2$
    \choice $3$
    \choice $4$
  \end{choices}
  \begin{solution}
    La descomposición de Schmidt ya está dada: $\ket{\Phi^+} = \frac{1}{\sqrt{2}}\ket{0}\otimes\ket{0} + \frac{1}{\sqrt{2}}\ket{1}\otimes\ket{1}$. Hay dos términos no nulos, por tanto el número de Schmidt es 2.
  \end{solution}

  \question El producto exterior $\ket{\psi}\bra{\phi}$ es:

  \begin{choices}
    \choice Un número complejo
    \choice Un vector
    \CorrectChoice Un operador lineal
    \choice Una matriz diagonal
  \end{choices}
  \begin{solution}
    El producto exterior $\ket{\psi}\bra{\phi}$ es un operador lineal que actúa como $(\ket{\psi}\bra{\phi})\ket{\chi} = \braket{\phi}{\chi}\ket{\psi}$.
  \end{solution}

  \question El teorema de Riesz-Fréchet establece que en un espacio de Hilbert:

  \begin{choices}
    \choice El espacio dual es de mayor dimensión
    \CorrectChoice Cada funcional lineal continuo puede representarse como producto interno
    \choice No existe isomorfismo con el dual
    \choice Los funcionales son todos discontinuos
  \end{choices}
  \begin{solution}
    El teorema de Riesz-Fréchet establece que para cada funcional lineal continuo $f$ existe único $y_f$ tal que $f(x) = \langle x, y_f \rangle$.
  \end{solution}

  \question Para un sistema de $n$ qubits, el espacio de estados tiene dimensión:

  \begin{choices}
    \choice $n$
    \choice $n^2$
    \CorrectChoice $2^n$
    \choice $2n$
  \end{choices}
  \begin{solution}
    El espacio de estados de $n$ qubits es $(\C^2)^{\otimes n} \cong \C^{2^n}$, que tiene dimensión $2^n$.
  \end{solution}

\end{questions}