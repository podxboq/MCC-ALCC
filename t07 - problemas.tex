
\unirsection{Problemas}

\begin{questions}

  \question Demuestre que las siguientes matrices son unitarias:
  \begin{parts}
    \part $U_1 = \frac{1}{\sqrt{2}}\begin{pmatrix} 1 & i \\ i & 1 \end{pmatrix}$
    \part $U_2 = \frac{1}{\sqrt{3}}\begin{pmatrix} 1 & \sqrt{2} \\ \sqrt{2} & -1 \end{pmatrix}$
  \end{parts}


  \question Exprese la puerta $Y$ como producto de rotaciones $R_x$ y $R_z$.

  \question Construya explícitamente la matriz $4 \times 4$ para la puerta Controlled-$R_y(\pi/3)$.

  \question Un qubit inicialmente en estado $\ket{+}$ evoluciona bajo el Hamiltoniano $H = \sigma_x + \sigma_z$.
  \begin{parts}
    \part Calcule $e^{-iHt}$
    \part Determine $\ket{\psi(t)}$
    \part ¿Cuál es el período de oscilación?
  \end{parts}

  \question Utilizando la notación de Dirac, demuestre que la puerta CSWAP puede escribirse como
  \[
    \text{CSWAP} = \ket{0}\bra{0} \otimes I \otimes I + \ket{1}\bra{1} \otimes \text{SWAP}\,.
  \]

  \question Demuestre que la puerta CNOT puede construirse a partir de puertas Hadamard y CZ.

  \question Muestre que la puerta SWAP puede construirse a partir de tres puertas CNOT.

  \question Comprueba que la matriz de Hadamard es equivalente a $\dfrac{X+Z}{\sqrt{2}}$.

  \question Demuestra que todo operador unitario y hermítico tiene autovalores $\pm 1$.

  \question Demuestra las siguientes igualdades:
  \begin{parts}
    \part $H X H = Z$.
    \part $H Z H = X$.
    \part $H Y H = -Y$.
    \part $X^2_1 = H_1 H_2 X^1_2 H_1 H_2$.
    \part $Z^2_1 = Z^1_2$.
  \end{parts}

  \question Considere el siguiente operador unitario sobre 3 cúbits
  \[
    U = H_1(\text{SWAP})^1_{2,3}H_1\,.
  \]

  Dado dos cúbits $\ket{a}$ y $\ket{b}$,
  \begin{parts}
    \part ¿cuál es el resultado de aplicar $U$ al estado $\ket{0ab}$?
    \part ¿Cual es la probabilidad de medir el primer cúbit en estado $\ket{0}$ tras aplicar $U$?
    \part Indica como podemos usar este operador para comparar si los cúbits $\ket{a}$ y $\ket{b}$ son iguales.
  \end{parts}

\end{questions}

