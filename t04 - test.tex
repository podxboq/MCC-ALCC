\begin{questions}

  \question El producto interno de $\mathbf{u} = \begin{pmatrix} 1 \\ i \end{pmatrix}$ y $\mathbf{v} = \begin{pmatrix} 2 \\ 1 \end{pmatrix}$ en $\C^2$ es:

  \begin{choices}
    \choice $2 + i$
    \CorrectChoice $2 - i$
    \choice $2 + 2i$
    \choice $3$
  \end{choices}
  \begin{solution}
    $\langle \mathbf{u}, \mathbf{v} \rangle = \conj{1} \cdot 2 + \conj{i} \cdot 1 = 2 + (-i) \cdot 1 = 2 - i$.
  \end{solution}

  \question La matriz $A = \begin{pmatrix} 1 & i \\ -i & 2 \end{pmatrix}$ es:

  \begin{choices}
    \choice Unitaria
    \CorrectChoice Hermitiana
    \choice Normal pero no hermitiana
    \choice Ninguna de las anteriores
  \end{choices}
  \begin{solution}
    $A^\dagger = \begin{pmatrix} 1 & i \\ -i & 2 \end{pmatrix}^* = \begin{pmatrix} 1 & -i \\ i & 2 \end{pmatrix}^T = \begin{pmatrix} 1 & i \\ -i & 2 \end{pmatrix} = A$, por tanto es hermitiana.
  \end{solution}

  \question Los valores propios de una matriz hermitiana son siempre:

  \begin{choices}
    \choice Complejos
    \CorrectChoice Reales
    \choice Imaginarios puros
    \choice De módulo 1
  \end{choices}
  \begin{solution}
    Es una propiedad fundamental de las matrices hermitianas: todos sus valores propios son reales, lo que las hace apropiadas para representar observables físicos.
  \end{solution}

  \question Si $U$ es un operador unitario y $\ket{\psi}$ un vector unitario, entonces $\|U\ket{\psi}\|$ es:

  \begin{choices}
    \choice $0$
    \choice Depende de $\ket{\psi}$
    \CorrectChoice $1$
    \choice $\|U\|$
  \end{choices}
  \begin{solution}
    Los operadores unitarios preservan la norma: $\|U\ket{\psi}\|^2 = \langle U\psi, U\psi \rangle = \langle \psi, U^\dagger U\psi \rangle = \langle \psi, \psi \rangle = 1$.
  \end{solution}

  \question Al medir el observable $\sigma_z$ en el estado $\ket{\psi} = \frac{1}{\sqrt{2}}(\ket{0} + \ket{1})$, la probabilidad de obtener $+1$ es:

  \begin{choices}
    \choice $0$
    \choice $1$
    \CorrectChoice $\frac{1}{2}$
    \choice $\frac{1}{\sqrt{2}}$
  \end{choices}
  \begin{solution}
    $\sigma_z$ tiene valor propio $+1$ para $\ket{0}$. La probabilidad es $P(+1) = |\braket{0}{\psi}|^2 = \left|\frac{1}{\sqrt{2}}\right|^2 = \frac{1}{2}$.
  \end{solution}

  \question Después de medir $\sigma_z$ en $\ket{\psi} = \frac{3}{5}\ket{0} + \frac{4i}{5}\ket{1}$ y obtener $-1$, el estado colapsa a:

  \begin{choices}
    \choice $\ket{0}$
    \CorrectChoice $\ket{1}$
    \choice $\frac{3}{5}\ket{0} + \frac{4i}{5}\ket{1}$
    \choice $\frac{3}{5}\ket{0}$
  \end{choices}
  \begin{solution}
    El valor propio $-1$ de $\sigma_z$ corresponde al vector propio $\ket{1}$, por tanto el estado colapsa a $\ket{1}$.
  \end{solution}

  \question El valor esperado $\langle \sigma_x \rangle$ en el estado $\ket{0}$ es:

  \begin{choices}
    \choice $1$
    \choice $-1$
    \CorrectChoice $0$
    \choice $\frac{1}{2}$
  \end{choices}
  \begin{solution}
    $\langle \sigma_x \rangle = \bra{0}\sigma_x\ket{0} = \bra{0}\ket{1} = 0$.
  \end{solution}

  \question La descomposición espectral de $\sigma_z$ es:

  \begin{choices}
    \choice $(+1)\ketbra{+}{+} + (-1)\ketbra{-}{-}$
    \CorrectChoice $(+1)\ketbra{0}{0} + (-1)\ketbra{1}{1}$
    \choice $(+1)\ketbra{1}{1} + (-1)\ketbra{0}{0}$
    \choice $\ketbra{0}{0} + \ketbra{1}{1}$
  \end{choices}
  \begin{solution}
    Los valores propios de $\sigma_z = \begin{pmatrix} 1 & 0 \\ 0 & -1 \end{pmatrix}$ son $+1$ (con $\ket{0}$) y $-1$ (con $\ket{1}$).
  \end{solution}

  \question Dos vectores propios de un operador hermitiano correspondientes a valores propios distintos son:

  \begin{choices}
    \choice Linealmente dependientes
    \choice Paralelos
    \CorrectChoice Ortogonales
    \choice Idénticos salvo fase global
  \end{choices}
  \begin{solution}
    Es una propiedad fundamental: vectores propios de un operador hermitiano con valores propios diferentes son automáticamente ortogonales.
  \end{solution}

  \question La identidad de Parseval para una base ortonormal $\{\ket{e_k}\}$ establece que:

  \begin{choices}
    \choice $\|\mathbf{v}\| = \sum_k \langle \mathbf{v}, \ket{e_k} \rangle$
    \CorrectChoice $\|\mathbf{v}\|^2 = \sum_k |\langle \mathbf{v}, \ket{e_k} \rangle|^2$
    \choice $\mathbf{v} = \sum_k \ket{e_k}$
    \choice $\sum_k \ket{e_k} = 1$
  \end{choices}
  \begin{solution}
    La identidad de Parseval expresa que la norma al cuadrado de un vector es la suma de los cuadrados de sus coeficientes de Fourier respecto a una base ortonormal.
  \end{solution}

\end{questions}