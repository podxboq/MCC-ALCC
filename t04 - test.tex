\begin{questions}

  \question El producto interno de $u = (1, i)$ y $v = (2, 1)$ en $\C^2$ es:

  \begin{choices}
    \choice $2 - i$
    \CorrectChoice $2 + i$
    \choice $2 + 2i$
    \choice $3$
  \end{choices}
  \begin{solution}
    $\langle u, v \rangle = 1 \cdot \conj{2} + i \cdot \conj{1} = 2 + i \cdot 1 = 2 + i$.
  \end{solution}

  \question La matriz $A = \begin{pmatrix} 1 & i \\ -i & 2 \end{pmatrix}$ es:

  \begin{choices}
    \choice Unitaria
    \CorrectChoice Hermitiana
    \choice Normal pero no hermitiana
    \choice Ninguna de las anteriores
  \end{choices}
  \begin{solution}
    $A^\dagger = \begin{pmatrix} 1 & i \\ -i & 2 \end{pmatrix}^* = \begin{pmatrix} 1 & -i \\ i & 2 \end{pmatrix}^T = \begin{pmatrix} 1 & i \\ -i & 2 \end{pmatrix} = A$, por tanto es hermitiana.
  \end{solution}

  \question Los valores propios de una matriz hermitiana son siempre:

  \begin{choices}
    \choice Complejos
    \CorrectChoice Reales
    \choice Imaginarios puros
    \choice De módulo 1
  \end{choices}
  \begin{solution}
    Es una propiedad fundamental de las matrices hermitianas: todos sus valores propios son reales, lo que las hace apropiadas para representar observables físicos.
  \end{solution}

  \question Si $U$ es un operador unitario y $u$ un vector unitario, entonces $\|uU\|$ es:

  \begin{choices}
    \choice $0$
    \choice Depende de $u$
    \CorrectChoice $1$
    \choice $\|U\|$
  \end{choices}
  \begin{solution}
    Los operadores unitarios preservan la norma: $\|uU\|^2 = \langle uU, uU \rangle = \langle u, U^\dagger Uu \rangle = \langle u, u \rangle = 1$.
  \end{solution}

  \question Dos vectores propios de un operador hermitiano correspondientes a valores propios distintos son:

  \begin{choices}
    \choice Linealmente dependientes
    \choice Paralelos
    \CorrectChoice Ortogonales
    \choice Idénticos
  \end{choices}
  \begin{solution}
    Es una propiedad fundamental: vectores propios de un operador hermitiano con valores propios diferentes son automáticamente ortogonales.
  \end{solution}

  \question La identidad de Parseval para una base ortonormal $\{\ket{e_k}\}$ establece que:

  \begin{choices}
    \choice $\|v\| = \sum_k \langle v, \ket{e_k} \rangle$
    \CorrectChoice $\|v\|^2 = \sum_k |\langle v, \ket{e_k} \rangle|^2$
    \choice $v = \sum_k \ket{e_k}$
    \choice $\sum_k \ket{e_k} = 1$
  \end{choices}
  \begin{solution}
    La identidad de Parseval expresa que la norma al cuadrado de un vector es la suma de los cuadrados de sus coeficientes de Fourier respecto a una base ortonormal.
  \end{solution}

  \question Si $A$ es una matriz normal, entonces:
  \begin{choices}
    \choice $A$ es diagonalizable mediante una matriz ortogonal
    \choice $A$ es hermitiana
    \CorrectChoice $A$ es diagonalizable mediante una matriz unitaria
    \choice $A$ es unitaria
  \end{choices}
  \begin{solution}
    Una matriz normal es diagonalizable mediante una matriz unitaria, es decir, existe una matriz unitaria $U$ tal que $A = UDU^\dagger$, donde $D$ es diagonal.
  \end{solution}

  \question El espacio de Hilbert $\ell^2$ consiste en:
  \begin{choices}
    \choice Todas las sucesiones de números complejos
    \choice Todas las sucesiones de números reales
    \CorrectChoice Todas las sucesiones de números complejos cuya serie de cuadrados es convergente
    \choice Todas las sucesiones de números complejos cuya serie es convergente
  \end{choices}
  \begin{solution}
    El espacio de Hilbert $\ell^2$ está formado por todas las sucesiones $(x_n)$ de números complejos tales que la serie $\sum_{n=1}^\infty |x_n|^2$ converge.
  \end{solution}

  \question En un espacio de Hilbert, una base ortonormal es:
  \begin{choices}
    \choice Un conjunto de vectores linealmente independientes
    \choice Un conjunto de vectores que generan el espacio
    \CorrectChoice Un conjunto de vectores ortogonales y normalizados que generan el espacio
    \choice Un conjunto de vectores que son todos unitarios
  \end{choices}
  \begin{solution}
    Una base ortonormal es un conjunto de vectores que son mutuamente ortogonales, cada uno de norma 1, y que generan todo el espacio.
  \end{solution}

  \question El teorema espectral para operadores hermitianos establece que:
  \begin{choices}
    \choice Todo operador hermitiano es invertible
    \choice Todo operador hermitiano tiene valores propios complejos
    \CorrectChoice Todo operador hermitiano puede ser diagonalizado mediante una base ortonormal
    \choice Todo operador hermitiano es unitario
  \end{choices}
  \begin{solution}
    El teorema espectral afirma que cualquier operador hermitiano en un espacio de Hilbertiano puede ser diagonalizado mediante una base ortonormal de vectores propios.
  \end{solution}

\end{questions}